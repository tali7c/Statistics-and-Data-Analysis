\documentclass{beamer}

\usetheme{Berlin}
\usecolortheme{Orchid}
\useoutertheme{miniframes}
\setbeamertemplate{navigation symbols}{}

\usepackage{amsmath}
\usepackage{booktabs}
\usepackage{graphicx}
\graphicspath{{../images/}}

\title[Statistics and Data Analysis]{Statistics and Data Analysis}
\subtitle{Unit 02 -- Lecture 04: Statistical Summaries for Data}
\author{Tofik Ali}
\institute{School of Computer Science, UPES Dehradun}
\date{\today}

\begin{document}

\begin{frame}
  \titlepage
  \vspace{-0.5em}
  \begin{center}
    \small \texttt{https://github.com/tali7c/Statistics-and-Data-Analysis}
  \end{center}
\end{frame}

\begin{frame}{Quick Links}
  \centering
  \hyperlink{sec:overview}{\beamerbutton{Overview}}\hspace{0.6em}
  \hyperlink{sec:tables}{\beamerbutton{Summary Tables}}\hspace{0.6em}
  \hyperlink{sec:group}{\beamerbutton{Grouped Summary}}\hspace{0.6em}
  \hyperlink{sec:demo}{\beamerbutton{Demo}}\hspace{0.6em}
  \hyperlink{sec:summary}{\beamerbutton{Summary}}
\end{frame}

\begin{frame}{Agenda}
  \tableofcontents
\end{frame}

\section{Overview}
\label{sec:overview}

\begin{frame}{Learning Outcomes}
  \begin{itemize}[<+->]
    \item Explain why we summarize data (communication and comparison)
    \item Interpret a standard summary table (count, mean, std, quartiles, min/max)
    \item Compute and interpret a five-number summary (min, Q1, median, Q3, max)
    \item Produce grouped summaries (mean/median by category)
    \item Explain what is lost when we compress data into a few numbers
  \end{itemize}
\end{frame}

\begin{frame}{Why Summaries?}
  A summary answers: ``What does the dataset look like in one page?''
  \vspace{0.6em}
  \begin{itemize}[<+->]
    \item We cannot read thousands of rows one-by-one
    \item We need quick \textbf{comparison} across groups (CSE vs ECE) or time (week 1 vs week 2)
    \item Summaries are used in reports, dashboards, and as a first step in analysis
  \end{itemize}
\end{frame}

\begin{frame}{A Standard Summary Table (Common Columns)}
  \small
  For a numeric feature (example: \texttt{final\_score}), a typical summary includes:
  \vspace{0.4em}
  \begin{itemize}
    \item \textbf{count} ($n$): number of non-missing values
    \item \textbf{mean}, \textbf{std} (sample standard deviation)
    \item \textbf{min}, \textbf{max}
    \item \textbf{25\% (Q1)}, \textbf{50\% (median)}, \textbf{75\% (Q3)}
  \end{itemize}
  \vspace{0.4em}
  \normalsize
  \textbf{Idea:} these numbers quickly describe center + spread + typical range.
\end{frame}

\section{Summary Tables}
\label{sec:tables}

\begin{frame}{Five-Number Summary (Very Important)}
  For a dataset $x$ (sorted), the five-number summary is:
  \[
    \min(x),\ Q_1,\ \mathrm{median},\ Q_3,\ \max(x)
  \]
  \begin{itemize}[<+->]
    \item It is the foundation of the boxplot
    \item It is more robust than mean/std when outliers exist
  \end{itemize}
\end{frame}

\begin{frame}{Exercise 1: Five-Number Summary}
  \small
  Dataset:
  \begin{center}
    \begin{tabular}{ccccccc}
      4 & 5 & 7 & 8 & 9 & 10 & 25
    \end{tabular}
  \end{center}
  \vspace{0.4em}
  \normalsize
  \textbf{Task:} Compute $\min$, $Q_1$, median, $Q_3$, $\max$ and IQR.
\end{frame}

\begin{frame}{Solution 1}
  Sorted: 4, 5, 7, 8, 9, 10, 25 (n=7)\\
  \vspace{0.4em}
  \begin{itemize}
    \item $\min=4,\ \max=25$
    \item median = 8
    \item lower half: 4, 5, 7 $\Rightarrow Q_1=5$
    \item upper half: 9, 10, 25 $\Rightarrow Q_3=10$
    \item IQR = $Q_3-Q_1 = 10-5 = 5$
  \end{itemize}
\end{frame}

\begin{frame}{Reading Quartiles (Interpretation)}
  Quartiles are percentiles:
  \begin{itemize}[<+->]
    \item $Q_1$ (25\%): 25\% of values are \emph{at or below} $Q_1$
    \item median (50\%): 50\% of values are at or below the median
    \item $Q_3$ (75\%): 75\% of values are at or below $Q_3$
  \end{itemize}
  \vspace{0.4em}
  \textbf{Checkpoint:} the middle 50\% of values lie between $Q_1$ and $Q_3$.
\end{frame}

\begin{frame}{Exercise 2: Interpret a Summary Row}
  \small
  Suppose a feature has summary:
  \begin{center}
    \begin{tabular}{lrrrrrrrr}
      \toprule
      & count & mean & std & min & 25\% & 50\% & 75\% & max \\
      \midrule
      final\_score & 24 & 71.0 & 12.0 & 40 & 65 & 74 & 82 & 92 \\
      \bottomrule
    \end{tabular}
  \end{center}
  \vspace{0.2em}
  \normalsize
  \textbf{Task:} What do 25\% and 75\% mean? What does std tell us?
\end{frame}

\begin{frame}{Solution 2}
  \begin{itemize}
    \item 25\% (Q1)=65: about 25\% of students scored 65 or less.
    \item 75\% (Q3)=82: about 75\% of students scored 82 or less.
    \item std=12: a typical score is about 12 points away from the mean (rough idea of spread).
  \end{itemize}
  \vspace{0.4em}
  \textbf{Important:} summaries do not show the full distribution shape.
\end{frame}

\begin{frame}{Mean vs Median in Summaries}
  \begin{itemize}[<+->]
    \item If mean $\approx$ median, distribution may be roughly symmetric
    \item If mean $>$ median, data is often right-skewed (high outliers pull mean up)
    \item If mean $<$ median, data is often left-skewed (low outliers pull mean down)
  \end{itemize}
  \vspace{0.4em}
  \textbf{Rule of thumb:} always confirm with a plot (histogram/boxplot).
\end{frame}

\begin{frame}{Exercise 3: Group Comparison (Outlier Effect)}
  \small
  Two groups:
  \begin{columns}[T]
    \begin{column}{0.48\textwidth}
      \textbf{Group A}\\
      60,\ 62,\ 65,\ 95
    \end{column}
    \begin{column}{0.48\textwidth}
      \textbf{Group B}\\
      70,\ 72,\ 73,\ 74
    \end{column}
  \end{columns}
  \vspace{0.4em}
  \normalsize
  \textbf{Task:} Compute mean and median for both groups. Which group is ``better''?
\end{frame}

\begin{frame}{Solution 3}
  \begin{itemize}
    \item Group A mean $= (60+62+65+95)/4 = 70.5$; median $= (62+65)/2 = 63.5$
    \item Group B mean $= (70+72+73+74)/4 = 72.25$; median $= (72+73)/2 = 72.5$
  \end{itemize}
  \vspace{0.4em}
  \textbf{Interpretation:} Group A has an outlier (95) that inflates its mean.
  Typical performance (median) is much lower in Group A.
\end{frame}

\section{Grouped Summary}
\label{sec:group}

\begin{frame}{Grouped Summaries (Stratification)}
  Instead of one summary for the entire dataset, we summarize \textbf{by group}:
  \begin{itemize}[<+->]
    \item mean/median \texttt{final\_score} by \texttt{program}
    \item mean attendance by section or batch
    \item revenue by category, etc.
  \end{itemize}
  \vspace{0.4em}
  \textbf{Why?} A single global average can hide important group differences.
\end{frame}

\begin{frame}{Exercise 4: Weighted Mean (Correct Overall Average)}
  \small
  Suppose:
  \begin{itemize}
    \item Section A: $n_A=10$, mean score = 70
    \item Section B: $n_B=5$, mean score = 80
  \end{itemize}
  \normalsize
  \textbf{Task:} Compute the overall mean score (all 15 students together).
\end{frame}

\begin{frame}{Solution 4}
  Overall mean is a \textbf{weighted mean}:
  \[
    \bar{x}=\frac{70\cdot 10 + 80\cdot 5}{10+5}=\frac{700+400}{15}=\frac{1100}{15}\approx 73.33
  \]
  \textbf{Note:} $(70+80)/2 = 75$ is wrong because group sizes are different.
\end{frame}

\begin{frame}{Exercise 5: Mean by Program (Small Table)}
  \small
  \begin{center}
  \begin{tabular}{lcc}
    \toprule
    Program & final\_score values \\
    \midrule
    CSE & 70,\ 75 \\
    ECE & 60,\ 65 \\
    AIML & 80,\ 85 \\
    \bottomrule
  \end{tabular}
  \end{center}
  \normalsize
  \textbf{Task:} Compute mean final\_score for each program.
\end{frame}

\begin{frame}{Solution 5}
  \begin{itemize}
    \item CSE mean = $(70+75)/2 = 72.5$
    \item ECE mean = $(60+65)/2 = 62.5$
    \item AIML mean = $(80+85)/2 = 82.5$
  \end{itemize}
  \vspace{0.4em}
  \textbf{Interpretation:} group summaries let us compare programs directly.
\end{frame}

\begin{frame}{Exercise 6: What Is Lost in a Summary Table?}
  \textbf{Question:} If we only report mean and std for a dataset, what could we miss?
  \vspace{0.6em}
  \begin{itemize}
    \item Think about outliers, skewness, and multi-modal distributions.
  \end{itemize}
\end{frame}

\begin{frame}{Solution 6}
  A small set of numbers can hide:
  \begin{itemize}
    \item outliers (one extreme value can distort mean/std)
    \item skewness (mean $\neq$ median) and long tails
    \item multi-modality (two peaks) where ``average'' is not typical
    \item subgroup differences (one group high, one group low)
  \end{itemize}
  \textbf{Takeaway:} summaries are useful, but always validate with plots.
\end{frame}

\section{Demo}
\label{sec:demo}

\begin{frame}{Mini Demo (Python)}
  Run from the lecture folder:
  \begin{center}
    \texttt{python demo/statistical\_summaries\_demo.py}
  \end{center}
  \vspace{0.4em}
  Uses:
  \begin{itemize}
    \item \texttt{data/student\_summary.csv}
  \end{itemize}
  Outputs:
  \begin{itemize}
    \item \texttt{data/overall\_summary.csv}
    \item \texttt{data/summary\_by\_program.csv}
    \item \texttt{images/mean\_final\_by\_program.png} (if matplotlib is installed)
  \end{itemize}
\end{frame}

\begin{frame}{Demo Output (Example Plot)}
  \begin{center}
  \IfFileExists{../images/mean_final_by_program.png}{
    \includegraphics[width=0.92\linewidth]{mean_final_by_program.png}
  }{
    \small (Run demo to generate: \texttt{mean\_final\_by\_program.png})
  }
  \end{center}
\end{frame}

\section{Summary}
\label{sec:summary}

\begin{frame}{Summary}
  \begin{itemize}[<+->]
    \item Summary tables compress data into center + spread + typical range (quartiles)
    \item Five-number summary is robust and supports boxplot thinking
    \item Grouped summaries (stratification) reveal differences hidden by global averages
    \item Summaries can hide distribution shape and outliers $\Rightarrow$ use plots too
  \end{itemize}
  \vspace{0.6em}
  \textbf{Exit question:} Why is a weighted mean needed when groups have different sizes?
\end{frame}

\end{document}

