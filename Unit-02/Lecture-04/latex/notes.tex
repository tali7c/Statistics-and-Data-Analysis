\documentclass[11pt]{article}
\usepackage[utf8]{inputenc}
\usepackage[T1]{fontenc}
\usepackage{geometry}
\usepackage{amsmath}
\usepackage{booktabs}
\usepackage{hyperref}
\geometry{margin=1in}

\title{Statistics and Data Analysis\\Unit 02 -- Lecture 04 Notes\\Statistical Summaries for Data}
\author{Tofik Ali}
\date{\today}

\begin{document}
\maketitle

\section*{What You Will Learn (Beginner-Friendly)}
In many real problems, the dataset is too large to read row-by-row. So we ask:
\begin{enumerate}
  \item What is a ``typical'' value of a feature (center)?
  \item How much do values vary (spread)?
  \item How can we compare groups (CSE vs ECE) or time (week 1 vs week 2)?
\end{enumerate}

By the end of this lecture, you should be able to:
\begin{itemize}
  \item interpret a common summary table (count, mean, std, quartiles, min/max),
  \item compute and interpret the five-number summary,
  \item produce and interpret grouped summaries,
  \item and explain what information is lost when we compress data into a few numbers.
\end{itemize}

\section*{1. Why Summaries Are Needed}
If a dataset has 10,000 rows, you cannot communicate it in a report by printing the raw table.
A \textbf{summary} compresses the data into a small set of numbers that still capture the important story.

\subsection*{1.1 ``Comparison across groups or time''}
Summaries let us compare:
\begin{itemize}
  \item \textbf{groups}: e.g., mean/median final score in CSE vs ECE,
  \item \textbf{time periods}: e.g., average weekly sales in January vs February.
\end{itemize}
Instead of comparing 10,000 raw values, we compare a few summary values.

\section*{2. Standard Summary Table (What Each Column Means)}
For a numeric variable (say \texttt{final\_score}), a typical summary contains:
\begin{itemize}
  \item \textbf{count} ($n$): how many values exist (after excluding missing values),
  \item \textbf{mean}: arithmetic average (sensitive to outliers),
  \item \textbf{std}: sample standard deviation (typical distance from mean),
  \item \textbf{min/max}: extremes,
  \item \textbf{25\%}, \textbf{50\%}, \textbf{75\%}: quartiles ($Q_1$, median, $Q_3$).
\end{itemize}

\paragraph{Important warning.}
A summary table does \textbf{not} show the full distribution shape.
Two datasets can have similar mean/std but look very different (skewed vs bimodal).

\section*{3. Five-Number Summary}
The five-number summary is:
\[
\min(x),\ Q_1,\ \text{median},\ Q_3,\ \max(x)
\]
It is used to create a \textbf{boxplot} and is often more robust than mean/std.

\subsection*{3.1 Quartile interpretation}
\begin{itemize}
  \item $Q_1$ (25th percentile): about 25\% of values are at or below $Q_1$.
  \item Median (50th percentile): about 50\% of values are at or below the median.
  \item $Q_3$ (75th percentile): about 75\% of values are at or below $Q_3$.
\end{itemize}
So, the middle 50\% of the data lies between $Q_1$ and $Q_3$.

\subsection*{Exercise 1 (solution)}
Dataset: 4, 5, 7, 8, 9, 10, 25\\
Five-number summary:
\begin{itemize}
  \item $\min=4,\ \max=25$
  \item median = 8
  \item lower half (4, 5, 7) $\Rightarrow Q_1=5$
  \item upper half (9, 10, 25) $\Rightarrow Q_3=10$
  \item IQR = $Q_3-Q_1=5$
\end{itemize}

\section*{4. Mean vs Median (Quick Skewness Clue)}
Mean and median are both measures of center, but:
\begin{itemize}
  \item mean uses all values and is pulled by outliers,
  \item median uses only ordering and is robust to outliers.
\end{itemize}

\paragraph{Rule of thumb (not a proof).}
\begin{itemize}
  \item mean $\approx$ median: might be roughly symmetric,
  \item mean $>$ median: often right-skewed,
  \item mean $<$ median: often left-skewed.
\end{itemize}
Always confirm with a plot.

\subsection*{Exercise 2 (solution)}
Given summary:
\begin{itemize}
  \item 25\% = 65 means about 25\% scored 65 or less.
  \item 75\% = 82 means about 75\% scored 82 or less.
  \item std = 12 means a typical score is roughly 12 points away from the mean (spread).
\end{itemize}

\subsection*{Exercise 3 (solution)}
Group A: 60, 62, 65, 95\\
Group B: 70, 72, 73, 74\\
\begin{itemize}
  \item Group A mean = 70.5; median = 63.5
  \item Group B mean = 72.25; median = 72.5
\end{itemize}
Interpretation: Group A has an outlier (95) that inflates its mean.
Typical performance (median) is much higher in Group B.

\section*{5. Grouped Summaries (Stratification)}
Sometimes one global summary is misleading.
We compute summaries \textbf{within groups} (by program, section, gender, etc.).

\subsection*{5.1 Weighted mean (why it matters)}
If groups have different sizes, the overall mean must weight by group size.

\subsection*{Exercise 4 (solution)}
Section A: $n_A=10$, mean=70; Section B: $n_B=5$, mean=80\\
\[
\bar{x}=\frac{70\cdot 10 + 80\cdot 5}{15}\approx 73.33
\]
Simple average of means (75) is incorrect here.

\subsection*{Exercise 5 (solution)}
Means:
\begin{itemize}
  \item CSE: 72.5
  \item ECE: 62.5
  \item AIML: 82.5
\end{itemize}

\subsection*{Exercise 6 (solution)}
When we only report mean and std, we can miss:
\begin{itemize}
  \item outliers and skewness,
  \item multi-modality (two peaks),
  \item differences between subgroups.
\end{itemize}

\section*{6. Mini Demo (Python)}
Run from the lecture folder:
\begin{verbatim}
python demo/statistical_summaries_demo.py
\end{verbatim}

It uses \texttt{data/student\_summary.csv} and prints:
\begin{itemize}
  \item an overall summary per numeric column,
  \item a grouped summary of \texttt{final\_score} by \texttt{program}.
\end{itemize}

It also saves:
\begin{itemize}
  \item \texttt{data/overall\_summary.csv}
  \item \texttt{data/summary\_by\_program.csv}
  \item \texttt{images/mean\_final\_by\_program.png} (if matplotlib is installed)
\end{itemize}

\section*{References}
\begin{itemize}
  \item Montgomery, D. C., \& Runger, G. C. \textit{Applied Statistics and Probability for Engineers}, Wiley, 7th ed., 2020.
  \item McKinney, W. \textit{Python for Data Analysis}, O'Reilly, 2022.
  \item Freedman, D., Pisani, R., \& Purves, R. \textit{Statistics}, W. W. Norton, 4th ed., 2007.
\end{itemize}

\end{document}

