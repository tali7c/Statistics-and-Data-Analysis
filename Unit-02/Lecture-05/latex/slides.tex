\documentclass{beamer}

\usetheme{Berlin}
\usecolortheme{Orchid}
\useoutertheme{miniframes}
\setbeamertemplate{navigation symbols}{}

\usepackage{amsmath}
\usepackage{booktabs}
\usepackage{graphicx}
\graphicspath{{../images/}}

\title[Statistics and Data Analysis]{Statistics and Data Analysis}
\subtitle{Unit 02 -- Lecture 05: Dimensional Summaries and Distributions}
\author{Tofik Ali}
\institute{School of Computer Science, UPES Dehradun}
\date{\today}

\begin{document}

\begin{frame}
  \titlepage
  \vspace{-0.5em}
  \begin{center}
    \small \texttt{https://github.com/tali7c/Statistics-and-Data-Analysis}
  \end{center}
\end{frame}

\begin{frame}{Quick Links}
  \centering
  \hyperlink{sec:dim}{\beamerbutton{Dimensional Summary}}\hspace{0.6em}
  \hyperlink{sec:shape}{\beamerbutton{Distribution Shape}}\hspace{0.6em}
  \hyperlink{sec:outliers}{\beamerbutton{Outliers}}\hspace{0.6em}
  \hyperlink{sec:demo}{\beamerbutton{Demo}}\hspace{0.6em}
  \hyperlink{sec:summary}{\beamerbutton{Summary}}
\end{frame}

\begin{frame}{Agenda}
  \tableofcontents
\end{frame}

\section{Dimensional Summaries}
\label{sec:dim}

\begin{frame}{Learning Outcomes}
  \begin{itemize}[<+->]
    \item Explain what a dimensional (per-feature) summary is
    \item Use mean/median/quartiles to get a quick idea of distribution shape
    \item Recognize common distribution shapes (symmetric, skewed, bimodal)
    \item Explain why distribution shape matters for interpretation and method choice
  \end{itemize}
\end{frame}

\begin{frame}{What is a Dimensional Summary?}
  A \textbf{dimensional summary} reports statistics \emph{for each feature}:
  \vspace{0.6em}
  \begin{itemize}[<+->]
    \item One dataset can have many columns (income, commute, sleep, score)
    \item We summarize each column separately: center + spread + quartiles
    \item This helps us quickly spot features that behave differently
  \end{itemize}
\end{frame}

\begin{frame}{Exercise 1: Mean vs Median (Shape Clue)}
  \small
  Three features (summary only):
  \begin{center}
  \begin{tabular}{lrr}
    \toprule
    Feature & Mean & Median \\
    \midrule
    A & 50 & 50 \\
    B & 80 & 60 \\
    C & 60 & 75 \\
    \bottomrule
  \end{tabular}
  \end{center}
  \normalsize
  \textbf{Task:} Which looks symmetric? Which is right-skewed? Which is left-skewed?
\end{frame}

\begin{frame}{Solution 1}
  \begin{itemize}
    \item A: mean $\approx$ median $\Rightarrow$ roughly symmetric (likely)
    \item B: mean $>$ median $\Rightarrow$ right-skewed (high values pull mean up)
    \item C: mean $<$ median $\Rightarrow$ left-skewed (low values pull mean down)
  \end{itemize}
  \vspace{0.4em}
  \textbf{Reminder:} this is a clue, not a guarantee. Confirm with a plot.
\end{frame}

\section{Distribution Shapes}
\label{sec:shape}

\begin{frame}{Common Distribution Shapes}
  \begin{itemize}[<+->]
    \item \textbf{Symmetric (approximately normal)}: mean $\approx$ median; bell-like
    \item \textbf{Right-skewed}: long tail to the right; mean $>$ median (often income)
    \item \textbf{Left-skewed}: long tail to the left; mean $<$ median (often scores near 100)
    \item \textbf{Bimodal}: two peaks (two sub-populations mixed together)
  \end{itemize}
\end{frame}

\begin{frame}{Exercise 2: Bimodal Example (Mean Can Be Misleading)}
  \small
  Commute times (minutes):
  \begin{center}
    \begin{tabular}{cccccccc}
      10 & 12 & 15 & 18 & 20 & 60 & 65 & 70
    \end{tabular}
  \end{center}
  \vspace{0.4em}
  \normalsize
  \textbf{Task:} Compute mean and median. Is the mean a ``typical'' commute time here?
\end{frame}

\begin{frame}{Solution 2}
  Sorted data: 10, 12, 15, 18, 20, 60, 65, 70
  \vspace{0.4em}
  \begin{itemize}
    \item mean $=270/8 = 33.75$
    \item median $=(18+20)/2 = 19$
  \end{itemize}
  \vspace{0.4em}
  \textbf{Interpretation:} the mean (33.75) is not typical because there are two clusters
  (short commuters and long commuters). The ``middle'' has almost no data.
\end{frame}

\begin{frame}{Exercise 3: Identify the Shape (Quick Reasoning)}
  \textbf{Question:} Which scenario is most likely right-skewed?
  \vspace{0.6em}
  \begin{enumerate}
    \item Heights of students
    \item Daily income of individuals
    \item Measurement error around zero
  \end{enumerate}
\end{frame}

\begin{frame}{Solution 3}
  \textbf{Daily income} is most likely right-skewed:
  most values are moderate, with a small number of very high values (long right tail).
\end{frame}

\section{Outliers and Robust Summaries}
\label{sec:outliers}

\begin{frame}{Outliers and Robustness}
  \begin{itemize}[<+->]
    \item Outliers can strongly affect mean and standard deviation
    \item Median and IQR are more robust (less sensitive to extremes)
    \item Always ask: error or true extreme?
  \end{itemize}
\end{frame}

\begin{frame}{Exercise 4: IQR Outlier Check}
  \small
  Dataset:
  \begin{center}
    \begin{tabular}{ccccccc}
      10 & 12 & 13 & 14 & 15 & 16 & 40
    \end{tabular}
  \end{center}
  \vspace{0.4em}
  \normalsize
  \textbf{Task:} Compute $Q_1$, $Q_3$, IQR, and check if 40 is an outlier using the IQR fences.
\end{frame}

\begin{frame}{Solution 4}
  Sorted: 10, 12, 13, 14, 15, 16, 40\\
  median = 14; lower half (10,12,13) $\Rightarrow Q_1=12$; upper half (15,16,40) $\Rightarrow Q_3=16$\\
  \vspace{0.4em}
  \begin{itemize}
    \item IQR = $16-12=4$
    \item Upper fence = $Q_3 + 1.5\cdot IQR = 16 + 6 = 22$
    \item Since $40>22$, 40 is an outlier by the IQR rule.
  \end{itemize}
\end{frame}

\begin{frame}{Robust Options (When Skew/Outliers Exist)}
  \begin{itemize}[<+->]
    \item Report median and IQR instead of mean and std
    \item Use trimmed mean (remove small \% of extremes)
    \item Transform the feature (e.g., $\log(1+x)$ for right-skewed positive values)
  \end{itemize}
\end{frame}

\begin{frame}{Exercise 5: Which Summary Would You Report?}
  \textbf{Question:} For income data (right-skewed), which pair is usually better?
  \vspace{0.6em}
  \begin{enumerate}
    \item mean + standard deviation
    \item median + IQR
  \end{enumerate}
\end{frame}

\begin{frame}{Solution 5}
  \textbf{Median + IQR} is usually better for right-skewed income:
  it represents the typical person and is less distorted by a few extremely high incomes.
\end{frame}

\begin{frame}{Exercise 6: Dimensional Summary (Tiny Table)}
  \small
  \begin{center}
  \begin{tabular}{lrr}
    \toprule
    Student & hours & score \\
    \midrule
    1 & 2 & 50 \\
    2 & 4 & 60 \\
    3 & 6 & 70 \\
    \bottomrule
  \end{tabular}
  \end{center}
  \normalsize
  \textbf{Task:} Compute mean(hours) and mean(score). This is a 2-feature dimensional summary.
\end{frame}

\begin{frame}{Solution 6}
  \begin{itemize}
    \item mean(hours) $=(2+4+6)/3 = 4$
    \item mean(score) $=(50+60+70)/3 = 60$
  \end{itemize}
\end{frame}

\section{Demo}
\label{sec:demo}

\begin{frame}{Mini Demo (Python)}
  Run from the lecture folder:
  \begin{center}
    \texttt{python demo/dimensional\_summaries\_distributions\_demo.py}
  \end{center}
  \vspace{0.4em}
  Uses:
  \begin{itemize}
    \item \texttt{data/multi\_feature\_distributions.csv}
  \end{itemize}
  Outputs:
  \begin{itemize}
    \item prints a per-feature summary (mean, median, std, quartiles, skewness)
    \item saves \texttt{images/hists\_grid.png} (if matplotlib is installed)
  \end{itemize}
\end{frame}

\begin{frame}{Demo Output (Histogram Grid)}
  \begin{center}
  \IfFileExists{../images/hists_grid.png}{
    \includegraphics[width=0.97\linewidth]{hists_grid.png}
  }{
    \small (Run demo to generate: \texttt{hists\_grid.png})
  }
  \end{center}
\end{frame}

\section{Summary}
\label{sec:summary}

\begin{frame}{Summary}
  \begin{itemize}[<+->]
    \item Dimensional summaries describe each feature (column) separately
    \item Mean vs median gives a fast clue about skewness; confirm with plots
    \item Bimodal data can make the mean ``not typical''
    \item For skew/outliers, use robust summaries (median/IQR) or transformations
  \end{itemize}
  \vspace{0.6em}
  \textbf{Exit question:} Why can the mean be misleading for a bimodal distribution?
\end{frame}

\end{document}

