\documentclass[11pt]{article}
\usepackage[utf8]{inputenc}
\usepackage[T1]{fontenc}
\usepackage{geometry}
\usepackage{amsmath}
\usepackage{listings}
\usepackage{xcolor}
\usepackage{graphicx}
\graphicspath{{../images/}}
\usepackage{amssymb}
\usepackage{booktabs}
\usepackage{hyperref}
\geometry{margin=1in}

\title{Statistics and Data Analysis\\Unit 02 -- Lecture 05 Notes\\Dimensional Summaries and Distributions}
\author{Tofik Ali}
\date{\today}

\begin{document}
\maketitle

\section*{What You Will Learn (Beginner-Friendly)}
In earlier lectures we learned measures of center (mean/median/mode) and spread (IQR, variance, std).
In this lecture we scale up that idea:
\begin{itemize}
  \item A dataset usually has many columns (dimensions/features).
  \item Each feature can have a different distribution shape.
  \item We need per-feature (dimensional) summaries and distribution thinking.
\end{itemize}

By the end, you should be able to:
\begin{itemize}
  \item compute and interpret per-feature summaries,
  \item recognize common distribution shapes (symmetric, skewed, bimodal),
  \item explain why shape matters for choosing the right summary statistic.
\end{itemize}

\section*{1. Dimensional (Per-Feature) Summaries}
\subsection*{1.1 Definition}
A \textbf{dimensional summary} means summarizing each feature/column separately using:
\begin{itemize}
  \item center: mean/median,
  \item spread: std/IQR,
  \item range: min/max,
  \item quartiles: $Q_1, Q_3$.
\end{itemize}

\paragraph{Why it helps.}
If you have 20 columns, you can quickly identify:
\begin{itemize}
  \item which features have large variability,
  \item which features have outliers,
  \item which features are likely skewed,
  \item which features might need transformation (like log).
\end{itemize}

\subsection*{Exercise 1 (solution)}
Given:
\begin{itemize}
  \item A: mean=50, median=50 $\Rightarrow$ roughly symmetric (likely)
  \item B: mean=80, median=60 $\Rightarrow$ right-skewed (high values pull mean upward)
  \item C: mean=60, median=75 $\Rightarrow$ left-skewed (low values pull mean downward)
\end{itemize}
This rule is a \textbf{heuristic}. Always confirm using a histogram or boxplot.

\section*{2. Distribution Shapes}
\subsection*{2.1 Symmetric distributions}
For symmetric distributions (often approximately normal):
\begin{itemize}
  \item mean $\approx$ median,
  \item left and right tails are similar,
  \item mean and std are often reasonable summaries.
\end{itemize}

\subsection*{2.2 Right-skewed distributions}
Right-skewed means there is a long tail on the right.
Example: income.
Most people have moderate incomes, but a few people have very high incomes.
This pulls the mean upward, so mean $>$ median is common.

\subsection*{2.3 Left-skewed distributions}
Left-skewed means there is a long tail on the left (a few very low values).
Example: marks on an easy exam where many students score very high.
Mean $<$ median can occur.

\subsection*{2.4 Bimodal distributions}
Bimodal means two peaks. This often happens when the data mixes two sub-populations.
Example: commute times might be short for hostel students and long for day scholars.

\subsection*{Exercise 2 (solution)}
Commute times: 10, 12, 15, 18, 20, 60, 65, 70\\
Mean:
\[
\frac{270}{8}=33.75
\]
Median:
\[
\frac{18+20}{2}=19
\]
Interpretation: the mean is not typical because the data has two clusters and very few values around 34.

\subsection*{Exercise 3 (solution)}
Daily income is most likely right-skewed.

\section*{3. Outliers and Robust Summaries}
\subsection*{3.1 Outliers}
Outliers are values that are unusually far from the rest.
They can be:
\begin{itemize}
  \item errors (wrong entry, sensor fault),
  \item or true extremes (rare but real cases).
\end{itemize}
So we should detect them and think, not blindly delete them.

\subsection*{3.2 IQR rule (recap)}
Compute:
\[
\mathrm{IQR}=Q_3-Q_1
\]
Fences:
\[
Q_1 - 1.5\mathrm{IQR}, \quad Q_3 + 1.5\mathrm{IQR}
\]
Values outside fences are flagged as potential outliers.

\subsection*{Exercise 4 (solution)}
Dataset: 10, 12, 13, 14, 15, 16, 40\\
Median = 14; $Q_1=12$; $Q_3=16$; IQR = 4\\
Upper fence = 16 + 1.5(4) = 22\\
So 40 is an outlier by the IQR rule.

\subsection*{Exercise 5 (solution)}
For income (right-skewed), median + IQR is usually better than mean + std because it is robust.

\subsection*{Exercise 6 (solution)}
Mean(hours) = 4 and mean(score) = 60.

\section*{4. Mini Demo (Python)}
Run from the lecture folder:
\begin{verbatim}
python demo/dimensional_summaries_distributions_demo.py
\end{verbatim}

It uses \texttt{data/multi\_feature\_distributions.csv} and prints a dimensional summary:
\begin{itemize}
  \item mean, median, std, min/max, quartiles, and a simple skewness estimate.
\end{itemize}
If matplotlib is installed, it also saves \texttt{images/hists\_grid.png}.

\section*{References}
\begin{itemize}
  \item Montgomery, D. C., \& Runger, G. C. \textit{Applied Statistics and Probability for Engineers}, Wiley, 7th ed., 2020.
  \item Freedman, D., Pisani, R., \& Purves, R. \textit{Statistics}, W. W. Norton, 4th ed., 2007.
  \item McKinney, W. \textit{Python for Data Analysis}, O'Reilly, 2022.
\end{itemize}





% BEGIN SLIDE APPENDIX (AUTO-GENERATED)
\clearpage
\section*{Appendix: Slide Deck Content (Reference)}
\noindent The material below is a reference copy of the slide deck content. Exercise solutions are explained in the main notes where applicable.

\subsection*{Title Slide}
\titlepage
  \vspace{-0.5em}
  \begin{center}
    \small \texttt{https://github.com/tali7c/Statistics-and-Data-Analysis}
  \end{center}
\subsection*{Quick Links}
\centering
  \textbf{Dimensional Summary}\hspace{0.6em}
  \textbf{Distribution Shape}\hspace{0.6em}
  \textbf{Outliers}\hspace{0.6em}
  \textbf{Demo}\hspace{0.6em}
  \textbf{Summary}
\subsection*{Agenda}
\begin{itemize}
  \item Dimensional Summaries
  \item Distribution Shapes
  \item Outliers and Robust Summaries
  \item Demo
  \item Summary
\end{itemize}
\subsection*{Learning Outcomes}
\begin{itemize}
    \item Explain what a dimensional (per-feature) summary is
    \item Use mean/median/quartiles to get a quick idea of distribution shape
    \item Recognize common distribution shapes (symmetric, skewed, bimodal)
    \item Explain why distribution shape matters for interpretation and method choice
  \end{itemize}
\subsection*{What is a Dimensional Summary?}
A \textbf{dimensional summary} reports statistics \emph{for each feature}:
  \vspace{0.6em}
  \begin{itemize}
    \item One dataset can have many columns (income, commute, sleep, score)
    \item We summarize each column separately: center + spread + quartiles
    \item This helps us quickly spot features that behave differently
  \end{itemize}
\subsection*{Exercise 1: Mean vs Median (Shape Clue)}
\small
  Three features (summary only):
  \begin{center}
  \begin{tabular}{lrr}
    \toprule
    Feature & Mean & Median \\
    \midrule
    A & 50 & 50 \\
    B & 80 & 60 \\
    C & 60 & 75 \\
    \bottomrule
  \end{tabular}
  \end{center}
  \normalsize
  \textbf{Task:} Which looks symmetric? Which is right-skewed? Which is left-skewed?
\subsection*{Solution 1}
\begin{itemize}
    \item A: mean $\approx$ median $\Rightarrow$ roughly symmetric (likely)
    \item B: mean $>$ median $\Rightarrow$ right-skewed (high values pull mean up)
    \item C: mean $<$ median $\Rightarrow$ left-skewed (low values pull mean down)
  \end{itemize}
  \vspace{0.4em}
  \textbf{Reminder:} this is a clue, not a guarantee. Confirm with a plot.
\subsection*{Common Distribution Shapes}
\begin{itemize}
    \item \textbf{Symmetric (approximately normal)}: mean $\approx$ median; bell-like
    \item \textbf{Right-skewed}: long tail to the right; mean $>$ median (often income)
    \item \textbf{Left-skewed}: long tail to the left; mean $<$ median (often scores near 100)
    \item \textbf{Bimodal}: two peaks (two sub-populations mixed together)
  \end{itemize}
\subsection*{Exercise 2: Bimodal Example (Mean Can Be Misleading)}
\small
  Commute times (minutes):
  \begin{center}
    \begin{tabular}{cccccccc}
      10 & 12 & 15 & 18 & 20 & 60 & 65 & 70
    \end{tabular}
  \end{center}
  \vspace{0.4em}
  \normalsize
  \textbf{Task:} Compute mean and median. Is the mean a ``typical'' commute time here?
\subsection*{Solution 2}
Sorted data: 10, 12, 15, 18, 20, 60, 65, 70
  \vspace{0.4em}
  \begin{itemize}
    \item mean $=270/8 = 33.75$
    \item median $=(18+20)/2 = 19$
  \end{itemize}
  \vspace{0.4em}
  \textbf{Interpretation:} the mean (33.75) is not typical because there are two clusters
  (short commuters and long commuters). The ``middle'' has almost no data.
\subsection*{Exercise 3: Identify the Shape (Quick Reasoning)}
\textbf{Question:} Which scenario is most likely right-skewed?
  \vspace{0.6em}
  \begin{enumerate}
    \item Heights of students
    \item Daily income of individuals
    \item Measurement error around zero
  \end{enumerate}
\subsection*{Solution 3}
\textbf{Daily income} is most likely right-skewed:
  most values are moderate, with a small number of very high values (long right tail).
\subsection*{Outliers and Robustness}
\begin{itemize}
    \item Outliers can strongly affect mean and standard deviation
    \item Median and IQR are more robust (less sensitive to extremes)
    \item Always ask: error or true extreme?
  \end{itemize}
\subsection*{Exercise 4: IQR Outlier Check}
\small
  Dataset:
  \begin{center}
    \begin{tabular}{ccccccc}
      10 & 12 & 13 & 14 & 15 & 16 & 40
    \end{tabular}
  \end{center}
  \vspace{0.4em}
  \normalsize
  \textbf{Task:} Compute $Q_1$, $Q_3$, IQR, and check if 40 is an outlier using the IQR fences.
\subsection*{Solution 4}
Sorted: 10, 12, 13, 14, 15, 16, 40\\
  median = 14; lower half (10,12,13) $\Rightarrow Q_1=12$; upper half (15,16,40) $\Rightarrow Q_3=16$\\
  \vspace{0.4em}
  \begin{itemize}
    \item IQR = $16-12=4$
    \item Upper fence = $Q_3 + 1.5\cdot IQR = 16 + 6 = 22$
    \item Since $40>22$, 40 is an outlier by the IQR rule.
  \end{itemize}
\subsection*{Robust Options (When Skew/Outliers Exist)}
\begin{itemize}
    \item Report median and IQR instead of mean and std
    \item Use trimmed mean (remove small \% of extremes)
    \item Transform the feature (e.g., $\log(1+x)$ for right-skewed positive values)
  \end{itemize}
\subsection*{Exercise 5: Which Summary Would You Report?}
\textbf{Question:} For income data (right-skewed), which pair is usually better?
  \vspace{0.6em}
  \begin{enumerate}
    \item mean + standard deviation
    \item median + IQR
  \end{enumerate}
\subsection*{Solution 5}
\textbf{Median + IQR} is usually better for right-skewed income:
  it represents the typical person and is less distorted by a few extremely high incomes.
\subsection*{Exercise 6: Dimensional Summary (Tiny Table)}
\small
  \begin{center}
  \begin{tabular}{lrr}
    \toprule
    Student & hours & score \\
    \midrule
    1 & 2 & 50 \\
    2 & 4 & 60 \\
    3 & 6 & 70 \\
    \bottomrule
  \end{tabular}
  \end{center}
  \normalsize
  \textbf{Task:} Compute mean(hours) and mean(score). This is a 2-feature dimensional summary.
\subsection*{Solution 6}
\begin{itemize}
    \item mean(hours) $=(2+4+6)/3 = 4$
    \item mean(score) $=(50+60+70)/3 = 60$
  \end{itemize}
\subsection*{Mini Demo (Python)}
Run from the lecture folder:
  \begin{center}
    \texttt{python demo/dimensional\_summaries\_distributions\_demo.py}
  \end{center}
  \vspace{0.4em}
  Uses:
  \begin{itemize}
    \item \texttt{data/multi\_feature\_distributions.csv}
  \end{itemize}
  Outputs:
  \begin{itemize}
    \item prints a per-feature summary (mean, median, std, quartiles, skewness)
    \item saves \texttt{images/hists\_grid.png} (if matplotlib is installed)
  \end{itemize}
\subsection*{Demo Output (Histogram Grid)}
\begin{center}
  \IfFileExists{../images/hists_grid.png}{
    \includegraphics[width=0.97\linewidth]{hists_grid.png}
  }{
    \small (Run demo to generate: \texttt{hists\_grid.png})
  }
  \end{center}
\subsection*{Summary}
\begin{itemize}
    \item Dimensional summaries describe each feature (column) separately
    \item Mean vs median gives a fast clue about skewness; confirm with plots
    \item Bimodal data can make the mean ``not typical''
    \item For skew/outliers, use robust summaries (median/IQR) or transformations
  \end{itemize}
  \vspace{0.6em}
  \textbf{Exit question:} Why can the mean be misleading for a bimodal distribution?
% END SLIDE APPENDIX (AUTO-GENERATED)

\end{document}

