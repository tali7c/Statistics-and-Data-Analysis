\documentclass{beamer}

\usetheme{Berlin}
\usecolortheme{Orchid}
\useoutertheme{miniframes}
\setbeamertemplate{navigation symbols}{}

\usepackage{amsmath}
\usepackage{booktabs}
\usepackage{graphicx}

\title[Statistics and Data Analysis]{Statistics and Data Analysis}
\subtitle{Unit 02 -- Lecture 03: Correlation, Skewness, Kurtosis}
\author{Tofik Ali}
\institute{School of Computer Science, UPES Dehradun}
\date{\today}

\begin{document}

\begin{frame}
  \titlepage
  \vspace{-0.5em}
  \begin{center}
    \small \texttt{https://github.com/tali7c/Statistics-and-Data-Analysis}
  \end{center}
\end{frame}

\begin{frame}{Quick Links}
  \centering
  \hyperlink{sec:corr}{\beamerbutton{Correlation}}\hspace{0.6em}
  \hyperlink{sec:skew}{\beamerbutton{Skewness}}\hspace{0.6em}
  \hyperlink{sec:kurt}{\beamerbutton{Kurtosis}}\hspace{0.6em}
  \hyperlink{sec:demo}{\beamerbutton{Demo}}\hspace{0.6em}
  \hyperlink{sec:summary}{\beamerbutton{Summary}}
\end{frame}

\begin{frame}{Agenda}
  \tableofcontents
\end{frame}

\section{Overview}

\begin{frame}{Learning Outcomes}
  \begin{itemize}[<+->]
    \item Explain correlation and compute Pearson correlation $r$
    \item Relate covariance and correlation, and explain why correlation is scale-free
    \item Identify common pitfalls: outliers, non-linearity, and correlation vs causation
    \item Interpret skewness (right/left skew) and kurtosis (tail heaviness)
  \end{itemize}
\end{frame}

\begin{frame}{From Covariance to Correlation (Recap)}
  \begin{itemize}[<+->]
    \item Covariance tells direction of joint variation, but it depends on units
    \item Correlation standardizes covariance to a unitless number in $[-1,1]$
    \item That makes correlation easier to compare across different datasets
  \end{itemize}
\end{frame}

\section{Correlation}
\label{sec:corr}

\begin{frame}{What is Correlation?}
  Correlation measures \textbf{linear association} between two variables.
  \vspace{0.6em}
  \begin{itemize}[<+->]
    \item $r>0$: as $x$ increases, $y$ tends to increase
    \item $r<0$: as $x$ increases, $y$ tends to decrease
    \item $r\approx 0$: no strong \emph{linear} pattern (could still be non-linear)
  \end{itemize}
\end{frame}

\begin{frame}{Pearson Correlation (Formula)}
  For paired data $(x_i,y_i)$:
  \[
    r = \frac{\sum_{i=1}^{n}(x_i-\bar{x})(y_i-\bar{y})}
             {\sqrt{\sum_{i=1}^{n}(x_i-\bar{x})^2}\sqrt{\sum_{i=1}^{n}(y_i-\bar{y})^2}}
  \]
  \begin{itemize}[<+->]
    \item Always between $-1$ and $1$
    \item Unitless (no units)
    \item Sensitive to outliers
  \end{itemize}
\end{frame}

\begin{frame}{Correlation vs Covariance}
  \[
    r = \frac{s_{xy}}{s_x s_y}
  \]
  \begin{itemize}[<+->]
    \item $s_{xy}$: sample covariance (Lecture 02)
    \item $s_x, s_y$: sample standard deviations
    \item Scaling a variable (changing units) does \textbf{not} change $r$
  \end{itemize}
\end{frame}

\begin{frame}{Interpreting $r$ (Rule of Thumb)}
  \begin{center}
    \begin{tabular}{ll}
      \toprule
      $|r|$ range & Common description \\
      \midrule
      0.00--0.19 & very weak \\
      0.20--0.39 & weak \\
      0.40--0.59 & moderate \\
      0.60--0.79 & strong \\
      0.80--1.00 & very strong \\
      \bottomrule
    \end{tabular}
  \end{center}
  \vspace{0.4em}
  \textit{Always confirm with a scatter plot.}
\end{frame}

\begin{frame}{Exercise 1: Pearson Correlation (Positive)}
  \small
  Hours studied vs Score:
  \vspace{0.4em}
  \begin{center}
    \begin{tabular}{cccccc}
      \toprule
      Hours ($x$) & 1 & 2 & 3 & 4 & 5 \\
      Score ($y$) & 52 & 55 & 60 & 65 & 68 \\
      \bottomrule
    \end{tabular}
  \end{center}
  \vspace{0.4em}
  \normalsize
  Given: $\bar{x}=3$, $\bar{y}=60$, $\sum (x-\bar{x})(y-\bar{y})=42$, $\sum (x-\bar{x})^2=10$, $\sum (y-\bar{y})^2=178$.\\
  \textbf{Task:} Compute $r$ and interpret it.
\end{frame}

\begin{frame}{Solution 1}
  \[
    r = \frac{42}{\sqrt{10}\sqrt{178}} = \frac{42}{\sqrt{1780}} \approx 0.9955
  \]
  \textbf{Interpretation:} Very strong positive linear association between hours and score.
\end{frame}

\begin{frame}{Exercise 2: Pearson Correlation (Negative)}
  \small
  Price vs Demand:
  \vspace{0.4em}
  \begin{center}
    \begin{tabular}{cccccc}
      \toprule
      Price ($x$) & 1 & 2 & 3 & 4 & 5 \\
      Demand ($y$) & 80 & 70 & 60 & 50 & 40 \\
      \bottomrule
    \end{tabular}
  \end{center}
  \vspace{0.4em}
  \normalsize
  \textbf{Task:} Compute $r$. What does the sign mean?
\end{frame}

\begin{frame}{Solution 2}
  Here $y = 90 - 10x$ is a perfect decreasing line.\\
  \[
    r = -1
  \]
  \textbf{Interpretation:} Perfect negative linear relationship.
\end{frame}

\begin{frame}{Exercise 3: $r=0$ Does Not Mean ``No Relationship''}
  \small
  Consider:
  \[
    x=[-2,-1,0,1,2],\quad y=x^2=[4,1,0,1,4]
  \]
  \normalsize
  \textbf{Task:} Compute $r$. Is there a relationship between $x$ and $y$?
\end{frame}

\begin{frame}{Solution 3}
  \small
  $\bar{x}=0$, $\bar{y}=2$. The numerator $\sum (x-\bar{x})(y-\bar{y})=0$, so:
  \[
    r = 0
  \]
  \normalsize
  \textbf{Key point:} $r=0$ means no \emph{linear} association; here the relationship is strong but non-linear.
\end{frame}

\begin{frame}{Correlation $\neq$ Causation}
  \begin{itemize}[<+->]
    \item Correlation only says ``they move together'' (linearly)
    \item A third variable can cause both (confounding)
    \item Sometimes correlation is accidental (spurious)
    \item Use domain knowledge + experiments/causal reasoning to claim causation
  \end{itemize}
\end{frame}

\begin{frame}{Exercise 4: Interpret a Correlation Claim}
  \small
  ``Ice cream sales and drowning incidents are positively correlated.''\\
  \vspace{0.6em}
  Which statement is most correct?
  \begin{enumerate}
    \item Ice cream causes drowning.
    \item Drowning causes ice cream sales.
    \item Both may increase due to a third factor (e.g., temperature/season).
  \end{enumerate}
\end{frame}

\begin{frame}{Solution 4}
  \textbf{Correct: (3)}. A confounder like hot weather can increase both swimming (risk) and ice cream sales.
\end{frame}

\section{Skewness}
\label{sec:skew}

\begin{frame}{Skewness (Distribution Asymmetry)}
  Skewness describes the \textbf{direction of the tail}.
  \vspace{0.6em}
  \begin{itemize}[<+->]
    \item \textbf{Right-skewed (positive):} long tail to the right (few very large values)
    \item \textbf{Left-skewed (negative):} long tail to the left (few very small values)
    \item Symmetric: tails are similar on both sides
  \end{itemize}
\end{frame}

\begin{frame}{Mean vs Median vs Mode (Heuristic)}
  \begin{itemize}[<+->]
    \item Right-skewed: mean $>$ median $>$ mode
    \item Left-skewed: mean $<$ median $<$ mode
    \item Symmetric: mean $\approx$ median $\approx$ mode
  \end{itemize}
  \vspace{0.4em}
  \textit{Reason: the mean is pulled toward the long tail.}
\end{frame}

\begin{frame}{Moment Skewness (One Common Formula)}
  Let $m_k$ be the $k$th central moment (divide by $n$):
  \[
    m_k = \frac{1}{n}\sum_{i=1}^{n}(x_i-\bar{x})^k
  \]
  Moment skewness:
  \[
    g_1 = \frac{m_3}{m_2^{3/2}}
  \]
  \begin{itemize}[<+->]
    \item $g_1>0$ right-skewed, $g_1<0$ left-skewed
    \item Different software may use small-sample corrections
  \end{itemize}
\end{frame}

\begin{frame}{Exercise 5: Identify Skewness Direction}
  \small
  Dataset A (Income, INR thousands):
  \begin{center}
    \begin{tabular}{ccccccccc}
      20 & 22 & 23 & 24 & 25 & 26 & 27 & 28 & 60
    \end{tabular}
  \end{center}
  Dataset B (Scores):
  \begin{center}
    \begin{tabular}{cccccccccc}
      50 & 80 & 85 & 88 & 90 & 92 & 93 & 94 & 95 & 96
    \end{tabular}
  \end{center}
  \normalsize
  \textbf{Task:} For each dataset, decide if it is right-skewed or left-skewed. Predict whether mean $>$ median or mean $<$ median.
\end{frame}

\begin{frame}{Solution 5}
  \begin{itemize}
    \item Dataset A: one large value (60) creates a right tail $\Rightarrow$ right-skewed, mean $>$ median
    \item Dataset B: one small value (50) creates a left tail $\Rightarrow$ left-skewed, mean $<$ median
  \end{itemize}
\end{frame}

\begin{frame}{Exercise 6: Compute Moment Skewness}
  \small
  For Dataset A (income), suppose:
  \[
    \bar{x}=28.33,\quad m_2=130.89,\quad m_3=3404.07
  \]
  \normalsize
  \textbf{Task:} Compute $g_1=\dfrac{m_3}{m_2^{3/2}}$ and interpret the sign.
\end{frame}

\begin{frame}{Solution 6}
  \[
    g_1 = \frac{3404.07}{(130.89)^{3/2}} \approx 2.27
  \]
  \textbf{Interpretation:} Positive and large $\Rightarrow$ strongly right-skewed distribution.
\end{frame}

\section{Kurtosis}
\label{sec:kurt}

\begin{frame}{Kurtosis (Tail Heaviness)}
  Kurtosis summarizes how heavy the tails are (and how often extreme values appear).
  \vspace{0.6em}
  \begin{itemize}[<+->]
    \item Often reported as \textbf{excess kurtosis} = kurtosis $-3$
    \item Normal distribution has excess kurtosis $0$
    \item Positive excess: heavier tails; negative excess: lighter tails
  \end{itemize}
\end{frame}

\begin{frame}{Moment Kurtosis (One Common Formula)}
  Moment kurtosis:
  \[
    g_2 = \frac{m_4}{m_2^2}
  \]
  Excess kurtosis:
  \[
    \text{Excess} = g_2 - 3
  \]
  \begin{itemize}[<+->]
    \item If excess $>0$, more extreme values than normal (heavy tails)
    \item If excess $<0$, fewer extremes than normal (light tails)
  \end{itemize}
\end{frame}

\begin{frame}{Exercise 7: Excess Kurtosis (Small Symmetric Data)}
  \small
  Dataset: 1, 2, 3, 4, 5. \\
  Mean = 3, deviations: $[-2,-1,0,1,2]$.
  \vspace{0.4em}

  \normalsize
  \textbf{Task:} Compute $m_2$, $m_4$, then $g_2$ and excess kurtosis.
\end{frame}

\begin{frame}{Solution 7}
  \small
  \[
    m_2 = \frac{4+1+0+1+4}{5} = 2
  \]
  \[
    m_4 = \frac{16+1+0+1+16}{5} = \frac{34}{5}=6.8
  \]
  \[
    g_2 = \frac{6.8}{2^2} = 1.7,\quad \text{excess} = 1.7-3=-1.3
  \]
  \normalsize
  \textbf{Interpretation:} Negative excess $\Rightarrow$ lighter tails than normal (platykurtic).
\end{frame}

\begin{frame}{Exercise 8: Excess Kurtosis (Income Example)}
  \small
  For the income dataset (Exercise 6), suppose:
  \[
    m_2 = 130.89,\quad m_4 = 112590.30
  \]
  \normalsize
  \textbf{Task:} Compute $g_2=\dfrac{m_4}{m_2^2}$ and excess kurtosis. Interpret.
\end{frame}

\begin{frame}{Solution 8}
  \[
    g_2 \approx \frac{112590.30}{(130.89)^2} \approx 6.57,\quad \text{excess}\approx 3.57
  \]
  \textbf{Interpretation:} Large positive excess $\Rightarrow$ heavy tails / extreme values (outliers).
\end{frame}

\begin{frame}{Common Pitfalls (Skewness \& Kurtosis)}
  \begin{itemize}[<+->]
    \item Small samples can give unstable skewness/kurtosis values
    \item Different formulas exist (bias corrections), so values may differ across tools
    \item Always verify with plots (histogram/boxplot) and context
  \end{itemize}
\end{frame}

\section{Demo}
\label{sec:demo}

\begin{frame}{Mini Demo (Python)}
  Run:
  \begin{center}
    \texttt{python demo/correlation\_skew\_kurt\_demo.py}
  \end{center}
  \vspace{0.4em}
  What it does:
  \begin{itemize}
    \item Computes Pearson correlation for three paired datasets
    \item Prints correlation matrix for \texttt{data/student\_metrics.csv}
    \item Computes moment skewness and excess kurtosis for example distributions
    \item (Optional) Saves plots to \texttt{images/} if matplotlib is installed
  \end{itemize}
\end{frame}

\section{Summary}
\label{sec:summary}

\begin{frame}{Summary}
  \begin{itemize}[<+->]
    \item Correlation standardizes covariance to $[-1,1]$ and measures linear association
    \item $r=0$ does not mean independence; it only indicates no linear relation
    \item Skewness describes tail direction; mean is pulled toward the tail
    \item Excess kurtosis relates to tail heaviness and outliers
  \end{itemize}
  \vspace{0.6em}
  \textbf{Exit question:} Give one real-life example where correlation might be misleading and explain why.
\end{frame}

\end{document}
