\documentclass[11pt]{article}
\usepackage[utf8]{inputenc}
\usepackage[T1]{fontenc}
\usepackage{geometry}
\usepackage{amsmath}
\usepackage{listings}
\usepackage{xcolor}
\usepackage{graphicx}
\graphicspath{{../images/}}
\usepackage{amssymb}
\usepackage{booktabs}
\usepackage{hyperref}
\geometry{margin=1in}

\title{Statistics and Data Analysis\\Unit 02 -- Lecture 06 Notes\\In-Class Activity: Summarization + Interpretation}
\author{Tofik Ali}
\date{\today}

\begin{document}
\maketitle

\section*{Purpose of This Activity}
This activity is designed to help you practice descriptive statistics end-to-end.
You will not only compute numbers, but also interpret what they mean and write a short conclusion.

\paragraph{Repository.}
\texttt{https://github.com/tali7c/Statistics-and-Data-Analysis}

\section*{Learning Outcomes}
After this activity, you should be able to:
\begin{enumerate}
  \item compute central tendency (mean, median, mode),
  \item compute dispersion (range, IQR, sample variance, sample std),
  \item compute Pearson correlation and interpret its sign and magnitude,
  \item compare groups using grouped summaries,
  \item write clear insights and limitations from statistical summaries.
\end{enumerate}

\section*{1. Dataset Description}
File: \texttt{data/activity\_student\_dataset.csv}

\subsection*{1.1 What the columns mean}
\begin{itemize}
  \item \texttt{program}: CSE / ECE / AIML (categorical).
  \item \texttt{attendance\_pct}: attendance percentage (numeric).
  \item \texttt{study\_hours}: study hours (numeric).
  \item \texttt{social\_media\_hours}: social media hours (numeric).
  \item \texttt{final\_score}: final score (numeric).
\end{itemize}

\section*{2. Tasks (What You Must Compute)}

\subsection*{Task 1: Central tendency}
For \texttt{final\_score}, compute:
\begin{itemize}
  \item mean: $\bar{x}=\frac{1}{n}\sum x_i$
  \item median (middle value after sorting)
  \item mode (most frequent value)
\end{itemize}

\paragraph{Interpretation tip.}
Compare mean vs median:
\begin{itemize}
  \item mean $>$ median often hints right skew,
  \item mean $<$ median often hints left skew.
\end{itemize}
This is only a clue; confirm with a plot.

\subsection*{Task 2: Dispersion}
For \texttt{final\_score}, compute:
\begin{itemize}
  \item range = $\max - \min$
  \item quartiles $Q_1, Q_3$ and IQR = $Q_3-Q_1$
  \item sample variance and sample standard deviation:
  \[
    s^2=\frac{1}{n-1}\sum (x_i-\bar{x})^2,\quad s=\sqrt{s^2}
  \]
\end{itemize}

\paragraph{Robustness.}
IQR is more robust to outliers than standard deviation.

\subsection*{Task 3: Correlation}
Compute Pearson correlation:
\[
r=\frac{\sum (x_i-\bar{x})(y_i-\bar{y})}{\sqrt{\sum (x_i-\bar{x})^2}\sqrt{\sum (y_i-\bar{y})^2}}
\]
for:
\begin{itemize}
  \item (\texttt{study\_hours}, \texttt{final\_score})
  \item (\texttt{social\_media\_hours}, \texttt{final\_score})
\end{itemize}

\paragraph{Very important.}
Correlation measures \textbf{linear association}. It does \textbf{not} prove causation.

\subsection*{Task 4: Grouped summaries}
Group by \texttt{program} and compute:
\begin{itemize}
  \item mean and median of \texttt{final\_score}
  \item mean of \texttt{attendance\_pct}
\end{itemize}

\subsection*{Task 5: Write-up}
Write:
\begin{itemize}
  \item 3 insights (what the dataset suggests)
  \item 2 limitations (why your conclusions might not generalize)
\end{itemize}

\section*{3. Expected Results (From the Provided Solution)}
If you run the provided solution script (next section), you should obtain:

\subsection*{3.1 Overall results}
\begin{itemize}
  \item mean(final\_score) = 65.50
  \item median(final\_score) = 65.50
  \item mode(final\_score) = 60
  \item $Q_1=60,\ Q_3=72,\ IQR=12$
  \item sample std(final\_score) $\approx$ 14.11
\end{itemize}

\subsection*{3.2 Correlations}
\begin{itemize}
  \item corr(study\_hours, final\_score) $\approx$ 0.5190 (moderate positive linear association)
  \item corr(social\_media\_hours, final\_score) $\approx$ -0.9771 (strong negative linear association)
\end{itemize}

\paragraph{Interpretation note.}
Even a strong correlation does not prove that one variable causes the other.
Other variables (motivation, prior knowledge, teaching quality) could be involved.

\subsection*{3.3 By program (mean final score)}
\begin{itemize}
  \item AIML: mean $\approx$ 75.83
  \item CSE: mean $\approx$ 59.17
  \item ECE: mean $\approx$ 61.50
\end{itemize}

\section*{4. Provided Solution Script (Mini Demo)}
After you attempt the activity yourself, run:
\begin{verbatim}
python demo/activity_solution.py
\end{verbatim}

It will save:
\begin{itemize}
  \item \texttt{data/overall\_results.csv}
  \item \texttt{data/summary\_by\_program.csv}
  \item plots in \texttt{images/} (scatter plots, bar chart, histogram)
\end{itemize}

\section*{5. Common Mistakes}
\begin{itemize}
  \item \textbf{Mixing up correlation and causation.} Correlation is not proof.
  \item \textbf{Using only the mean.} Always look at median and IQR too.
  \item \textbf{Ignoring groups.} A global average can hide group differences.
  \item \textbf{Not writing limitations.} Every conclusion must mention what could be wrong.
  \item \textbf{No plots.} Tables alone can hide skewness and outliers.
\end{itemize}

\section*{6. Extension Questions (Optional)}
If you finish early, try any two:
\begin{enumerate}
  \item Identify the lowest and highest scoring students and comment on their study/social media hours.
  \item Compare correlation within each program separately (CSE vs ECE vs AIML).
  \item Replace mean with median for program comparison and see if ranking changes.
  \item Use IQR fences to flag potential outliers in \texttt{final\_score}.
\end{enumerate}

\section*{References}
\begin{itemize}
  \item Montgomery, D. C., \& Runger, G. C. \textit{Applied Statistics and Probability for Engineers}, Wiley, 7th ed., 2020.
  \item Freedman, D., Pisani, R., \& Purves, R. \textit{Statistics}, W. W. Norton, 4th ed., 2007.
  \item McKinney, W. \textit{Python for Data Analysis}, O'Reilly, 2022.
\end{itemize}





% BEGIN SLIDE APPENDIX (AUTO-GENERATED)
\clearpage
\section*{Appendix: Slide Deck Content (Reference)}
\noindent The material below is a reference copy of the slide deck content. Exercise solutions are explained in the main notes where applicable.

\subsection*{Title Slide}
\titlepage
  \vspace{-0.5em}
  \begin{center}
    \small \texttt{https://github.com/tali7c/Statistics-and-Data-Analysis}
  \end{center}
\subsection*{Quick Links}
\centering
  \textbf{Activity}\hspace{0.6em}
  \textbf{Tasks}\hspace{0.6em}
  \textbf{Solution}\hspace{0.6em}
  \textbf{Wrap-up}
\subsection*{Agenda}
\begin{itemize}
  \item Activity Brief
  \item Tasks and Deliverables
  \item Solution and Discussion
  \item Wrap-up
\end{itemize}
\subsection*{What We Will Do Today}
You will complete a small end-to-end descriptive statistics task:
  \begin{itemize}
    \item compute central tendency (mean/median/mode)
    \item compute dispersion (range, IQR, variance, std)
    \item compute correlation for two variable pairs
    \item create grouped summaries by program
    \item write 3 insights + 2 limitations
  \end{itemize}
\subsection*{Time Plan (55 minutes)}
\begin{itemize}
    \item 10 min: attendance + setup
    \item 25 min: activity work (in pairs)
    \item 10 min: discussion (compare results and assumptions)
    \item 5 min: wrap-up + exit question
  \end{itemize}
\subsection*{Dataset}
\small
  File: \texttt{data/activity\_student\_dataset.csv}
  \vspace{0.4em}

  Columns:
  \begin{itemize}
    \item \texttt{program} (CSE/ECE/AIML)
    \item \texttt{attendance\_pct}, \texttt{study\_hours}, \texttt{social\_media\_hours} (numeric)
    \item \texttt{final\_score} (numeric)
  \end{itemize}

  \vspace{0.4em}
  \normalsize
  \textbf{Goal:} summarize and interpret what these numbers suggest.
\subsection*{Task 1: Central Tendency (5 minutes)}
Compute for \texttt{final\_score}:
  \begin{itemize}
    \item mean
    \item median
    \item mode
  \end{itemize}
  \vspace{0.4em}
  \textbf{Checkpoint:} If mean $\ne$ median, what does that hint about skewness?
\subsection*{Task 2: Dispersion (10 minutes)}
Compute for \texttt{final\_score}:
  \begin{itemize}
    \item range
    \item $Q_1$, $Q_3$, IQR
    \item sample variance and sample standard deviation
  \end{itemize}
  \vspace{0.4em}
  \textbf{Checkpoint:} Which is more robust to outliers: std or IQR?
\subsection*{Task 3: Correlation (10 minutes)}
Compute Pearson correlation:
  \begin{itemize}
    \item corr(\texttt{study\_hours}, \texttt{final\_score})
    \item corr(\texttt{social\_media\_hours}, \texttt{final\_score})
  \end{itemize}
  \vspace{0.4em}
  \textbf{Checkpoint:} Correlation measures what kind of relationship?
\subsection*{Task 4: Grouped Summaries (10 minutes)}
Group by \texttt{program} and compute:
  \begin{itemize}
    \item mean and median of \texttt{final\_score}
    \item mean attendance
  \end{itemize}
  \vspace{0.4em}
  \textbf{Checkpoint:} Which program looks strongest by mean? By median?
\subsection*{Task 5: Write Insights + Limitations (5 minutes)}
Deliver:
  \begin{itemize}
    \item 3 insights (what the numbers suggest)
    \item 2 limitations (why the conclusion may be weak)
  \end{itemize}
  \vspace{0.4em}
  \textbf{Example limitation:} small dataset $\Rightarrow$ results may not generalize.
\subsection*{Final Deliverables (Submit/Show)}
\begin{itemize}
    \item computed values (central tendency, dispersion, correlations)
    \item 1 grouped summary table by program
    \item 2 scatter plots OR 1 histogram + 1 bar chart
    \item short write-up (3 insights + 2 limitations)
  \end{itemize}
\subsection*{Solution Script (Python)}
After attempting yourself, run:
  \begin{center}
    \texttt{python demo/activity\_solution.py}
  \end{center}
  Outputs:
  \begin{itemize}
    \item \texttt{data/overall\_results.csv}
    \item \texttt{data/summary\_by\_program.csv}
    \item plots in \texttt{images/} (scatter, bar, histogram)
  \end{itemize}
\subsection*{Expected Key Results (Overall)}
\small
  \begin{center}
    \begin{tabular}{lr}
      \toprule
      Statistic & Value \\
      \midrule
      Mean(final\_score) & 65.50 \\
      Median(final\_score) & 65.50 \\
      Mode(final\_score) & 60 \\
      Range(final\_score) & 65 \\
      $Q_1$ / $Q_3$ & 60 / 72 \\
      IQR & 12 \\
      Sample std (final\_score) & 14.11 \\
      \bottomrule
    \end{tabular}
  \end{center}
\subsection*{Expected Key Results (Correlation)}
\small
  \begin{center}
    \begin{tabular}{lr}
      \toprule
      Pair & Pearson $r$ \\
      \midrule
      (study\_hours, final\_score) & 0.5190 \\
      (social\_media\_hours, final\_score) & -0.9771 \\
      \bottomrule
    \end{tabular}
  \end{center}
  \vspace{0.2em}
  \normalsize
  \textbf{Question:} Does this prove causation? Why/why not?
\subsection*{Expected Key Results (By Program)}
\small
  \begin{center}
    \begin{tabular}{lrr}
      \toprule
      Program & Mean(final\_score) & Median(final\_score) \\
      \midrule
      CSE & 59.17 & 62.50 \\
      ECE & 61.50 & 62.00 \\
      AIML & 75.83 & 75.00 \\
      \bottomrule
    \end{tabular}
  \end{center}
\subsection*{Example Plots}
\textbf{Hours vs Final Score}
      \begin{center}
      \IfFileExists{../images/hours_vs_final_score.png}{
        \includegraphics[width=\linewidth]{hours_vs_final_score.png}
      }{
        \small (Run solution to generate \texttt{hours\_vs\_final\_score.png})
      }
      \end{center}
    
    
      \textbf{Mean Final by Program}
      \begin{center}
      \IfFileExists{../images/mean_final_by_program.png}{
        \includegraphics[width=\linewidth]{mean_final_by_program.png}
      }{
        \small (Run solution to generate \texttt{mean\_final\_by\_program.png})
      }
      \end{center}
\subsection*{Wrap-up}
\begin{itemize}
    \item A good descriptive analysis combines: center + spread + relationships + group comparisons
    \item Always state assumptions and limitations
    \item Never confuse correlation with causation
  \end{itemize}
  \vspace{0.6em}
  \textbf{Exit question:} Which statistic changed your interpretation the most (mean, median, IQR, or correlation)? Why?
% END SLIDE APPENDIX (AUTO-GENERATED)

\end{document}

