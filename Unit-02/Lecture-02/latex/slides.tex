\documentclass{beamer}

\usetheme{Berlin}
\usecolortheme{Orchid}
\useoutertheme{miniframes}
\setbeamertemplate{navigation symbols}{}

\usepackage{amsmath}
\usepackage{booktabs}
\usepackage{graphicx}

\title[Statistics and Data Analysis]{Statistics and Data Analysis}
\subtitle{Unit 02 -- Lecture 02: Dispersion and Covariance}
\author{Tofik Ali}
\institute{School of Computer Science, UPES Dehradun}
\date{\today}

\begin{document}

\begin{frame}
  \titlepage
  \vspace{-0.5em}
  \begin{center}
    \small \texttt{https://github.com/tali7c/Statistics-and-Data-Analysis}
  \end{center}
\end{frame}

\begin{frame}{Quick Links}
  \centering
  \hyperlink{sec:overview}{\beamerbutton{Overview}}\hspace{0.6em}
  \hyperlink{sec:dispersion}{\beamerbutton{Dispersion}}\hspace{0.6em}
  \hyperlink{sec:covariance}{\beamerbutton{Covariance}}\hspace{0.6em}
  \hyperlink{sec:demo}{\beamerbutton{Demo}}\hspace{0.6em}
  \hyperlink{sec:summary}{\beamerbutton{Summary}}
\end{frame}

\begin{frame}{Agenda}
  \tableofcontents
\end{frame}

\section{Overview}
\label{sec:overview}

\begin{frame}{Learning Outcomes}
  \begin{itemize}[<+->]
    \item Explain why dispersion is needed (beyond the mean/median/mode)
    \item Compute and interpret range and IQR
    \item Compute sample variance and standard deviation
    \item Compute coefficient of variation (CV) and simple z-scores
    \item Use the IQR rule to flag potential outliers
    \item Define covariance and interpret its sign and units
  \end{itemize}
\end{frame}

\begin{frame}{Warm-up: Same Mean, Different Spread}
  Two datasets can have the same mean but different variability:
  \vspace{0.6em}
  \begin{columns}[T]
    \begin{column}{0.48\textwidth}
      \textbf{Dataset A}\\
      \begin{center}
        \begin{tabular}{ccc}
          10 & 15 & 20
        \end{tabular}
      \end{center}
    \end{column}
    \begin{column}{0.48\textwidth}
      \textbf{Dataset B}\\
      \begin{center}
        \begin{tabular}{ccc}
          14 & 15 & 16
        \end{tabular}
      \end{center}
    \end{column}
  \end{columns}
  \vspace{0.6em}
  \textbf{Checkpoint:} Which dataset is more variable? Why?
\end{frame}

\section{Dispersion}
\label{sec:dispersion}

\begin{frame}{What is Dispersion?}
  Dispersion describes how spread out the data is around the center.
  \vspace{0.6em}
  \begin{itemize}[<+->]
    \item \textbf{Range:} max -- min
    \item \textbf{IQR:} spread of the middle 50\% (Q3 -- Q1)
    \item \textbf{Variance/SD:} average squared deviation / typical deviation
  \end{itemize}
\end{frame}

\begin{frame}{Range and Interquartile Range (IQR)}
  \begin{itemize}[<+->]
    \item \textbf{Range} = $\max(x) - \min(x)$ (very sensitive to outliers)
    \item \textbf{Quartiles} split sorted data into quarters
    \item \textbf{IQR} = Q3 -- Q1 (more robust than range)
  \end{itemize}
\end{frame}

\begin{frame}{Exercise 1: Range and IQR}
  \small
  Dataset (Scores):
  \vspace{0.4em}
  \begin{center}
    \begin{tabular}{cccccc}
      11 & 13 & 15 & 15 & 17 & 19
    \end{tabular}
  \end{center}
  \vspace{0.4em}
  \normalsize
  \textbf{Task:} Compute Range, Q1, Q3, and IQR.
\end{frame}

\begin{frame}{Solution 1}
  Sorted data: 11, 13, 15, 15, 17, 19
  \vspace{0.4em}
  \begin{itemize}
    \item Range = $19 - 11 = 8$
    \item Lower half: 11, 13, 15 $\Rightarrow$ Q1 = 13
    \item Upper half: 15, 17, 19 $\Rightarrow$ Q3 = 17
    \item IQR = $17 - 13 = 4$
  \end{itemize}
\end{frame}

\begin{frame}{Variance and Standard Deviation}
  \begin{itemize}[<+->]
    \item Variance measures average squared deviation from the mean
    \item Standard deviation is the square root of variance
    \item \textbf{Units:} variance has squared units; SD has original units
  \end{itemize}
\end{frame}

\begin{frame}{Sample Variance (Why $n-1$?)}
  \[
    s^2 = \frac{1}{n-1}\sum_{i=1}^{n}(x_i-\bar{x})^2
  \]
  \begin{itemize}[<+->]
    \item Using $n-1$ helps correct bias when estimating population variance
    \item We lose one ``degree of freedom'' because $\bar{x}$ is estimated from the data
  \end{itemize}
\end{frame}

\begin{frame}{Exercise 2: Sample Variance and SD}
  \small
  Use the same dataset: 11, 13, 15, 15, 17, 19 \\
  Mean: $\bar{x}=15$
  \vspace{0.4em}
  \begin{itemize}
    \item Compute $s^2$ and $s$
    \item Hint: $\sum (x_i-\bar{x})^2 = 40$
  \end{itemize}
\end{frame}

\begin{frame}{Solution 2}
  \begin{itemize}
    \item $n=6$
    \item $s^2 = \frac{40}{6-1} = \frac{40}{5} = 8$
    \item $s = \sqrt{8} \approx 2.83$
  \end{itemize}
  \vspace{0.5em}
  \textbf{Interpretation:} A typical score is about 2.8 points away from the mean.
\end{frame}

\begin{frame}{Coefficient of Variation (CV)}
  CV compares spread \emph{relative to the mean}:
  \[
    \mathrm{CV} = \frac{s}{\bar{x}} \times 100\%
  \]
  \begin{itemize}[<+->]
    \item Unitless (percentage) $\Rightarrow$ useful to compare variability across different scales
    \item Works best when the mean is positive and not near zero
  \end{itemize}
\end{frame}

\begin{frame}{Exercise 3: Coefficient of Variation}
  \small
  Use the same dataset: 11, 13, 15, 15, 17, 19 \\
  From Exercise 2: $\bar{x}=15$, $s \approx 2.83$
  \vspace{0.4em}

  \normalsize
  \textbf{Task:} Compute CV (in \%).
\end{frame}

\begin{frame}{Solution 3}
  \[
    \mathrm{CV} = \frac{2.83}{15}\times 100\% \approx 18.9\%
  \]
  \textbf{Interpretation:} The typical spread is about 19\% of the mean.
\end{frame}

\begin{frame}{Standardization (z-score)}
  A z-score tells how many standard deviations a value is from the mean:
  \[
    z = \frac{x - \bar{x}}{s}
  \]
  \begin{itemize}[<+->]
    \item $z>0$: value is above the mean; $z<0$: below the mean
    \item $|z|$ near 2 or 3 often indicates an unusually large/small value
  \end{itemize}
\end{frame}

\begin{frame}{Exercise 4: z-score}
  \small
  Using $\bar{x}=15$ and $s \approx 2.83$, compute the z-score of $x=19$.
  \vspace{0.4em}

  \normalsize
  \textbf{Task:} Compute $z$ and interpret it in one sentence.
\end{frame}

\begin{frame}{Solution 4}
  \[
    z = \frac{19-15}{2.83} \approx 1.41
  \]
  \textbf{Interpretation:} 19 is about 1.4 standard deviations above the mean.
\end{frame}

\begin{frame}{Outlier Detection (IQR Rule)}
  A common rule to flag potential outliers uses \textbf{fences}:
  \[
    \text{Lower fence} = Q_1 - 1.5\times \mathrm{IQR},\quad
    \text{Upper fence} = Q_3 + 1.5\times \mathrm{IQR}
  \]
  \vspace{0.4em}
  Values outside the fences are possible outliers.
\end{frame}

\begin{frame}{Exercise 5: IQR Outlier Check}
  \small
  Monthly income (INR thousands):
  \vspace{0.4em}
  \begin{center}
    \begin{tabular}{ccccccccc}
      20 & 22 & 23 & 24 & 25 & 26 & 27 & 28 & 60
    \end{tabular}
  \end{center}
  \vspace{0.4em}
  \normalsize
  \textbf{Task:} Compute $Q_1$, $Q_3$, IQR and decide if 60 is an outlier.
\end{frame}

\begin{frame}{Solution 5}
  \small
  Median = 25 (since $n=9$).
  \begin{itemize}
    \item Lower half: 20, 22, 23, 24 $\Rightarrow Q_1 = (22+23)/2 = 22.5$
    \item Upper half: 26, 27, 28, 60 $\Rightarrow Q_3 = (27+28)/2 = 27.5$
    \item IQR = $27.5 - 22.5 = 5$
    \item Fences: $22.5-7.5=15$ and $27.5+7.5=35$
  \end{itemize}
  \textbf{Conclusion:} $60>35$ $\Rightarrow$ 60 is an outlier.
\end{frame}

\begin{frame}{Think-Pair-Share (2 minutes)}
  \textbf{Prompt:} Suppose you have a dataset with a few extreme outliers.\\
  Which spread measure would you report first: \textbf{IQR} or \textbf{SD}? Why?
\end{frame}

\section{Covariance}
\label{sec:covariance}

\begin{frame}{What is Covariance?}
  Covariance measures how two variables vary together.
  \vspace{0.6em}
  \begin{itemize}[<+->]
    \item Positive covariance: both tend to increase together
    \item Negative covariance: one increases while the other decreases
    \item Near zero: no \emph{linear} co-variation (could still be non-linear)
  \end{itemize}
\end{frame}

\begin{frame}{Sample Covariance}
  For paired data $(x_i,y_i)$:
  \[
    s_{xy} = \frac{1}{n-1}\sum_{i=1}^{n}(x_i-\bar{x})(y_i-\bar{y})
  \]
  \begin{itemize}[<+->]
    \item \textbf{Units:} (units of $x$) $\times$ (units of $y$)
    \item Covariance depends on scale (change units $\Rightarrow$ covariance changes)
  \end{itemize}
\end{frame}

\begin{frame}{Exercise 6: Covariance (Positive)}
  \small
  Dataset (Hours studied vs Score):
  \vspace{0.4em}
  \begin{center}
    \begin{tabular}{cccccc}
      \toprule
      Hours ($x$) & 1 & 2 & 3 & 4 & 5 \\
      Score ($y$) & 52 & 55 & 60 & 65 & 68 \\
      \bottomrule
    \end{tabular}
  \end{center}
  \vspace{0.4em}
  \normalsize
  \textbf{Task:} Compute sample covariance $s_{xy}$. Interpret the sign.\\
  (Means: $\bar{x}=3$, $\bar{y}=60$)
\end{frame}

\begin{frame}{Solution 6}
  Deviations: $x-\bar{x} = [-2,-1,0,1,2]$ \\
  Deviations: $y-\bar{y} = [-8,-5,0,5,8]$
  \vspace{0.4em}
  \[
    \sum (x_i-\bar{x})(y_i-\bar{y}) = 42
  \]
  \[
    s_{xy} = \frac{42}{5-1} = 10.5
  \]
  \textbf{Interpretation:} Positive covariance $\Rightarrow$ as hours increase, scores tend to increase.
\end{frame}

\begin{frame}{Scale Dependence (Important)}
  \begin{itemize}[<+->]
    \item If we multiply $y$ by 10 (change units), covariance multiplies by 10
    \item So covariance is hard to compare across different unit scales
    \item Next lecture: \textbf{correlation} standardizes covariance to $[-1,1]$
  \end{itemize}
\end{frame}

\begin{frame}{Exercise 7: Covariance (Negative)}
  \small
  Dataset (Price vs Demand):
  \vspace{0.4em}
  \begin{center}
    \begin{tabular}{cccccc}
      \toprule
      Price ($x$) & 1 & 2 & 3 & 4 & 5 \\
      Demand ($y$) & 80 & 70 & 60 & 50 & 40 \\
      \bottomrule
    \end{tabular}
  \end{center}
  \vspace{0.4em}
  \normalsize
  \textbf{Task:} Compute $s_{xy}$ and interpret the sign.\\
  (Means: $\bar{x}=3$, $\bar{y}=60$)
\end{frame}

\begin{frame}{Solution 7}
  Deviations: $x-\bar{x} = [-2,-1,0,1,2]$ \\
  Deviations: $y-\bar{y} = [20,10,0,-10,-20]$
  \vspace{0.4em}
  \[
    \sum (x_i-\bar{x})(y_i-\bar{y}) = -100
  \]
  \[
    s_{xy} = \frac{-100}{5-1} = -25
  \]
  \textbf{Interpretation:} Negative covariance $\Rightarrow$ higher price tends to reduce demand.
\end{frame}

\begin{frame}{Exercise 8: Unit Change and Covariance}
  \small
  From Exercise 6, covariance (hours, score) is $10.5$. \\
  Suppose we measure time in minutes: $x' = 60x$.
  \vspace{0.4em}

  \normalsize
  \textbf{Task:} What is covariance of $(x',y)$? (No re-calculation needed.)
\end{frame}

\begin{frame}{Solution 8}
  Property: $\mathrm{cov}(aX,Y)=a\,\mathrm{cov}(X,Y)$.
  \[
    \mathrm{cov}(60X,Y)=60\times 10.5 = 630
  \]
  \textbf{Interpretation:} Units changed $\Rightarrow$ covariance changed (scale-dependent).
\end{frame}

\begin{frame}{Exercise 9: Covariance 0 but Strong Relationship}
  \small
  Consider:
  \[
    x = [-2,-1,0,1,2],\quad y=x^2 = [4,1,0,1,4]
  \]
  \normalsize
  \textbf{Task:} Compute sample covariance. Are $x$ and $y$ independent?
\end{frame}

\begin{frame}{Solution 9}
  \small
  $\bar{x}=0$, $\bar{y}=2$. \\
  Products $(x-\bar{x})(y-\bar{y})$: $-4, 1, 0, -1, 4$ $\Rightarrow$ sum $=0$.
  \[
    s_{xy} = \frac{0}{5-1} = 0
  \]
  \normalsize
  \textbf{Key point:} Covariance 0 $\neq$ independence (here $y$ is determined by $x$).
\end{frame}

\section{Demo}
\label{sec:demo}

\begin{frame}{Mini Demo (Python)}
  Run:
  \begin{center}
    \texttt{python demo/dispersion\_covariance\_demo.py}
  \end{center}
  \vspace{0.4em}
  What it does:
  \begin{itemize}
    \item Computes range, IQR, variance, SD for \texttt{data/scores\_small.csv}
    \item Flags outliers for \texttt{data/incomes\_outlier.csv} using the IQR rule
    \item Computes covariance for two paired datasets
    \item (Optional) Saves plots to \texttt{images/} if matplotlib is installed
  \end{itemize}
\end{frame}

\section{Summary}
\label{sec:summary}

\begin{frame}{Summary}
  \begin{itemize}[<+->]
    \item Mean alone is not enough; dispersion describes spread
    \item IQR is robust; variance/SD quantify typical deviation
    \item Covariance captures joint variation (sign matters; scale matters)
  \end{itemize}
  \vspace{0.6em}
  \textbf{Exit question:} For a dataset with strong outliers, which spread measure would you report first and why?
\end{frame}

\end{document}
