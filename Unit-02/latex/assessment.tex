\documentclass[11pt,a4paper]{article}

\usepackage[utf8]{inputenc}
\usepackage[T1]{fontenc}
\usepackage{geometry}
\usepackage{hyperref}
\usepackage{enumitem}
\usepackage{xcolor}
\usepackage{fancyhdr}
\usepackage{amsmath}
\usepackage{booktabs}
\usepackage{listings}

\geometry{margin=1in}
\hypersetup{
  colorlinks=true,
  linkcolor=blue,
  urlcolor=blue
}

\setlist{itemsep=0.35em, topsep=0.35em}

\pagestyle{fancy}
\fancyhf{}
\lhead{Statistics and Data Analysis (B.Tech)}
\rhead{Unit 02: Diagnostic Assessment}
\cfoot{\thepage}
\setlength{\headheight}{14pt}

\lstset{
  basicstyle=\ttfamily\small,
  keywordstyle=\color{blue},
  commentstyle=\color{gray},
  stringstyle=\color{teal},
  showstringspaces=false,
  columns=fullflexible,
  frame=single
}

% Marks per question (total = 100)
\newcommand{\EMarks}{\textbf{[2 marks]} }
\newcommand{\MMarks}{\textbf{[3 marks]} }
\newcommand{\HMarks}{\textbf{[5 marks]} }

\begin{document}

\begin{center}
  {\LARGE \textbf{Diagnostic Assessment -- Unit 02}}\\[0.25em]
  {\large Statistics and Data Analysis}\\[0.25em]
  \normalsize (Descriptive Statistics; designed to identify learning gaps)
\end{center}

\noindent
\textbf{Instructor:} Tofik Ali \hfill \textbf{Time Limit:} None\\
\textbf{Submission Deadline:} March 31, 2026\\
\textbf{Student Name:} \_\_\_\_\_\_\_\_\_\_\_\_\_\_\_\_\_ \hfill \textbf{Roll No.:} \_\_\_\_\_\_\_\_\\

\vspace{0.6em}
\hrule
\vspace{0.8em}

\textbf{Instructions}
\begin{itemize}[leftmargin=*]
  \item \textbf{This is a diagnostic assessment, not a quiz/test.} The goal is to identify learning gaps so you know what to revise next.
  \item \textbf{Total marks: 100.} Easy: 10 $\times$ 2 = 20, Medium: 10 $\times$ 3 = 30, Hard: 10 $\times$ 5 = 50.
  \item \textbf{No time limit.} Submit on or before \textbf{March 31, 2026}.
  \item \textbf{Important (70\% rule): To have your assignment marks counted, you must score at least 70\% (70/100 or more) in this assessment.}
    If you score below 70\%, you may reattempt and resubmit. Your latest submission will be considered.
  \item \textbf{Marking policy (honest attempt):} Marks are deducted only for (i) not attempting a question and (ii) high similarity with other submissions (copying).
    Correctness is \textbf{not} the main focus; your reasoning and effort matter more.
  \item \textbf{Show your real effort:} if you tried multiple approaches for any question, write them all (Attempt 1, Attempt 2, ...). Partial/incorrect attempts are acceptable.
  \item \textbf{Many questions are open-ended by design:} multiple solutions/variants are acceptable. Use your own example data and explain your choices.
  \item Unless specified, you may choose \textbf{sample} or \textbf{population} formulas, but you must mention your choice.
  \item Round final numeric answers to \textbf{2 decimal places}.
\end{itemize}

\vspace{0.6em}

\section*{Easy (10 Questions)}
\begin{enumerate}[label=\textbf{E\arabic*.}, leftmargin=*]
  \item \EMarks \textbf{[Subjective]} Create a dataset of \textbf{8 numbers} using your roll number digits (you may repeat digits).
    Compute mean and median. Write 2--3 lines: which measure is more robust to outliers and why?

  \item \EMarks \textbf{[MCQ -- Select all that apply]} Which measures are generally \textbf{more robust to outliers}?
    \begin{enumerate}[label=(\Alph*), leftmargin=*]
      \item Mean
      \item Median
      \item Mode
      \item 10\% trimmed mean
      \item Standard deviation
    \end{enumerate}

  \item \EMarks \textbf{[Subjective]} For the dataset $x=\{5, 6, 6, 7, 9, 10\}$ compute mean, median, and mode.

  \item \EMarks \textbf{[Subjective]} Explain the difference between \textbf{population variance} and \textbf{sample variance}.
    Why do we use $(n-1)$ in the sample variance formula? (4--6 lines)

  \item \EMarks \textbf{[Subjective]} Compute the \textbf{range} and \textbf{IQR} for $x=\{2, 3, 3, 6, 7, 9, 10, 12\}$.
    State your quartile method.

  \item \EMarks \textbf{[Subjective]} A dataset has mean $=60$ and standard deviation $=8$.
    Compute the z-score for $x=70$ and interpret in one line.

  \item \EMarks \textbf{[MCQ -- Select all that apply]} Which statements are \textbf{true}?
    \begin{enumerate}[label=(\Alph*), leftmargin=*]
      \item Covariance is scale dependent.
      \item Correlation is always between $-1$ and $+1$.
      \item Correlation changes if we convert height from cm to meters.
      \item A correlation of $0$ means there is no relationship at all.
      \item Negative correlation means as $x$ increases, $y$ tends to decrease (linearly).
    \end{enumerate}

  \item \EMarks \textbf{[Subjective]} Write 4--6 lines explaining \textbf{``correlation does not imply causation''} and give one example.

  \item \EMarks \textbf{[Subjective]} Identify the likely skewness type (right-skewed / left-skewed / symmetric) and justify briefly:
    \begin{enumerate}[label=(\alph*), leftmargin=*]
      \item Salaries in a large company
      \item Marks in a very easy exam (most score high)
      \item Adult heights
    \end{enumerate}

  \item \EMarks \textbf{[Subjective]} In 4--6 lines, explain what \textbf{kurtosis} tells you.
    Mention one common misunderstanding about kurtosis.
\end{enumerate}

\section*{Medium (10 Questions)}
\begin{enumerate}[label=\textbf{M\arabic*.}, leftmargin=*]
  \item \MMarks \textbf{[Subjective]} Compute variance and standard deviation for $x=\{4, 6, 8, 10, 12\}$.
    Use either sample or population formula (mention your choice).

  \item \MMarks \textbf{[Subjective]} Given paired data $(x,y)=\{(1,2),(2,3),(3,5),(4,4),(5,6)\}$:
    compute covariance and Pearson correlation and interpret the sign and strength (3--5 lines).

  \item \MMarks \textbf{[Subjective]} Create two datasets $A$ and $B$ of size 8 such that:
    mean($A$) = mean($B$) but std($A$) $>$ std($B$).
    Show your datasets and verify using calculations (many answers possible).

  \item \MMarks \textbf{[Subjective]} For $x=\{4, 5, 7, 8, 9, 10, 12, 13, 14, 30\}$ compute the five-number summary.
    Identify outliers using the 1.5$\times$IQR rule.

  \item \MMarks \textbf{[Subjective]} Explain (with a short numeric example) why covariance is scale dependent but correlation is not.
    (Example: $x' = 10x$, $y$ unchanged.)

  \item \MMarks \textbf{[Subjective]} Create a small dataset (at least 12 points) where correlation is \textbf{positive} overall,
    but one outlier can strongly change the value. Explain in 4--6 lines what happened.

  \item \MMarks \textbf{[Subjective]} Using Python (recommended), generate a summary table (count, mean, std, min, quartiles, max)
    for a dataset you choose (at least 30 numbers). Provide code and write 4--6 lines interpreting the results.

  \item \MMarks \textbf{[Subjective]} A feature has skewness $=+1.1$ and kurtosis $=5.0$.
    Interpret what this suggests about the distribution shape and tails (6--8 lines).

  \item \MMarks \textbf{[Subjective]} Explain in 6--8 lines: when can the mean be misleading?
    Provide one example dataset and explain which summary you would report instead (median/IQR, trimmed mean, etc.).

  \item \MMarks \textbf{[Subjective]} Choose an appropriate plot for each task and justify (2--3 lines each):
    \begin{enumerate}[label=(\alph*), leftmargin=*]
      \item Compare distributions of marks for two sections
      \item Detect outliers in delivery time data
      \item Check linear relationship between study time and marks
    \end{enumerate}
\end{enumerate}

\section*{Hard (10 Questions)}
\begin{enumerate}[label=\textbf{H\arabic*.}, leftmargin=*]
  \item \HMarks \textbf{[Subjective]} Robust reporting: for $x=\{12, 13, 13, 14, 15, 16, 16, 17, 18, 120\}$ compute:
    mean, median, IQR, and a 10\% trimmed mean. Write a 8--12 line paragraph explaining what you would report and why.

  \item \HMarks \textbf{[Subjective]} Construct a dataset (size $\ge$ 20) that is clearly right-skewed.
    Justify using mean vs median and at least one plot (histogram or boxplot). Provide your code or calculations.

  \item \HMarks \textbf{[Subjective]} Build a correlation matrix for three variables you design (at least 12 observations).
    Explain in 6--10 lines what the matrix tells you, and mention one limitation of correlation.

  \item \HMarks \textbf{[Subjective]} Nonlinear relationship challenge: create a dataset (at least 10 points) where
    $x$ and $y$ have a strong nonlinear relationship but Pearson correlation is near 0.
    Explain why and show the plot that reveals the pattern.

  \item \HMarks \textbf{[Subjective]} Confounding variable: Give a realistic example where correlation is high but not causal.
    Identify at least one confounder and describe how you would test/validate the relationship (8--12 lines).

  \item \HMarks \textbf{[Subjective]} Dimensional summaries: Suppose you have a dataset with 8 numeric features and 500 rows.
    List at least 10 per-feature statistics you would compute. Mark which ones help detect outliers and non-normality.

  \item \HMarks \textbf{[Subjective]} Grouped summaries: Design a summary table for exam marks grouped by branch and gender.
    Mention 3 pitfalls (e.g., small group size, outliers, imbalance) and how your summary handles them.

  \item \HMarks \textbf{[Subjective]} Write a short ``data story'' (10--14 lines) for a stakeholder using a dataset you choose.
    Include: one central tendency measure, one dispersion measure, one shape measure (skewness or kurtosis), and one plot.
    Keep it clear and non-technical.

  \item \HMarks \textbf{[Subjective]} Compare two datasets that have the same mean and standard deviation but differ in distribution shape.
    Construct the datasets, verify the statistics, and explain (6--10 lines) what the summaries miss.

  \item \HMarks \textbf{[Subjective]} Mini case study: You are analyzing delivery times (minutes) for an online service.
    Write an analysis plan (10--14 steps) using Unit 02 measures and visualizations.
    Include how you will treat outliers and how you will communicate results.
\end{enumerate}

\vfill
\noindent\textit{End of Assessment}

\end{document}

