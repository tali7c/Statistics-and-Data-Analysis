\documentclass[11pt,a4paper]{article}

\usepackage[utf8]{inputenc}
\usepackage[T1]{fontenc}
\usepackage{geometry}
\usepackage{hyperref}
\usepackage{enumitem}
\usepackage{xcolor}
\usepackage{fancyhdr}
\usepackage{amsmath}
\usepackage{booktabs}
\usepackage{listings}

\geometry{margin=1in}
\hypersetup{
  colorlinks=true,
  linkcolor=blue,
  urlcolor=blue
}

\setlist{itemsep=0.35em, topsep=0.35em}

\pagestyle{fancy}
\fancyhf{}
\lhead{Statistics and Data Analysis (B.Tech)}
\rhead{Unit 02: Descriptive Statistics}
\cfoot{\thepage}
\setlength{\headheight}{14pt}

\lstset{
  basicstyle=\ttfamily\small,
  keywordstyle=\color{blue},
  commentstyle=\color{gray},
  stringstyle=\color{teal},
  showstringspaces=false,
  columns=fullflexible,
  frame=single
}

\begin{document}

\begin{center}
  {\LARGE \textbf{Assignment -- Unit 02}}\\[0.25em]
  {\large Statistics and Data Analysis}\\[0.25em]
  \normalsize (Central Tendency, Dispersion, Covariance, Correlation, Skewness, Kurtosis, Summaries, Distributions)
\end{center}

\noindent
\textbf{Instructor:} Tofik Ali \hfill \textbf{Submission Deadline:} March 31, 2026\\
\textbf{Student Name:} \_\_\_\_\_\_\_\_\_\_\_\_\_\_\_\_\_ \hfill \textbf{Roll No.:} \_\_\_\_\_\_\_\_\\

\vspace{0.6em}
\hrule
\vspace{0.8em}

\textbf{Instructions}
\begin{itemize}[leftmargin=*]
  \item Answer \textbf{all} questions. Show calculations and write assumptions.
  \item There is \textbf{no time limit}. Submit on or before \textbf{March 31, 2026}.
  \item Unless specified, you may choose \textbf{sample} or \textbf{population} formulas, but you must clearly mention your choice.
  \item Round final numeric answers to \textbf{2 decimal places}.
  \item You may use a calculator or Python for verification, but show key steps/formulas in your solution.
\end{itemize}

\vspace{0.6em}

\section*{Easy (10 Questions)}
\begin{enumerate}[label=E\arabic*., leftmargin=*]
  \item \textbf{Mean/Median/Mode:} For the data
    \[
      x = \{7, 9, 9, 10, 12, 13\}
    \]
    compute mean, median, and mode. Which measure is most representative here, and why (2--3 lines)?

  \item \textbf{Effect of an outlier:} Consider
    \[
      x = \{10, 11, 12, 13, 14\}
    \]
    Compute the mean and median. Now add an outlier value 100 and recompute mean and median.
    Write one line: which measure changed more and why?

  \item \textbf{Trimmed mean:} Compute the 20\% trimmed mean for:
    \[
      x = \{2, 3, 4, 7, 9, 10, 12, 50, 60, 80\}
    \]
    (Sort first. Remove equal number of values from both ends.)

  \item \textbf{Range and IQR:} For the data
    \[
      x = \{3, 5, 7, 8, 8, 10, 12, 15\}
    \]
    compute range, $Q_1$, $Q_3$, and IQR. State the method you used for quartiles.

  \item \textbf{Variance and standard deviation:} For
    \[
      x = \{4, 6, 8, 10\}
    \]
    compute variance and standard deviation (sample or population, mention your choice).

  \item \textbf{Z-scores:} Using your mean and standard deviation from E5, compute the z-score of $x=10$.
    Interpret the z-score in one line.

  \item \textbf{Covariance sign:} Without calculating exact numbers, predict whether covariance is positive, negative,
    or near zero for each pair, and justify in 1--2 lines:
    \begin{enumerate}[label=(\alph*), leftmargin=*]
      \item Height and weight
      \item Price and demand (in general)
      \item Shoe size and exam marks
    \end{enumerate}

  \item \textbf{Correlation vs causation:} Write 4--6 lines explaining why ``correlation does not imply causation'',
    and give one real-world example.

  \item \textbf{Skewness (conceptual):} For each dataset, identify whether it is likely \textbf{right-skewed},
    \textbf{left-skewed}, or \textbf{approximately symmetric}. Justify briefly.
    \begin{enumerate}[label=(\alph*), leftmargin=*]
      \item Salaries in a large company
      \item Marks in a very easy test
      \item Heights of adult humans
    \end{enumerate}

  \item \textbf{Kurtosis (conceptual):} In 4--6 lines, explain what \textbf{high kurtosis} indicates about a distribution.
    Mention one common misunderstanding about kurtosis.
\end{enumerate}

\section*{Medium (10 Questions)}
\begin{enumerate}[label=M\arabic*., leftmargin=*]
  \item \textbf{Five-number summary + outliers:} For the dataset:
    \[
      x = \{4, 5, 7, 8, 9, 10, 12, 13, 14, 30\}
    \]
    compute the five-number summary and identify outliers using the 1.5$\times$IQR rule.

  \item \textbf{Compare two groups:} Section A marks:
    \[
      A=\{55, 60, 62, 65, 70, 72, 75\}
    \]
    Section B marks:
    \[
      B=\{45, 50, 58, 60, 68, 80, 90\}
    \]
    Compute mean and standard deviation for both and write 4--6 lines comparing central tendency and dispersion.

  \item \textbf{Covariance and correlation:} Given paired data:
    \[
      (x,y)=\{(1,2),(2,3),(3,5),(4,4),(5,6)\}
    \]
    compute covariance and Pearson correlation. Interpret the result in 3--5 lines.

  \item \textbf{Scale dependence:} Show (with a short calculation) how covariance changes if we transform
    $x' = 10x$ while keeping $y$ same. Use the dataset from M3.

  \item \textbf{Correlation matrix (small):} Consider the dataset below (5 observations):
    \[
      x=\{1,2,3,4,5\},\quad y=\{2,4,6,8,10\},\quad z=\{5,4,3,2,1\}
    \]
    Compute the correlation matrix among $(x,y,z)$ and interpret the signs/magnitudes.

  \item \textbf{Interpret skewness/kurtosis values:} A feature has skewness $= +1.2$ and kurtosis $= 4.8$.
    Write 5--8 lines interpreting what this suggests about the distribution shape and tails.

  \item \textbf{Summary table (by hand):} For the data:
    \[
      x = \{2, 4, 4, 5, 7, 9, 10, 10\}
    \]
    create a summary table containing: count, mean, std, min, $Q_1$, median, $Q_3$, max.
    (You may compute quartiles using your chosen method; state it.)

  \item \textbf{Python summary (recommended):} Using Python (pandas), compute \texttt{describe()} for the same dataset in M7.
    Write the code and paste the output table (or write the key values). Mention one value that surprised you.

  \item \textbf{When summaries mislead:} Give an example (you may invent small data) where two datasets have the same mean and std
    but look very different. Explain in 5--8 lines what statistics miss and what plot you would use.

  \item \textbf{Choice of measure:} For each scenario, choose an appropriate measure of central tendency and dispersion
    and justify (3--4 lines each):
    \begin{enumerate}[label=(\alph*), leftmargin=*]
      \item Monthly income data
      \item Daily temperature in a city
      \item Delivery times for an online service (with occasional delays)
    \end{enumerate}
\end{enumerate}

\section*{Hard (10 Questions)}
\begin{enumerate}[label=H\arabic*., leftmargin=*]
  \item \textbf{Robust summary report:} You are given the dataset:
    \[
      x=\{12, 13, 13, 14, 15, 16, 16, 17, 18, 120\}
    \]
    Compute mean, median, IQR, and a 10\% trimmed mean.
    Write a 6--10 line ``data summary'' paragraph explaining what you would report and why.

  \item \textbf{Design a dataset:} Construct two datasets $A$ and $B$ of size 8 such that:
    \begin{itemize}[leftmargin=*]
      \item mean($A$) = mean($B$)
      \item std($A$) $>$ std($B$)
    \end{itemize}
    Show your datasets and verify the conditions with calculations.

  \item \textbf{Covariance matrix:} For the 3-variable dataset (6 observations) below:
    \[
      (x,y,z)=\{(1,2,3),(2,1,4),(3,3,2),(4,5,1),(5,4,2),(6,6,0)\}
    \]
    compute the sample covariance matrix. Identify which pair has the strongest linear relationship.

  \item \textbf{Correlation pitfalls:} Give a real or realistic example where correlation is high but the relationship is
    \textbf{not} causal. Explain a confounding variable in 6--10 lines.

  \item \textbf{Nonlinear relationship:} Create a small dataset (at least 8 points) where $x$ and $y$ have a strong nonlinear
    relationship but Pearson correlation is near zero. Explain why this happens and what plot reveals it.

  \item \textbf{Dimensional summaries:} Suppose you have a dataset with 8 features (columns) and 500 rows.
    List the key summary statistics you would compute per feature (at least 10 items).
    Mention which ones you would use to detect outliers and non-normality.

  \item \textbf{Grouped summaries:} You have exam marks for two branches (CSE and AI) and two genders.
    Design a table (columns) that summarizes the data fairly and avoids misleading conclusions.
    Mention at least 3 pitfalls and how your table avoids them.

  \item \textbf{Skewness vs median/mean:} Explain the relationship between skewness and the relative positions of mean and median.
    Provide two small example datasets: one right-skewed and one left-skewed, and verify mean vs median.

  \item \textbf{Kurtosis comparison:} Two distributions have the same mean and variance but different kurtosis.
    In 8--12 lines, explain what might differ in their shape and why it matters for outliers/risk.

  \item \textbf{Mini case study:} You are analyzing delivery times (in minutes) for an online service.
    Suggest a complete analysis plan (8--12 steps) using Unit 02 measures and visualizations.
    Include which statistics you will compute, how you will treat outliers, and how you will communicate results.
\end{enumerate}

\vfill
\noindent\textit{End of Assignment}

\end{document}
