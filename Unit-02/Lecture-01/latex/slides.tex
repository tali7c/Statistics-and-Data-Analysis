\documentclass{beamer}

\usetheme{Berlin}
\usecolortheme{Orchid}
\useoutertheme{miniframes}
\setbeamertemplate{navigation symbols}{}

\usepackage{amsmath}
\usepackage{booktabs}
\usepackage{graphicx}

\title[Statistics and Data Analysis]{Statistics and Data Analysis}
\subtitle{Unit 02 -- Lecture 01: Measures of Central Tendency}
\author{Tofik Ali}
\institute{School of Computer Science, UPES Dehradun}
\date{\today}

\begin{document}

\begin{frame}
  \titlepage
  \vspace{-0.5em}
  \begin{center}
    \small \texttt{https://github.com/tali7c/Statistics-and-Data-Analysis}
  \end{center}
\end{frame}

\begin{frame}{Quick Links}
  \centering
  \hyperlink{sec:core}{\beamerbutton{Core Concepts}}\hspace{1em}
  \hyperlink{sec:visuals}{\beamerbutton{Visuals}}\hspace{1em}
  \hyperlink{sec:interactive}{\beamerbutton{Interactive}}
\end{frame}

\begin{frame}{Agenda}
  \tableofcontents
\end{frame}

\section{Overview}

\section{Core Concepts}
\label{sec:core}

\begin{frame}{Learning Outcomes}
  \begin{itemize}[<+->]
    \item Differentiate mean, median, and mode
    \item Select an appropriate measure for context
    \item Explain robustness and outlier effects
    \item Apply measures to real datasets
  \end{itemize}
\end{frame}

\begin{frame}{Why Central Tendency?}
  \begin{itemize}[<+->]
    \item Summarizes a dataset with one representative value
    \item Enables comparison across groups or time
    \item Foundation for dispersion and inference
  \end{itemize}
\end{frame}

\begin{frame}{Mean}
  \[
    \bar{x} = \frac{1}{n}\sum_{i=1}^{n} x_i
  \]
  \begin{itemize}[<+->]
    \item Sensitive to extreme values
    \item Best for symmetric distributions
  \end{itemize}
\end{frame}

\begin{frame}{Median}
  \begin{itemize}[<+->]
    \item Middle value after sorting
    \item Robust to outliers and skew
    \item Preferred for income or skewed data
  \end{itemize}
\end{frame}

\begin{frame}{Mode}
  \begin{itemize}[<+->]
    \item Most frequent value
    \item Useful for categorical data
    \item Can be multi-modal
  \end{itemize}
\end{frame}

\begin{frame}{Categorical Example (Mode)}
  \small
  Dataset (Major Specialization):
  \vspace{0.4em}
  \begin{center}
    \begin{tabular}{cccccc}
      CSE & CSE & AI & CSE & DS & AI \\
      CSE & DS & AI & AI & CSE & DS \\
    \end{tabular}
  \end{center}
  \vspace{0.4em}
  \normalsize
  \textbf{Mode:} CSE (most frequent category).
\end{frame}

\begin{frame}{Decimal Real-Value Example}
  \small
  Dataset (CGPA values):
  \vspace{0.4em}
  \begin{center}
    \begin{tabular}{cccccc}
      7.2 & 7.5 & 7.8 & 8.1 & 8.3 & 8.6 \\
      7.4 & 7.9 & 8.0 & 8.2 & 8.5 & 8.7 \\
    \end{tabular}
  \end{center}
  \vspace{0.4em}
  \normalsize
  \textbf{Use:} Mean/median for central tendency of continuous data.
\end{frame}

\begin{frame}{Multi-Dimensional Example}
  \small
  Dataset (Student Attributes):
  \vspace{0.4em}
  \begin{center}
    \begin{tabular}{lccc}
      \toprule
      Student & Attendance (\%) & Quiz Score & Project Score \\
      \midrule
      A & 92 & 8.5 & 9.0 \\
      B & 85 & 7.8 & 8.2 \\
      C & 88 & 9.1 & 8.7 \\
      D & 76 & 6.9 & 7.5 \\
      \bottomrule
    \end{tabular}
  \end{center}
  \vspace{0.4em}
  \normalsize
  \textbf{Use:} Compute mean/median per feature (column-wise).
\end{frame}

\begin{frame}{Multi-Dimensional Categorical Example}
  \small
  Dataset (Student Profile):
  \vspace{0.4em}
  \begin{center}
    \begin{tabular}{lccc}
      \toprule
      Student & Major & Section & Club \\
      \midrule
      A & CSE & A & Robotics \\
      B & AI & B & AI \\
      C & CSE & A & Coding \\
      D & DS & B & Robotics \\
      E & CSE & A & AI \\
      \bottomrule
    \end{tabular}
  \end{center}
  \vspace{0.4em}
  \normalsize
  \textbf{Use:} Mode per column (most common major/section/club).
\end{frame}

\begin{frame}{Dataset for Calculation Exercise}
  \small
  Use this sample income data (in thousands) for mean, median, and mode:
  \vspace{0.5em}
  \begin{center}
    \begin{tabular}{cccccc}
      12 & 14 & 15 & 15 & 16 & 18 \\
      19 & 20 & 22 & 25 & 27 & 30 \\
      32 & 35 & 40 &  &  &  \\
    \end{tabular}
  \end{center}
  \vspace{0.3em}
  \normalsize
  \textbf{Exercise:} Compute mean, median, and mode. Comment on skew.
\end{frame}

\begin{frame}{Exercise Solution (Summary)}
  \small
  Sorted data (n=15): 12, 14, 15, 15, 16, 18, 19, 20, 22, 25, 27, 30, 32, 35, 40
  \vspace{0.4em}
  \begin{itemize}
    \item \textbf{Mean} = 22.67
    \item \textbf{Median} = 20
    \item \textbf{Mode} = 15
    \item \textbf{Skew} = right-skewed (mean $>$ median)
  \end{itemize}
\end{frame}

\section{Visuals}
\label{sec:visuals}

\begin{frame}{Comparison Table}
  \begin{tabular}{lccc}
    \toprule
    Measure & Mean & Median & Mode \\
    \midrule
    Symmetric data & \checkmark & \checkmark & \checkmark \\
    Skewed data &  & \checkmark & \checkmark \\
    Outliers present &  & \checkmark & \checkmark \\
    \bottomrule
  \end{tabular}
\end{frame}

\begin{frame}{Outlier Effect (Step 1)}
  \small
  Dataset (in thousands):
  \vspace{0.3em}
  \begin{center}
    \begin{tabular}{cccccc}
      12 & 14 & 15 & 15 & 16 & 18 \\
      19 & 20 & 22 & 25 & 27 & 30 \\
      32 & 35 & 40 &  &  &  \\
    \end{tabular}
  \end{center}
  \vspace{0.4em}
  \textbf{Add a large outlier to the dataset:} 200
\end{frame}

\begin{frame}{Outlier Effect (Step 2)}
  \begin{itemize}
    \item \textbf{Mean:} 22.67 $\rightarrow$ 33.75 \hfill \textit{Mean shifts strongly}
    \item \textbf{Median:} 20 $\rightarrow$ 21 \hfill \textit{Median changes little}
  \end{itemize}
\end{frame}

\begin{frame}{Equation Sample}
  \[
    \bar{x}_{\text{trim}} = \frac{1}{n-2k}\sum_{i=k+1}^{n-k} x_{(i)}
  \]
  \small Trimmed mean for robustness.
\end{frame}

\section{Interactive}
\label{sec:interactive}

\begin{frame}{Checkpoint Question}
  Which measure best describes income data, and why?
\end{frame}

\begin{frame}{Mini Demo (Python)}
  \begin{itemize}
    \item Load sample dataset
    \item Compute mean, median, mode
    \item Visualize distribution
  \end{itemize}
\end{frame}

\section{Summary}

\begin{frame}{Summary}
  \begin{itemize}[<+->]
    \item Mean, median, mode capture different centers
    \item Robustness matters for skew/outliers
    \item Always check the distribution shape
  \end{itemize}
  \vspace{0.5em}
  \textbf{Exit question:} What would you report first and why?
\end{frame}

\end{document}
