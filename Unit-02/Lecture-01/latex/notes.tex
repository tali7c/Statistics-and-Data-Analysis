\documentclass[11pt]{article}
\usepackage[utf8]{inputenc}
\usepackage[T1]{fontenc}
\usepackage{geometry}
\usepackage{amsmath}
\usepackage{booktabs}
\usepackage{hyperref}
\geometry{margin=1in}

\title{Statistics and Data Analysis\\Unit 02 -- Lecture 01 Notes}
\author{Tofik Ali}
\date{\today}

\begin{document}
\maketitle

\section*{Measures of Central Tendency}

\subsection*{Learning Outcomes}
\begin{itemize}
  \item Differentiate mean, median, and mode.
  \item Choose the most appropriate measure for a given context.
  \item Explain robustness and the effect of outliers.
  \item Apply these measures to real datasets.
\end{itemize}

\section*{Slide-by-slide Notes}

\subsection*{Title Slide}
Introduce the unit and the focus on central tendency as the ``center'' of data.
Explain that many real datasets are large and messy, so we need a simple way to
summarize what a typical value looks like. Central tendency gives us a single
number that represents the dataset.

\subsection*{Quick Links / Agenda}
Explain the flow: core concepts, examples and visuals, then a short demo and
exercise. Tell students the aim is to understand when each measure is useful,
not just how to calculate it.

\subsection*{Learning Outcomes}
Central tendency is the foundation for descriptive statistics and later inference.
By the end of the lecture, students should be able to pick the correct measure
for a given context and justify their choice.

\subsection*{Why Central Tendency?}
Central tendency compresses data into a representative value for comparison
and communication. In practice, people ask: ``What is a typical salary?'' or
``What is the typical score?'' The answer depends on the data distribution.
No single measure is ``best'' in all cases because different datasets have
different shapes and outliers.
When we say it ``enables comparison across groups or time,'' we mean we can
reduce each group (e.g., Section A vs. Section B) or each time period (e.g.,
this year vs. last year) to a single representative value and then compare
those values. For example, if the median score in Section A is 72 and in
Section B is 68, we can quickly see which group performed better. Similarly,
if the mean weekly sales were 50 units in January and 65 units in February, we
can say sales increased over time. Without a central measure, comparing large
datasets is slower and harder to communicate.

\subsection*{Mean}
The arithmetic mean is the average:
\[
\bar{x} = \frac{1}{n}\sum_{i=1}^{n} x_i
\]
Explain each symbol: $n$ is the number of data points, $x_i$ is the $i$-th value,
and $\bar{x}$ is read as ``x-bar.'' The mean adds all values and divides by the
count. It uses every data point, so it changes when any value changes.\\
\textbf{Strengths:} stable for symmetric distributions, easy to compute, and used
in later formulas (variance, covariance, regression).\\
\textbf{Limitations:} sensitive to outliers. A single extremely large or small
value can pull the mean away from the ``typical'' value.

\subsection*{Median}
The median is the middle value after sorting the data. If there are an odd
number of values, the median is the middle one. If there are an even number,
the median is the average of the two middle values.\\
\textbf{Strengths:} robust to extreme values; preferred for skewed data
(e.g., income, house prices).\\
\textbf{Limitations:} ignores the magnitude of extreme values, so it is less
sensitive to tail changes.

\subsection*{Mode}
The mode is the most frequent value or category. A dataset can have one mode,
multiple modes (bi-modal, multi-modal), or no clear mode.\\
\textbf{Strengths:} useful for categorical or discrete data (e.g., most common
major, most common defect type).\\
\textbf{Limitations:} can be unstable and may not represent the ``center'' for
continuous data.

\subsection*{Categorical Example (Mode)}
Use a categorical dataset like ``Major Specialization'' to show that mean and
median are not meaningful for categories, while mode is meaningful. The mode
answers: which category appears most often?

\subsection*{Decimal Real-Value Example}
For continuous values such as CGPA, mean and median are both valid. Emphasize
that these values are real numbers and represent typical performance. The mode
may be less informative if values are mostly unique.

\subsection*{Multi-Dimensional Example}
For a table with multiple columns (e.g., attendance, quiz score, project score),
central tendency is computed column-wise. This shows that a dataset can have
several ``centers,'' one for each feature.

\subsection*{Dataset for Calculation Exercise}
Provide the small income dataset. Ask students to compute mean, median, and
mode manually. This helps them practice sorting and calculating.

\subsection*{Exercise Solution}
Explain the solution step-by-step:
\begin{itemize}
  \item Count values: $n=15$.
  \item Sort the data and identify the middle value (median).
  \item Compute the mean by summing all values and dividing by 15.
  \item Identify the most frequent value (mode).
\end{itemize}
Highlight that the mean is larger than the median, which indicates right-skew.

\subsection*{Comparison Table}
Selection rules:
\begin{itemize}
  \item Symmetric data: mean/median/mode are all acceptable.
  \item Skewed data or outliers: median (or mode) is more stable.
\end{itemize}
Explain that this is a practical checklist for choosing the correct measure.

\subsection*{Outlier Effect}
Demonstrate with a simple example:
\begin{itemize}
  \item Dataset A: 10, 12, 12, 13, 14 $\rightarrow$ mean $\approx$ 12.2, median = 12
  \item Dataset B: add outlier 100 $\rightarrow$ mean $\approx$ 30.2, median = 12
\end{itemize}
Conclusion: median is robust. This is why reports for skewed data often use
median instead of mean.

\subsection*{Trimmed Mean}
The trimmed mean removes a small percentage of the smallest and largest values
before computing the mean. It is a compromise between mean and median and is
useful when there are mild outliers but we still want to use most of the data.

\subsection*{Checkpoint Question}
Which measure best describes income data, and why?\\
Expected: median, because income distributions are typically right-skewed.
Right-skewed means a long tail to the right: a small number of very high incomes
pull the mean upward, while most values remain lower. The median resists this
pull and better represents the ``typical'' income for the majority of people.

\subsection*{Mini Demo (Python)}
Walk through:
\begin{enumerate}
  \item Load dataset
  \item Compute mean/median/mode
  \item Visualize histogram and show effect of outlier
\end{enumerate}
Explain each step briefly, and remind students to compare mean and median after
adding the outlier.

Demo assets:
\begin{itemize}
  \item \texttt{demo/mini\_demo.py}
  \item \texttt{data/income\_small.csv}
\end{itemize}

\subsection*{Summary / Exit Question}
Reinforce the selection logic and ask students to justify a choice in context.
Encourage them to always look at the distribution shape before deciding on a
single summary value.

\section*{Demo (Python)}
Run:
\begin{verbatim}
python demo/mini_demo.py
\end{verbatim}

What it shows:
\begin{itemize}
  \item Mean/median/mode on a small income dataset
  \item Mean shift after injecting an outlier
  \item Histogram of values
\end{itemize}

\section*{References}
\begin{itemize}
  \item Montgomery, D. C., \& Runger, G. C. \textit{Applied Statistics and Probability for Engineers}, Wiley, 7th ed., 2020.
  \item Gupta, S. C., \& Kapoor, V. K. \textit{Fundamentals of Applied Statistics}, Sultan Chand \& Sons, 4th rev. ed., 2007.
  \item McKinney, W. \textit{Python for Data Analysis}, O'Reilly, 2022.
\end{itemize}

\end{document}
