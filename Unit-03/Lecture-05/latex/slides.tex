\documentclass{beamer}

      \usetheme{Berlin}
      \usecolortheme{Orchid}
      \useoutertheme{miniframes}
      \setbeamertemplate{navigation symbols}{}

      \usepackage{amsmath}
      \usepackage{amssymb}
      \usepackage{booktabs}
      \usepackage{graphicx}
      \graphicspath{{../images/}}

      \title[Statistics and Data Analysis]{Statistics and Data Analysis}
      \subtitle{Unit 03 -- Lecture 05: ANOVA (One-Way) and Post-hoc Intuition}
      \author{Tofik Ali}
      \institute{School of Computer Science, UPES Dehradun}
      \date{\today}

      \begin{document}

      \begin{frame}
        \titlepage
        \vspace{-0.5em}
        \begin{center}
          \small \texttt{https://github.com/tali7c/Statistics-and-Data-Analysis}
        \end{center}
      \end{frame}

      \begin{frame}{Quick Links}
        \centering
        \hyperlink{sec:overview}{\beamerbutton{Overview}}\hspace{0.6em}
\hyperlink{sec:anova}{\beamerbutton{ANOVA Concept}}\hspace{0.6em}
\hyperlink{sec:assum}{\beamerbutton{Assumptions}}\hspace{0.6em}
\hyperlink{sec:exercises}{\beamerbutton{Exercises}}\hspace{0.6em}
\hyperlink{sec:demo}{\beamerbutton{Demo}}\hspace{0.6em}
\hyperlink{sec:summary}{\beamerbutton{Summary}}\hspace{0.6em}
      \end{frame}

      \begin{frame}{Agenda}
        \tableofcontents
      \end{frame}

\section{Overview}
\label{sec:overview}


\begin{frame}{Learning Outcomes}
      \begin{itemize}[<+->]
        \item Explain why ANOVA is used for comparing 3+ means
\item Describe between-group vs within-group variation
\item Interpret F statistic at a high level
\item State main assumptions of one-way ANOVA
\item Explain what a post-hoc test is
      \end{itemize}
    \end{frame}

\section{ANOVA Concept}
\label{sec:anova}


\begin{frame}{ANOVA Concept: Key Points}
      \begin{itemize}[<+->]
        \item One global test for equality of means
\item Avoids inflating Type I error vs many t-tests
\item If significant, follow with post-hoc
      \end{itemize}
    \end{frame}

\begin{frame}{ANOVA Concept: Key Formula}
  \[ F = \frac{\text{between-group variation}}{\text{within-group variation}} \]
\end{frame}

\section{Assumptions}
\label{sec:assum}


\begin{frame}{Assumptions: Key Points}
      \begin{itemize}[<+->]
        \item Independent observations
\item Rough normality within groups (or robust with n)
\item Similar variances across groups
      \end{itemize}
    \end{frame}

\section{Exercises}
\label{sec:exercises}


\begin{frame}{Exercise 1: Write H0}
  \small
  Compare 3 group means. What is H0?
\end{frame}

\begin{frame}{Solution 1}
  \begin{itemize}
    \item H0: mu1 = mu2 = mu3
  \end{itemize}
\end{frame}

\begin{frame}{Exercise 2: Within variance}
  \small
  If within-group variance increases, what happens to F (all else equal)?
\end{frame}

\begin{frame}{Solution 2}
  \begin{itemize}
    \item F tends to decrease; harder to detect differences.
  \end{itemize}
\end{frame}

\begin{frame}{Exercise 3: Next step}
  \small
  ANOVA p-value is 0.01 at alpha=0.05. What next?
\end{frame}

\begin{frame}{Solution 3}
      \begin{itemize}
        \item Reject H0.
\item Run post-hoc to find which pairs differ.
      \end{itemize}
    \end{frame}

\section{Demo}
\label{sec:demo}


\begin{frame}{Mini Demo (Python)}
  Run from the lecture folder:
  \begin{center}
    \texttt{python demo/demo.py}
  \end{center}
  \vspace{0.4em}
  Outputs:
  \begin{itemize}
    \item \texttt{images/demo.png}
    \item \texttt{data/results.txt}
  \end{itemize}
\end{frame}

\begin{frame}{Demo Output (Example)}
  \begin{center}
  \IfFileExists{../images/demo.png}{
    \includegraphics[width=0.92\linewidth]{demo.png}
  }{
    \small (Run demo to generate: \texttt{demo.png})
  }
  \end{center}
\end{frame}

\section{Summary}
\label{sec:summary}


\begin{frame}{Summary}
      \begin{itemize}[<+->]
        \item Key definitions and the main formula.
\item How to interpret results in context.
\item How the demo connects to the theory.
      \end{itemize}
    \end{frame}

\begin{frame}{Exit Question}
  \small
  Why are several pairwise t-tests not equivalent to one ANOVA?
\end{frame}

\end{document}
