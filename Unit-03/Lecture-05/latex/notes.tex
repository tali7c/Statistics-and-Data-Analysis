\documentclass[11pt]{article}
\usepackage[utf8]{inputenc}
\usepackage[T1]{fontenc}
\usepackage{geometry}
\usepackage{amsmath}
\usepackage{booktabs}
\usepackage{hyperref}
\geometry{margin=1in}

\title{Statistics and Data Analysis\\Unit 03 -- Lecture 05 Notes}
\author{Tofik Ali}
\date{\today}

\begin{document}
\maketitle

\section*{Topic}
One-way ANOVA; F statistic; assumptions; post-hoc comparisons.

\subsection*{Learning Outcomes}
\begin{itemize}
  \item Explain why ANOVA is used for comparing 3+ means
  \item Describe between-group vs within-group variation
  \item Interpret F statistic at a high level
  \item State main assumptions of one-way ANOVA
  \item Explain what a post-hoc test is
\end{itemize}

\section*{Detailed Notes}
These notes are designed to be read alongside the slides. They expand each slide bullet into
plain-language explanations, small worked examples, and common pitfalls. When a formula
appears, emphasize (1) what each symbol means, (2) the assumptions needed to use it, and (3)
how to interpret the final number in the problem context.

\section*{ANOVA Concept}
\begin{itemize}
  \item One global test for equality of means
  \item Avoids inflating Type I error vs many t-tests
  \item If significant, follow with post-hoc
\end{itemize}

\section*{Assumptions}
\begin{itemize}
  \item Independent observations
  \item Rough normality within groups (or robust with n)
  \item Similar variances across groups
\end{itemize}

\section*{Exercises (with Solutions)}
\subsection*{Exercise 1: Write H0}
Compare 3 group means. What is H0?
\subsection*{Solution}
\begin{itemize}
  \item H0: mu1 = mu2 = mu3
\end{itemize}

\subsection*{Exercise 2: Within variance}
If within-group variance increases, what happens to F (all else equal)?
\subsection*{Solution}
\begin{itemize}
  \item F tends to decrease; harder to detect differences.
\end{itemize}

\subsection*{Exercise 3: Next step}
ANOVA p-value is 0.01 at alpha=0.05. What next?
\subsection*{Solution}
\begin{itemize}
  \item Reject H0.
  \item Run post-hoc to find which pairs differ.
\end{itemize}

\section*{Exit Question}
Why are several pairwise t-tests not equivalent to one ANOVA?

\section*{Demo (Python)}
Run from the lecture folder:
\begin{verbatim}
python demo/demo.py
\end{verbatim}

Output files:
\begin{itemize}
  \item \texttt{images/demo.png}
  \item \texttt{data/results.txt}
\end{itemize}

\section*{References}
\begin{itemize}
  \item Montgomery, D. C., \& Runger, G. C. \textit{Applied Statistics and Probability for Engineers}, Wiley.
  \item Devore, J. L. \textit{Probability and Statistics for Engineering and the Sciences}, Cengage.
  \item McKinney, W. \textit{Python for Data Analysis}, O'Reilly.
\end{itemize}
\end{document}
