\documentclass[11pt]{article}
\usepackage[utf8]{inputenc}
\usepackage[T1]{fontenc}
\usepackage{geometry}
\usepackage{amsmath}
\usepackage{listings}
\usepackage{xcolor}
\usepackage{amssymb}
\usepackage{booktabs}
\usepackage{graphicx}
\graphicspath{{../images/}}
\usepackage{hyperref}
\geometry{margin=1in}

\title{Statistics and Data Analysis\\Unit 03 -- Lecture 05 Notes\\ANOVA (One-Way) and Post-hoc Intuition}
\author{Tofik Ali}
\date{\today}

\begin{document}
\maketitle

\section*{Topic}
One-way ANOVA; F statistic; assumptions; post-hoc comparisons.
\section*{How to Use These Notes}
These notes are written for students who are seeing the topic for the first time. They
follow the slide order, but add the missing 'why', interpretation, and common mistakes. If
you get stuck, look at the worked exercises and then run the Python demo.

Course repository (slides, demos, datasets): \url{https://github.com/tali7c/Statistics-and-Data-Analysis}

\section*{Time Plan (55 minutes)}
\begin{itemize}
  \item 0--10 min: Attendance + recap of previous lecture
  \item 10--35 min: Core concepts (this lecture's sections)
  \item 35--45 min: Exercises (solve 1--2 in class, rest as practice)
  \item 45--50 min: Mini demo + interpretation of output
  \item 50--55 min: Buffer / wrap-up (leave 5 minutes early)
\end{itemize}

\section*{Slide-by-slide Notes}
\subsection*{Title Slide}
State the lecture title clearly and connect it to what students already know.
Tell students what they will be able to do by the end (not just what you will cover).

\subsection*{Quick Links / Agenda}
Explain the structure of the lecture and where the exercises and demo appear.
\begin{itemize}
  \item Overview
  \item ANOVA Concept
  \item Assumptions
  \item Exercises
  \item Demo
  \item Summary
\end{itemize}

\subsection*{Learning Outcomes}
\begin{itemize}
  \item Explain why ANOVA is used for comparing 3+ means
  \item Describe between-group vs within-group variation
  \item Interpret F statistic at a high level
  \item State main assumptions of one-way ANOVA
  \item Explain what a post-hoc test is
\end{itemize}
\paragraph{Why these outcomes matter.}
\textbf{ANOVA} compares means of 3+ groups using variance decomposition. The idea is: if
group means are truly the same, between-group variability should look similar to within-
group variability. The F-statistic is essentially a ratio of these two sources of variation.
If ANOVA is significant, you usually follow with a \textbf{post-hoc} procedure to identify
which specific pairs of groups differ. Post-hoc methods control the family-wise error rate
so that 'looking at many pairs' does not create many false positives.

\subsection*{ANOVA Concept: Key Points}
\begin{itemize}
  \item One global test for equality of means
  \item Avoids inflating Type I error vs many t-tests
  \item If significant, follow with post-hoc
\end{itemize}
\paragraph{Explanation.}
A \textbf{Type I error} is a false positive: you conclude there is an effect/difference when
there is none. A \textbf{Type II error} is a false negative: you miss a real effect.
Lowering $\alpha$ reduces Type I errors but can increase Type II errors unless you also
increase sample size.
\textbf{ANOVA} compares means of 3+ groups using variance decomposition. The idea is: if
group means are truly the same, between-group variability should look similar to within-
group variability. The F-statistic is essentially a ratio of these two sources of variation.
If ANOVA is significant, you usually follow with a \textbf{post-hoc} procedure to identify
which specific pairs of groups differ. Post-hoc methods control the family-wise error rate
so that 'looking at many pairs' does not create many false positives.

\subsection*{ANOVA Concept: Key Formula}
\[ F = \frac{\text{between-group variation}}{\text{within-group variation}} \]
\paragraph{How to read the formula.}
\textbf{ANOVA} compares means of 3+ groups using variance decomposition. The idea is: if
group means are truly the same, between-group variability should look similar to within-
group variability. The F-statistic is essentially a ratio of these two sources of variation.

\subsection*{Assumptions: Key Points}
\begin{itemize}
  \item Independent observations
  \item Rough normality within groups (or robust with n)
  \item Similar variances across groups
\end{itemize}
\paragraph{Explanation.}
Always state assumptions clearly. Common assumptions in classical tests include independence
of observations, roughly normal errors (or a large-sample justification), and similar
variances across groups. Violations do not automatically invalidate a result, but they
change how much you should trust the p-value and confidence interval.

\subsection*{Exercises (with Solutions)}
Attempt the exercise first, then compare with the solution. Focus on interpretation, not
only arithmetic.

\subsection*{Exercise 1: Write H0}
Compare 3 group means. What is H0?
\subsubsection*{Solution}
\begin{itemize}
  \item H0: mu1 = mu2 = mu3
\end{itemize}
\paragraph{Walkthrough.}
The \textbf{null hypothesis $H_0$} usually represents 'no effect' or a baseline value (e.g.,
$\mu=60$). The \textbf{alternative $H_1$} represents the effect you are looking for (e.g.,
$\mu\neq 60$ or $\mu>60$). We compute a test statistic and a p-value assuming $H_0$ is true.

\subsection*{Exercise 2: Within variance}
If within-group variance increases, what happens to F (all else equal)?
\subsubsection*{Solution}
\begin{itemize}
  \item F tends to decrease; harder to detect differences.
\end{itemize}
\paragraph{Walkthrough.}
\textbf{Differencing} transforms a series by subtracting the previous value: $y_t -
y_{t-1}$. It removes trend and can help achieve stationarity. Over-differencing can add
noise, so use the smallest differencing order that works.

\subsection*{Exercise 3: Next step}
ANOVA p-value is 0.01 at alpha=0.05. What next?
\subsubsection*{Solution}
\begin{itemize}
  \item Reject H0.
  \item Run post-hoc to find which pairs differ.
\end{itemize}
\paragraph{Walkthrough.}
The \textbf{null hypothesis $H_0$} usually represents 'no effect' or a baseline value (e.g.,
$\mu=60$). The \textbf{alternative $H_1$} represents the effect you are looking for (e.g.,
$\mu\neq 60$ or $\mu>60$). We compute a test statistic and a p-value assuming $H_0$ is true.
A \textbf{p-value} is computed assuming the null hypothesis $H_0$ is true. It measures how
surprising the observed data (or something more extreme) would be under $H_0$. A small
p-value suggests the data is hard to explain by $H_0$ alone, but it does not tell you how
large the effect is or whether it is practically important.

\subsection*{Mini Demo (Python)}
Run from the lecture folder:
\begin{verbatim}
python demo/demo.py
\end{verbatim}

Output files:
\begin{itemize}
  \item \texttt{images/demo.png}
  \item \texttt{data/results.txt}
\end{itemize}
\paragraph{What to show and say.}
\begin{itemize}
  \item Generates 3 groups with different means and runs one-way ANOVA (F-test).
  \item Shows a boxplot of the groups to connect F to 'between vs within' variation.
  \item Use the p-value to explain what ANOVA can/cannot tell (needs post-hoc for pairs).
\end{itemize}

\subsection*{Demo Output (Example)}
\begin{center}
\IfFileExists{../images/demo.png}{
  \includegraphics[width=0.95\linewidth]{../images/demo.png}
}{
  \small (Run the demo to generate \texttt{images/demo.png})
}
\end{center}

\subsection*{Summary}
\begin{itemize}
  \item Key definitions and the main formula.
  \item How to interpret results in context.
  \item How the demo connects to the theory.
\end{itemize}

\subsection*{Exit Question}
Why are several pairwise t-tests not equivalent to one ANOVA?
\paragraph{Suggested answer (for revision).}
Many pairwise t-tests inflate the family-wise Type I error rate; ANOVA performs one global
test to control false positives before post-hoc comparisons.

\section*{References}
\begin{itemize}
  \item Montgomery, D. C., \& Runger, G. C. \textit{Applied Statistics and Probability for Engineers}, Wiley.
  \item Devore, J. L. \textit{Probability and Statistics for Engineering and the Sciences}, Cengage.
  \item McKinney, W. \textit{Python for Data Analysis}, O'Reilly.
\end{itemize}

% BEGIN SLIDE APPENDIX (AUTO-GENERATED)
\clearpage
\section*{Appendix: Slide Deck Content (Reference)}
\noindent The material below is a reference copy of the slide deck content. Exercise solutions are explained in the main notes where applicable.

\subsection*{Title Slide}
\titlepage
        \vspace{-0.5em}
        \begin{center}
          \small \texttt{https://github.com/tali7c/Statistics-and-Data-Analysis}
        \end{center}
\subsection*{Quick Links}
\centering
        \textbf{Overview}\hspace{0.6em}
\textbf{ANOVA Concept}\hspace{0.6em}
\textbf{Assumptions}\hspace{0.6em}
\textbf{Exercises}\hspace{0.6em}
\textbf{Demo}\hspace{0.6em}
\textbf{Summary}\hspace{0.6em}
\subsection*{Agenda}
\begin{itemize}
  \item Overview
  \item ANOVA Concept
  \item Assumptions
  \item Exercises
  \item Demo
  \item Summary
\end{itemize}
\subsection*{Learning Outcomes}
\begin{itemize}
        \item Explain why ANOVA is used for comparing 3+ means
\item Describe between-group vs within-group variation
\item Interpret F statistic at a high level
\item State main assumptions of one-way ANOVA
\item Explain what a post-hoc test is
      \end{itemize}
\subsection*{ANOVA Concept: Key Points}
\begin{itemize}
        \item One global test for equality of means
\item Avoids inflating Type I error vs many t-tests
\item If significant, follow with post-hoc
      \end{itemize}
\subsection*{ANOVA Concept: Key Formula}
\[ F = \frac{\text{between-group variation}}{\text{within-group variation}} \]
\subsection*{Assumptions: Key Points}
\begin{itemize}
        \item Independent observations
\item Rough normality within groups (or robust with n)
\item Similar variances across groups
      \end{itemize}
\subsection*{Exercise 1: Write H0}
\small
  Compare 3 group means. What is H0?
\subsection*{Solution 1}
\begin{itemize}
    \item H0: mu1 = mu2 = mu3
  \end{itemize}
\subsection*{Exercise 2: Within variance}
\small
  If within-group variance increases, what happens to F (all else equal)?
\subsection*{Solution 2}
\begin{itemize}
    \item F tends to decrease; harder to detect differences.
  \end{itemize}
\subsection*{Exercise 3: Next step}
\small
  ANOVA p-value is 0.01 at alpha=0.05. What next?
\subsection*{Solution 3}
\begin{itemize}
        \item Reject H0.
\item Run post-hoc to find which pairs differ.
      \end{itemize}
\subsection*{Mini Demo (Python)}
Run from the lecture folder:
  \begin{center}
    \texttt{python demo/demo.py}
  \end{center}
  \vspace{0.4em}
  Outputs:
  \begin{itemize}
    \item \texttt{images/demo.png}
    \item \texttt{data/results.txt}
  \end{itemize}
\subsection*{Demo Output (Example)}
\begin{center}
  \IfFileExists{../images/demo.png}{
    \includegraphics[width=0.92\linewidth]{demo.png}
  }{
    \small (Run demo to generate: \texttt{demo.png})
  }
  \end{center}
\subsection*{Summary}
\begin{itemize}
        \item Key definitions and the main formula.
\item How to interpret results in context.
\item How the demo connects to the theory.
      \end{itemize}
\subsection*{Exit Question}
\small
  Why are several pairwise t-tests not equivalent to one ANOVA?
% END SLIDE APPENDIX (AUTO-GENERATED)

\end{document}
