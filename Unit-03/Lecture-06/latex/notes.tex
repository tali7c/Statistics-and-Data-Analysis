\documentclass[11pt]{article}
\usepackage[utf8]{inputenc}
\usepackage[T1]{fontenc}
\usepackage{geometry}
\usepackage{amsmath}
\usepackage{listings}
\usepackage{xcolor}
\usepackage{amssymb}
\usepackage{booktabs}
\usepackage{graphicx}
\graphicspath{{../images/}}
\usepackage{hyperref}
\geometry{margin=1in}

\title{Statistics and Data Analysis\\Unit 03 -- Lecture 06 Notes\\Non-parametric Tests and p-value Interpretation}
\author{Tofik Ali}
\date{\today}

\begin{document}
\maketitle

\section*{Topic}
Rank-based tests and p-value interpretation; statistical vs practical significance.
\section*{How to Use These Notes}
These notes are written for students who are seeing the topic for the first time. They
follow the slide order, but add the missing 'why', interpretation, and common mistakes. If
you get stuck, look at the worked exercises and then run the Python demo.

Course repository (slides, demos, datasets): \url{https://github.com/tali7c/Statistics-and-Data-Analysis}

\section*{Time Plan (55 minutes)}
\begin{itemize}
  \item 0--10 min: Attendance + recap of previous lecture
  \item 10--35 min: Core concepts (this lecture's sections)
  \item 35--45 min: Exercises (solve 1--2 in class, rest as practice)
  \item 45--50 min: Mini demo + interpretation of output
  \item 50--55 min: Buffer / wrap-up (leave 5 minutes early)
\end{itemize}

\section*{Slide-by-slide Notes}
\subsection*{Title Slide}
State the lecture title clearly and connect it to what students already know.
Tell students what they will be able to do by the end (not just what you will cover).

\subsection*{Quick Links / Agenda}
Explain the structure of the lecture and where the exercises and demo appear.
\begin{itemize}
  \item Overview
  \item When to Use
  \item Common Tests
  \item Exercises
  \item Demo
  \item Summary
\end{itemize}

\subsection*{Learning Outcomes}
\begin{itemize}
  \item Explain why non-parametric tests are used
  \item Choose Mann-Whitney / Wilcoxon / Kruskal-Wallis
  \item Interpret p-values carefully
  \item Discuss statistical vs practical significance
  \item Explain multiple testing risk
\end{itemize}
\paragraph{Why these outcomes matter.}
A \textbf{p-value} is computed assuming the null hypothesis $H_0$ is true. It measures how
surprising the observed data (or something more extreme) would be under $H_0$. A small
p-value suggests the data is hard to explain by $H_0$ alone, but it does not tell you how
large the effect is or whether it is practically important.
\textbf{Non-parametric} tests rely on ranks rather than raw values. They are useful when
data is skewed, has outliers, or is ordinal (e.g., ratings). The trade-off is that they may
be less powerful than parametric tests when assumptions of parametric tests actually hold.

\subsection*{When to Use: Key Points}
\begin{itemize}
  \item Skewed data/outliers
  \item Ordinal scales
  \item Small sample and doubtful normality
\end{itemize}

\subsection*{Common Tests: Key Points}
\begin{itemize}
  \item Two independent groups: Mann-Whitney U
  \item Paired samples: Wilcoxon signed-rank
  \item 3+ groups: Kruskal-Wallis
\end{itemize}

\subsection*{Exercises (with Solutions)}
Attempt the exercise first, then compare with the solution. Focus on interpretation, not
only arithmetic.

\subsection*{Exercise 1: Choose test}
Same students before/after training (skewed). Which test?
\subsubsection*{Solution}
\begin{itemize}
  \item Wilcoxon signed-rank
\end{itemize}

\subsection*{Exercise 2: Practical vs statistical}
Very small p-value but tiny difference: what should you report?
\subsubsection*{Solution}
\begin{itemize}
  \item Report effect size and context; significance != importance.
\end{itemize}
\paragraph{Walkthrough.}
A \textbf{p-value} is computed assuming the null hypothesis $H_0$ is true. It measures how
surprising the observed data (or something more extreme) would be under $H_0$. A small
p-value suggests the data is hard to explain by $H_0$ alone, but it does not tell you how
large the effect is or whether it is practically important.
\textbf{Effect size} quantifies \emph{how big} a difference/relationship is (e.g., Cohen's
$d$, correlation $r$). With large samples, even tiny effects can be statistically
significant, so reporting effect size prevents over-claiming.

\subsection*{Exercise 3: Multiple testing}
20 tests at alpha=0.05: expected false positives?
\subsubsection*{Solution}
\begin{itemize}
  \item About 1 on average.
\end{itemize}
\paragraph{Walkthrough.}
The \textbf{significance level} $\alpha$ is the maximum Type I error rate you are willing to
tolerate: the probability of rejecting $H_0$ when $H_0$ is actually true. Common choices are
0.05 or 0.01, but the right value depends on consequences of false alarms vs missed
detections.
In chi-square tests, \textbf{expected counts} are what you would expect to see if $H_0$ were
true (e.g., independence). Very small expected counts can break the approximation used by
the test; a common rule of thumb is that most expected counts should be at least 5.

\subsection*{Mini Demo (Python)}
Run from the lecture folder:
\begin{verbatim}
python demo/demo.py
\end{verbatim}

Output files:
\begin{itemize}
  \item \texttt{images/demo.png}
  \item \texttt{data/results.txt}
\end{itemize}
\paragraph{What to show and say.}
\begin{itemize}
  \item Generates two skewed groups and runs Mann-Whitney U (rank-based) test.
  \item Shows a boxplot and reports medians to emphasize robustness under skew/outliers.
  \item Compare with a mean-based test to discuss when non-parametric is preferred.
\end{itemize}

\subsection*{Demo Output (Example)}
\begin{center}
\IfFileExists{../images/demo.png}{
  \includegraphics[width=0.95\linewidth]{../images/demo.png}
}{
  \small (Run the demo to generate \texttt{images/demo.png})
}
\end{center}

\subsection*{Summary}
\begin{itemize}
  \item Key definitions and the main formula.
  \item How to interpret results in context.
  \item How the demo connects to the theory.
\end{itemize}

\subsection*{Exit Question}
Give one reason to prefer a rank-based test over a mean-based test.
\paragraph{Suggested answer (for revision).}
Rank-based tests reduce sensitivity to outliers/skew and work for ordinal data, so they are
safer when normality is doubtful.

\section*{References}
\begin{itemize}
  \item Montgomery, D. C., \& Runger, G. C. \textit{Applied Statistics and Probability for Engineers}, Wiley.
  \item Devore, J. L. \textit{Probability and Statistics for Engineering and the Sciences}, Cengage.
  \item McKinney, W. \textit{Python for Data Analysis}, O'Reilly.
\end{itemize}

% BEGIN SLIDE APPENDIX (AUTO-GENERATED)
\clearpage
\section*{Appendix: Slide Deck Content (Reference)}
\noindent The material below is a reference copy of the slide deck content. Exercise solutions are explained in the main notes where applicable.

\subsection*{Title Slide}
\titlepage
        \vspace{-0.5em}
        \begin{center}
          \small \texttt{https://github.com/tali7c/Statistics-and-Data-Analysis}
        \end{center}
\subsection*{Quick Links}
\centering
        \textbf{Overview}\hspace{0.6em}
\textbf{When to Use}\hspace{0.6em}
\textbf{Common Tests}\hspace{0.6em}
\textbf{Exercises}\hspace{0.6em}
\textbf{Demo}\hspace{0.6em}
\textbf{Summary}\hspace{0.6em}
\subsection*{Agenda}
\begin{itemize}
  \item Overview
  \item When to Use
  \item Common Tests
  \item Exercises
  \item Demo
  \item Summary
\end{itemize}
\subsection*{Learning Outcomes}
\begin{itemize}
        \item Explain why non-parametric tests are used
\item Choose Mann-Whitney / Wilcoxon / Kruskal-Wallis
\item Interpret p-values carefully
\item Discuss statistical vs practical significance
\item Explain multiple testing risk
      \end{itemize}
\subsection*{When to Use: Key Points}
\begin{itemize}
        \item Skewed data/outliers
\item Ordinal scales
\item Small sample and doubtful normality
      \end{itemize}
\subsection*{Common Tests: Key Points}
\begin{itemize}
        \item Two independent groups: Mann-Whitney U
\item Paired samples: Wilcoxon signed-rank
\item 3+ groups: Kruskal-Wallis
      \end{itemize}
\subsection*{Exercise 1: Choose test}
\small
  Same students before/after training (skewed). Which test?
\subsection*{Solution 1}
\begin{itemize}
    \item Wilcoxon signed-rank
  \end{itemize}
\subsection*{Exercise 2: Practical vs statistical}
\small
  Very small p-value but tiny difference: what should you report?
\subsection*{Solution 2}
\begin{itemize}
    \item Report effect size and context; significance != importance.
  \end{itemize}
\subsection*{Exercise 3: Multiple testing}
\small
  20 tests at alpha=0.05: expected false positives?
\subsection*{Solution 3}
\begin{itemize}
    \item About 1 on average.
  \end{itemize}
\subsection*{Mini Demo (Python)}
Run from the lecture folder:
  \begin{center}
    \texttt{python demo/demo.py}
  \end{center}
  \vspace{0.4em}
  Outputs:
  \begin{itemize}
    \item \texttt{images/demo.png}
    \item \texttt{data/results.txt}
  \end{itemize}
\subsection*{Demo Output (Example)}
\begin{center}
  \IfFileExists{../images/demo.png}{
    \includegraphics[width=0.92\linewidth]{demo.png}
  }{
    \small (Run demo to generate: \texttt{demo.png})
  }
  \end{center}
\subsection*{Summary}
\begin{itemize}
        \item Key definitions and the main formula.
\item How to interpret results in context.
\item How the demo connects to the theory.
      \end{itemize}
\subsection*{Exit Question}
\small
  Give one reason to prefer a rank-based test over a mean-based test.
% END SLIDE APPENDIX (AUTO-GENERATED)

\end{document}
