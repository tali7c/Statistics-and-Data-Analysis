\documentclass{beamer}

      \usetheme{Berlin}
      \usecolortheme{Orchid}
      \useoutertheme{miniframes}
      \setbeamertemplate{navigation symbols}{}

      \usepackage{amsmath}
      \usepackage{amssymb}
      \usepackage{booktabs}
      \usepackage{graphicx}
      \graphicspath{{../images/}}

      \title[Statistics and Data Analysis]{Statistics and Data Analysis}
      \subtitle{Unit 03 -- Lecture 07: Case Exercise: Interpreting Hypothesis Testing Results}
      \author{Tofik Ali}
      \institute{School of Computer Science, UPES Dehradun}
      \date{\today}

      \begin{document}

      \begin{frame}
        \titlepage
        \vspace{-0.5em}
        \begin{center}
          \small \texttt{https://github.com/tali7c/Statistics-and-Data-Analysis}
        \end{center}
      \end{frame}

      \begin{frame}{Quick Links}
        \centering
        \hyperlink{sec:overview}{\beamerbutton{Overview}}\hspace{0.6em}
\hyperlink{sec:read}{\beamerbutton{Reading Results}}\hspace{0.6em}
\hyperlink{sec:pitfalls}{\beamerbutton{Pitfalls}}\hspace{0.6em}
\hyperlink{sec:exercises}{\beamerbutton{Exercises}}\hspace{0.6em}
\hyperlink{sec:demo}{\beamerbutton{Demo}}\hspace{0.6em}
\hyperlink{sec:summary}{\beamerbutton{Summary}}\hspace{0.6em}
      \end{frame}

      \begin{frame}{Agenda}
        \tableofcontents
      \end{frame}

\section{Overview}
\label{sec:overview}


\begin{frame}{Learning Outcomes}
      \begin{itemize}[<+->]
        \item Interpret p-values and confidence intervals correctly
\item Compute a simple effect size from summary statistics
\item Identify common red flags (only p-values, many tests, no effect size)
\item Write a cautious conclusion in plain language
\item Avoid correlation-causation confusion
      \end{itemize}
    \end{frame}

\section{Reading Results}
\label{sec:read}


\begin{frame}{Reading Results: Key Points}
      \begin{itemize}[<+->]
        \item Check n, center, spread
\item Prefer CI + effect size
\item Ask: what does it mean in the real world?
      \end{itemize}
    \end{frame}

\section{Pitfalls}
\label{sec:pitfalls}


\begin{frame}{Pitfalls: Key Points}
      \begin{itemize}[<+->]
        \item Multiple comparisons
\item Selective reporting (p-hacking)
\item Over-claiming causation
      \end{itemize}
    \end{frame}

\section{Exercises}
\label{sec:exercises}


\begin{frame}{Exercise 1: Interpret CI}
  \small
  95\% CI for (new-old) is (1.2, 3.8). What does it suggest?
\end{frame}

\begin{frame}{Solution 1}
      \begin{itemize}
        \item Likely positive effect (CI above 0).
\item Magnitude between 1.2 and 3.8 units.
      \end{itemize}
    \end{frame}

\begin{frame}{Exercise 2: Compute d}
  \small
  A: n=20 mean=72 SD=10; B: n=20 mean=68 SD=10. Compute Cohen's d.
\end{frame}

\begin{frame}{Solution 2}
      \begin{itemize}
        \item Pooled SD=10
\item d=(72-68)/10=0.4
      \end{itemize}
    \end{frame}

\begin{frame}{Exercise 3: Cautious conclusion}
  \small
  p-value=0.03 but effect size is tiny. What should you conclude?
\end{frame}

\begin{frame}{Solution 3}
      \begin{itemize}
        \item Evidence of difference, but small magnitude.
\item May not justify action without cost/benefit.
      \end{itemize}
    \end{frame}

\section{Demo}
\label{sec:demo}


\begin{frame}{Mini Demo (Python)}
  Run from the lecture folder:
  \begin{center}
    \texttt{python demo/demo.py}
  \end{center}
  \vspace{0.4em}
  Outputs:
  \begin{itemize}
    \item \texttt{images/demo.png}
    \item \texttt{data/results.txt}
  \end{itemize}
\end{frame}

\begin{frame}{Demo Output (Example)}
  \begin{center}
  \IfFileExists{../images/demo.png}{
    \includegraphics[width=0.92\linewidth]{demo.png}
  }{
    \small (Run demo to generate: \texttt{demo.png})
  }
  \end{center}
\end{frame}

\section{Summary}
\label{sec:summary}


\begin{frame}{Summary}
      \begin{itemize}[<+->]
        \item Key definitions and the main formula.
\item How to interpret results in context.
\item How the demo connects to the theory.
      \end{itemize}
    \end{frame}

\begin{frame}{Exit Question}
  \small
  What is one red flag when a paper reports only p-values and no effect sizes?
\end{frame}

\end{document}
