\documentclass[11pt]{article}
\usepackage[utf8]{inputenc}
\usepackage[T1]{fontenc}
\usepackage{geometry}
\usepackage{amsmath}
\usepackage{listings}
\usepackage{xcolor}
\usepackage{amssymb}
\usepackage{booktabs}
\usepackage{graphicx}
\graphicspath{{../images/}}
\usepackage{hyperref}
\geometry{margin=1in}

\title{Statistics and Data Analysis\\Unit 03 -- Lecture 07 Notes\\Case Exercise: Interpreting Hypothesis Testing Results}
\author{Tofik Ali}
\date{\today}

\begin{document}
\maketitle

\section*{Topic}
Interpret published hypothesis testing results; emphasize CI, effect size, and pitfalls.
\section*{How to Use These Notes}
These notes are written for students who are seeing the topic for the first time. They
follow the slide order, but add the missing 'why', interpretation, and common mistakes. If
you get stuck, look at the worked exercises and then run the Python demo.

Course repository (slides, demos, datasets): \url{https://github.com/tali7c/Statistics-and-Data-Analysis}

\section*{Time Plan (55 minutes)}
\begin{itemize}
  \item 0--10 min: Attendance + recap of previous lecture
  \item 10--35 min: Core concepts (this lecture's sections)
  \item 35--45 min: Exercises (solve 1--2 in class, rest as practice)
  \item 45--50 min: Mini demo + interpretation of output
  \item 50--55 min: Buffer / wrap-up (leave 5 minutes early)
\end{itemize}

\section*{Slide-by-slide Notes}
\subsection*{Title Slide}
State the lecture title clearly and connect it to what students already know.
Tell students what they will be able to do by the end (not just what you will cover).

\subsection*{Quick Links / Agenda}
Explain the structure of the lecture and where the exercises and demo appear.
\begin{itemize}
  \item Overview
  \item Reading Results
  \item Pitfalls
  \item Exercises
  \item Demo
  \item Summary
\end{itemize}

\subsection*{Learning Outcomes}
\begin{itemize}
  \item Interpret p-values and confidence intervals correctly
  \item Compute a simple effect size from summary statistics
  \item Identify common red flags (only p-values, many tests, no effect size)
  \item Write a cautious conclusion in plain language
  \item Avoid correlation-causation confusion
\end{itemize}
\paragraph{Why these outcomes matter.}
A \textbf{confidence interval (CI)} is an interval estimate, not a single number. The
correct interpretation is long-run: if we repeated the same sampling procedure many times,
about 95\% of the computed 95\% intervals would contain the true population value. It is
\textbf{not} correct to say there is a 95\% probability the parameter lies in a particular
computed interval.
A \textbf{p-value} is computed assuming the null hypothesis $H_0$ is true. It measures how
surprising the observed data (or something more extreme) would be under $H_0$. A small
p-value suggests the data is hard to explain by $H_0$ alone, but it does not tell you how
large the effect is or whether it is practically important.

\subsection*{Reading Results: Key Points}
\begin{itemize}
  \item Check n, center, spread
  \item Prefer CI + effect size
  \item Ask: what does it mean in the real world?
\end{itemize}
\paragraph{Explanation.}
\textbf{Effect size} quantifies \emph{how big} a difference/relationship is (e.g., Cohen's
$d$, correlation $r$). With large samples, even tiny effects can be statistically
significant, so reporting effect size prevents over-claiming.

\subsection*{Pitfalls: Key Points}
\begin{itemize}
  \item Multiple comparisons
  \item Selective reporting (p-hacking)
  \item Over-claiming causation
\end{itemize}

\subsection*{Exercises (with Solutions)}
Attempt the exercise first, then compare with the solution. Focus on interpretation, not
only arithmetic.

\subsection*{Exercise 1: Interpret CI}
95\% CI for (new-old) is (1.2, 3.8). What does it suggest?
\subsubsection*{Solution}
\begin{itemize}
  \item Likely positive effect (CI above 0).
  \item Magnitude between 1.2 and 3.8 units.
\end{itemize}

\subsection*{Exercise 2: Compute d}
A: n=20 mean=72 SD=10; B: n=20 mean=68 SD=10. Compute Cohen's d.
\subsubsection*{Solution}
\begin{itemize}
  \item Pooled SD=10
  \item d=(72-68)/10=0.4
\end{itemize}

\subsection*{Exercise 3: Cautious conclusion}
p-value=0.03 but effect size is tiny. What should you conclude?
\subsubsection*{Solution}
\begin{itemize}
  \item Evidence of difference, but small magnitude.
  \item May not justify action without cost/benefit.
\end{itemize}
\paragraph{Walkthrough.}
A \textbf{p-value} is computed assuming the null hypothesis $H_0$ is true. It measures how
surprising the observed data (or something more extreme) would be under $H_0$. A small
p-value suggests the data is hard to explain by $H_0$ alone, but it does not tell you how
large the effect is or whether it is practically important.
\textbf{Effect size} quantifies \emph{how big} a difference/relationship is (e.g., Cohen's
$d$, correlation $r$). With large samples, even tiny effects can be statistically
significant, so reporting effect size prevents over-claiming.

\subsection*{Mini Demo (Python)}
Run from the lecture folder:
\begin{verbatim}
python demo/demo.py
\end{verbatim}

Output files:
\begin{itemize}
  \item \texttt{images/demo.png}
  \item \texttt{data/results.txt}
\end{itemize}
\paragraph{What to show and say.}
\begin{itemize}
  \item Computes simple effect sizes (Cohen's d) from multiple study summaries.
  \item Plots effect sizes around 0 to practice interpretation and cautious conclusions.
  \item Use it to discuss why we want CI/effect size, not only p-values.
\end{itemize}

\subsection*{Demo Output (Example)}
\begin{center}
\IfFileExists{../images/demo.png}{
  \includegraphics[width=0.95\linewidth]{../images/demo.png}
}{
  \small (Run the demo to generate \texttt{images/demo.png})
}
\end{center}

\subsection*{Summary}
\begin{itemize}
  \item Key definitions and the main formula.
  \item How to interpret results in context.
  \item How the demo connects to the theory.
\end{itemize}

\subsection*{Exit Question}
What is one red flag when a paper reports only p-values and no effect sizes?
\paragraph{Suggested answer (for revision).}
Only reporting p-values hides the magnitude; without effect sizes/CIs you cannot judge
whether a result matters in practice.

\section*{References}
\begin{itemize}
  \item Montgomery, D. C., \& Runger, G. C. \textit{Applied Statistics and Probability for Engineers}, Wiley.
  \item Devore, J. L. \textit{Probability and Statistics for Engineering and the Sciences}, Cengage.
  \item McKinney, W. \textit{Python for Data Analysis}, O'Reilly.
\end{itemize}

% BEGIN SLIDE APPENDIX (AUTO-GENERATED)
\clearpage
\section*{Appendix: Slide Deck Content (Reference)}
\noindent The material below is a reference copy of the slide deck content. Exercise solutions are explained in the main notes where applicable.

\subsection*{Title Slide}
\titlepage
        \vspace{-0.5em}
        \begin{center}
          \small \texttt{https://github.com/tali7c/Statistics-and-Data-Analysis}
        \end{center}
\subsection*{Quick Links}
\centering
        \textbf{Overview}\hspace{0.6em}
\textbf{Reading Results}\hspace{0.6em}
\textbf{Pitfalls}\hspace{0.6em}
\textbf{Exercises}\hspace{0.6em}
\textbf{Demo}\hspace{0.6em}
\textbf{Summary}\hspace{0.6em}
\subsection*{Agenda}
\begin{itemize}
  \item Overview
  \item Reading Results
  \item Pitfalls
  \item Exercises
  \item Demo
  \item Summary
\end{itemize}
\subsection*{Learning Outcomes}
\begin{itemize}
        \item Interpret p-values and confidence intervals correctly
\item Compute a simple effect size from summary statistics
\item Identify common red flags (only p-values, many tests, no effect size)
\item Write a cautious conclusion in plain language
\item Avoid correlation-causation confusion
      \end{itemize}
\subsection*{Reading Results: Key Points}
\begin{itemize}
        \item Check n, center, spread
\item Prefer CI + effect size
\item Ask: what does it mean in the real world?
      \end{itemize}
\subsection*{Pitfalls: Key Points}
\begin{itemize}
        \item Multiple comparisons
\item Selective reporting (p-hacking)
\item Over-claiming causation
      \end{itemize}
\subsection*{Exercise 1: Interpret CI}
\small
  95\% CI for (new-old) is (1.2, 3.8). What does it suggest?
\subsection*{Solution 1}
\begin{itemize}
        \item Likely positive effect (CI above 0).
\item Magnitude between 1.2 and 3.8 units.
      \end{itemize}
\subsection*{Exercise 2: Compute d}
\small
  A: n=20 mean=72 SD=10; B: n=20 mean=68 SD=10. Compute Cohen's d.
\subsection*{Solution 2}
\begin{itemize}
        \item Pooled SD=10
\item d=(72-68)/10=0.4
      \end{itemize}
\subsection*{Exercise 3: Cautious conclusion}
\small
  p-value=0.03 but effect size is tiny. What should you conclude?
\subsection*{Solution 3}
\begin{itemize}
        \item Evidence of difference, but small magnitude.
\item May not justify action without cost/benefit.
      \end{itemize}
\subsection*{Mini Demo (Python)}
Run from the lecture folder:
  \begin{center}
    \texttt{python demo/demo.py}
  \end{center}
  \vspace{0.4em}
  Outputs:
  \begin{itemize}
    \item \texttt{images/demo.png}
    \item \texttt{data/results.txt}
  \end{itemize}
\subsection*{Demo Output (Example)}
\begin{center}
  \IfFileExists{../images/demo.png}{
    \includegraphics[width=0.92\linewidth]{demo.png}
  }{
    \small (Run demo to generate: \texttt{demo.png})
  }
  \end{center}
\subsection*{Summary}
\begin{itemize}
        \item Key definitions and the main formula.
\item How to interpret results in context.
\item How the demo connects to the theory.
      \end{itemize}
\subsection*{Exit Question}
\small
  What is one red flag when a paper reports only p-values and no effect sizes?
% END SLIDE APPENDIX (AUTO-GENERATED)

\end{document}
