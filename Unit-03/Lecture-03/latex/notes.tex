\documentclass[11pt]{article}
\usepackage[utf8]{inputenc}
\usepackage[T1]{fontenc}
\usepackage{geometry}
\usepackage{amsmath}
\usepackage{listings}
\usepackage{xcolor}
\usepackage{amssymb}
\usepackage{booktabs}
\usepackage{graphicx}
\graphicspath{{../images/}}
\usepackage{hyperref}
\geometry{margin=1in}

\title{Statistics and Data Analysis\\Unit 03 -- Lecture 03 Notes\\Hypothesis Testing (t-test): Paired Test and Effect Size}
\author{Tofik Ali}
\date{\today}

\begin{document}
\maketitle

\section*{Topic}
Paired t-test, mean difference, effect size, and interpretation.
\section*{How to Use These Notes}
These notes are written for students who are seeing the topic for the first time. They
follow the slide order, but add the missing 'why', interpretation, and common mistakes. If
you get stuck, look at the worked exercises and then run the Python demo.

Course repository (slides, demos, datasets): \url{https://github.com/tali7c/Statistics-and-Data-Analysis}

\section*{Time Plan (55 minutes)}
\begin{itemize}
  \item 0--10 min: Attendance + recap of previous lecture
  \item 10--35 min: Core concepts (this lecture's sections)
  \item 35--45 min: Exercises (solve 1--2 in class, rest as practice)
  \item 45--50 min: Mini demo + interpretation of output
  \item 50--55 min: Buffer / wrap-up (leave 5 minutes early)
\end{itemize}

\section*{Slide-by-slide Notes}
\subsection*{Title Slide}
State the lecture title clearly and connect it to what students already know.
Tell students what they will be able to do by the end (not just what you will cover).

\subsection*{Quick Links / Agenda}
Explain the structure of the lecture and where the exercises and demo appear.
\begin{itemize}
  \item Overview
  \item Paired Design
  \item Effect Size
  \item Exercises
  \item Demo
  \item Summary
\end{itemize}

\subsection*{Learning Outcomes}
\begin{itemize}
  \item Differentiate paired vs independent designs
  \item Compute within-pair differences di
  \item Run a paired t-test (conceptually)
  \item Explain effect size and why we report it
  \item Interpret results in context (not only p-value)
\end{itemize}
\paragraph{Why these outcomes matter.}
A \textbf{p-value} is computed assuming the null hypothesis $H_0$ is true. It measures how
surprising the observed data (or something more extreme) would be under $H_0$. A small
p-value suggests the data is hard to explain by $H_0$ alone, but it does not tell you how
large the effect is or whether it is practically important.
\textbf{Effect size} quantifies \emph{how big} a difference/relationship is (e.g., Cohen's
$d$, correlation $r$). With large samples, even tiny effects can be statistically
significant, so reporting effect size prevents over-claiming.

\subsection*{Paired Design: Key Points}
\begin{itemize}
  \item Same unit measured twice (before/after)
  \item Analyze differences di = after - before
  \item Pairing reduces noise from individual differences
\end{itemize}
\paragraph{Explanation.}
\textbf{Degrees of freedom (df)} roughly represent how much independent information is
available to estimate variability. For a one-sample t-test, $\mathrm{df}=n-1$ because one
constraint is used to estimate the sample mean. df affects the critical values and the shape
of the t-distribution (small df -> heavier tails).
\textbf{Differencing} transforms a series by subtracting the previous value: $y_t -
y_{t-1}$. It removes trend and can help achieve stationarity. Over-differencing can add
noise, so use the smallest differencing order that works.

\subsection*{Paired Design: Key Formula}
\[ t = \frac{\bar{d}}{s_d/\sqrt{n}},\quad \mathrm{df}=n-1 \]
\paragraph{How to read the formula.}
\textbf{Degrees of freedom (df)} roughly represent how much independent information is
available to estimate variability. For a one-sample t-test, $\mathrm{df}=n-1$ because one
constraint is used to estimate the sample mean. df affects the critical values and the shape
of the t-distribution (small df -> heavier tails).

\subsection*{Effect Size: Key Points}
\begin{itemize}
  \item p-value answers: evidence?
  \item Effect size answers: how big?
  \item Large n can make tiny effects significant
\end{itemize}
\paragraph{Explanation.}
A \textbf{p-value} is computed assuming the null hypothesis $H_0$ is true. It measures how
surprising the observed data (or something more extreme) would be under $H_0$. A small
p-value suggests the data is hard to explain by $H_0$ alone, but it does not tell you how
large the effect is or whether it is practically important.
\textbf{Effect size} quantifies \emph{how big} a difference/relationship is (e.g., Cohen's
$d$, correlation $r$). With large samples, even tiny effects can be statistically
significant, so reporting effect size prevents over-claiming.

\subsection*{Effect Size: Key Formula}
\[ d = \frac{\bar{x}_1-\bar{x}_2}{s_{\mathrm{pooled}}} \]
\paragraph{How to read the formula.}
\textbf{Effect size} quantifies \emph{how big} a difference/relationship is (e.g., Cohen's
$d$, correlation $r$). With large samples, even tiny effects can be statistically
significant, so reporting effect size prevents over-claiming.

\subsection*{Exercises (with Solutions)}
Attempt the exercise first, then compare with the solution. Focus on interpretation, not
only arithmetic.

\subsection*{Exercise 1: Compute differences}
Before/After: (10,12), (12,12), (11,14), (9,10). Compute di and dbar.
\subsubsection*{Solution}
\begin{itemize}
  \item di: 2,0,3,1
  \item dbar = 1.5
\end{itemize}
\paragraph{Walkthrough.}
\textbf{Differencing} transforms a series by subtracting the previous value: $y_t -
y_{t-1}$. It removes trend and can help achieve stationarity. Over-differencing can add
noise, so use the smallest differencing order that works.

\subsection*{Exercise 2: CI idea}
If the 95\% CI for mean difference excludes 0, what does it suggest?
\subsubsection*{Solution}
\begin{itemize}
  \item Evidence of a change (difference likely non-zero).
  \item Check magnitude and context.
\end{itemize}
\paragraph{Walkthrough.}
\textbf{Differencing} transforms a series by subtracting the previous value: $y_t -
y_{t-1}$. It removes trend and can help achieve stationarity. Over-differencing can add
noise, so use the smallest differencing order that works.

\subsection*{Exercise 3: Interpret d}
If Cohen's d=0.3, what does it suggest (rule of thumb)?
\subsubsection*{Solution}
\begin{itemize}
  \item Small effect (context dependent).
  \item Still may matter if cheap/safe to adopt.
\end{itemize}

\subsection*{Mini Demo (Python)}
Run from the lecture folder:
\begin{verbatim}
python demo/demo.py
\end{verbatim}

Output files:
\begin{itemize}
  \item \texttt{images/demo.png}
  \item \texttt{data/results.txt}
\end{itemize}
\paragraph{What to show and say.}
\begin{itemize}
  \item Generates before/after scores for the same individuals and runs a paired t-test.
  \item Shows paired lines so students see why pairing reduces noise.
  \item Reports Cohen's d (paired) to emphasize effect size, not only p-value.
\end{itemize}

\subsection*{Demo Output (Example)}
\begin{center}
\IfFileExists{../images/demo.png}{
  \includegraphics[width=0.95\linewidth]{../images/demo.png}
}{
  \small (Run the demo to generate \texttt{images/demo.png})
}
\end{center}

\subsection*{Summary}
\begin{itemize}
  \item Key definitions and the main formula.
  \item How to interpret results in context.
  \item How the demo connects to the theory.
\end{itemize}

\subsection*{Exit Question}
Why can paired designs be more powerful than independent designs?
\paragraph{Suggested answer (for revision).}
Pairing removes between-subject variation by comparing each person to themselves, so the
test has higher power.

\section*{References}
\begin{itemize}
  \item Montgomery, D. C., \& Runger, G. C. \textit{Applied Statistics and Probability for Engineers}, Wiley.
  \item Devore, J. L. \textit{Probability and Statistics for Engineering and the Sciences}, Cengage.
  \item McKinney, W. \textit{Python for Data Analysis}, O'Reilly.
\end{itemize}

% BEGIN SLIDE APPENDIX (AUTO-GENERATED)
\clearpage
\section*{Appendix: Slide Deck Content (Reference)}
\noindent The material below is a reference copy of the slide deck content. Exercise solutions are explained in the main notes where applicable.

\subsection*{Title Slide}
\titlepage
        \vspace{-0.5em}
        \begin{center}
          \small \texttt{https://github.com/tali7c/Statistics-and-Data-Analysis}
        \end{center}
\subsection*{Quick Links}
\centering
        \textbf{Overview}\hspace{0.6em}
\textbf{Paired Design}\hspace{0.6em}
\textbf{Effect Size}\hspace{0.6em}
\textbf{Exercises}\hspace{0.6em}
\textbf{Demo}\hspace{0.6em}
\textbf{Summary}\hspace{0.6em}
\subsection*{Agenda}
\begin{itemize}
  \item Overview
  \item Paired Design
  \item Effect Size
  \item Exercises
  \item Demo
  \item Summary
\end{itemize}
\subsection*{Learning Outcomes}
\begin{itemize}
        \item Differentiate paired vs independent designs
\item Compute within-pair differences di
\item Run a paired t-test (conceptually)
\item Explain effect size and why we report it
\item Interpret results in context (not only p-value)
      \end{itemize}
\subsection*{Paired Design: Key Points}
\begin{itemize}
        \item Same unit measured twice (before/after)
\item Analyze differences di = after - before
\item Pairing reduces noise from individual differences
      \end{itemize}
\subsection*{Paired Design: Key Formula}
\[ t = \frac{\bar{d}}{s_d/\sqrt{n}},\quad \mathrm{df}=n-1 \]
\subsection*{Effect Size: Key Points}
\begin{itemize}
        \item p-value answers: evidence?
\item Effect size answers: how big?
\item Large n can make tiny effects significant
      \end{itemize}
\subsection*{Effect Size: Key Formula}
\[ d = \frac{\bar{x}_1-\bar{x}_2}{s_{\mathrm{pooled}}} \]
\subsection*{Exercise 1: Compute differences}
\small
  Before/After: (10,12), (12,12), (11,14), (9,10). Compute di and dbar.
\subsection*{Solution 1}
\begin{itemize}
        \item di: 2,0,3,1
\item dbar = 1.5
      \end{itemize}
\subsection*{Exercise 2: CI idea}
\small
  If the 95\% CI for mean difference excludes 0, what does it suggest?
\subsection*{Solution 2}
\begin{itemize}
        \item Evidence of a change (difference likely non-zero).
\item Check magnitude and context.
      \end{itemize}
\subsection*{Exercise 3: Interpret d}
\small
  If Cohen's d=0.3, what does it suggest (rule of thumb)?
\subsection*{Solution 3}
\begin{itemize}
        \item Small effect (context dependent).
\item Still may matter if cheap/safe to adopt.
      \end{itemize}
\subsection*{Mini Demo (Python)}
Run from the lecture folder:
  \begin{center}
    \texttt{python demo/demo.py}
  \end{center}
  \vspace{0.4em}
  Outputs:
  \begin{itemize}
    \item \texttt{images/demo.png}
    \item \texttt{data/results.txt}
  \end{itemize}
\subsection*{Demo Output (Example)}
\begin{center}
  \IfFileExists{../images/demo.png}{
    \includegraphics[width=0.92\linewidth]{demo.png}
  }{
    \small (Run demo to generate: \texttt{demo.png})
  }
  \end{center}
\subsection*{Summary}
\begin{itemize}
        \item Key definitions and the main formula.
\item How to interpret results in context.
\item How the demo connects to the theory.
      \end{itemize}
\subsection*{Exit Question}
\small
  Why can paired designs be more powerful than independent designs?
% END SLIDE APPENDIX (AUTO-GENERATED)

\end{document}
