\documentclass[11pt]{article}
\usepackage[utf8]{inputenc}
\usepackage[T1]{fontenc}
\usepackage{geometry}
\usepackage{amsmath}
\usepackage{listings}
\usepackage{xcolor}
\usepackage{amssymb}
\usepackage{booktabs}
\usepackage{graphicx}
\graphicspath{{../images/}}
\usepackage{hyperref}
\geometry{margin=1in}

\title{Statistics and Data Analysis\\Unit 03 -- Lecture 02 Notes\\Hypothesis Testing (t-test): Concepts and Setup}
\author{Tofik Ali}
\date{\today}

\begin{document}
\maketitle

\section*{Topic}
Null vs alternative; one-sample and two-sample t-tests; p-values and interpretation.
\section*{How to Use These Notes}
These notes are written for students who are seeing the topic for the first time. They
follow the slide order, but add the missing 'why', interpretation, and common mistakes. If
you get stuck, look at the worked exercises and then run the Python demo.

Course repository (slides, demos, datasets): \url{https://github.com/tali7c/Statistics-and-Data-Analysis}

\section*{Time Plan (55 minutes)}
\begin{itemize}
  \item 0--10 min: Attendance + recap of previous lecture
  \item 10--35 min: Core concepts (this lecture's sections)
  \item 35--45 min: Exercises (solve 1--2 in class, rest as practice)
  \item 45--50 min: Mini demo + interpretation of output
  \item 50--55 min: Buffer / wrap-up (leave 5 minutes early)
\end{itemize}

\section*{Slide-by-slide Notes}
\subsection*{Title Slide}
State the lecture title clearly and connect it to what students already know.
Tell students what they will be able to do by the end (not just what you will cover).

\subsection*{Quick Links / Agenda}
Explain the structure of the lecture and where the exercises and demo appear.
\begin{itemize}
  \item Overview
  \item t-test Basics
  \item p-values
  \item Exercises
  \item Demo
  \item Summary
\end{itemize}

\subsection*{Learning Outcomes}
\begin{itemize}
  \item Define null and alternative hypotheses clearly
  \item Compute a one-sample t statistic (given summary)
  \item Explain p-value and significance level alpha
  \item Distinguish one-tailed vs two-tailed tests
  \item State key assumptions behind the t-test
\end{itemize}
\paragraph{Why these outcomes matter.}
\textbf{Hypothesis testing} is a decision framework. You start with a default claim (the
null hypothesis $H_0$) and ask whether the sample evidence is strong enough to reject it in
favor of an alternative $H_1$. The key idea is to control error rates: a test can be wrong,
so we choose a significance level $\alpha$ and interpret results with that risk in mind.
A \textbf{one-tailed} test is used when only one direction matters (only increase or only
decrease). A \textbf{two-tailed} test is used when deviations in both directions matter.
Choosing one-tailed after seeing the data is not valid; decide the tail based on the
research question before looking at results.

\subsection*{t-test Basics: Key Points}
\begin{itemize}
  \item H0/H1 setup
  \item Test statistic measures how far the sample is from H0
  \item Assumptions: independence, outliers, normality/CLT
\end{itemize}
\paragraph{Explanation.}
The \textbf{null hypothesis $H_0$} usually represents 'no effect' or a baseline value (e.g.,
$\mu=60$). The \textbf{alternative $H_1$} represents the effect you are looking for (e.g.,
$\mu\neq 60$ or $\mu>60$). We compute a test statistic and a p-value assuming $H_0$ is true.
\textbf{Degrees of freedom (df)} roughly represent how much independent information is
available to estimate variability. For a one-sample t-test, $\mathrm{df}=n-1$ because one
constraint is used to estimate the sample mean. df affects the critical values and the shape
of the t-distribution (small df -> heavier tails).
The \textbf{Central Limit Theorem (CLT)} explains why many methods work for large samples:
sample means tend to become approximately normal even when the population is not perfectly
normal. With small samples, you must be more careful about outliers and skewness.

\subsection*{t-test Basics: Key Formula}
\[ t = \frac{\bar{x}-\mu_0}{s/\sqrt{n}},\quad \mathrm{df}=n-1 \]
\paragraph{How to read the formula.}
\textbf{Degrees of freedom (df)} roughly represent how much independent information is
available to estimate variability. For a one-sample t-test, $\mathrm{df}=n-1$ because one
constraint is used to estimate the sample mean. df affects the critical values and the shape
of the t-distribution (small df -> heavier tails).

\subsection*{p-values: Key Points}
\begin{itemize}
  \item p-value = probability of data (or more extreme) assuming H0
  \item Small p-value -> evidence against H0
  \item p-value is not effect size
\end{itemize}
\paragraph{Explanation.}
The \textbf{null hypothesis $H_0$} usually represents 'no effect' or a baseline value (e.g.,
$\mu=60$). The \textbf{alternative $H_1$} represents the effect you are looking for (e.g.,
$\mu\neq 60$ or $\mu>60$). We compute a test statistic and a p-value assuming $H_0$ is true.
A \textbf{p-value} is computed assuming the null hypothesis $H_0$ is true. It measures how
surprising the observed data (or something more extreme) would be under $H_0$. A small
p-value suggests the data is hard to explain by $H_0$ alone, but it does not tell you how
large the effect is or whether it is practically important.
\textbf{Effect size} quantifies \emph{how big} a difference/relationship is (e.g., Cohen's
$d$, correlation $r$). With large samples, even tiny effects can be statistically
significant, so reporting effect size prevents over-claiming.

\subsection*{Exercises (with Solutions)}
Attempt the exercise first, then compare with the solution. Focus on interpretation, not
only arithmetic.

\subsection*{Exercise 1: Write hypotheses}
Claim: mean score is 60. Write H0 and H1 for a two-sided test.
\subsubsection*{Solution}
\begin{itemize}
  \item H0: mu = 60
  \item H1: mu != 60
\end{itemize}
\paragraph{Walkthrough.}
\textbf{Hypothesis testing} is a decision framework. You start with a default claim (the
null hypothesis $H_0$) and ask whether the sample evidence is strong enough to reject it in
favor of an alternative $H_1$. The key idea is to control error rates: a test can be wrong,
so we choose a significance level $\alpha$ and interpret results with that risk in mind.
The \textbf{null hypothesis $H_0$} usually represents 'no effect' or a baseline value (e.g.,
$\mu=60$). The \textbf{alternative $H_1$} represents the effect you are looking for (e.g.,
$\mu\neq 60$ or $\mu>60$). We compute a test statistic and a p-value assuming $H_0$ is true.

\subsection*{Exercise 2: Compute t}
Given n=25, xbar=53, s=10, test H0: mu=50. Compute t.
\subsubsection*{Solution}
\begin{itemize}
  \item SE = 10/sqrt(25) = 2
  \item t = (53-50)/2 = 1.5
  \item df = 24
\end{itemize}
\paragraph{Walkthrough.}
The \textbf{null hypothesis $H_0$} usually represents 'no effect' or a baseline value (e.g.,
$\mu=60$). The \textbf{alternative $H_1$} represents the effect you are looking for (e.g.,
$\mu\neq 60$ or $\mu>60$). We compute a test statistic and a p-value assuming $H_0$ is true.
\textbf{Degrees of freedom (df)} roughly represent how much independent information is
available to estimate variability. For a one-sample t-test, $\mathrm{df}=n-1$ because one
constraint is used to estimate the sample mean. df affects the critical values and the shape
of the t-distribution (small df -> heavier tails).

\subsection*{Exercise 3: Tail choice}
You want to show a new method increases mean score. One-tailed or two-tailed?
\subsubsection*{Solution}
\begin{itemize}
  \item One-tailed (right): H1: mu > mu0
\end{itemize}
\paragraph{Walkthrough.}
A \textbf{one-tailed} test is used when only one direction matters (only increase or only
decrease). A \textbf{two-tailed} test is used when deviations in both directions matter.
Choosing one-tailed after seeing the data is not valid; decide the tail based on the
research question before looking at results.

\subsection*{Mini Demo (Python)}
Run from the lecture folder:
\begin{verbatim}
python demo/demo.py
\end{verbatim}

Output files:
\begin{itemize}
  \item \texttt{images/demo.png}
  \item \texttt{data/results.txt}
\end{itemize}
\paragraph{What to show and say.}
\begin{itemize}
  \item Generates two independent groups and runs a Welch two-sample t-test.
  \item Shows a boxplot so students can connect the test to the distribution shape/outliers.
  \item Open data/results.txt to see the t-statistic and p-value and interpret them.
\end{itemize}

\subsection*{Demo Output (Example)}
\begin{center}
\IfFileExists{../images/demo.png}{
  \includegraphics[width=0.95\linewidth]{../images/demo.png}
}{
  \small (Run the demo to generate \texttt{images/demo.png})
}
\end{center}

\subsection*{Summary}
\begin{itemize}
  \item Key definitions and the main formula.
  \item How to interpret results in context.
  \item How the demo connects to the theory.
\end{itemize}

\subsection*{Exit Question}
Why can a very small p-value still be unimportant in practice?
\paragraph{Suggested answer (for revision).}
With large n, even tiny differences can yield very small p-values; always check effect size
and practical impact.

\section*{References}
\begin{itemize}
  \item Montgomery, D. C., \& Runger, G. C. \textit{Applied Statistics and Probability for Engineers}, Wiley.
  \item Devore, J. L. \textit{Probability and Statistics for Engineering and the Sciences}, Cengage.
  \item McKinney, W. \textit{Python for Data Analysis}, O'Reilly.
\end{itemize}

% BEGIN SLIDE APPENDIX (AUTO-GENERATED)
\clearpage
\section*{Appendix: Slide Deck Content (Reference)}
\noindent The material below is a reference copy of the slide deck content. Exercise solutions are explained in the main notes where applicable.

\subsection*{Title Slide}
\titlepage
        \vspace{-0.5em}
        \begin{center}
          \small \texttt{https://github.com/tali7c/Statistics-and-Data-Analysis}
        \end{center}
\subsection*{Quick Links}
\centering
        \textbf{Overview}\hspace{0.6em}
\textbf{t-test Basics}\hspace{0.6em}
\textbf{p-values}\hspace{0.6em}
\textbf{Exercises}\hspace{0.6em}
\textbf{Demo}\hspace{0.6em}
\textbf{Summary}\hspace{0.6em}
\subsection*{Agenda}
\begin{itemize}
  \item Overview
  \item t-test Basics
  \item p-values
  \item Exercises
  \item Demo
  \item Summary
\end{itemize}
\subsection*{Learning Outcomes}
\begin{itemize}
        \item Define null and alternative hypotheses clearly
\item Compute a one-sample t statistic (given summary)
\item Explain p-value and significance level alpha
\item Distinguish one-tailed vs two-tailed tests
\item State key assumptions behind the t-test
      \end{itemize}
\subsection*{t-test Basics: Key Points}
\begin{itemize}
        \item H0/H1 setup
\item Test statistic measures how far the sample is from H0
\item Assumptions: independence, outliers, normality/CLT
      \end{itemize}
\subsection*{t-test Basics: Key Formula}
\[ t = \frac{\bar{x}-\mu_0}{s/\sqrt{n}},\quad \mathrm{df}=n-1 \]
\subsection*{p-values: Key Points}
\begin{itemize}
        \item p-value = probability of data (or more extreme) assuming H0
\item Small p-value -> evidence against H0
\item p-value is not effect size
      \end{itemize}
\subsection*{Exercise 1: Write hypotheses}
\small
  Claim: mean score is 60. Write H0 and H1 for a two-sided test.
\subsection*{Solution 1}
\begin{itemize}
        \item H0: mu = 60
\item H1: mu != 60
      \end{itemize}
\subsection*{Exercise 2: Compute t}
\small
  Given n=25, xbar=53, s=10, test H0: mu=50. Compute t.
\subsection*{Solution 2}
\begin{itemize}
        \item SE = 10/sqrt(25) = 2
\item t = (53-50)/2 = 1.5
\item df = 24
      \end{itemize}
\subsection*{Exercise 3: Tail choice}
\small
  You want to show a new method increases mean score. One-tailed or two-tailed?
\subsection*{Solution 3}
\begin{itemize}
    \item One-tailed (right): H1: mu > mu0
  \end{itemize}
\subsection*{Mini Demo (Python)}
Run from the lecture folder:
  \begin{center}
    \texttt{python demo/demo.py}
  \end{center}
  \vspace{0.4em}
  Outputs:
  \begin{itemize}
    \item \texttt{images/demo.png}
    \item \texttt{data/results.txt}
  \end{itemize}
\subsection*{Demo Output (Example)}
\begin{center}
  \IfFileExists{../images/demo.png}{
    \includegraphics[width=0.92\linewidth]{demo.png}
  }{
    \small (Run demo to generate: \texttt{demo.png})
  }
  \end{center}
\subsection*{Summary}
\begin{itemize}
        \item Key definitions and the main formula.
\item How to interpret results in context.
\item How the demo connects to the theory.
      \end{itemize}
\subsection*{Exit Question}
\small
  Why can a very small p-value still be unimportant in practice?
% END SLIDE APPENDIX (AUTO-GENERATED)

\end{document}
