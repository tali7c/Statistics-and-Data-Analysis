\documentclass[11pt]{article}
\usepackage[utf8]{inputenc}
\usepackage[T1]{fontenc}
\usepackage{geometry}
\usepackage{amsmath}
\usepackage{booktabs}
\usepackage{hyperref}
\geometry{margin=1in}

\title{Statistics and Data Analysis\\Unit 03 -- Lecture 02 Notes}
\author{Tofik Ali}
\date{\today}

\begin{document}
\maketitle

\section*{Topic}
Null vs alternative; one-sample and two-sample t-tests; p-values and interpretation.

\subsection*{Learning Outcomes}
\begin{itemize}
  \item Define null and alternative hypotheses clearly
  \item Compute a one-sample t statistic (given summary)
  \item Explain p-value and significance level alpha
  \item Distinguish one-tailed vs two-tailed tests
  \item State key assumptions behind the t-test
\end{itemize}

\section*{Detailed Notes}
These notes are designed to be read alongside the slides. They expand each slide bullet into
plain-language explanations, small worked examples, and common pitfalls. When a formula
appears, emphasize (1) what each symbol means, (2) the assumptions needed to use it, and (3)
how to interpret the final number in the problem context.

\section*{t-test Basics}
\begin{itemize}
  \item H0/H1 setup
  \item Test statistic measures how far the sample is from H0
  \item Assumptions: independence, outliers, normality/CLT
\end{itemize}

\section*{p-values}
\begin{itemize}
  \item p-value = probability of data (or more extreme) assuming H0
  \item Small p-value -> evidence against H0
  \item p-value is not effect size
\end{itemize}

\section*{Exercises (with Solutions)}
\subsection*{Exercise 1: Write hypotheses}
Claim: mean score is 60. Write H0 and H1 for a two-sided test.
\subsection*{Solution}
\begin{itemize}
  \item H0: mu = 60
  \item H1: mu != 60
\end{itemize}

\subsection*{Exercise 2: Compute t}
Given n=25, xbar=53, s=10, test H0: mu=50. Compute t.
\subsection*{Solution}
\begin{itemize}
  \item SE = 10/sqrt(25) = 2
  \item t = (53-50)/2 = 1.5
  \item df = 24
\end{itemize}

\subsection*{Exercise 3: Tail choice}
You want to show a new method increases mean score. One-tailed or two-tailed?
\subsection*{Solution}
\begin{itemize}
  \item One-tailed (right): H1: mu > mu0
\end{itemize}

\section*{Exit Question}
Why can a very small p-value still be unimportant in practice?

\section*{Demo (Python)}
Run from the lecture folder:
\begin{verbatim}
python demo/demo.py
\end{verbatim}

Output files:
\begin{itemize}
  \item \texttt{images/demo.png}
  \item \texttt{data/results.txt}
\end{itemize}

\section*{References}
\begin{itemize}
  \item Montgomery, D. C., \& Runger, G. C. \textit{Applied Statistics and Probability for Engineers}, Wiley.
  \item Devore, J. L. \textit{Probability and Statistics for Engineering and the Sciences}, Cengage.
  \item McKinney, W. \textit{Python for Data Analysis}, O'Reilly.
\end{itemize}
\end{document}
