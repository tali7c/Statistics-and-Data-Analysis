\documentclass{beamer}

      \usetheme{Berlin}
      \usecolortheme{Orchid}
      \useoutertheme{miniframes}
      \setbeamertemplate{navigation symbols}{}

      \usepackage{amsmath}
      \usepackage{amssymb}
      \usepackage{booktabs}
      \usepackage{graphicx}
      \graphicspath{{../images/}}

      \title[Statistics and Data Analysis]{Statistics and Data Analysis}
      \subtitle{Unit 03 -- Lecture 02: Hypothesis Testing (t-test): Concepts and Setup}
      \author{Tofik Ali}
      \institute{School of Computer Science, UPES Dehradun}
      \date{\today}

      \begin{document}

      \begin{frame}
        \titlepage
        \vspace{-0.5em}
        \begin{center}
          \small \texttt{https://github.com/tali7c/Statistics-and-Data-Analysis}
        \end{center}
      \end{frame}

      \begin{frame}{Quick Links}
        \centering
        \hyperlink{sec:overview}{\beamerbutton{Overview}}\hspace{0.6em}
\hyperlink{sec:ttest}{\beamerbutton{t-test Basics}}\hspace{0.6em}
\hyperlink{sec:pvalues}{\beamerbutton{p-values}}\hspace{0.6em}
\hyperlink{sec:exercises}{\beamerbutton{Exercises}}\hspace{0.6em}
\hyperlink{sec:demo}{\beamerbutton{Demo}}\hspace{0.6em}
\hyperlink{sec:summary}{\beamerbutton{Summary}}\hspace{0.6em}
      \end{frame}

      \begin{frame}{Agenda}
        \tableofcontents
      \end{frame}

\section{Overview}
\label{sec:overview}


\begin{frame}{Learning Outcomes}
      \begin{itemize}[<+->]
        \item Define null and alternative hypotheses clearly
\item Compute a one-sample t statistic (given summary)
\item Explain p-value and significance level alpha
\item Distinguish one-tailed vs two-tailed tests
\item State key assumptions behind the t-test
      \end{itemize}
    \end{frame}

\section{t-test Basics}
\label{sec:ttest}


\begin{frame}{t-test Basics: Key Points}
      \begin{itemize}[<+->]
        \item H0/H1 setup
\item Test statistic measures how far the sample is from H0
\item Assumptions: independence, outliers, normality/CLT
      \end{itemize}
    \end{frame}

\begin{frame}{t-test Basics: Key Formula}
  \[ t = \frac{\bar{x}-\mu_0}{s/\sqrt{n}},\quad \mathrm{df}=n-1 \]
\end{frame}

\section{p-values}
\label{sec:pvalues}


\begin{frame}{p-values: Key Points}
      \begin{itemize}[<+->]
        \item p-value = probability of data (or more extreme) assuming H0
\item Small p-value -> evidence against H0
\item p-value is not effect size
      \end{itemize}
    \end{frame}

\section{Exercises}
\label{sec:exercises}


\begin{frame}{Exercise 1: Write hypotheses}
  \small
  Claim: mean score is 60. Write H0 and H1 for a two-sided test.
\end{frame}

\begin{frame}{Solution 1}
      \begin{itemize}
        \item H0: mu = 60
\item H1: mu != 60
      \end{itemize}
    \end{frame}

\begin{frame}{Exercise 2: Compute t}
  \small
  Given n=25, xbar=53, s=10, test H0: mu=50. Compute t.
\end{frame}

\begin{frame}{Solution 2}
      \begin{itemize}
        \item SE = 10/sqrt(25) = 2
\item t = (53-50)/2 = 1.5
\item df = 24
      \end{itemize}
    \end{frame}

\begin{frame}{Exercise 3: Tail choice}
  \small
  You want to show a new method increases mean score. One-tailed or two-tailed?
\end{frame}

\begin{frame}{Solution 3}
  \begin{itemize}
    \item One-tailed (right): H1: mu > mu0
  \end{itemize}
\end{frame}

\section{Demo}
\label{sec:demo}


\begin{frame}{Mini Demo (Python)}
  Run from the lecture folder:
  \begin{center}
    \texttt{python demo/demo.py}
  \end{center}
  \vspace{0.4em}
  Outputs:
  \begin{itemize}
    \item \texttt{images/demo.png}
    \item \texttt{data/results.txt}
  \end{itemize}
\end{frame}

\begin{frame}{Demo Output (Example)}
  \begin{center}
  \IfFileExists{../images/demo.png}{
    \includegraphics[width=0.92\linewidth]{demo.png}
  }{
    \small (Run demo to generate: \texttt{demo.png})
  }
  \end{center}
\end{frame}

\section{Summary}
\label{sec:summary}


\begin{frame}{Summary}
      \begin{itemize}[<+->]
        \item Key definitions and the main formula.
\item How to interpret results in context.
\item How the demo connects to the theory.
      \end{itemize}
    \end{frame}

\begin{frame}{Exit Question}
  \small
  Why can a very small p-value still be unimportant in practice?
\end{frame}

\end{document}
