\documentclass{beamer}

      \usetheme{Berlin}
      \usecolortheme{Orchid}
      \useoutertheme{miniframes}
      \setbeamertemplate{navigation symbols}{}

      \usepackage{amsmath}
      \usepackage{amssymb}
      \usepackage{booktabs}
      \usepackage{graphicx}
      \graphicspath{{../images/}}

      \title[Statistics and Data Analysis]{Statistics and Data Analysis}
      \subtitle{Unit 03 -- Lecture 01: Population, Sample, Sampling, Estimation and Confidence Intervals}
      \author{Tofik Ali}
      \institute{School of Computer Science, UPES Dehradun}
      \date{\today}

      \begin{document}

      \begin{frame}
        \titlepage
        \vspace{-0.5em}
        \begin{center}
          \small \texttt{https://github.com/tali7c/Statistics-and-Data-Analysis}
        \end{center}
      \end{frame}

      \begin{frame}{Quick Links}
        \centering
        \hyperlink{sec:overview}{\beamerbutton{Overview}}\hspace{0.6em}
\hyperlink{sec:sampling}{\beamerbutton{Sampling}}\hspace{0.6em}
\hyperlink{sec:ci}{\beamerbutton{Confidence Intervals}}\hspace{0.6em}
\hyperlink{sec:exercises}{\beamerbutton{Exercises}}\hspace{0.6em}
\hyperlink{sec:demo}{\beamerbutton{Demo}}\hspace{0.6em}
\hyperlink{sec:summary}{\beamerbutton{Summary}}\hspace{0.6em}
      \end{frame}

      \begin{frame}{Agenda}
        \tableofcontents
      \end{frame}

\section{Overview}
\label{sec:overview}


\begin{frame}{Learning Outcomes}
      \begin{itemize}[<+->]
        \item Differentiate population parameters and sample statistics
\item Explain sampling bias vs random error
\item Describe common sampling methods (SRS, stratified, cluster)
\item Compute and interpret a basic confidence interval for a mean
      \end{itemize}
    \end{frame}

\section{Sampling}
\label{sec:sampling}


\begin{frame}{Sampling: Key Points}
      \begin{itemize}[<+->]
        \item Population vs sample
\item Bias vs random error
\item Representative sampling matters
      \end{itemize}
    \end{frame}

\section{Confidence Intervals}
\label{sec:ci}


\begin{frame}{Confidence Intervals: Key Points}
      \begin{itemize}[<+->]
        \item Interpretation: long-run coverage
\item Width depends on n and variability
\item CIs support decision-making with uncertainty
      \end{itemize}
    \end{frame}

\begin{frame}{Confidence Intervals: Key Formula}
  \[ \bar{x} \pm t_{\alpha/2,\,n-1}\,\frac{s}{\sqrt{n}} \]
\end{frame}

\section{Exercises}
\label{sec:exercises}


\begin{frame}{Exercise 1: Parameter vs Statistic}
  \small
  Give one example of a parameter and one of a statistic.
\end{frame}

\begin{frame}{Solution 1}
      \begin{itemize}
        \item Parameter: population mean
\item Statistic: sample mean
      \end{itemize}
    \end{frame}

\begin{frame}{Exercise 2: CI Interpretation}
  \small
  In one sentence, what does a 95\% CI mean (correctly)?
\end{frame}

\begin{frame}{Solution 2}
  \begin{itemize}
    \item About 95\% of such intervals contain the true mean in repeated sampling.
  \end{itemize}
\end{frame}

\begin{frame}{Exercise 3: Bias Scenario}
  \small
  Why does convenience sampling create bias?
\end{frame}

\begin{frame}{Solution 3}
  \begin{itemize}
    \item Because some groups are over/under-represented systematically.
  \end{itemize}
\end{frame}

\section{Demo}
\label{sec:demo}


\begin{frame}{Mini Demo (Python)}
  Run from the lecture folder:
  \begin{center}
    \texttt{python demo/demo.py}
  \end{center}
  \vspace{0.4em}
  Outputs:
  \begin{itemize}
    \item \texttt{images/demo.png}
    \item \texttt{data/results.txt}
  \end{itemize}
\end{frame}

\begin{frame}{Demo Output (Example)}
  \begin{center}
  \IfFileExists{../images/demo.png}{
    \includegraphics[width=0.92\linewidth]{demo.png}
  }{
    \small (Run demo to generate: \texttt{demo.png})
  }
  \end{center}
\end{frame}

\section{Summary}
\label{sec:summary}


\begin{frame}{Summary}
      \begin{itemize}[<+->]
        \item Key definitions and the main formula.
\item How to interpret results in context.
\item How the demo connects to the theory.
      \end{itemize}
    \end{frame}

\begin{frame}{Exit Question}
  \small
  If your CI is too wide, what two actions reduce its width (without cheating)?
\end{frame}

\end{document}
