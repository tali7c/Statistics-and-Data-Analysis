\documentclass[11pt]{article}
\usepackage[utf8]{inputenc}
\usepackage[T1]{fontenc}
\usepackage{geometry}
\usepackage{amsmath}
\usepackage{listings}
\usepackage{xcolor}
\usepackage{amssymb}
\usepackage{booktabs}
\usepackage{graphicx}
\graphicspath{{../images/}}
\usepackage{hyperref}
\geometry{margin=1in}

\title{Statistics and Data Analysis\\Unit 03 -- Lecture 01 Notes\\Population, Sample, Sampling, Estimation and Confidence Intervals}
\author{Tofik Ali}
\date{\today}

\begin{document}
\maketitle

\section*{Topic}
Population vs sample; sampling techniques; estimation; confidence intervals (overview).
\section*{How to Use These Notes}
These notes are written for students who are seeing the topic for the first time. They
follow the slide order, but add the missing 'why', interpretation, and common mistakes. If
you get stuck, look at the worked exercises and then run the Python demo.

Course repository (slides, demos, datasets): \url{https://github.com/tali7c/Statistics-and-Data-Analysis}

\section*{Time Plan (55 minutes)}
\begin{itemize}
  \item 0--10 min: Attendance + recap of previous lecture
  \item 10--35 min: Core concepts (this lecture's sections)
  \item 35--45 min: Exercises (solve 1--2 in class, rest as practice)
  \item 45--50 min: Mini demo + interpretation of output
  \item 50--55 min: Buffer / wrap-up (leave 5 minutes early)
\end{itemize}

\section*{Slide-by-slide Notes}
\subsection*{Title Slide}
State the lecture title clearly and connect it to what students already know.
Tell students what they will be able to do by the end (not just what you will cover).

\subsection*{Quick Links / Agenda}
Explain the structure of the lecture and where the exercises and demo appear.
\begin{itemize}
  \item Overview
  \item Sampling
  \item Confidence Intervals
  \item Exercises
  \item Demo
  \item Summary
\end{itemize}

\subsection*{Learning Outcomes}
\begin{itemize}
  \item Differentiate population parameters and sample statistics
  \item Explain sampling bias vs random error
  \item Describe common sampling methods (SRS, stratified, cluster)
  \item Compute and interpret a basic confidence interval for a mean
\end{itemize}
\paragraph{Why these outcomes matter.}
A \textbf{population} is the entire group we care about (all students, all sensors, all
invoices, etc.). A \textbf{sample} is the smaller subset we actually observe. Most mistakes
in statistics happen when we forget that we only see a sample, but we want to make a
statement about the population.
A \textbf{parameter} is a fixed (but usually unknown) number that describes the population
(e.g., $\mu$, $\sigma$). A \textbf{statistic} is a number computed from the sample (e.g.,
$\bar{x}$, $s$). Statistics vary from sample to sample, which is why we talk about
uncertainty.

\subsection*{Sampling: Key Points}
\begin{itemize}
  \item Population vs sample
  \item Bias vs random error
  \item Representative sampling matters
\end{itemize}
\paragraph{Explanation.}
A \textbf{population} is the entire group we care about (all students, all sensors, all
invoices, etc.). A \textbf{sample} is the smaller subset we actually observe. Most mistakes
in statistics happen when we forget that we only see a sample, but we want to make a
statement about the population.
\textbf{Bias} is a systematic error: your method tends to be wrong in the same direction
again and again (too high or too low). It does not disappear by taking more samples if the
sampling process is flawed. Fixing bias usually requires changing the data collection
procedure (sampling frame, selection method, non-response handling).
\textbf{Random error} is natural variation due to sampling and measurement noise. Unlike
bias, random error can be reduced by increasing sample size or improving measurement
quality. In many formulas you see a $\sqrt{n}$ in the denominator: that is the mathematical
reason larger samples give more stable estimates.

\subsection*{Confidence Intervals: Key Points}
\begin{itemize}
  \item Interpretation: long-run coverage
  \item Width depends on n and variability
  \item CIs support decision-making with uncertainty
\end{itemize}
\paragraph{Explanation.}
A \textbf{confidence interval (CI)} is an interval estimate, not a single number. The
correct interpretation is long-run: if we repeated the same sampling procedure many times,
about 95\% of the computed 95\% intervals would contain the true population value. It is
\textbf{not} correct to say there is a 95\% probability the parameter lies in a particular
computed interval.
The \textbf{significance level} $\alpha$ is the maximum Type I error rate you are willing to
tolerate: the probability of rejecting $H_0$ when $H_0$ is actually true. Common choices are
0.05 or 0.01, but the right value depends on consequences of false alarms vs missed
detections.

\subsection*{Confidence Intervals: Key Formula}
\[ \bar{x} \pm t_{\alpha/2,\,n-1}\,\frac{s}{\sqrt{n}} \]
\paragraph{How to read the formula.}
A \textbf{confidence interval (CI)} is an interval estimate, not a single number. The
correct interpretation is long-run: if we repeated the same sampling procedure many times,
about 95\% of the computed 95\% intervals would contain the true population value. It is
\textbf{not} correct to say there is a 95\% probability the parameter lies in a particular
computed interval.
The \textbf{significance level} $\alpha$ is the maximum Type I error rate you are willing to
tolerate: the probability of rejecting $H_0$ when $H_0$ is actually true. Common choices are
0.05 or 0.01, but the right value depends on consequences of false alarms vs missed
detections.

\subsection*{Exercises (with Solutions)}
Attempt the exercise first, then compare with the solution. Focus on interpretation, not
only arithmetic.

\subsection*{Exercise 1: Parameter vs Statistic}
Give one example of a parameter and one of a statistic.
\subsubsection*{Solution}
\begin{itemize}
  \item Parameter: population mean
  \item Statistic: sample mean
\end{itemize}
\paragraph{Walkthrough.}
A \textbf{population} is the entire group we care about (all students, all sensors, all
invoices, etc.). A \textbf{sample} is the smaller subset we actually observe. Most mistakes
in statistics happen when we forget that we only see a sample, but we want to make a
statement about the population.
A \textbf{parameter} is a fixed (but usually unknown) number that describes the population
(e.g., $\mu$, $\sigma$). A \textbf{statistic} is a number computed from the sample (e.g.,
$\bar{x}$, $s$). Statistics vary from sample to sample, which is why we talk about
uncertainty.

\subsection*{Exercise 2: CI Interpretation}
In one sentence, what does a 95\% CI mean (correctly)?
\subsubsection*{Solution}
\begin{itemize}
  \item About 95\% of such intervals contain the true mean in repeated sampling.
\end{itemize}

\subsection*{Exercise 3: Bias Scenario}
Why does convenience sampling create bias?
\subsubsection*{Solution}
\begin{itemize}
  \item Because some groups are over/under-represented systematically.
\end{itemize}
\paragraph{Walkthrough.}
\textbf{Bias} is a systematic error: your method tends to be wrong in the same direction
again and again (too high or too low). It does not disappear by taking more samples if the
sampling process is flawed. Fixing bias usually requires changing the data collection
procedure (sampling frame, selection method, non-response handling).

\subsection*{Mini Demo (Python)}
Run from the lecture folder:
\begin{verbatim}
python demo/demo.py
\end{verbatim}

Output files:
\begin{itemize}
  \item \texttt{images/demo.png}
  \item \texttt{data/results.txt}
\end{itemize}
\paragraph{What to show and say.}
\begin{itemize}
  \item Creates a random sample and computes a 95\% CI for the mean (t-interval).
  \item Plots a histogram of many sample means to show the sampling distribution.
  \item Use it to discuss why larger n gives a narrower CI (smaller standard error).
\end{itemize}

\subsection*{Demo Output (Example)}
\begin{center}
\IfFileExists{../images/demo.png}{
  \includegraphics[width=0.95\linewidth]{../images/demo.png}
}{
  \small (Run the demo to generate \texttt{images/demo.png})
}
\end{center}

\subsection*{Summary}
\begin{itemize}
  \item Key definitions and the main formula.
  \item How to interpret results in context.
  \item How the demo connects to the theory.
\end{itemize}

\subsection*{Exit Question}
If your CI is too wide, what two actions reduce its width (without cheating)?
\paragraph{Suggested answer (for revision).}
Increase sample size n and reduce variability s (better measurement or a more
homogeneous/stratified sample).

\section*{References}
\begin{itemize}
  \item Montgomery, D. C., \& Runger, G. C. \textit{Applied Statistics and Probability for Engineers}, Wiley.
  \item Devore, J. L. \textit{Probability and Statistics for Engineering and the Sciences}, Cengage.
  \item McKinney, W. \textit{Python for Data Analysis}, O'Reilly.
\end{itemize}

% BEGIN SLIDE APPENDIX (AUTO-GENERATED)
\clearpage
\section*{Appendix: Slide Deck Content (Reference)}
\noindent The material below is a reference copy of the slide deck content. Exercise solutions are explained in the main notes where applicable.

\subsection*{Title Slide}
\titlepage
        \vspace{-0.5em}
        \begin{center}
          \small \texttt{https://github.com/tali7c/Statistics-and-Data-Analysis}
        \end{center}
\subsection*{Quick Links}
\centering
        \textbf{Overview}\hspace{0.6em}
\textbf{Sampling}\hspace{0.6em}
\textbf{Confidence Intervals}\hspace{0.6em}
\textbf{Exercises}\hspace{0.6em}
\textbf{Demo}\hspace{0.6em}
\textbf{Summary}\hspace{0.6em}
\subsection*{Agenda}
\begin{itemize}
  \item Overview
  \item Sampling
  \item Confidence Intervals
  \item Exercises
  \item Demo
  \item Summary
\end{itemize}
\subsection*{Learning Outcomes}
\begin{itemize}
        \item Differentiate population parameters and sample statistics
\item Explain sampling bias vs random error
\item Describe common sampling methods (SRS, stratified, cluster)
\item Compute and interpret a basic confidence interval for a mean
      \end{itemize}
\subsection*{Sampling: Key Points}
\begin{itemize}
        \item Population vs sample
\item Bias vs random error
\item Representative sampling matters
      \end{itemize}
\subsection*{Confidence Intervals: Key Points}
\begin{itemize}
        \item Interpretation: long-run coverage
\item Width depends on n and variability
\item CIs support decision-making with uncertainty
      \end{itemize}
\subsection*{Confidence Intervals: Key Formula}
\[ \bar{x} \pm t_{\alpha/2,\,n-1}\,\frac{s}{\sqrt{n}} \]
\subsection*{Exercise 1: Parameter vs Statistic}
\small
  Give one example of a parameter and one of a statistic.
\subsection*{Solution 1}
\begin{itemize}
        \item Parameter: population mean
\item Statistic: sample mean
      \end{itemize}
\subsection*{Exercise 2: CI Interpretation}
\small
  In one sentence, what does a 95\% CI mean (correctly)?
\subsection*{Solution 2}
\begin{itemize}
    \item About 95\% of such intervals contain the true mean in repeated sampling.
  \end{itemize}
\subsection*{Exercise 3: Bias Scenario}
\small
  Why does convenience sampling create bias?
\subsection*{Solution 3}
\begin{itemize}
    \item Because some groups are over/under-represented systematically.
  \end{itemize}
\subsection*{Mini Demo (Python)}
Run from the lecture folder:
  \begin{center}
    \texttt{python demo/demo.py}
  \end{center}
  \vspace{0.4em}
  Outputs:
  \begin{itemize}
    \item \texttt{images/demo.png}
    \item \texttt{data/results.txt}
  \end{itemize}
\subsection*{Demo Output (Example)}
\begin{center}
  \IfFileExists{../images/demo.png}{
    \includegraphics[width=0.92\linewidth]{demo.png}
  }{
    \small (Run demo to generate: \texttt{demo.png})
  }
  \end{center}
\subsection*{Summary}
\begin{itemize}
        \item Key definitions and the main formula.
\item How to interpret results in context.
\item How the demo connects to the theory.
      \end{itemize}
\subsection*{Exit Question}
\small
  If your CI is too wide, what two actions reduce its width (without cheating)?
% END SLIDE APPENDIX (AUTO-GENERATED)

\end{document}
