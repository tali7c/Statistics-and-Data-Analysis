\documentclass{beamer}

      \usetheme{Berlin}
      \usecolortheme{Orchid}
      \useoutertheme{miniframes}
      \setbeamertemplate{navigation symbols}{}

      \usepackage{amsmath}
      \usepackage{amssymb}
      \usepackage{booktabs}
      \usepackage{graphicx}
      \graphicspath{{../images/}}

      \title[Statistics and Data Analysis]{Statistics and Data Analysis}
      \subtitle{Unit 03 -- Lecture 04: Chi-square Tests (Goodness-of-Fit and Independence)}
      \author{Tofik Ali}
      \institute{School of Computer Science, UPES Dehradun}
      \date{\today}

      \begin{document}

      \begin{frame}
        \titlepage
        \vspace{-0.5em}
        \begin{center}
          \small \texttt{https://github.com/tali7c/Statistics-and-Data-Analysis}
        \end{center}
      \end{frame}

      \begin{frame}{Quick Links}
        \centering
        \hyperlink{sec:overview}{\beamerbutton{Overview}}\hspace{0.6em}
\hyperlink{sec:chi2}{\beamerbutton{Chi-square Tests}}\hspace{0.6em}
\hyperlink{sec:expected}{\beamerbutton{Expected Counts}}\hspace{0.6em}
\hyperlink{sec:exercises}{\beamerbutton{Exercises}}\hspace{0.6em}
\hyperlink{sec:demo}{\beamerbutton{Demo}}\hspace{0.6em}
\hyperlink{sec:summary}{\beamerbutton{Summary}}\hspace{0.6em}
      \end{frame}

      \begin{frame}{Agenda}
        \tableofcontents
      \end{frame}

\section{Overview}
\label{sec:overview}


\begin{frame}{Learning Outcomes}
      \begin{itemize}[<+->]
        \item Explain when chi-square tests are used (counts/frequencies)
\item Compute expected counts for a contingency table
\item Compute chi-square statistic (basic)
\item State assumptions (expected counts not too small)
\item Interpret independence vs association
      \end{itemize}
    \end{frame}

\section{Chi-square Tests}
\label{sec:chi2}


\begin{frame}{Chi-square Tests: Key Points}
      \begin{itemize}[<+->]
        \item GOF: one categorical variable
\item Independence: two categorical variables
\item Compare observed O to expected E under H0
      \end{itemize}
    \end{frame}

\begin{frame}{Chi-square Tests: Key Formula}
  \[ \chi^2 = \sum \frac{(O-E)^2}{E} \]
\end{frame}

\section{Expected Counts}
\label{sec:expected}


\begin{frame}{Expected Counts: Key Points}
      \begin{itemize}[<+->]
        \item E\_rc = (row total)(col total)/N
\item df = (R-1)(C-1) for independence
      \end{itemize}
    \end{frame}

\section{Exercises}
\label{sec:exercises}


\begin{frame}{Exercise 1: Expected counts}
  \small
  Row totals 60/40, column totals 70/30, N=100. Compute E11.
\end{frame}

\begin{frame}{Solution 1}
  \begin{itemize}
    \item E11 = 60*70/100 = 42
  \end{itemize}
\end{frame}

\begin{frame}{Exercise 2: Interpret reject}
  \small
  If you reject H0 in independence test, what do you conclude?
\end{frame}

\begin{frame}{Solution 2}
      \begin{itemize}
        \item Evidence of association between variables.
\item Not direction; not causation.
      \end{itemize}
    \end{frame}

\begin{frame}{Exercise 3: Assumption}
  \small
  Why do we worry about small expected counts?
\end{frame}

\begin{frame}{Solution 3}
  \begin{itemize}
    \item Chi-square approximation can break when E is very small.
  \end{itemize}
\end{frame}

\section{Demo}
\label{sec:demo}


\begin{frame}{Mini Demo (Python)}
  Run from the lecture folder:
  \begin{center}
    \texttt{python demo/demo.py}
  \end{center}
  \vspace{0.4em}
  Outputs:
  \begin{itemize}
    \item \texttt{images/demo.png}
    \item \texttt{data/results.txt}
  \end{itemize}
\end{frame}

\begin{frame}{Demo Output (Example)}
  \begin{center}
  \IfFileExists{../images/demo.png}{
    \includegraphics[width=0.92\linewidth]{demo.png}
  }{
    \small (Run demo to generate: \texttt{demo.png})
  }
  \end{center}
\end{frame}

\section{Summary}
\label{sec:summary}


\begin{frame}{Summary}
      \begin{itemize}[<+->]
        \item Key definitions and the main formula.
\item How to interpret results in context.
\item How the demo connects to the theory.
      \end{itemize}
    \end{frame}

\begin{frame}{Exit Question}
  \small
  Why do chi-square tests use expected counts instead of only raw percentages?
\end{frame}

\end{document}
