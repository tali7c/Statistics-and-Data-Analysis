\documentclass[11pt]{article}
\usepackage[utf8]{inputenc}
\usepackage[T1]{fontenc}
\usepackage{geometry}
\usepackage{amsmath}
\usepackage{listings}
\usepackage{xcolor}
\usepackage{amssymb}
\usepackage{booktabs}
\usepackage{graphicx}
\graphicspath{{../images/}}
\usepackage{hyperref}
\geometry{margin=1in}

\title{Statistics and Data Analysis\\Unit 03 -- Lecture 04 Notes\\Chi-square Tests (Goodness-of-Fit and Independence)}
\author{Tofik Ali}
\date{\today}

\begin{document}
\maketitle

\section*{Topic}
Chi-square tests for counts: GOF and independence; expected counts; assumptions.
\section*{How to Use These Notes}
These notes are written for students who are seeing the topic for the first time. They
follow the slide order, but add the missing 'why', interpretation, and common mistakes. If
you get stuck, look at the worked exercises and then run the Python demo.

Course repository (slides, demos, datasets): \url{https://github.com/tali7c/Statistics-and-Data-Analysis}

\section*{Time Plan (55 minutes)}
\begin{itemize}
  \item 0--10 min: Attendance + recap of previous lecture
  \item 10--35 min: Core concepts (this lecture's sections)
  \item 35--45 min: Exercises (solve 1--2 in class, rest as practice)
  \item 45--50 min: Mini demo + interpretation of output
  \item 50--55 min: Buffer / wrap-up (leave 5 minutes early)
\end{itemize}

\section*{Slide-by-slide Notes}
\subsection*{Title Slide}
State the lecture title clearly and connect it to what students already know.
Tell students what they will be able to do by the end (not just what you will cover).

\subsection*{Quick Links / Agenda}
Explain the structure of the lecture and where the exercises and demo appear.
\begin{itemize}
  \item Overview
  \item Chi-square Tests
  \item Expected Counts
  \item Exercises
  \item Demo
  \item Summary
\end{itemize}

\subsection*{Learning Outcomes}
\begin{itemize}
  \item Explain when chi-square tests are used (counts/frequencies)
  \item Compute expected counts for a contingency table
  \item Compute chi-square statistic (basic)
  \item State assumptions (expected counts not too small)
  \item Interpret independence vs association
\end{itemize}
\paragraph{Why these outcomes matter.}
A \textbf{chi-square test} is used for \textbf{counts/frequencies} (categorical data). It
compares observed counts $O$ to expected counts $E$ under the null hypothesis. Large
deviations relative to $E$ increase the $\chi^2$ statistic and provide evidence against the
null.
In chi-square tests, \textbf{expected counts} are what you would expect to see if $H_0$ were
true (e.g., independence). Very small expected counts can break the approximation used by
the test; a common rule of thumb is that most expected counts should be at least 5.

\subsection*{Chi-square Tests: Key Points}
\begin{itemize}
  \item GOF: one categorical variable
  \item Independence: two categorical variables
  \item Compare observed O to expected E under H0
\end{itemize}
\paragraph{Explanation.}
The \textbf{null hypothesis $H_0$} usually represents 'no effect' or a baseline value (e.g.,
$\mu=60$). The \textbf{alternative $H_1$} represents the effect you are looking for (e.g.,
$\mu\neq 60$ or $\mu>60$). We compute a test statistic and a p-value assuming $H_0$ is true.
A \textbf{chi-square test} is used for \textbf{counts/frequencies} (categorical data). It
compares observed counts $O$ to expected counts $E$ under the null hypothesis. Large
deviations relative to $E$ increase the $\chi^2$ statistic and provide evidence against the
null.
In chi-square tests, \textbf{expected counts} are what you would expect to see if $H_0$ were
true (e.g., independence). Very small expected counts can break the approximation used by
the test; a common rule of thumb is that most expected counts should be at least 5.

\subsection*{Chi-square Tests: Key Formula}
\[ \chi^2 = \sum \frac{(O-E)^2}{E} \]
\paragraph{How to read the formula.}
A \textbf{chi-square test} is used for \textbf{counts/frequencies} (categorical data). It
compares observed counts $O$ to expected counts $E$ under the null hypothesis. Large
deviations relative to $E$ increase the $\chi^2$ statistic and provide evidence against the
null.

\subsection*{Expected Counts: Key Points}
\begin{itemize}
  \item E\_rc = (row total)(col total)/N
  \item df = (R-1)(C-1) for independence
\end{itemize}
\paragraph{Explanation.}
\textbf{Degrees of freedom (df)} roughly represent how much independent information is
available to estimate variability. For a one-sample t-test, $\mathrm{df}=n-1$ because one
constraint is used to estimate the sample mean. df affects the critical values and the shape
of the t-distribution (small df -> heavier tails).
In chi-square tests, \textbf{expected counts} are what you would expect to see if $H_0$ were
true (e.g., independence). Very small expected counts can break the approximation used by
the test; a common rule of thumb is that most expected counts should be at least 5.

\subsection*{Exercises (with Solutions)}
Attempt the exercise first, then compare with the solution. Focus on interpretation, not
only arithmetic.

\subsection*{Exercise 1: Expected counts}
Row totals 60/40, column totals 70/30, N=100. Compute E11.
\subsubsection*{Solution}
\begin{itemize}
  \item E11 = 60*70/100 = 42
\end{itemize}
\paragraph{Walkthrough.}
In chi-square tests, \textbf{expected counts} are what you would expect to see if $H_0$ were
true (e.g., independence). Very small expected counts can break the approximation used by
the test; a common rule of thumb is that most expected counts should be at least 5.

\subsection*{Exercise 2: Interpret reject}
If you reject H0 in independence test, what do you conclude?
\subsubsection*{Solution}
\begin{itemize}
  \item Evidence of association between variables.
  \item Not direction; not causation.
\end{itemize}
\paragraph{Walkthrough.}
The \textbf{null hypothesis $H_0$} usually represents 'no effect' or a baseline value (e.g.,
$\mu=60$). The \textbf{alternative $H_1$} represents the effect you are looking for (e.g.,
$\mu\neq 60$ or $\mu>60$). We compute a test statistic and a p-value assuming $H_0$ is true.

\subsection*{Exercise 3: Assumption}
Why do we worry about small expected counts?
\subsubsection*{Solution}
\begin{itemize}
  \item Chi-square approximation can break when E is very small.
\end{itemize}
\paragraph{Walkthrough.}
A \textbf{chi-square test} is used for \textbf{counts/frequencies} (categorical data). It
compares observed counts $O$ to expected counts $E$ under the null hypothesis. Large
deviations relative to $E$ increase the $\chi^2$ statistic and provide evidence against the
null.
In chi-square tests, \textbf{expected counts} are what you would expect to see if $H_0$ were
true (e.g., independence). Very small expected counts can break the approximation used by
the test; a common rule of thumb is that most expected counts should be at least 5.

\subsection*{Mini Demo (Python)}
Run from the lecture folder:
\begin{verbatim}
python demo/demo.py
\end{verbatim}

Output files:
\begin{itemize}
  \item \texttt{images/demo.png}
  \item \texttt{data/results.txt}
\end{itemize}
\paragraph{What to show and say.}
\begin{itemize}
  \item Builds a small contingency table (counts) and runs a chi-square independence test.
  \item Visualizes counts as a heatmap to make 'observed vs expected' intuitive.
  \item Use the output to discuss df and why expected counts should not be too small.
\end{itemize}

\subsection*{Demo Output (Example)}
\begin{center}
\IfFileExists{../images/demo.png}{
  \includegraphics[width=0.95\linewidth]{../images/demo.png}
}{
  \small (Run the demo to generate \texttt{images/demo.png})
}
\end{center}

\subsection*{Summary}
\begin{itemize}
  \item Key definitions and the main formula.
  \item How to interpret results in context.
  \item How the demo connects to the theory.
\end{itemize}

\subsection*{Exit Question}
Why do chi-square tests use expected counts instead of only raw percentages?
\paragraph{Suggested answer (for revision).}
Expected counts incorporate the row/column totals and sample size; they define what 'no
association' predicts so we can measure deviations fairly.

\section*{References}
\begin{itemize}
  \item Montgomery, D. C., \& Runger, G. C. \textit{Applied Statistics and Probability for Engineers}, Wiley.
  \item Devore, J. L. \textit{Probability and Statistics for Engineering and the Sciences}, Cengage.
  \item McKinney, W. \textit{Python for Data Analysis}, O'Reilly.
\end{itemize}

% BEGIN SLIDE APPENDIX (AUTO-GENERATED)
\clearpage
\section*{Appendix: Slide Deck Content (Reference)}
\noindent The material below is a reference copy of the slide deck content. Exercise solutions are explained in the main notes where applicable.

\subsection*{Title Slide}
\titlepage
        \vspace{-0.5em}
        \begin{center}
          \small \texttt{https://github.com/tali7c/Statistics-and-Data-Analysis}
        \end{center}
\subsection*{Quick Links}
\centering
        \textbf{Overview}\hspace{0.6em}
\textbf{Chi-square Tests}\hspace{0.6em}
\textbf{Expected Counts}\hspace{0.6em}
\textbf{Exercises}\hspace{0.6em}
\textbf{Demo}\hspace{0.6em}
\textbf{Summary}\hspace{0.6em}
\subsection*{Agenda}
\begin{itemize}
  \item Overview
  \item Chi-square Tests
  \item Expected Counts
  \item Exercises
  \item Demo
  \item Summary
\end{itemize}
\subsection*{Learning Outcomes}
\begin{itemize}
        \item Explain when chi-square tests are used (counts/frequencies)
\item Compute expected counts for a contingency table
\item Compute chi-square statistic (basic)
\item State assumptions (expected counts not too small)
\item Interpret independence vs association
      \end{itemize}
\subsection*{Chi-square Tests: Key Points}
\begin{itemize}
        \item GOF: one categorical variable
\item Independence: two categorical variables
\item Compare observed O to expected E under H0
      \end{itemize}
\subsection*{Chi-square Tests: Key Formula}
\[ \chi^2 = \sum \frac{(O-E)^2}{E} \]
\subsection*{Expected Counts: Key Points}
\begin{itemize}
        \item E\_rc = (row total)(col total)/N
\item df = (R-1)(C-1) for independence
      \end{itemize}
\subsection*{Exercise 1: Expected counts}
\small
  Row totals 60/40, column totals 70/30, N=100. Compute E11.
\subsection*{Solution 1}
\begin{itemize}
    \item E11 = 60*70/100 = 42
  \end{itemize}
\subsection*{Exercise 2: Interpret reject}
\small
  If you reject H0 in independence test, what do you conclude?
\subsection*{Solution 2}
\begin{itemize}
        \item Evidence of association between variables.
\item Not direction; not causation.
      \end{itemize}
\subsection*{Exercise 3: Assumption}
\small
  Why do we worry about small expected counts?
\subsection*{Solution 3}
\begin{itemize}
    \item Chi-square approximation can break when E is very small.
  \end{itemize}
\subsection*{Mini Demo (Python)}
Run from the lecture folder:
  \begin{center}
    \texttt{python demo/demo.py}
  \end{center}
  \vspace{0.4em}
  Outputs:
  \begin{itemize}
    \item \texttt{images/demo.png}
    \item \texttt{data/results.txt}
  \end{itemize}
\subsection*{Demo Output (Example)}
\begin{center}
  \IfFileExists{../images/demo.png}{
    \includegraphics[width=0.92\linewidth]{demo.png}
  }{
    \small (Run demo to generate: \texttt{demo.png})
  }
  \end{center}
\subsection*{Summary}
\begin{itemize}
        \item Key definitions and the main formula.
\item How to interpret results in context.
\item How the demo connects to the theory.
      \end{itemize}
\subsection*{Exit Question}
\small
  Why do chi-square tests use expected counts instead of only raw percentages?
% END SLIDE APPENDIX (AUTO-GENERATED)

\end{document}
