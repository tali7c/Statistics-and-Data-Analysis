\documentclass[11pt]{article}
\usepackage[utf8]{inputenc}
\usepackage[T1]{fontenc}
\usepackage{geometry}
\usepackage{amsmath}
\usepackage{booktabs}
\usepackage{hyperref}
\geometry{margin=1in}

\title{Statistics and Data Analysis\\Unit 03 -- Lecture 04 Notes}
\author{Tofik Ali}
\date{\today}

\begin{document}
\maketitle

\section*{Topic}
Chi-square tests for counts: GOF and independence; expected counts; assumptions.

\subsection*{Learning Outcomes}
\begin{itemize}
  \item Explain when chi-square tests are used (counts/frequencies)
  \item Compute expected counts for a contingency table
  \item Compute chi-square statistic (basic)
  \item State assumptions (expected counts not too small)
  \item Interpret independence vs association
\end{itemize}

\section*{Detailed Notes}
These notes are designed to be read alongside the slides. They expand each slide bullet into
plain-language explanations, small worked examples, and common pitfalls. When a formula
appears, emphasize (1) what each symbol means, (2) the assumptions needed to use it, and (3)
how to interpret the final number in the problem context.

\section*{Chi-square Tests}
\begin{itemize}
  \item GOF: one categorical variable
  \item Independence: two categorical variables
  \item Compare observed O to expected E under H0
\end{itemize}

\section*{Expected Counts}
\begin{itemize}
  \item E\_rc = (row total)(col total)/N
  \item df = (R-1)(C-1) for independence
\end{itemize}

\section*{Exercises (with Solutions)}
\subsection*{Exercise 1: Expected counts}
Row totals 60/40, column totals 70/30, N=100. Compute E11.
\subsection*{Solution}
\begin{itemize}
  \item E11 = 60*70/100 = 42
\end{itemize}

\subsection*{Exercise 2: Interpret reject}
If you reject H0 in independence test, what do you conclude?
\subsection*{Solution}
\begin{itemize}
  \item Evidence of association between variables.
  \item Not direction; not causation.
\end{itemize}

\subsection*{Exercise 3: Assumption}
Why do we worry about small expected counts?
\subsection*{Solution}
\begin{itemize}
  \item Chi-square approximation can break when E is very small.
\end{itemize}

\section*{Exit Question}
Why do chi-square tests use expected counts instead of only raw percentages?

\section*{Demo (Python)}
Run from the lecture folder:
\begin{verbatim}
python demo/demo.py
\end{verbatim}

Output files:
\begin{itemize}
  \item \texttt{images/demo.png}
  \item \texttt{data/results.txt}
\end{itemize}

\section*{References}
\begin{itemize}
  \item Montgomery, D. C., \& Runger, G. C. \textit{Applied Statistics and Probability for Engineers}, Wiley.
  \item Devore, J. L. \textit{Probability and Statistics for Engineering and the Sciences}, Cengage.
  \item McKinney, W. \textit{Python for Data Analysis}, O'Reilly.
\end{itemize}
\end{document}
