\documentclass[11pt]{article}
\usepackage[utf8]{inputenc}
\usepackage[T1]{fontenc}
\usepackage{geometry}
\usepackage{amsmath}
\usepackage{listings}
\usepackage{xcolor}
\usepackage{amssymb}
\usepackage{booktabs}
\usepackage{graphicx}
\graphicspath{{../images/}}
\usepackage{hyperref}
\geometry{margin=1in}

\title{Statistics and Data Analysis\\Unit 04 -- Lecture 01 Notes\\Correlation and Regression: Concepts}
\author{Tofik Ali}
\date{\today}

\begin{document}
\maketitle

\section*{Topic}
Correlation vs regression concepts; causation warning; residual idea.
\section*{How to Use These Notes}
These notes are written for students who are seeing the topic for the first time. They
follow the slide order, but add the missing 'why', interpretation, and common mistakes. If
you get stuck, look at the worked exercises and then run the Python demo.

Course repository (slides, demos, datasets): \url{https://github.com/tali7c/Statistics-and-Data-Analysis}

\section*{Time Plan (55 minutes)}
\begin{itemize}
  \item 0--10 min: Attendance + recap of previous lecture
  \item 10--35 min: Core concepts (this lecture's sections)
  \item 35--45 min: Exercises (solve 1--2 in class, rest as practice)
  \item 45--50 min: Mini demo + interpretation of output
  \item 50--55 min: Buffer / wrap-up (leave 5 minutes early)
\end{itemize}

\section*{Slide-by-slide Notes}
\subsection*{Title Slide}
State the lecture title clearly and connect it to what students already know.
Tell students what they will be able to do by the end (not just what you will cover).

\subsection*{Quick Links / Agenda}
Explain the structure of the lecture and where the exercises and demo appear.
\begin{itemize}
  \item Overview
  \item Concepts
  \item Causation Warning
  \item Exercises
  \item Demo
  \item Summary
\end{itemize}

\subsection*{Learning Outcomes}
\begin{itemize}
  \item Differentiate correlation and regression
  \item Explain why correlation does not imply causation
  \item Interpret a scatter plot (trend, outliers)
  \item Define residual and why residuals matter
\end{itemize}
\paragraph{Why these outcomes matter.}
\textbf{Correlation} measures the strength of a linear association between two variables. It
is symmetric (no X/Y direction) and does not imply causation. Outliers can inflate or hide
correlation, so always look at the scatter plot.
\textbf{Regression} models a response variable $Y$ as a function of predictor(s) $X$. It has
direction (predictors -> response), produces a fitted equation, and lets you predict and
explain. Regression is not automatically causal; causality needs design or strong
assumptions.

\subsection*{Concepts: Key Points}
\begin{itemize}
  \item Correlation measures linear association
  \item Regression models Y as a function of X
  \item Regression has roles: predictors vs response
\end{itemize}
\paragraph{Explanation.}
\textbf{Correlation} measures the strength of a linear association between two variables. It
is symmetric (no X/Y direction) and does not imply causation. Outliers can inflate or hide
correlation, so always look at the scatter plot.
\textbf{Regression} models a response variable $Y$ as a function of predictor(s) $X$. It has
direction (predictors -> response), produces a fitted equation, and lets you predict and
explain. Regression is not automatically causal; causality needs design or strong
assumptions.

\subsection*{Causation Warning: Key Points}
\begin{itemize}
  \item Confounding can create misleading correlation
  \item Reverse causality is possible
  \item Causal claims need design or strong assumptions
\end{itemize}
\paragraph{Explanation.}
Always state assumptions clearly. Common assumptions in classical tests include independence
of observations, roughly normal errors (or a large-sample justification), and similar
variances across groups. Violations do not automatically invalidate a result, but they
change how much you should trust the p-value and confidence interval.
\textbf{Correlation} measures the strength of a linear association between two variables. It
is symmetric (no X/Y direction) and does not imply causation. Outliers can inflate or hide
correlation, so always look at the scatter plot.
A \textbf{confounder} is a third variable that influences both $X$ and $Y$, creating a
misleading association. Example: ice-cream sales and drowning both increase in summer;
temperature is the confounder. In practice, confounding is the main reason correlation is
not causation.

\subsection*{Exercises (with Solutions)}
Attempt the exercise first, then compare with the solution. Focus on interpretation, not
only arithmetic.

\subsection*{Exercise 1: Pick response variable}
Predict house price using size and location. What is the response variable?
\subsubsection*{Solution}
\begin{itemize}
  \item House price is the response (Y).
\end{itemize}

\subsection*{Exercise 2: Interpret r}
If r=0.7 between study hours and score, what does it mean?
\subsubsection*{Solution}
\begin{itemize}
  \item Strong positive linear association.
  \item Not proof of causation.
\end{itemize}

\subsection*{Exercise 3: Residual sign}
If y=74 and yhat=80, what is residual?
\subsubsection*{Solution}
\begin{itemize}
  \item Residual = y - yhat = -6 (over-prediction).
\end{itemize}
\paragraph{Walkthrough.}
A \textbf{residual} is $y - \hat{y}$. Residual plots tell you what the model failed to
explain. Patterns in residuals (trend, curvature, changing variance) are warnings that your
model form is inadequate or assumptions are violated.

\subsection*{Mini Demo (Python)}
Run from the lecture folder:
\begin{verbatim}
python demo/demo.py
\end{verbatim}

Output files:
\begin{itemize}
  \item \texttt{images/demo.png}
  \item \texttt{data/results.txt}
\end{itemize}
\paragraph{What to show and say.}
\begin{itemize}
  \item Creates a noisy linear relationship and plots scatter with fitted line.
  \item Reports correlation and a simple regression fit to compare the two ideas.
  \item Use the plot to talk about outliers and why correlation is not causation.
\end{itemize}

\subsection*{Demo Output (Example)}
\begin{center}
\IfFileExists{../images/demo.png}{
  \includegraphics[width=0.95\linewidth]{../images/demo.png}
}{
  \small (Run the demo to generate \texttt{images/demo.png})
}
\end{center}

\subsection*{Summary}
\begin{itemize}
  \item Key definitions and the main formula.
  \item How to interpret results in context.
  \item How the demo connects to the theory.
\end{itemize}

\subsection*{Exit Question}
Give one example of a confounder that can create a misleading correlation.
\paragraph{Suggested answer (for revision).}
A confounder can influence both variables (e.g., temperature affects ice-cream sales and
swimming), creating correlation without causation.

\section*{References}
\begin{itemize}
  \item Montgomery, D. C., \& Runger, G. C. \textit{Applied Statistics and Probability for Engineers}, Wiley.
  \item Devore, J. L. \textit{Probability and Statistics for Engineering and the Sciences}, Cengage.
  \item McKinney, W. \textit{Python for Data Analysis}, O'Reilly.
\end{itemize}

% BEGIN SLIDE APPENDIX (AUTO-GENERATED)
\clearpage
\section*{Appendix: Slide Deck Content (Reference)}
\noindent The material below is a reference copy of the slide deck content. Exercise solutions are explained in the main notes where applicable.

\subsection*{Title Slide}
\titlepage
        \vspace{-0.5em}
        \begin{center}
          \small \texttt{https://github.com/tali7c/Statistics-and-Data-Analysis}
        \end{center}
\subsection*{Quick Links}
\centering
        \textbf{Overview}\hspace{0.6em}
\textbf{Concepts}\hspace{0.6em}
\textbf{Causation Warning}\hspace{0.6em}
\textbf{Exercises}\hspace{0.6em}
\textbf{Demo}\hspace{0.6em}
\textbf{Summary}\hspace{0.6em}
\subsection*{Agenda}
\begin{itemize}
  \item Overview
  \item Concepts
  \item Causation Warning
  \item Exercises
  \item Demo
  \item Summary
\end{itemize}
\subsection*{Learning Outcomes}
\begin{itemize}
        \item Differentiate correlation and regression
\item Explain why correlation does not imply causation
\item Interpret a scatter plot (trend, outliers)
\item Define residual and why residuals matter
      \end{itemize}
\subsection*{Concepts: Key Points}
\begin{itemize}
        \item Correlation measures linear association
\item Regression models Y as a function of X
\item Regression has roles: predictors vs response
      \end{itemize}
\subsection*{Causation Warning: Key Points}
\begin{itemize}
        \item Confounding can create misleading correlation
\item Reverse causality is possible
\item Causal claims need design or strong assumptions
      \end{itemize}
\subsection*{Exercise 1: Pick response variable}
\small
  Predict house price using size and location. What is the response variable?
\subsection*{Solution 1}
\begin{itemize}
    \item House price is the response (Y).
  \end{itemize}
\subsection*{Exercise 2: Interpret r}
\small
  If r=0.7 between study hours and score, what does it mean?
\subsection*{Solution 2}
\begin{itemize}
        \item Strong positive linear association.
\item Not proof of causation.
      \end{itemize}
\subsection*{Exercise 3: Residual sign}
\small
  If y=74 and yhat=80, what is residual?
\subsection*{Solution 3}
\begin{itemize}
    \item Residual = y - yhat = -6 (over-prediction).
  \end{itemize}
\subsection*{Mini Demo (Python)}
Run from the lecture folder:
  \begin{center}
    \texttt{python demo/demo.py}
  \end{center}
  \vspace{0.4em}
  Outputs:
  \begin{itemize}
    \item \texttt{images/demo.png}
    \item \texttt{data/results.txt}
  \end{itemize}
\subsection*{Demo Output (Example)}
\begin{center}
  \IfFileExists{../images/demo.png}{
    \includegraphics[width=0.92\linewidth]{demo.png}
  }{
    \small (Run demo to generate: \texttt{demo.png})
  }
  \end{center}
\subsection*{Summary}
\begin{itemize}
        \item Key definitions and the main formula.
\item How to interpret results in context.
\item How the demo connects to the theory.
      \end{itemize}
\subsection*{Exit Question}
\small
  Give one example of a confounder that can create a misleading correlation.
% END SLIDE APPENDIX (AUTO-GENERATED)

\end{document}
