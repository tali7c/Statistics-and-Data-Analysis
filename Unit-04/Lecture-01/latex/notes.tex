\documentclass[11pt]{article}
\usepackage[utf8]{inputenc}
\usepackage[T1]{fontenc}
\usepackage{geometry}
\usepackage{amsmath}
\usepackage{booktabs}
\usepackage{hyperref}
\geometry{margin=1in}

\title{Statistics and Data Analysis\\Unit 04 -- Lecture 01 Notes}
\author{Tofik Ali}
\date{\today}

\begin{document}
\maketitle

\section*{Topic}
Correlation vs regression concepts; causation warning; residual idea.

\subsection*{Learning Outcomes}
\begin{itemize}
  \item Differentiate correlation and regression
  \item Explain why correlation does not imply causation
  \item Interpret a scatter plot (trend, outliers)
  \item Define residual and why residuals matter
\end{itemize}

\section*{Detailed Notes}
These notes are designed to be read alongside the slides. They expand each slide bullet into
plain-language explanations, small worked examples, and common pitfalls. When a formula
appears, emphasize (1) what each symbol means, (2) the assumptions needed to use it, and (3)
how to interpret the final number in the problem context.

\section*{Concepts}
\begin{itemize}
  \item Correlation measures linear association
  \item Regression models Y as a function of X
  \item Regression has roles: predictors vs response
\end{itemize}

\section*{Causation Warning}
\begin{itemize}
  \item Confounding can create misleading correlation
  \item Reverse causality is possible
  \item Causal claims need design or strong assumptions
\end{itemize}

\section*{Exercises (with Solutions)}
\subsection*{Exercise 1: Pick response variable}
Predict house price using size and location. What is the response variable?
\subsection*{Solution}
\begin{itemize}
  \item House price is the response (Y).
\end{itemize}

\subsection*{Exercise 2: Interpret r}
If r=0.7 between study hours and score, what does it mean?
\subsection*{Solution}
\begin{itemize}
  \item Strong positive linear association.
  \item Not proof of causation.
\end{itemize}

\subsection*{Exercise 3: Residual sign}
If y=74 and yhat=80, what is residual?
\subsection*{Solution}
\begin{itemize}
  \item Residual = y - yhat = -6 (over-prediction).
\end{itemize}

\section*{Exit Question}
Give one example of a confounder that can create a misleading correlation.

\section*{Demo (Python)}
Run from the lecture folder:
\begin{verbatim}
python demo/demo.py
\end{verbatim}

Output files:
\begin{itemize}
  \item \texttt{images/demo.png}
  \item \texttt{data/results.txt}
\end{itemize}

\section*{References}
\begin{itemize}
  \item Montgomery, D. C., \& Runger, G. C. \textit{Applied Statistics and Probability for Engineers}, Wiley.
  \item Devore, J. L. \textit{Probability and Statistics for Engineering and the Sciences}, Cengage.
  \item McKinney, W. \textit{Python for Data Analysis}, O'Reilly.
\end{itemize}
\end{document}
