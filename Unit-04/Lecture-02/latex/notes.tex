\documentclass[11pt]{article}
\usepackage[utf8]{inputenc}
\usepackage[T1]{fontenc}
\usepackage{geometry}
\usepackage{amsmath}
\usepackage{listings}
\usepackage{xcolor}
\usepackage{amssymb}
\usepackage{booktabs}
\usepackage{graphicx}
\graphicspath{{../images/}}
\usepackage{hyperref}
\geometry{margin=1in}

\title{Statistics and Data Analysis\\Unit 04 -- Lecture 02 Notes\\Simple Linear Regression (OLS)}
\author{Tofik Ali}
\date{\today}

\begin{document}
\maketitle

\section*{Topic}
Simple linear regression model, interpretation, residuals and R-squared.
\section*{How to Use These Notes}
These notes are written for students who are seeing the topic for the first time. They
follow the slide order, but add the missing 'why', interpretation, and common mistakes. If
you get stuck, look at the worked exercises and then run the Python demo.

Course repository (slides, demos, datasets): \url{https://github.com/tali7c/Statistics-and-Data-Analysis}

\section*{Time Plan (55 minutes)}
\begin{itemize}
  \item 0--10 min: Attendance + recap of previous lecture
  \item 10--35 min: Core concepts (this lecture's sections)
  \item 35--45 min: Exercises (solve 1--2 in class, rest as practice)
  \item 45--50 min: Mini demo + interpretation of output
  \item 50--55 min: Buffer / wrap-up (leave 5 minutes early)
\end{itemize}

\section*{Slide-by-slide Notes}
\subsection*{Title Slide}
State the lecture title clearly and connect it to what students already know.
Tell students what they will be able to do by the end (not just what you will cover).

\subsection*{Quick Links / Agenda}
Explain the structure of the lecture and where the exercises and demo appear.
\begin{itemize}
  \item Overview
  \item Model
  \item Fit and Diagnostics
  \item Exercises
  \item Demo
  \item Summary
\end{itemize}

\subsection*{Learning Outcomes}
\begin{itemize}
  \item Write the simple linear regression model
  \item Interpret slope and intercept in context
  \item Compute a prediction and a residual
  \item Explain R-squared (intuition)
\end{itemize}
\paragraph{Why these outcomes matter.}
\textbf{Regression} models a response variable $Y$ as a function of predictor(s) $X$. It has
direction (predictors -> response), produces a fitted equation, and lets you predict and
explain. Regression is not automatically causal; causality needs design or strong
assumptions.
A \textbf{residual} is $y - \hat{y}$. Residual plots tell you what the model failed to
explain. Patterns in residuals (trend, curvature, changing variance) are warnings that your
model form is inadequate or assumptions are violated.

\subsection*{Model: Key Points}
\begin{itemize}
  \item y = b0 + b1 x + error
  \item Slope: expected change in y for 1-unit increase in x
  \item Intercept: predicted y at x=0 (interpret carefully)
\end{itemize}
\paragraph{Explanation.}
In chi-square tests, \textbf{expected counts} are what you would expect to see if $H_0$ were
true (e.g., independence). Very small expected counts can break the approximation used by
the test; a common rule of thumb is that most expected counts should be at least 5.
In simple linear regression, the \textbf{slope} is the expected change in $Y$ for a one-unit
increase in $X$ (on average). Always state units (e.g., 'marks increase by 2 points per
extra hour of study').
The \textbf{intercept} is the predicted value when $X=0$. It may or may not be meaningful
depending on whether $X=0$ is realistic in your context. If $X=0$ is outside the observed
range, do not over-interpret the intercept.

\subsection*{Model: Key Formula}
\[ y = \beta_0 + \beta_1 x + \epsilon \]

\subsection*{Fit and Diagnostics: Key Points}
\begin{itemize}
  \item Look at residual plots for patterns
  \item Outliers can dominate the fitted line
  \item High $R^2$ does not guarantee a good model
\end{itemize}
\paragraph{Explanation.}
A \textbf{residual} is $y - \hat{y}$. Residual plots tell you what the model failed to
explain. Patterns in residuals (trend, curvature, changing variance) are warnings that your
model form is inadequate or assumptions are violated.
$R^2$ is the fraction of variance in $Y$ explained by the model (in-sample). It increases
when you add predictors, even useless ones, so it is not a guarantee of a good model. Use
residual diagnostics and out-of-sample evaluation to judge model quality.

\subsection*{Exercises (with Solutions)}
Attempt the exercise first, then compare with the solution. Focus on interpretation, not
only arithmetic.

\subsection*{Exercise 1: Prediction}
Model: yhat = 10 + 2x. Predict y when x=7.
\subsubsection*{Solution}
\begin{itemize}
  \item yhat = 24
\end{itemize}

\subsection*{Exercise 2: Residual}
If actual y=20 at x=7, compute residual.
\subsubsection*{Solution}
\begin{itemize}
  \item 20-24 = -4
\end{itemize}
\paragraph{Walkthrough.}
A \textbf{residual} is $y - \hat{y}$. Residual plots tell you what the model failed to
explain. Patterns in residuals (trend, curvature, changing variance) are warnings that your
model form is inadequate or assumptions are violated.

\subsection*{Exercise 3: Interpret slope}
Slope is 5 thousand INR per extra room. Interpret.
\subsubsection*{Solution}
\begin{itemize}
  \item Each extra room increases predicted price by ~5k INR (on average).
\end{itemize}
\paragraph{Walkthrough.}
In simple linear regression, the \textbf{slope} is the expected change in $Y$ for a one-unit
increase in $X$ (on average). Always state units (e.g., 'marks increase by 2 points per
extra hour of study').

\subsection*{Mini Demo (Python)}
Run from the lecture folder:
\begin{verbatim}
python demo/demo.py
\end{verbatim}

Output files:
\begin{itemize}
  \item \texttt{images/demo.png}
  \item \texttt{data/results.txt}
\end{itemize}
\paragraph{What to show and say.}
\begin{itemize}
  \item Fits a simple linear regression on synthetic data (one predictor).
  \item Shows scatter + fitted line and reports slope/intercept and $R^2$.
  \item Use residual behavior (in results) to motivate diagnostics.
\end{itemize}

\subsection*{Demo Output (Example)}
\begin{center}
\IfFileExists{../images/demo.png}{
  \includegraphics[width=0.95\linewidth]{../images/demo.png}
}{
  \small (Run the demo to generate \texttt{images/demo.png})
}
\end{center}

\subsection*{Summary}
\begin{itemize}
  \item Key definitions and the main formula.
  \item How to interpret results in context.
  \item How the demo connects to the theory.
\end{itemize}

\subsection*{Exit Question}
Why do we check residual plots even if $R^2$ is high?
\paragraph{Suggested answer (for revision).}
Residual plots reveal nonlinearity, outliers, or changing variance; $R^2$ alone can look
good even when assumptions are violated.

\section*{References}
\begin{itemize}
  \item Montgomery, D. C., \& Runger, G. C. \textit{Applied Statistics and Probability for Engineers}, Wiley.
  \item Devore, J. L. \textit{Probability and Statistics for Engineering and the Sciences}, Cengage.
  \item McKinney, W. \textit{Python for Data Analysis}, O'Reilly.
\end{itemize}

% BEGIN SLIDE APPENDIX (AUTO-GENERATED)
\clearpage
\section*{Appendix: Slide Deck Content (Reference)}
\noindent The material below is a reference copy of the slide deck content. Exercise solutions are explained in the main notes where applicable.

\subsection*{Title Slide}
\titlepage
        \vspace{-0.5em}
        \begin{center}
          \small \texttt{https://github.com/tali7c/Statistics-and-Data-Analysis}
        \end{center}
\subsection*{Quick Links}
\centering
        \textbf{Overview}\hspace{0.6em}
\textbf{Model}\hspace{0.6em}
\textbf{Fit and Diagnostics}\hspace{0.6em}
\textbf{Exercises}\hspace{0.6em}
\textbf{Demo}\hspace{0.6em}
\textbf{Summary}\hspace{0.6em}
\subsection*{Agenda}
\begin{itemize}
  \item Overview
  \item Model
  \item Fit and Diagnostics
  \item Exercises
  \item Demo
  \item Summary
\end{itemize}
\subsection*{Learning Outcomes}
\begin{itemize}
        \item Write the simple linear regression model
\item Interpret slope and intercept in context
\item Compute a prediction and a residual
\item Explain R-squared (intuition)
      \end{itemize}
\subsection*{Model: Key Points}
\begin{itemize}
        \item y = b0 + b1 x + error
\item Slope: expected change in y for 1-unit increase in x
\item Intercept: predicted y at x=0 (interpret carefully)
      \end{itemize}
\subsection*{Model: Key Formula}
\[ y = \beta_0 + \beta_1 x + \epsilon \]
\subsection*{Fit and Diagnostics: Key Points}
\begin{itemize}
        \item Look at residual plots for patterns
\item Outliers can dominate the fitted line
\item High $R^2$ does not guarantee a good model
      \end{itemize}
\subsection*{Exercise 1: Prediction}
\small
  Model: yhat = 10 + 2x. Predict y when x=7.
\subsection*{Solution 1}
\begin{itemize}
    \item yhat = 24
  \end{itemize}
\subsection*{Exercise 2: Residual}
\small
  If actual y=20 at x=7, compute residual.
\subsection*{Solution 2}
\begin{itemize}
    \item 20-24 = -4
  \end{itemize}
\subsection*{Exercise 3: Interpret slope}
\small
  Slope is 5 thousand INR per extra room. Interpret.
\subsection*{Solution 3}
\begin{itemize}
    \item Each extra room increases predicted price by ~5k INR (on average).
  \end{itemize}
\subsection*{Mini Demo (Python)}
Run from the lecture folder:
  \begin{center}
    \texttt{python demo/demo.py}
  \end{center}
  \vspace{0.4em}
  Outputs:
  \begin{itemize}
    \item \texttt{images/demo.png}
    \item \texttt{data/results.txt}
  \end{itemize}
\subsection*{Demo Output (Example)}
\begin{center}
  \IfFileExists{../images/demo.png}{
    \includegraphics[width=0.92\linewidth]{demo.png}
  }{
    \small (Run demo to generate: \texttt{demo.png})
  }
  \end{center}
\subsection*{Summary}
\begin{itemize}
        \item Key definitions and the main formula.
\item How to interpret results in context.
\item How the demo connects to the theory.
      \end{itemize}
\subsection*{Exit Question}
\small
  Why do we check residual plots even if $R^2$ is high?
% END SLIDE APPENDIX (AUTO-GENERATED)

\end{document}
