\documentclass{beamer}

      \usetheme{Berlin}
      \usecolortheme{Orchid}
      \useoutertheme{miniframes}
      \setbeamertemplate{navigation symbols}{}

      \usepackage{amsmath}
      \usepackage{amssymb}
      \usepackage{booktabs}
      \usepackage{graphicx}
      \graphicspath{{../images/}}

      \title[Statistics and Data Analysis]{Statistics and Data Analysis}
      \subtitle{Unit 04 -- Lecture 04: Polynomial Regression and Logistic Regression}
      \author{Tofik Ali}
      \institute{School of Computer Science, UPES Dehradun}
      \date{\today}

      \begin{document}

      \begin{frame}
        \titlepage
        \vspace{-0.5em}
        \begin{center}
          \small \texttt{https://github.com/tali7c/Statistics-and-Data-Analysis}
        \end{center}
      \end{frame}

      \begin{frame}{Quick Links}
        \centering
        \hyperlink{sec:overview}{\beamerbutton{Overview}}\hspace{0.6em}
\hyperlink{sec:poly}{\beamerbutton{Polynomial Regression}}\hspace{0.6em}
\hyperlink{sec:logit}{\beamerbutton{Logistic Regression}}\hspace{0.6em}
\hyperlink{sec:exercises}{\beamerbutton{Exercises}}\hspace{0.6em}
\hyperlink{sec:demo}{\beamerbutton{Demo}}\hspace{0.6em}
\hyperlink{sec:summary}{\beamerbutton{Summary}}\hspace{0.6em}
      \end{frame}

      \begin{frame}{Agenda}
        \tableofcontents
      \end{frame}

\section{Overview}
\label{sec:overview}


\begin{frame}{Learning Outcomes}
      \begin{itemize}[<+->]
        \item Explain polynomial features for modeling curvature
\item Recognize overfitting risk with high degree
\item Write logistic regression probability model (sigmoid)
\item Compute precision and recall from a confusion matrix
      \end{itemize}
    \end{frame}

\section{Polynomial Regression}
\label{sec:poly}


\begin{frame}{Polynomial Regression: Key Points}
      \begin{itemize}[<+->]
        \item Add features $x, x^2, x^3, \dots$
\item Still linear in parameters
\item Choose degree using validation
      \end{itemize}
    \end{frame}

\section{Logistic Regression}
\label{sec:logit}


\begin{frame}{Logistic Regression: Key Points}
      \begin{itemize}[<+->]
        \item Outputs probability in (0,1)
\item Threshold converts probability to class label
\item Evaluate using confusion matrix / ROC
      \end{itemize}
    \end{frame}

\begin{frame}{Logistic Regression: Key Formula}
  \[ P(y=1\mid x)=\frac{1}{1+e^{-(\beta_0+\beta^T x)}} \]
\end{frame}

\section{Exercises}
\label{sec:exercises}


\begin{frame}{Exercise 1: Polynomial features}
  \small
  For degree-2 polynomial, what features do we use from $x$?
\end{frame}

\begin{frame}{Solution 1}
  \begin{itemize}
    \item Use $1, x, x^2$ (intercept + linear + quadratic).
  \end{itemize}
\end{frame}

\begin{frame}{Exercise 2: Precision/recall}
  \small
  TP=30 FP=10 FN=20 TN=40. Compute precision and recall.
\end{frame}

\begin{frame}{Solution 2}
      \begin{itemize}
        \item Precision=30/(30+10)=0.75
\item Recall=30/(30+20)=0.60
      \end{itemize}
    \end{frame}

\begin{frame}{Exercise 3: Threshold effect}
  \small
  If threshold increases from 0.5 to 0.8, what tends to happen to precision and recall?
\end{frame}

\begin{frame}{Solution 3}
  \begin{itemize}
    \item Precision often increases, recall often decreases.
  \end{itemize}
\end{frame}

\section{Demo}
\label{sec:demo}


\begin{frame}{Mini Demo (Python)}
  Run from the lecture folder:
  \begin{center}
    \texttt{python demo/demo.py}
  \end{center}
  \vspace{0.4em}
  Outputs:
  \begin{itemize}
    \item \texttt{images/demo.png}
    \item \texttt{data/results.txt}
  \end{itemize}
\end{frame}

\begin{frame}{Demo Output (Example)}
  \begin{center}
  \IfFileExists{../images/demo.png}{
    \includegraphics[width=0.92\linewidth]{demo.png}
  }{
    \small (Run demo to generate: \texttt{demo.png})
  }
  \end{center}
\end{frame}

\section{Summary}
\label{sec:summary}


\begin{frame}{Summary}
      \begin{itemize}[<+->]
        \item Key definitions and the main formula.
\item How to interpret results in context.
\item How the demo connects to the theory.
      \end{itemize}
    \end{frame}

\begin{frame}{Exit Question}
  \small
  Why is ROC curve useful when classes are imbalanced?
\end{frame}

\end{document}
