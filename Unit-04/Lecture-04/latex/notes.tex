\documentclass[11pt]{article}
\usepackage[utf8]{inputenc}
\usepackage[T1]{fontenc}
\usepackage{geometry}
\usepackage{amsmath}
\usepackage{booktabs}
\usepackage{hyperref}
\geometry{margin=1in}

\title{Statistics and Data Analysis\\Unit 04 -- Lecture 04 Notes}
\author{Tofik Ali}
\date{\today}

\begin{document}
\maketitle

\section*{Topic}
Polynomial regression for curvature; logistic regression for classification; basic evaluation metrics.

\subsection*{Learning Outcomes}
\begin{itemize}
  \item Explain polynomial features for modeling curvature
  \item Recognize overfitting risk with high degree
  \item Write logistic regression probability model (sigmoid)
  \item Compute precision and recall from a confusion matrix
\end{itemize}

\section*{Detailed Notes}
These notes are designed to be read alongside the slides. They expand each slide bullet into
plain-language explanations, small worked examples, and common pitfalls. When a formula
appears, emphasize (1) what each symbol means, (2) the assumptions needed to use it, and (3)
how to interpret the final number in the problem context.

\section*{Polynomial Regression}
\begin{itemize}
  \item Add features $x, x^2, x^3, \dots$
  \item Still linear in parameters
  \item Choose degree using validation
\end{itemize}

\section*{Logistic Regression}
\begin{itemize}
  \item Outputs probability in (0,1)
  \item Threshold converts probability to class label
  \item Evaluate using confusion matrix / ROC
\end{itemize}

\section*{Exercises (with Solutions)}
\subsection*{Exercise 1: Polynomial features}
For degree-2 polynomial, what features do we use from $x$?
\subsection*{Solution}
\begin{itemize}
  \item Use $1, x, x^2$ (intercept + linear + quadratic).
\end{itemize}

\subsection*{Exercise 2: Precision/recall}
TP=30 FP=10 FN=20 TN=40. Compute precision and recall.
\subsection*{Solution}
\begin{itemize}
  \item Precision=30/(30+10)=0.75
  \item Recall=30/(30+20)=0.60
\end{itemize}

\subsection*{Exercise 3: Threshold effect}
If threshold increases from 0.5 to 0.8, what tends to happen to precision and recall?
\subsection*{Solution}
\begin{itemize}
  \item Precision often increases, recall often decreases.
\end{itemize}

\section*{Exit Question}
Why is ROC curve useful when classes are imbalanced?

\section*{Demo (Python)}
Run from the lecture folder:
\begin{verbatim}
python demo/demo.py
\end{verbatim}

Output files:
\begin{itemize}
  \item \texttt{images/demo.png}
  \item \texttt{data/results.txt}
\end{itemize}

\section*{References}
\begin{itemize}
  \item Montgomery, D. C., \& Runger, G. C. \textit{Applied Statistics and Probability for Engineers}, Wiley.
  \item Devore, J. L. \textit{Probability and Statistics for Engineering and the Sciences}, Cengage.
  \item McKinney, W. \textit{Python for Data Analysis}, O'Reilly.
\end{itemize}
\end{document}
