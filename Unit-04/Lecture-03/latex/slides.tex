\documentclass{beamer}

      \usetheme{Berlin}
      \usecolortheme{Orchid}
      \useoutertheme{miniframes}
      \setbeamertemplate{navigation symbols}{}

      \usepackage{amsmath}
      \usepackage{amssymb}
      \usepackage{booktabs}
      \usepackage{graphicx}
      \graphicspath{{../images/}}

      \title[Statistics and Data Analysis]{Statistics and Data Analysis}
      \subtitle{Unit 04 -- Lecture 03: Multiple Linear Regression}
      \author{Tofik Ali}
      \institute{School of Computer Science, UPES Dehradun}
      \date{\today}

      \begin{document}

      \begin{frame}
        \titlepage
        \vspace{-0.5em}
        \begin{center}
          \small \texttt{https://github.com/tali7c/Statistics-and-Data-Analysis}
        \end{center}
      \end{frame}

      \begin{frame}{Quick Links}
        \centering
        \hyperlink{sec:overview}{\beamerbutton{Overview}}\hspace{0.6em}
\hyperlink{sec:model}{\beamerbutton{Model}}\hspace{0.6em}
\hyperlink{sec:interp}{\beamerbutton{Interpretation}}\hspace{0.6em}
\hyperlink{sec:exercises}{\beamerbutton{Exercises}}\hspace{0.6em}
\hyperlink{sec:demo}{\beamerbutton{Demo}}\hspace{0.6em}
\hyperlink{sec:summary}{\beamerbutton{Summary}}\hspace{0.6em}
      \end{frame}

      \begin{frame}{Agenda}
        \tableofcontents
      \end{frame}

\section{Overview}
\label{sec:overview}


\begin{frame}{Learning Outcomes}
      \begin{itemize}[<+->]
        \item Write the multiple linear regression model
\item Interpret a coefficient as a partial effect
\item Explain dummy variables for categories (basic)
\item Explain adjusted R-squared (intuition)
      \end{itemize}
    \end{frame}

\section{Model}
\label{sec:model}


\begin{frame}{Model: Key Points}
      \begin{itemize}[<+->]
        \item y = b0 + b1 x1 + b2 x2 + ...
\item Each coefficient is a partial effect (others fixed)
\item Scaling helps when using regularization
      \end{itemize}
    \end{frame}

\begin{frame}{Model: Key Formula}
  \[ y = \beta_0 + \beta_1 x_1 + \beta_2 x_2 + \cdots + \epsilon \]
\end{frame}

\section{Interpretation}
\label{sec:interp}


\begin{frame}{Interpretation: Key Points}
      \begin{itemize}[<+->]
        \item Dummy variables encode categories
\item Adjusted $R^2$ penalizes unnecessary predictors
\item Multicollinearity can harm interpretability
      \end{itemize}
    \end{frame}

\section{Exercises}
\label{sec:exercises}


\begin{frame}{Exercise 1: Partial effect}
  \small
  Model: yhat=5 + 0.8x1 + 2.0x2. Interpret coefficient 2.0.
\end{frame}

\begin{frame}{Solution 1}
  \begin{itemize}
    \item Holding x1 fixed, +1 in x2 increases yhat by 2.0 units.
  \end{itemize}
\end{frame}

\begin{frame}{Exercise 2: Dummy variable}
  \small
  Urban=1, Rural=0. If coef(Urban)=10, interpret.
\end{frame}

\begin{frame}{Solution 2}
  \begin{itemize}
    \item Urban has predicted y about 10 units higher than Rural (all else equal).
  \end{itemize}
\end{frame}

\begin{frame}{Exercise 3: Adjusted $R^2$}
  \small
  Why use adjusted $R^2$ when comparing models with different number of predictors?
\end{frame}

\begin{frame}{Solution 3}
  \begin{itemize}
    \item Because $R^2$ never decreases, adjusted $R^2$ penalizes extra predictors.
  \end{itemize}
\end{frame}

\section{Demo}
\label{sec:demo}


\begin{frame}{Mini Demo (Python)}
  Run from the lecture folder:
  \begin{center}
    \texttt{python demo/demo.py}
  \end{center}
  \vspace{0.4em}
  Outputs:
  \begin{itemize}
    \item \texttt{images/demo.png}
    \item \texttt{data/results.txt}
  \end{itemize}
\end{frame}

\begin{frame}{Demo Output (Example)}
  \begin{center}
  \IfFileExists{../images/demo.png}{
    \includegraphics[width=0.92\linewidth]{demo.png}
  }{
    \small (Run demo to generate: \texttt{demo.png})
  }
  \end{center}
\end{frame}

\section{Summary}
\label{sec:summary}


\begin{frame}{Summary}
      \begin{itemize}[<+->]
        \item Key definitions and the main formula.
\item How to interpret results in context.
\item How the demo connects to the theory.
      \end{itemize}
    \end{frame}

\begin{frame}{Exit Question}
  \small
  Why does adding a useless feature still increase (or keep) $R^2$?
\end{frame}

\end{document}
