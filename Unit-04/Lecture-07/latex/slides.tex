\documentclass{beamer}

      \usetheme{Berlin}
      \usecolortheme{Orchid}
      \useoutertheme{miniframes}
      \setbeamertemplate{navigation symbols}{}

      \usepackage{amsmath}
      \usepackage{amssymb}
      \usepackage{booktabs}
      \usepackage{graphicx}
      \graphicspath{{../images/}}

      \title[Statistics and Data Analysis]{Statistics and Data Analysis}
      \subtitle{Unit 04 -- Lecture 07: VIF, AIC/BIC, Ridge and Lasso (Part 2)}
      \author{Tofik Ali}
      \institute{School of Computer Science, UPES Dehradun}
      \date{\today}

      \begin{document}

      \begin{frame}
        \titlepage
        \vspace{-0.5em}
        \begin{center}
          \small \texttt{https://github.com/tali7c/Statistics-and-Data-Analysis}
        \end{center}
      \end{frame}

      \begin{frame}{Quick Links}
        \centering
        \hyperlink{sec:overview}{\beamerbutton{Overview}}\hspace{0.6em}
\hyperlink{sec:vif}{\beamerbutton{VIF}}\hspace{0.6em}
\hyperlink{sec:reg}{\beamerbutton{Ridge/Lasso}}\hspace{0.6em}
\hyperlink{sec:exercises}{\beamerbutton{Exercises}}\hspace{0.6em}
\hyperlink{sec:demo}{\beamerbutton{Demo}}\hspace{0.6em}
\hyperlink{sec:summary}{\beamerbutton{Summary}}\hspace{0.6em}
      \end{frame}

      \begin{frame}{Agenda}
        \tableofcontents
      \end{frame}

\section{Overview}
\label{sec:overview}


\begin{frame}{Learning Outcomes}
      \begin{itemize}[<+->]
        \item Compute and interpret VIF (basic)
\item Explain AIC/BIC as model selection criteria (intuition)
\item Write ridge and lasso objectives
\item Explain coefficient shrinkage and feature selection idea
      \end{itemize}
    \end{frame}

\section{VIF}
\label{sec:vif}


\begin{frame}{VIF: Key Points}
      \begin{itemize}[<+->]
        \item Definition: $\mathrm{VIF}_j = 1/(1-R_j^2)$
\item Higher VIF -> more multicollinearity
\item Rule of thumb thresholds (5/10)
      \end{itemize}
    \end{frame}

\begin{frame}{VIF: Key Formula}
  \[ \mathrm{VIF}_j = \frac{1}{1-R_j^2} \]
\end{frame}

\section{Ridge/Lasso}
\label{sec:reg}


\begin{frame}{Ridge/Lasso: Key Points}
      \begin{itemize}[<+->]
        \item Ridge uses L2 penalty (shrinks)
\item Lasso uses L1 penalty (can set some to 0)
\item Scale features before regularization
      \end{itemize}
    \end{frame}

\begin{frame}{Ridge/Lasso: Key Formula}
  \[ \min \sum (y-\hat{y})^2 + \lambda \sum \beta_j^2 \quad \text{(ridge)} \]
\end{frame}

\section{Exercises}
\label{sec:exercises}


\begin{frame}{Exercise 1: Compute VIF}
  \small
  If $R_j^2=0.9$, compute $\mathrm{VIF}_j$.
\end{frame}

\begin{frame}{Solution 1}
  \begin{itemize}
    \item $\mathrm{VIF}_j = 1/(1-0.9)=10$ (high).
  \end{itemize}
\end{frame}

\begin{frame}{Exercise 2: Ridge vs lasso}
  \small
  Which can produce exact zero coefficients?
\end{frame}

\begin{frame}{Solution 2}
  \begin{itemize}
    \item Lasso (L1) can set some coefficients to 0.
  \end{itemize}
\end{frame}

\begin{frame}{Exercise 3: AIC/BIC meaning}
  \small
  Lower AIC/BIC means what (conceptually)?
\end{frame}

\begin{frame}{Solution 3}
  \begin{itemize}
    \item Better trade-off between fit and complexity (relative).
  \end{itemize}
\end{frame}

\section{Demo}
\label{sec:demo}


\begin{frame}{Mini Demo (Python)}
  Run from the lecture folder:
  \begin{center}
    \texttt{python demo/demo.py}
  \end{center}
  \vspace{0.4em}
  Outputs:
  \begin{itemize}
    \item \texttt{images/demo.png}
    \item \texttt{data/results.txt}
  \end{itemize}
\end{frame}

\begin{frame}{Demo Output (Example)}
  \begin{center}
  \IfFileExists{../images/demo.png}{
    \includegraphics[width=0.92\linewidth]{demo.png}
  }{
    \small (Run demo to generate: \texttt{demo.png})
  }
  \end{center}
\end{frame}

\section{Summary}
\label{sec:summary}


\begin{frame}{Summary}
      \begin{itemize}[<+->]
        \item Key definitions and the main formula.
\item How to interpret results in context.
\item How the demo connects to the theory.
      \end{itemize}
    \end{frame}

\begin{frame}{Exit Question}
  \small
  Why can ridge help when predictors are highly correlated?
\end{frame}

\end{document}
