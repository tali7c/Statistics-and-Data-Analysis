\documentclass[11pt]{article}
\usepackage[utf8]{inputenc}
\usepackage[T1]{fontenc}
\usepackage{geometry}
\usepackage{amsmath}
\usepackage{listings}
\usepackage{xcolor}
\usepackage{amssymb}
\usepackage{booktabs}
\usepackage{graphicx}
\graphicspath{{../images/}}
\usepackage{hyperref}
\geometry{margin=1in}

\title{Statistics and Data Analysis\\Unit 04 -- Lecture 05 Notes\\Multicollinearity}
\author{Tofik Ali}
\date{\today}

\begin{document}
\maketitle

\section*{Topic}
Multicollinearity: definition, symptoms, detection, and fixes.
\section*{How to Use These Notes}
These notes are written for students who are seeing the topic for the first time. They
follow the slide order, but add the missing 'why', interpretation, and common mistakes. If
you get stuck, look at the worked exercises and then run the Python demo.

Course repository (slides, demos, datasets): \url{https://github.com/tali7c/Statistics-and-Data-Analysis}

\section*{Time Plan (55 minutes)}
\begin{itemize}
  \item 0--10 min: Attendance + recap of previous lecture
  \item 10--35 min: Core concepts (this lecture's sections)
  \item 35--45 min: Exercises (solve 1--2 in class, rest as practice)
  \item 45--50 min: Mini demo + interpretation of output
  \item 50--55 min: Buffer / wrap-up (leave 5 minutes early)
\end{itemize}

\section*{Slide-by-slide Notes}
\subsection*{Title Slide}
State the lecture title clearly and connect it to what students already know.
Tell students what they will be able to do by the end (not just what you will cover).

\subsection*{Quick Links / Agenda}
Explain the structure of the lecture and where the exercises and demo appear.
\begin{itemize}
  \item Overview
  \item What and Why
  \item Detection
  \item Exercises
  \item Demo
  \item Summary
\end{itemize}

\subsection*{Learning Outcomes}
\begin{itemize}
  \item Define multicollinearity (high correlation among predictors)
  \item Explain why it harms interpretation (unstable coefficients)
  \item Recognize symptoms (large SEs, unstable signs)
  \item List common fixes (drop/combine/regularize)
\end{itemize}
\paragraph{Why these outcomes matter.}
\textbf{Correlation} measures the strength of a linear association between two variables. It
is symmetric (no X/Y direction) and does not imply causation. Outliers can inflate or hide
correlation, so always look at the scatter plot.
\textbf{Multicollinearity} means predictors overlap strongly (high correlation among $X$'s).
It makes individual coefficients unstable and standard errors large, so interpretation
suffers. Prediction can still be good, but explanations like 'feature X causes Y' become
unreliable.

\subsection*{What and Why: Key Points}
\begin{itemize}
  \item Predictors overlap in information
  \item Coefficients become unstable
  \item Prediction may still be OK but interpretation suffers
\end{itemize}

\subsection*{Detection: Key Points}
\begin{itemize}
  \item Correlation matrix/heatmap (screening)
  \item VIF (next)
  \item Condition number (advanced)
\end{itemize}
\paragraph{Explanation.}
\textbf{Correlation} measures the strength of a linear association between two variables. It
is symmetric (no X/Y direction) and does not imply causation. Outliers can inflate or hide
correlation, so always look at the scatter plot.
\textbf{VIF} measures how much the variance of a coefficient is inflated due to
multicollinearity. High VIF indicates a predictor can be explained by other predictors, so
its coefficient becomes unstable.

\subsection*{Exercises (with Solutions)}
Attempt the exercise first, then compare with the solution. Focus on interpretation, not
only arithmetic.

\subsection*{Exercise 1: Identify}
If corr(x1,x2)=0.98, what risk do you expect?
\subsubsection*{Solution}
\begin{itemize}
  \item High multicollinearity; unstable coefficients.
\end{itemize}
\paragraph{Walkthrough.}
\textbf{Multicollinearity} means predictors overlap strongly (high correlation among $X$'s).
It makes individual coefficients unstable and standard errors large, so interpretation
suffers. Prediction can still be good, but explanations like 'feature X causes Y' become
unreliable.

\subsection*{Exercise 2: Fix}
Name one fix for multicollinearity.
\subsubsection*{Solution}
\begin{itemize}
  \item Drop one feature, combine features, or use ridge/PCA.
\end{itemize}
\paragraph{Walkthrough.}
\textbf{Multicollinearity} means predictors overlap strongly (high correlation among $X$'s).
It makes individual coefficients unstable and standard errors large, so interpretation
suffers. Prediction can still be good, but explanations like 'feature X causes Y' become
unreliable.
\textbf{Ridge regression (L2)} shrinks coefficients toward zero, which reduces variance and
helps with multicollinearity. It usually keeps all features but with smaller magnitudes.
Always scale features before using ridge/lasso so the penalty is fair.

\subsection*{Exercise 3: Prediction vs interpretation}
Can multicollinearity still allow good prediction?
\subsubsection*{Solution}
\begin{itemize}
  \item Yes, but individual coefficients are unreliable.
\end{itemize}
\paragraph{Walkthrough.}
\textbf{Multicollinearity} means predictors overlap strongly (high correlation among $X$'s).
It makes individual coefficients unstable and standard errors large, so interpretation
suffers. Prediction can still be good, but explanations like 'feature X causes Y' become
unreliable.

\subsection*{Mini Demo (Python)}
Run from the lecture folder:
\begin{verbatim}
python demo/demo.py
\end{verbatim}

Output files:
\begin{itemize}
  \item \texttt{images/demo.png}
  \item \texttt{data/results.txt}
\end{itemize}
\paragraph{What to show and say.}
\begin{itemize}
  \item Creates correlated predictors and fits regression to show unstable coefficients.
  \item Reports correlation structure and coefficient variability across runs.
  \item Use it to motivate VIF and regularization as fixes.
\end{itemize}

\subsection*{Demo Output (Example)}
\begin{center}
\IfFileExists{../images/demo.png}{
  \includegraphics[width=0.95\linewidth]{../images/demo.png}
}{
  \small (Run the demo to generate \texttt{images/demo.png})
}
\end{center}

\subsection*{Summary}
\begin{itemize}
  \item Key definitions and the main formula.
  \item How to interpret results in context.
  \item How the demo connects to the theory.
\end{itemize}

\subsection*{Exit Question}
What does multicollinearity break first: prediction or interpretation (and why)?
\paragraph{Suggested answer (for revision).}
Interpretation breaks first: coefficients become unstable and hard to trust, while
prediction can remain reasonable due to redundant information.

\section*{References}
\begin{itemize}
  \item Montgomery, D. C., \& Runger, G. C. \textit{Applied Statistics and Probability for Engineers}, Wiley.
  \item Devore, J. L. \textit{Probability and Statistics for Engineering and the Sciences}, Cengage.
  \item McKinney, W. \textit{Python for Data Analysis}, O'Reilly.
\end{itemize}

% BEGIN SLIDE APPENDIX (AUTO-GENERATED)
\clearpage
\section*{Appendix: Slide Deck Content (Reference)}
\noindent The material below is a reference copy of the slide deck content. Exercise solutions are explained in the main notes where applicable.

\subsection*{Title Slide}
\titlepage
        \vspace{-0.5em}
        \begin{center}
          \small \texttt{https://github.com/tali7c/Statistics-and-Data-Analysis}
        \end{center}
\subsection*{Quick Links}
\centering
        \textbf{Overview}\hspace{0.6em}
\textbf{What and Why}\hspace{0.6em}
\textbf{Detection}\hspace{0.6em}
\textbf{Exercises}\hspace{0.6em}
\textbf{Demo}\hspace{0.6em}
\textbf{Summary}\hspace{0.6em}
\subsection*{Agenda}
\begin{itemize}
  \item Overview
  \item What and Why
  \item Detection
  \item Exercises
  \item Demo
  \item Summary
\end{itemize}
\subsection*{Learning Outcomes}
\begin{itemize}
        \item Define multicollinearity (high correlation among predictors)
\item Explain why it harms interpretation (unstable coefficients)
\item Recognize symptoms (large SEs, unstable signs)
\item List common fixes (drop/combine/regularize)
      \end{itemize}
\subsection*{What and Why: Key Points}
\begin{itemize}
        \item Predictors overlap in information
\item Coefficients become unstable
\item Prediction may still be OK but interpretation suffers
      \end{itemize}
\subsection*{Detection: Key Points}
\begin{itemize}
        \item Correlation matrix/heatmap (screening)
\item VIF (next)
\item Condition number (advanced)
      \end{itemize}
\subsection*{Exercise 1: Identify}
\small
  If corr(x1,x2)=0.98, what risk do you expect?
\subsection*{Solution 1}
\begin{itemize}
    \item High multicollinearity; unstable coefficients.
  \end{itemize}
\subsection*{Exercise 2: Fix}
\small
  Name one fix for multicollinearity.
\subsection*{Solution 2}
\begin{itemize}
    \item Drop one feature, combine features, or use ridge/PCA.
  \end{itemize}
\subsection*{Exercise 3: Prediction vs interpretation}
\small
  Can multicollinearity still allow good prediction?
\subsection*{Solution 3}
\begin{itemize}
    \item Yes, but individual coefficients are unreliable.
  \end{itemize}
\subsection*{Mini Demo (Python)}
Run from the lecture folder:
  \begin{center}
    \texttt{python demo/demo.py}
  \end{center}
  \vspace{0.4em}
  Outputs:
  \begin{itemize}
    \item \texttt{images/demo.png}
    \item \texttt{data/results.txt}
  \end{itemize}
\subsection*{Demo Output (Example)}
\begin{center}
  \IfFileExists{../images/demo.png}{
    \includegraphics[width=0.92\linewidth]{demo.png}
  }{
    \small (Run demo to generate: \texttt{demo.png})
  }
  \end{center}
\subsection*{Summary}
\begin{itemize}
        \item Key definitions and the main formula.
\item How to interpret results in context.
\item How the demo connects to the theory.
      \end{itemize}
\subsection*{Exit Question}
\small
  What does multicollinearity break first: prediction or interpretation (and why)?
% END SLIDE APPENDIX (AUTO-GENERATED)

\end{document}
