\documentclass[11pt]{article}
\usepackage[utf8]{inputenc}
\usepackage[T1]{fontenc}
\usepackage{geometry}
\usepackage{amsmath}
\usepackage{booktabs}
\usepackage{hyperref}
\geometry{margin=1in}

\title{Statistics and Data Analysis\\Unit 05 -- Lecture 03 Notes}
\author{Tofik Ali}
\date{\today}

\begin{document}
\maketitle

\section*{Topic}
PCA: variance-maximizing projection; explained variance; scaling.

\subsection*{Learning Outcomes}
\begin{itemize}
  \item Explain PCA as a variance-maximizing linear projection
  \item State why scaling is important before PCA
  \item Interpret explained variance ratio and scree plot
  \item Use PCA for visualization and noise reduction
\end{itemize}

\section*{Detailed Notes}
These notes are designed to be read alongside the slides. They expand each slide bullet into
plain-language explanations, small worked examples, and common pitfalls. When a formula
appears, emphasize (1) what each symbol means, (2) the assumptions needed to use it, and (3)
how to interpret the final number in the problem context.

\section*{PCA Intuition}
\begin{itemize}
  \item Find new axes (components) capturing maximum variance
  \item Components are orthogonal
  \item PC1 captures most variance
\end{itemize}

\section*{Explained Variance}
\begin{itemize}
  \item Explained variance ratio per component
  \item Choose k via scree plot / cumulative variance target
  \item Validate downstream performance
\end{itemize}

\section*{Exercises (with Solutions)}
\subsection*{Exercise 1: Scaling}
Why scale features before PCA?
\subsection*{Solution}
\begin{itemize}
  \item To prevent large-unit features dominating variance.
\end{itemize}

\subsection*{Exercise 2: Components}
Are PCA components original features?
\subsection*{Solution}
\begin{itemize}
  \item No; they are linear combinations.
\end{itemize}

\subsection*{Exercise 3: Choosing k}
If first 2 PCs explain 88\% and you need 90\%, what do you do?
\subsection*{Solution}
\begin{itemize}
  \item Add next PC(s) until target reached.
\end{itemize}

\section*{Exit Question}
Why might PCA improve a model even though it discards some variance?

\section*{Demo (Python)}
Run from the lecture folder:
\begin{verbatim}
python demo/demo.py
\end{verbatim}

Output files:
\begin{itemize}
  \item \texttt{images/demo.png}
  \item \texttt{data/results.txt}
\end{itemize}

\section*{References}
\begin{itemize}
  \item Montgomery, D. C., \& Runger, G. C. \textit{Applied Statistics and Probability for Engineers}, Wiley.
  \item Devore, J. L. \textit{Probability and Statistics for Engineering and the Sciences}, Cengage.
  \item McKinney, W. \textit{Python for Data Analysis}, O'Reilly.
\end{itemize}
\end{document}
