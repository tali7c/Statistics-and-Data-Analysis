\documentclass[11pt]{article}
\usepackage[utf8]{inputenc}
\usepackage[T1]{fontenc}
\usepackage{geometry}
\usepackage{amsmath}
\usepackage{listings}
\usepackage{xcolor}
\usepackage{amssymb}
\usepackage{booktabs}
\usepackage{graphicx}
\graphicspath{{../images/}}
\usepackage{hyperref}
\geometry{margin=1in}

\title{Statistics and Data Analysis\\Unit 05 -- Lecture 07 Notes\\Case Study: PCA + Clustering}
\author{Tofik Ali}
\date{\today}

\begin{document}
\maketitle

\section*{Topic}
Case study: PCA + KMeans clustering; visualize and interpret clusters.
\section*{How to Use These Notes}
These notes are written for students who are seeing the topic for the first time. They
follow the slide order, but add the missing 'why', interpretation, and common mistakes. If
you get stuck, look at the worked exercises and then run the Python demo.

Course repository (slides, demos, datasets): \url{https://github.com/tali7c/Statistics-and-Data-Analysis}

\section*{Time Plan (55 minutes)}
\begin{itemize}
  \item 0--10 min: Attendance + recap of previous lecture
  \item 10--35 min: Core concepts (this lecture's sections)
  \item 35--45 min: Exercises (solve 1--2 in class, rest as practice)
  \item 45--50 min: Mini demo + interpretation of output
  \item 50--55 min: Buffer / wrap-up (leave 5 minutes early)
\end{itemize}

\section*{Slide-by-slide Notes}
\subsection*{Title Slide}
State the lecture title clearly and connect it to what students already know.
Tell students what they will be able to do by the end (not just what you will cover).

\subsection*{Quick Links / Agenda}
Explain the structure of the lecture and where the exercises and demo appear.
\begin{itemize}
  \item Overview
  \item Pipeline
  \item Interpretation
  \item Exercises
  \item Demo
  \item Summary
\end{itemize}

\subsection*{Learning Outcomes}
\begin{itemize}
  \item Run PCA before clustering for visualization/stability
  \item Explain why scaling matters for clustering
  \item Use KMeans and interpret clusters cautiously
  \item Visualize clusters in PCA space
\end{itemize}
\paragraph{Why these outcomes matter.}
\textbf{PCA} finds new axes (principal components) that capture maximum variance. It is a
rotation of the feature space. Because PCA is variance-based, it is sensitive to scaling:
standardize features first unless all features are already comparable.

\subsection*{Pipeline: Key Points}
\begin{itemize}
  \item Scale features
  \item Run PCA (2D for visualization)
  \item Cluster (KMeans) and visualize
\end{itemize}
\paragraph{Explanation.}
\textbf{PCA} finds new axes (principal components) that capture maximum variance. It is a
rotation of the feature space. Because PCA is variance-based, it is sensitive to scaling:
standardize features first unless all features are already comparable.

\subsection*{Interpretation: Key Points}
\begin{itemize}
  \item Clusters are patterns, not truth
  \item Check stability across seeds/k
  \item Explain clusters using original variables
\end{itemize}

\subsection*{Exercises (with Solutions)}
Attempt the exercise first, then compare with the solution. Focus on interpretation, not
only arithmetic.

\subsection*{Exercise 1: Scaling}
Why scale before KMeans?
\subsubsection*{Solution}
\begin{itemize}
  \item Distance-based; scale dominates otherwise.
\end{itemize}

\subsection*{Exercise 2: Choose k}
Name one heuristic to choose k.
\subsubsection*{Solution}
\begin{itemize}
  \item Elbow, silhouette, domain knowledge.
\end{itemize}

\subsection*{Exercise 3: Explain cluster}
How to explain cluster to non-technical audience?
\subsubsection*{Solution}
\begin{itemize}
  \item Describe in original variables (high spend, frequent visits, etc.).
\end{itemize}

\subsection*{Mini Demo (Python)}
Run from the lecture folder:
\begin{verbatim}
python demo/demo.py
\end{verbatim}

Output files:
\begin{itemize}
  \item \texttt{images/demo.png}
  \item \texttt{data/results.txt}
\end{itemize}
\paragraph{What to show and say.}
\begin{itemize}
  \item Scales features, runs PCA for 2D visualization, then clusters with KMeans.
  \item Plots clusters in PCA space to discuss patterns vs ground truth.
  \item Use it to talk about choosing k (elbow/silhouette) and stability checks.
\end{itemize}

\subsection*{Demo Output (Example)}
\begin{center}
\IfFileExists{../images/demo.png}{
  \includegraphics[width=0.95\linewidth]{../images/demo.png}
}{
  \small (Run the demo to generate \texttt{images/demo.png})
}
\end{center}

\subsection*{Summary}
\begin{itemize}
  \item Key definitions and the main formula.
  \item How to interpret results in context.
  \item How the demo connects to the theory.
\end{itemize}

\subsection*{Exit Question}
Why should you validate cluster stability before using clusters for decisions?
\paragraph{Suggested answer (for revision).}
Clusters can change with seed/k; stability checks prevent making decisions based on
accidental patterns in one run.

\section*{References}
\begin{itemize}
  \item Montgomery, D. C., \& Runger, G. C. \textit{Applied Statistics and Probability for Engineers}, Wiley.
  \item Devore, J. L. \textit{Probability and Statistics for Engineering and the Sciences}, Cengage.
  \item McKinney, W. \textit{Python for Data Analysis}, O'Reilly.
\end{itemize}

% BEGIN SLIDE APPENDIX (AUTO-GENERATED)
\clearpage
\section*{Appendix: Slide Deck Content (Reference)}
\noindent The material below is a reference copy of the slide deck content. Exercise solutions are explained in the main notes where applicable.

\subsection*{Title Slide}
\titlepage
        \vspace{-0.5em}
        \begin{center}
          \small \texttt{https://github.com/tali7c/Statistics-and-Data-Analysis}
        \end{center}
\subsection*{Quick Links}
\centering
        \textbf{Overview}\hspace{0.6em}
\textbf{Pipeline}\hspace{0.6em}
\textbf{Interpretation}\hspace{0.6em}
\textbf{Exercises}\hspace{0.6em}
\textbf{Demo}\hspace{0.6em}
\textbf{Summary}\hspace{0.6em}
\subsection*{Agenda}
\begin{itemize}
  \item Overview
  \item Pipeline
  \item Interpretation
  \item Exercises
  \item Demo
  \item Summary
\end{itemize}
\subsection*{Learning Outcomes}
\begin{itemize}
        \item Run PCA before clustering for visualization/stability
\item Explain why scaling matters for clustering
\item Use KMeans and interpret clusters cautiously
\item Visualize clusters in PCA space
      \end{itemize}
\subsection*{Pipeline: Key Points}
\begin{itemize}
        \item Scale features
\item Run PCA (2D for visualization)
\item Cluster (KMeans) and visualize
      \end{itemize}
\subsection*{Interpretation: Key Points}
\begin{itemize}
        \item Clusters are patterns, not truth
\item Check stability across seeds/k
\item Explain clusters using original variables
      \end{itemize}
\subsection*{Exercise 1: Scaling}
\small
  Why scale before KMeans?
\subsection*{Solution 1}
\begin{itemize}
    \item Distance-based; scale dominates otherwise.
  \end{itemize}
\subsection*{Exercise 2: Choose k}
\small
  Name one heuristic to choose k.
\subsection*{Solution 2}
\begin{itemize}
    \item Elbow, silhouette, domain knowledge.
  \end{itemize}
\subsection*{Exercise 3: Explain cluster}
\small
  How to explain cluster to non-technical audience?
\subsection*{Solution 3}
\begin{itemize}
    \item Describe in original variables (high spend, frequent visits, etc.).
  \end{itemize}
\subsection*{Mini Demo (Python)}
Run from the lecture folder:
  \begin{center}
    \texttt{python demo/demo.py}
  \end{center}
  \vspace{0.4em}
  Outputs:
  \begin{itemize}
    \item \texttt{images/demo.png}
    \item \texttt{data/results.txt}
  \end{itemize}
\subsection*{Demo Output (Example)}
\begin{center}
  \IfFileExists{../images/demo.png}{
    \includegraphics[width=0.92\linewidth]{demo.png}
  }{
    \small (Run demo to generate: \texttt{demo.png})
  }
  \end{center}
\subsection*{Summary}
\begin{itemize}
        \item Key definitions and the main formula.
\item How to interpret results in context.
\item How the demo connects to the theory.
      \end{itemize}
\subsection*{Exit Question}
\small
  Why should you validate cluster stability before using clusters for decisions?
% END SLIDE APPENDIX (AUTO-GENERATED)

\end{document}
