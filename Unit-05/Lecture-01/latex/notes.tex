\documentclass[11pt]{article}
\usepackage[utf8]{inputenc}
\usepackage[T1]{fontenc}
\usepackage{geometry}
\usepackage{amsmath}
\usepackage{listings}
\usepackage{xcolor}
\usepackage{amssymb}
\usepackage{booktabs}
\usepackage{graphicx}
\graphicspath{{../images/}}
\usepackage{hyperref}
\geometry{margin=1in}

\title{Statistics and Data Analysis\\Unit 05 -- Lecture 01 Notes\\Feature Selection, Engineering and Dimensionality Reduction (Intro)}
\author{Tofik Ali}
\date{\today}

\begin{document}
\maketitle

\section*{Topic}
Intro to feature selection, feature engineering, and dimensionality reduction.
\section*{How to Use These Notes}
These notes are written for students who are seeing the topic for the first time. They
follow the slide order, but add the missing 'why', interpretation, and common mistakes. If
you get stuck, look at the worked exercises and then run the Python demo.

Course repository (slides, demos, datasets): \url{https://github.com/tali7c/Statistics-and-Data-Analysis}

\section*{Time Plan (55 minutes)}
\begin{itemize}
  \item 0--10 min: Attendance + recap of previous lecture
  \item 10--35 min: Core concepts (this lecture's sections)
  \item 35--45 min: Exercises (solve 1--2 in class, rest as practice)
  \item 45--50 min: Mini demo + interpretation of output
  \item 50--55 min: Buffer / wrap-up (leave 5 minutes early)
\end{itemize}

\section*{Slide-by-slide Notes}
\subsection*{Title Slide}
State the lecture title clearly and connect it to what students already know.
Tell students what they will be able to do by the end (not just what you will cover).

\subsection*{Quick Links / Agenda}
Explain the structure of the lecture and where the exercises and demo appear.
\begin{itemize}
  \item Overview
  \item Why Features
  \item Selection vs Reduction
  \item Exercises
  \item Demo
  \item Summary
\end{itemize}

\subsection*{Learning Outcomes}
\begin{itemize}
  \item Differentiate feature selection vs dimensionality reduction
  \item Explain why too many features can hurt (overfitting, cost)
  \item Describe a simple feature engineering pipeline
  \item Identify target leakage in engineered features
\end{itemize}
\paragraph{Why these outcomes matter.}
\textbf{Data leakage} happens when information from the future or from the test set
influences training. Typical examples: scaling before splitting, using target-related
features, or using random splits for time series. Leakage can produce very good-looking
accuracy that disappears in real deployment.

\subsection*{Why Features: Key Points}
\begin{itemize}
  \item Features are how models see data
  \item Goal: represent signal and reduce noise
  \item Bad features -> bad models
\end{itemize}

\subsection*{Selection vs Reduction: Key Points}
\begin{itemize}
  \item Selection keeps a subset of original features
  \item Reduction creates new components (e.g., PCA)
  \item Validate choices using CV
\end{itemize}
\paragraph{Explanation.}
\textbf{PCA} finds new axes (principal components) that capture maximum variance. It is a
rotation of the feature space. Because PCA is variance-based, it is sensitive to scaling:
standardize features first unless all features are already comparable.

\subsection*{Exercises (with Solutions)}
Attempt the exercise first, then compare with the solution. Focus on interpretation, not
only arithmetic.

\subsection*{Exercise 1: Selection or reduction}
Dropping 30 out of 100 features is selection or reduction?
\subsubsection*{Solution}
\begin{itemize}
  \item Feature selection (subset).
\end{itemize}

\subsection*{Exercise 2: Leakage}
Is using final exam score to predict final grade leakage?
\subsubsection*{Solution}
\begin{itemize}
  \item Yes; it contains future/target information.
\end{itemize}
\paragraph{Walkthrough.}
\textbf{Data leakage} happens when information from the future or from the test set
influences training. Typical examples: scaling before splitting, using target-related
features, or using random splits for time series. Leakage can produce very good-looking
accuracy that disappears in real deployment.

\subsection*{Exercise 3: Engineering example}
Give one time-based engineered feature.
\subsubsection*{Solution}
\begin{itemize}
  \item Day-of-week, month, time-since-last-event, rolling average, etc.
\end{itemize}

\subsection*{Mini Demo (Python)}
Run from the lecture folder:
\begin{verbatim}
python demo/demo.py
\end{verbatim}

Output files:
\begin{itemize}
  \item \texttt{images/demo.png}
  \item \texttt{data/results.txt}
\end{itemize}
\paragraph{What to show and say.}
\begin{itemize}
  \item Shows how feature quality affects model performance on a toy dataset.
  \item Creates simple engineered features and compares baseline vs improved model.
  \item Use it to motivate selection/reduction to fight overfitting and cost.
\end{itemize}

\subsection*{Demo Output (Example)}
\begin{center}
\IfFileExists{../images/demo.png}{
  \includegraphics[width=0.95\linewidth]{../images/demo.png}
}{
  \small (Run the demo to generate \texttt{images/demo.png})
}
\end{center}

\subsection*{Summary}
\begin{itemize}
  \item Key definitions and the main formula.
  \item How to interpret results in context.
  \item How the demo connects to the theory.
\end{itemize}

\subsection*{Exit Question}
Why can adding more features sometimes reduce test accuracy?
\paragraph{Suggested answer (for revision).}
Extra features can add noise and increase overfitting (curse of dimensionality), reducing
test performance even if training fit improves.

\section*{References}
\begin{itemize}
  \item Montgomery, D. C., \& Runger, G. C. \textit{Applied Statistics and Probability for Engineers}, Wiley.
  \item Devore, J. L. \textit{Probability and Statistics for Engineering and the Sciences}, Cengage.
  \item McKinney, W. \textit{Python for Data Analysis}, O'Reilly.
\end{itemize}

% BEGIN SLIDE APPENDIX (AUTO-GENERATED)
\clearpage
\section*{Appendix: Slide Deck Content (Reference)}
\noindent The material below is a reference copy of the slide deck content. Exercise solutions are explained in the main notes where applicable.

\subsection*{Title Slide}
\titlepage
        \vspace{-0.5em}
        \begin{center}
          \small \texttt{https://github.com/tali7c/Statistics-and-Data-Analysis}
        \end{center}
\subsection*{Quick Links}
\centering
        \textbf{Overview}\hspace{0.6em}
\textbf{Why Features}\hspace{0.6em}
\textbf{Selection vs Reduction}\hspace{0.6em}
\textbf{Exercises}\hspace{0.6em}
\textbf{Demo}\hspace{0.6em}
\textbf{Summary}\hspace{0.6em}
\subsection*{Agenda}
\begin{itemize}
  \item Overview
  \item Why Features
  \item Selection vs Reduction
  \item Exercises
  \item Demo
  \item Summary
\end{itemize}
\subsection*{Learning Outcomes}
\begin{itemize}
        \item Differentiate feature selection vs dimensionality reduction
\item Explain why too many features can hurt (overfitting, cost)
\item Describe a simple feature engineering pipeline
\item Identify target leakage in engineered features
      \end{itemize}
\subsection*{Why Features: Key Points}
\begin{itemize}
        \item Features are how models see data
\item Goal: represent signal and reduce noise
\item Bad features -> bad models
      \end{itemize}
\subsection*{Selection vs Reduction: Key Points}
\begin{itemize}
        \item Selection keeps a subset of original features
\item Reduction creates new components (e.g., PCA)
\item Validate choices using CV
      \end{itemize}
\subsection*{Exercise 1: Selection or reduction}
\small
  Dropping 30 out of 100 features is selection or reduction?
\subsection*{Solution 1}
\begin{itemize}
    \item Feature selection (subset).
  \end{itemize}
\subsection*{Exercise 2: Leakage}
\small
  Is using final exam score to predict final grade leakage?
\subsection*{Solution 2}
\begin{itemize}
    \item Yes; it contains future/target information.
  \end{itemize}
\subsection*{Exercise 3: Engineering example}
\small
  Give one time-based engineered feature.
\subsection*{Solution 3}
\begin{itemize}
    \item Day-of-week, month, time-since-last-event, rolling average, etc.
  \end{itemize}
\subsection*{Mini Demo (Python)}
Run from the lecture folder:
  \begin{center}
    \texttt{python demo/demo.py}
  \end{center}
  \vspace{0.4em}
  Outputs:
  \begin{itemize}
    \item \texttt{images/demo.png}
    \item \texttt{data/results.txt}
  \end{itemize}
\subsection*{Demo Output (Example)}
\begin{center}
  \IfFileExists{../images/demo.png}{
    \includegraphics[width=0.92\linewidth]{demo.png}
  }{
    \small (Run demo to generate: \texttt{demo.png})
  }
  \end{center}
\subsection*{Summary}
\begin{itemize}
        \item Key definitions and the main formula.
\item How to interpret results in context.
\item How the demo connects to the theory.
      \end{itemize}
\subsection*{Exit Question}
\small
  Why can adding more features sometimes reduce test accuracy?
% END SLIDE APPENDIX (AUTO-GENERATED)

\end{document}
