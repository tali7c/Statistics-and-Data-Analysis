\documentclass[11pt]{article}
\usepackage[utf8]{inputenc}
\usepackage[T1]{fontenc}
\usepackage{geometry}
\usepackage{amsmath}
\usepackage{booktabs}
\usepackage{hyperref}
\geometry{margin=1in}

\title{Statistics and Data Analysis\\Unit 05 -- Lecture 01 Notes}
\author{Tofik Ali}
\date{\today}

\begin{document}
\maketitle

\section*{Topic}
Intro to feature selection, feature engineering, and dimensionality reduction.

\subsection*{Learning Outcomes}
\begin{itemize}
  \item Differentiate feature selection vs dimensionality reduction
  \item Explain why too many features can hurt (overfitting, cost)
  \item Describe a simple feature engineering pipeline
  \item Identify target leakage in engineered features
\end{itemize}

\section*{Detailed Notes}
These notes are designed to be read alongside the slides. They expand each slide bullet into
plain-language explanations, small worked examples, and common pitfalls. When a formula
appears, emphasize (1) what each symbol means, (2) the assumptions needed to use it, and (3)
how to interpret the final number in the problem context.

\section*{Why Features}
\begin{itemize}
  \item Features are how models see data
  \item Goal: represent signal and reduce noise
  \item Bad features -> bad models
\end{itemize}

\section*{Selection vs Reduction}
\begin{itemize}
  \item Selection keeps a subset of original features
  \item Reduction creates new components (e.g., PCA)
  \item Validate choices using CV
\end{itemize}

\section*{Exercises (with Solutions)}
\subsection*{Exercise 1: Selection or reduction}
Dropping 30 out of 100 features is selection or reduction?
\subsection*{Solution}
\begin{itemize}
  \item Feature selection (subset).
\end{itemize}

\subsection*{Exercise 2: Leakage}
Is using final exam score to predict final grade leakage?
\subsection*{Solution}
\begin{itemize}
  \item Yes; it contains future/target information.
\end{itemize}

\subsection*{Exercise 3: Engineering example}
Give one time-based engineered feature.
\subsection*{Solution}
\begin{itemize}
  \item Day-of-week, month, time-since-last-event, rolling average, etc.
\end{itemize}

\section*{Exit Question}
Why can adding more features sometimes reduce test accuracy?

\section*{Demo (Python)}
Run from the lecture folder:
\begin{verbatim}
python demo/demo.py
\end{verbatim}

Output files:
\begin{itemize}
  \item \texttt{images/demo.png}
  \item \texttt{data/results.txt}
\end{itemize}

\section*{References}
\begin{itemize}
  \item Montgomery, D. C., \& Runger, G. C. \textit{Applied Statistics and Probability for Engineers}, Wiley.
  \item Devore, J. L. \textit{Probability and Statistics for Engineering and the Sciences}, Cengage.
  \item McKinney, W. \textit{Python for Data Analysis}, O'Reilly.
\end{itemize}
\end{document}
