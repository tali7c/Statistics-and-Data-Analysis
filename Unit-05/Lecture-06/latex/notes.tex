\documentclass[11pt]{article}
\usepackage[utf8]{inputenc}
\usepackage[T1]{fontenc}
\usepackage{geometry}
\usepackage{amsmath}
\usepackage{listings}
\usepackage{xcolor}
\usepackage{amssymb}
\usepackage{booktabs}
\usepackage{graphicx}
\graphicspath{{../images/}}
\usepackage{hyperref}
\geometry{margin=1in}

\title{Statistics and Data Analysis\\Unit 05 -- Lecture 06 Notes\\Advanced Feature Engineering for Multivariate Data}
\author{Tofik Ali}
\date{\today}

\begin{document}
\maketitle

\section*{Topic}
Interactions, aggregations, time features, and leakage avoidance.
\section*{How to Use These Notes}
These notes are written for students who are seeing the topic for the first time. They
follow the slide order, but add the missing 'why', interpretation, and common mistakes. If
you get stuck, look at the worked exercises and then run the Python demo.

Course repository (slides, demos, datasets): \url{https://github.com/tali7c/Statistics-and-Data-Analysis}

\section*{Time Plan (55 minutes)}
\begin{itemize}
  \item 0--10 min: Attendance + recap of previous lecture
  \item 10--35 min: Core concepts (this lecture's sections)
  \item 35--45 min: Exercises (solve 1--2 in class, rest as practice)
  \item 45--50 min: Mini demo + interpretation of output
  \item 50--55 min: Buffer / wrap-up (leave 5 minutes early)
\end{itemize}

\section*{Slide-by-slide Notes}
\subsection*{Title Slide}
State the lecture title clearly and connect it to what students already know.
Tell students what they will be able to do by the end (not just what you will cover).

\subsection*{Quick Links / Agenda}
Explain the structure of the lecture and where the exercises and demo appear.
\begin{itemize}
  \item Overview
  \item Interactions
  \item Aggregations
  \item Exercises
  \item Demo
  \item Summary
\end{itemize}

\subsection*{Learning Outcomes}
\begin{itemize}
  \item Create interaction features when meaningful
  \item Create aggregation features from transactional data
  \item Engineer time-based features (lags/rolling)
  \item Avoid leakage and look-ahead bias
\end{itemize}
\paragraph{Why these outcomes matter.}
\textbf{Bias} is a systematic error: your method tends to be wrong in the same direction
again and again (too high or too low). It does not disappear by taking more samples if the
sampling process is flawed. Fixing bias usually requires changing the data collection
procedure (sampling frame, selection method, non-response handling).
\textbf{Data leakage} happens when information from the future or from the test set
influences training. Typical examples: scaling before splitting, using target-related
features, or using random splits for time series. Leakage can produce very good-looking
accuracy that disappears in real deployment.

\subsection*{Interactions: Key Points}
\begin{itemize}
  \item Products and ratios capture combined effects
  \item Use domain knowledge
  \item Validate with CV
\end{itemize}

\subsection*{Aggregations: Key Points}
\begin{itemize}
  \item Per-user totals/means/counts
  \item Rolling windows (last 7/30 days)
  \item Avoid using future data
\end{itemize}

\subsection*{Exercises (with Solutions)}
Attempt the exercise first, then compare with the solution. Focus on interpretation, not
only arithmetic.

\subsection*{Exercise 1: Interaction}
Give one interaction feature for house price.
\subsubsection*{Solution}
\begin{itemize}
  \item size\_m2 * location\_score (example).
\end{itemize}

\subsection*{Exercise 2: Aggregation}
Name one per-user aggregation for churn prediction.
\subsubsection*{Solution}
\begin{itemize}
  \item days\_since\_last\_purchase (example).
\end{itemize}

\subsection*{Exercise 3: Leakage}
Is using next-30-days spend to predict churn today leakage?
\subsubsection*{Solution}
\begin{itemize}
  \item Yes; it uses future info.
\end{itemize}
\paragraph{Walkthrough.}
\textbf{Data leakage} happens when information from the future or from the test set
influences training. Typical examples: scaling before splitting, using target-related
features, or using random splits for time series. Leakage can produce very good-looking
accuracy that disappears in real deployment.

\subsection*{Mini Demo (Python)}
Run from the lecture folder:
\begin{verbatim}
python demo/demo.py
\end{verbatim}

Output files:
\begin{itemize}
  \item \texttt{images/demo.png}
  \item \texttt{data/results.txt}
\end{itemize}
\paragraph{What to show and say.}
\begin{itemize}
  \item Creates interaction and aggregation features from a toy transactional dataset.
  \item Shows how leakage can happen if you use future information in features.
  \item Use it to emphasize validation and time-aware splits when needed.
\end{itemize}

\subsection*{Demo Output (Example)}
\begin{center}
\IfFileExists{../images/demo.png}{
  \includegraphics[width=0.95\linewidth]{../images/demo.png}
}{
  \small (Run the demo to generate \texttt{images/demo.png})
}
\end{center}

\subsection*{Summary}
\begin{itemize}
  \item Key definitions and the main formula.
  \item How to interpret results in context.
  \item How the demo connects to the theory.
\end{itemize}

\subsection*{Exit Question}
How does cross-validation help detect whether engineered features overfit?
\paragraph{Suggested answer (for revision).}
Cross-validation (and proper time splits) reveal whether engineered features generalize; if
score drops on validation, features may overfit/leak.

\section*{References}
\begin{itemize}
  \item Montgomery, D. C., \& Runger, G. C. \textit{Applied Statistics and Probability for Engineers}, Wiley.
  \item Devore, J. L. \textit{Probability and Statistics for Engineering and the Sciences}, Cengage.
  \item McKinney, W. \textit{Python for Data Analysis}, O'Reilly.
\end{itemize}

% BEGIN SLIDE APPENDIX (AUTO-GENERATED)
\clearpage
\section*{Appendix: Slide Deck Content (Reference)}
\noindent The material below is a reference copy of the slide deck content. Exercise solutions are explained in the main notes where applicable.

\subsection*{Title Slide}
\titlepage
        \vspace{-0.5em}
        \begin{center}
          \small \texttt{https://github.com/tali7c/Statistics-and-Data-Analysis}
        \end{center}
\subsection*{Quick Links}
\centering
        \textbf{Overview}\hspace{0.6em}
\textbf{Interactions}\hspace{0.6em}
\textbf{Aggregations}\hspace{0.6em}
\textbf{Exercises}\hspace{0.6em}
\textbf{Demo}\hspace{0.6em}
\textbf{Summary}\hspace{0.6em}
\subsection*{Agenda}
\begin{itemize}
  \item Overview
  \item Interactions
  \item Aggregations
  \item Exercises
  \item Demo
  \item Summary
\end{itemize}
\subsection*{Learning Outcomes}
\begin{itemize}
        \item Create interaction features when meaningful
\item Create aggregation features from transactional data
\item Engineer time-based features (lags/rolling)
\item Avoid leakage and look-ahead bias
      \end{itemize}
\subsection*{Interactions: Key Points}
\begin{itemize}
        \item Products and ratios capture combined effects
\item Use domain knowledge
\item Validate with CV
      \end{itemize}
\subsection*{Aggregations: Key Points}
\begin{itemize}
        \item Per-user totals/means/counts
\item Rolling windows (last 7/30 days)
\item Avoid using future data
      \end{itemize}
\subsection*{Exercise 1: Interaction}
\small
  Give one interaction feature for house price.
\subsection*{Solution 1}
\begin{itemize}
    \item size\_m2 * location\_score (example).
  \end{itemize}
\subsection*{Exercise 2: Aggregation}
\small
  Name one per-user aggregation for churn prediction.
\subsection*{Solution 2}
\begin{itemize}
    \item days\_since\_last\_purchase (example).
  \end{itemize}
\subsection*{Exercise 3: Leakage}
\small
  Is using next-30-days spend to predict churn today leakage?
\subsection*{Solution 3}
\begin{itemize}
    \item Yes; it uses future info.
  \end{itemize}
\subsection*{Mini Demo (Python)}
Run from the lecture folder:
  \begin{center}
    \texttt{python demo/demo.py}
  \end{center}
  \vspace{0.4em}
  Outputs:
  \begin{itemize}
    \item \texttt{images/demo.png}
    \item \texttt{data/results.txt}
  \end{itemize}
\subsection*{Demo Output (Example)}
\begin{center}
  \IfFileExists{../images/demo.png}{
    \includegraphics[width=0.92\linewidth]{demo.png}
  }{
    \small (Run demo to generate: \texttt{demo.png})
  }
  \end{center}
\subsection*{Summary}
\begin{itemize}
        \item Key definitions and the main formula.
\item How to interpret results in context.
\item How the demo connects to the theory.
      \end{itemize}
\subsection*{Exit Question}
\small
  How does cross-validation help detect whether engineered features overfit?
% END SLIDE APPENDIX (AUTO-GENERATED)

\end{document}
