\documentclass[11pt]{article}
\usepackage[utf8]{inputenc}
\usepackage[T1]{fontenc}
\usepackage{geometry}
\usepackage{amsmath}
\usepackage{booktabs}
\usepackage{hyperref}
\geometry{margin=1in}

\title{Statistics and Data Analysis\\Unit 05 -- Lecture 04 Notes}
\author{Tofik Ali}
\date{\today}

\begin{document}
\maketitle

\section*{Topic}
Factor analysis (latent factors) and LDA (supervised separation).

\subsection*{Learning Outcomes}
\begin{itemize}
  \item Explain factor analysis as latent-factor modeling (intuition)
  \item Differentiate PCA vs factor analysis (goal/assumptions)
  \item Explain LDA as supervised dimensionality reduction/classifier
  \item Interpret a 2D LDA projection
\end{itemize}

\section*{Detailed Notes}
These notes are designed to be read alongside the slides. They expand each slide bullet into
plain-language explanations, small worked examples, and common pitfalls. When a formula
appears, emphasize (1) what each symbol means, (2) the assumptions needed to use it, and (3)
how to interpret the final number in the problem context.

\section*{Factor Analysis}
\begin{itemize}
  \item Observed variables driven by a few latent factors
  \item Goal: explain correlations via factors
  \item Used for surveys/constructs
\end{itemize}

\section*{LDA}
\begin{itemize}
  \item Supervised: uses labels
  \item Finds projection maximizing class separation
  \item Can classify and visualize
\end{itemize}

\section*{Exercises (with Solutions)}
\subsection*{Exercise 1: Supervised?}
Is PCA supervised? Is LDA supervised?
\subsection*{Solution}
\begin{itemize}
  \item PCA is unsupervised; LDA is supervised.
\end{itemize}

\subsection*{Exercise 2: Goal}
What does PCA optimize vs LDA (intuition)?
\subsection*{Solution}
\begin{itemize}
  \item PCA: variance captured; LDA: class separability.
\end{itemize}

\subsection*{Exercise 3: Use case}
Labeled A/B/C data, want 2D plot separating classes. PCA or LDA?
\subsection*{Solution}
\begin{itemize}
  \item LDA (uses labels for separation).
\end{itemize}

\section*{Exit Question}
Why can LDA separate classes better than PCA on labeled data?

\section*{Demo (Python)}
Run from the lecture folder:
\begin{verbatim}
python demo/demo.py
\end{verbatim}

Output files:
\begin{itemize}
  \item \texttt{images/demo.png}
  \item \texttt{data/results.txt}
\end{itemize}

\section*{References}
\begin{itemize}
  \item Montgomery, D. C., \& Runger, G. C. \textit{Applied Statistics and Probability for Engineers}, Wiley.
  \item Devore, J. L. \textit{Probability and Statistics for Engineering and the Sciences}, Cengage.
  \item McKinney, W. \textit{Python for Data Analysis}, O'Reilly.
\end{itemize}
\end{document}
