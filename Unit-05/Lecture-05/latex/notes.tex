\documentclass[11pt]{article}
\usepackage[utf8]{inputenc}
\usepackage[T1]{fontenc}
\usepackage{geometry}
\usepackage{amsmath}
\usepackage{listings}
\usepackage{xcolor}
\usepackage{amssymb}
\usepackage{booktabs}
\usepackage{graphicx}
\graphicspath{{../images/}}
\usepackage{hyperref}
\geometry{margin=1in}

\title{Statistics and Data Analysis\\Unit 05 -- Lecture 05 Notes\\Kernel PCA and t-SNE}
\author{Tofik Ali}
\date{\today}

\begin{document}
\maketitle

\section*{Topic}
Nonlinear dimensionality reduction: kernel PCA and t-SNE (visualization).
\section*{How to Use These Notes}
These notes are written for students who are seeing the topic for the first time. They
follow the slide order, but add the missing 'why', interpretation, and common mistakes. If
you get stuck, look at the worked exercises and then run the Python demo.

Course repository (slides, demos, datasets): \url{https://github.com/tali7c/Statistics-and-Data-Analysis}

\section*{Time Plan (55 minutes)}
\begin{itemize}
  \item 0--10 min: Attendance + recap of previous lecture
  \item 10--35 min: Core concepts (this lecture's sections)
  \item 35--45 min: Exercises (solve 1--2 in class, rest as practice)
  \item 45--50 min: Mini demo + interpretation of output
  \item 50--55 min: Buffer / wrap-up (leave 5 minutes early)
\end{itemize}

\section*{Slide-by-slide Notes}
\subsection*{Title Slide}
State the lecture title clearly and connect it to what students already know.
Tell students what they will be able to do by the end (not just what you will cover).

\subsection*{Quick Links / Agenda}
Explain the structure of the lecture and where the exercises and demo appear.
\begin{itemize}
  \item Overview
  \item Kernel PCA
  \item t-SNE
  \item Exercises
  \item Demo
  \item Summary
\end{itemize}

\subsection*{Learning Outcomes}
\begin{itemize}
  \item Explain why nonlinear methods are sometimes needed
  \item Describe kernel PCA idea (high level)
  \item Describe t-SNE purpose (visualization) and pitfalls
  \item Choose PCA vs t-SNE appropriately
\end{itemize}
\paragraph{Why these outcomes matter.}
\textbf{PCA} finds new axes (principal components) that capture maximum variance. It is a
rotation of the feature space. Because PCA is variance-based, it is sensitive to scaling:
standardize features first unless all features are already comparable.
\textbf{t-SNE} is mainly for visualization. It preserves local neighborhood structure but
can distort global distances. Do not treat t-SNE plots as proof of clusters; treat them as
exploratory pictures that need validation.

\subsection*{Kernel PCA: Key Points}
\begin{itemize}
  \item Implicitly map to higher-dimensional space via kernel
  \item Apply PCA in that space
  \item Captures nonlinear structure
\end{itemize}
\paragraph{Explanation.}
\textbf{PCA} finds new axes (principal components) that capture maximum variance. It is a
rotation of the feature space. Because PCA is variance-based, it is sensitive to scaling:
standardize features first unless all features are already comparable.

\subsection*{t-SNE: Key Points}
\begin{itemize}
  \item Mainly for 2D/3D visualization
  \item Preserves local neighborhoods
  \item Global distances can be misleading
\end{itemize}
\paragraph{Explanation.}
\textbf{t-SNE} is mainly for visualization. It preserves local neighborhood structure but
can distort global distances. Do not treat t-SNE plots as proof of clusters; treat them as
exploratory pictures that need validation.

\subsection*{Exercises (with Solutions)}
Attempt the exercise first, then compare with the solution. Focus on interpretation, not
only arithmetic.

\subsection*{Exercise 1: Use case}
Name one warning when interpreting t-SNE plots.
\subsubsection*{Solution}
\begin{itemize}
  \item Global distances and cluster sizes can be misleading.
\end{itemize}
\paragraph{Walkthrough.}
\textbf{t-SNE} is mainly for visualization. It preserves local neighborhood structure but
can distort global distances. Do not treat t-SNE plots as proof of clusters; treat them as
exploratory pictures that need validation.

\subsection*{Exercise 2: Randomness}
What should you do if t-SNE changes across runs?
\subsubsection*{Solution}
\begin{itemize}
  \item Set seed and check stability.
\end{itemize}
\paragraph{Walkthrough.}
\textbf{t-SNE} is mainly for visualization. It preserves local neighborhood structure but
can distort global distances. Do not treat t-SNE plots as proof of clusters; treat them as
exploratory pictures that need validation.

\subsection*{Exercise 3: Kernel PCA benefit}
Why kernel PCA can help on circular data?
\subsubsection*{Solution}
\begin{itemize}
  \item It can capture nonlinear manifold structure.
\end{itemize}
\paragraph{Walkthrough.}
\textbf{PCA} finds new axes (principal components) that capture maximum variance. It is a
rotation of the feature space. Because PCA is variance-based, it is sensitive to scaling:
standardize features first unless all features are already comparable.

\subsection*{Mini Demo (Python)}
Run from the lecture folder:
\begin{verbatim}
python demo/demo.py
\end{verbatim}

Output files:
\begin{itemize}
  \item \texttt{images/demo.png}
  \item \texttt{data/results.txt}
\end{itemize}
\paragraph{What to show and say.}
\begin{itemize}
  \item Creates a nonlinear dataset and compares PCA-like vs nonlinear embeddings.
  \item Produces a 2D visualization to discuss neighborhood preservation.
  \item Use it to warn that t-SNE is mainly for visualization, not modeling features.
\end{itemize}

\subsection*{Demo Output (Example)}
\begin{center}
\IfFileExists{../images/demo.png}{
  \includegraphics[width=0.95\linewidth]{../images/demo.png}
}{
  \small (Run the demo to generate \texttt{images/demo.png})
}
\end{center}

\subsection*{Summary}
\begin{itemize}
  \item Key definitions and the main formula.
  \item How to interpret results in context.
  \item How the demo connects to the theory.
\end{itemize}

\subsection*{Exit Question}
Why should we avoid using t-SNE coordinates directly as model features (usually)?
\paragraph{Suggested answer (for revision).}
t-SNE coordinates are not stable/global-metric features and can distort distances; it is
primarily a visualization tool, not a feature generator.

\section*{References}
\begin{itemize}
  \item Montgomery, D. C., \& Runger, G. C. \textit{Applied Statistics and Probability for Engineers}, Wiley.
  \item Devore, J. L. \textit{Probability and Statistics for Engineering and the Sciences}, Cengage.
  \item McKinney, W. \textit{Python for Data Analysis}, O'Reilly.
\end{itemize}

% BEGIN SLIDE APPENDIX (AUTO-GENERATED)
\clearpage
\section*{Appendix: Slide Deck Content (Reference)}
\noindent The material below is a reference copy of the slide deck content. Exercise solutions are explained in the main notes where applicable.

\subsection*{Title Slide}
\titlepage
        \vspace{-0.5em}
        \begin{center}
          \small \texttt{https://github.com/tali7c/Statistics-and-Data-Analysis}
        \end{center}
\subsection*{Quick Links}
\centering
        \textbf{Overview}\hspace{0.6em}
\textbf{Kernel PCA}\hspace{0.6em}
\textbf{t-SNE}\hspace{0.6em}
\textbf{Exercises}\hspace{0.6em}
\textbf{Demo}\hspace{0.6em}
\textbf{Summary}\hspace{0.6em}
\subsection*{Agenda}
\begin{itemize}
  \item Overview
  \item Kernel PCA
  \item t-SNE
  \item Exercises
  \item Demo
  \item Summary
\end{itemize}
\subsection*{Learning Outcomes}
\begin{itemize}
        \item Explain why nonlinear methods are sometimes needed
\item Describe kernel PCA idea (high level)
\item Describe t-SNE purpose (visualization) and pitfalls
\item Choose PCA vs t-SNE appropriately
      \end{itemize}
\subsection*{Kernel PCA: Key Points}
\begin{itemize}
        \item Implicitly map to higher-dimensional space via kernel
\item Apply PCA in that space
\item Captures nonlinear structure
      \end{itemize}
\subsection*{t-SNE: Key Points}
\begin{itemize}
        \item Mainly for 2D/3D visualization
\item Preserves local neighborhoods
\item Global distances can be misleading
      \end{itemize}
\subsection*{Exercise 1: Use case}
\small
  Name one warning when interpreting t-SNE plots.
\subsection*{Solution 1}
\begin{itemize}
    \item Global distances and cluster sizes can be misleading.
  \end{itemize}
\subsection*{Exercise 2: Randomness}
\small
  What should you do if t-SNE changes across runs?
\subsection*{Solution 2}
\begin{itemize}
    \item Set seed and check stability.
  \end{itemize}
\subsection*{Exercise 3: Kernel PCA benefit}
\small
  Why kernel PCA can help on circular data?
\subsection*{Solution 3}
\begin{itemize}
    \item It can capture nonlinear manifold structure.
  \end{itemize}
\subsection*{Mini Demo (Python)}
Run from the lecture folder:
  \begin{center}
    \texttt{python demo/demo.py}
  \end{center}
  \vspace{0.4em}
  Outputs:
  \begin{itemize}
    \item \texttt{images/demo.png}
    \item \texttt{data/results.txt}
  \end{itemize}
\subsection*{Demo Output (Example)}
\begin{center}
  \IfFileExists{../images/demo.png}{
    \includegraphics[width=0.92\linewidth]{demo.png}
  }{
    \small (Run demo to generate: \texttt{demo.png})
  }
  \end{center}
\subsection*{Summary}
\begin{itemize}
        \item Key definitions and the main formula.
\item How to interpret results in context.
\item How the demo connects to the theory.
      \end{itemize}
\subsection*{Exit Question}
\small
  Why should we avoid using t-SNE coordinates directly as model features (usually)?
% END SLIDE APPENDIX (AUTO-GENERATED)

\end{document}
