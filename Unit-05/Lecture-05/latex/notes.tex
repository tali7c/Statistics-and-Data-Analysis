\documentclass[11pt]{article}
\usepackage[utf8]{inputenc}
\usepackage[T1]{fontenc}
\usepackage{geometry}
\usepackage{amsmath}
\usepackage{booktabs}
\usepackage{hyperref}
\geometry{margin=1in}

\title{Statistics and Data Analysis\\Unit 05 -- Lecture 05 Notes}
\author{Tofik Ali}
\date{\today}

\begin{document}
\maketitle

\section*{Topic}
Nonlinear dimensionality reduction: kernel PCA and t-SNE (visualization).

\subsection*{Learning Outcomes}
\begin{itemize}
  \item Explain why nonlinear methods are sometimes needed
  \item Describe kernel PCA idea (high level)
  \item Describe t-SNE purpose (visualization) and pitfalls
  \item Choose PCA vs t-SNE appropriately
\end{itemize}

\section*{Detailed Notes}
These notes are designed to be read alongside the slides. They expand each slide bullet into
plain-language explanations, small worked examples, and common pitfalls. When a formula
appears, emphasize (1) what each symbol means, (2) the assumptions needed to use it, and (3)
how to interpret the final number in the problem context.

\section*{Kernel PCA}
\begin{itemize}
  \item Implicitly map to higher-dimensional space via kernel
  \item Apply PCA in that space
  \item Captures nonlinear structure
\end{itemize}

\section*{t-SNE}
\begin{itemize}
  \item Mainly for 2D/3D visualization
  \item Preserves local neighborhoods
  \item Global distances can be misleading
\end{itemize}

\section*{Exercises (with Solutions)}
\subsection*{Exercise 1: Use case}
Name one warning when interpreting t-SNE plots.
\subsection*{Solution}
\begin{itemize}
  \item Global distances and cluster sizes can be misleading.
\end{itemize}

\subsection*{Exercise 2: Randomness}
What should you do if t-SNE changes across runs?
\subsection*{Solution}
\begin{itemize}
  \item Set seed and check stability.
\end{itemize}

\subsection*{Exercise 3: Kernel PCA benefit}
Why kernel PCA can help on circular data?
\subsection*{Solution}
\begin{itemize}
  \item It can capture nonlinear manifold structure.
\end{itemize}

\section*{Exit Question}
Why should we avoid using t-SNE coordinates directly as model features (usually)?

\section*{Demo (Python)}
Run from the lecture folder:
\begin{verbatim}
python demo/demo.py
\end{verbatim}

Output files:
\begin{itemize}
  \item \texttt{images/demo.png}
  \item \texttt{data/results.txt}
\end{itemize}

\section*{References}
\begin{itemize}
  \item Montgomery, D. C., \& Runger, G. C. \textit{Applied Statistics and Probability for Engineers}, Wiley.
  \item Devore, J. L. \textit{Probability and Statistics for Engineering and the Sciences}, Cengage.
  \item McKinney, W. \textit{Python for Data Analysis}, O'Reilly.
\end{itemize}
\end{document}
