\documentclass{beamer}

\usetheme{Berlin}
\usecolortheme{Orchid}
\useoutertheme{miniframes}
\setbeamertemplate{navigation symbols}{}

\usepackage{amsmath}
\usepackage{booktabs}
\usepackage{graphicx}
\graphicspath{{../images/}}

\title[Statistics and Data Analysis]{Statistics and Data Analysis}
\subtitle{Unit 01 -- Lecture 01: Data Types, Sources, and Cleaning Basics}
\author{Tofik Ali}
\institute{School of Computer Science, UPES Dehradun}
\date{\today}

\begin{document}

\begin{frame}
  \titlepage
  \vspace{-0.5em}
  \begin{center}
    \small \texttt{https://github.com/tali7c/Statistics-and-Data-Analysis}
  \end{center}
\end{frame}

\begin{frame}{Quick Links}
  \centering
  \hyperlink{sec:types}{\beamerbutton{Types \& Formats}}\hspace{0.6em}
  \hyperlink{sec:sources}{\beamerbutton{Sources}}\hspace{0.6em}
  \hyperlink{sec:cleaning}{\beamerbutton{Cleaning}}\hspace{0.6em}
  \hyperlink{sec:demo}{\beamerbutton{Demo}}\hspace{0.6em}
  \hyperlink{sec:summary}{\beamerbutton{Summary}}
\end{frame}

\begin{frame}{Agenda}
  \tableofcontents
\end{frame}

\section{Overview}

\begin{frame}{Learning Outcomes}
  \begin{itemize}[<+->]
    \item Identify common data types and formats used in analytics
    \item List common data sources and acquisition methods
    \item Detect typical data quality issues (missing values, duplicates, outliers)
    \item Apply basic cleaning steps in Python and save a cleaned dataset
  \end{itemize}
\end{frame}

\section{Data Types and Formats}
\label{sec:types}

\begin{frame}{Dataset, Observation, Variable}
  \begin{itemize}[<+->]
    \item \textbf{Dataset:} a collection of observations (rows) and variables (columns)
    \item \textbf{Observation:} one record (e.g., one student)
    \item \textbf{Variable/Feature:} one attribute (e.g., attendance, CGPA)
  \end{itemize}
  \vspace{0.4em}
  \textbf{Goal:} convert raw data into a form suitable for analysis and modeling.
\end{frame}

\begin{frame}{Common Data Types (Practical View)}
  \begin{itemize}[<+->]
    \item \textbf{Numeric:} integers (count), real values (measurements)
    \item \textbf{Categorical:} nominal (branch), ordinal (rating: low/med/high)
    \item \textbf{Binary:} yes/no, pass/fail
    \item \textbf{Date/Time:} join date, timestamp
    \item \textbf{Text:} feedback, comments
  \end{itemize}
\end{frame}

\begin{frame}{Exercise 1: Classify Variable Types}
  \small
  For each variable, write the type (numeric/categorical/binary/datetime/text):
  \begin{enumerate}
    \item Age
    \item Program/Branch (CSE, ECE, \ldots)
    \item Attendance (\%)
    \item Join date
    \item Feedback comment
  \end{enumerate}
\end{frame}

\begin{frame}{Solution 1}
  \begin{itemize}
    \item Age: numeric (integer)
    \item Program/Branch: categorical (nominal)
    \item Attendance (\%): numeric (real)
    \item Join date: datetime
    \item Feedback: text (unstructured)
  \end{itemize}
\end{frame}

\begin{frame}{Data Formats}
  \begin{itemize}[<+->]
    \item \textbf{Structured:} fixed schema (tables) \\
      \hspace{1.2em}Examples: CSV, SQL tables
    \item \textbf{Semi-structured:} flexible schema with tags/keys \\
      \hspace{1.2em}Examples: JSON, XML
    \item \textbf{Unstructured:} free-form content \\
      \hspace{1.2em}Examples: text documents, images, audio
  \end{itemize}
\end{frame}

\begin{frame}{Structured Example (Table)}
  \small
  \begin{center}
    \begin{tabular}{lccc}
      \toprule
      student\_id & program & attendance\_pct & cgpa \\
      \midrule
      1001 & CSE & 92 & 8.2 \\
      1002 & CSE & 85 & 7.5 \\
      1003 & ECE & 105 & 8.9 \\
      \bottomrule
    \end{tabular}
  \end{center}
  \vspace{0.4em}
  \normalsize
  \textbf{Note:} 105\% attendance is an example of an out-of-range value.
\end{frame}

\begin{frame}[fragile]{Semi-structured Example (JSON)}
\small
\begin{verbatim}
{
  "student_id": 1001,
  "program": "CSE",
  "attendance_pct": 92,
  "courses": ["Math", "DSA", "Stats"]
}
\end{verbatim}
\normalsize
Keys may vary from record to record (flexible schema).
\end{frame}

\begin{frame}[fragile]{Unstructured Example (Text/Log)}
\small
\begin{verbatim}
2026-02-08 10:02:11 INFO login user=1007 device=android city=Delhi
\end{verbatim}
\normalsize
Useful information exists, but it requires parsing and feature extraction.
\end{frame}

\section{Data Sources and Acquisition}
\label{sec:sources}

\begin{frame}{Common Data Sources}
  \begin{itemize}[<+->]
    \item Surveys and forms (Google Forms, LMS exports)
    \item Databases (student records, attendance systems)
    \item Web and app logs (clickstream)
    \item Sensors/IoT (temperature, GPS)
    \item Public datasets (government portals, research repositories)
  \end{itemize}
\end{frame}

\begin{frame}{Acquisition Methods}
  \begin{itemize}[<+->]
    \item Files: CSV/Excel export $\rightarrow$ \texttt{read\_csv}, \texttt{read\_excel}
    \item Database query: SQL $\rightarrow$ extract tables
    \item API calls: JSON responses $\rightarrow$ parse and store
    \item Manual entry: small datasets (careful with errors)
  \end{itemize}
\end{frame}

\begin{frame}{Exercise 2: Choose a Source}
  \small
  For each case, suggest a likely source (survey/database/log/API):
  \begin{enumerate}
    \item Daily attendance of students
    \item Online learning platform clicks
    \item Student feedback comments
    \item Weather readings every minute
  \end{enumerate}
\end{frame}

\begin{frame}{Solution 2}
  \begin{itemize}
    \item Attendance: database export (or CSV from attendance system)
    \item Clicks: logs (web/app logs)
    \item Feedback: survey + text field (unstructured text)
    \item Weather readings: sensors/IoT or API
  \end{itemize}
\end{frame}

\section{Data Cleaning}
\label{sec:cleaning}

\begin{frame}{Why Cleaning Matters}
  \begin{itemize}[<+->]
    \item Models and statistics assume data is meaningful and consistent
    \item ``Garbage in, garbage out'' $\rightarrow$ wrong conclusions
    \item Cleaning improves: accuracy, fairness, and reproducibility
  \end{itemize}
\end{frame}

\begin{frame}{Common Data Quality Issues}
  \begin{itemize}[<+->]
    \item Missing values (blank, NaN, NULL)
    \item Duplicates (same record repeated)
    \item Inconsistent categories (\texttt{cse}, \texttt{CSE}, \texttt{ CSE})
    \item Out-of-range values (attendance 105\%, CGPA 12)
    \item Wrong data type (``nine'' instead of 9.0)
  \end{itemize}
\end{frame}

\begin{frame}{Handling Missing Values (Basic Options)}
  \begin{itemize}[<+->]
    \item \textbf{Drop:} remove rows/columns (only if few missing and safe)
    \item \textbf{Impute:} fill with mean/median/mode (simple baseline)
    \item \textbf{Domain rule:} fill with a meaningful default (carefully)
    \item \textbf{Flag:} create an indicator feature ``was\_missing''
  \end{itemize}
\end{frame}

\begin{frame}{Exercise 3: Missingness Decision}
  \small
  In a dataset of 20 students, the column \texttt{cgpa} has 2 missing values.
  \begin{itemize}
    \item What is the missingness percentage?
    \item Suggest one reasonable action for this column.
  \end{itemize}
\end{frame}

\begin{frame}{Solution 3}
  \begin{itemize}
    \item Missingness = $2/20 \times 100\% = 10\%$
    \item Action: impute using \textbf{median} CGPA (robust) and optionally add a flag
  \end{itemize}
\end{frame}

\begin{frame}{Outliers (Basic Idea)}
  An outlier is a value that is unusually far from typical values.
  \vspace{0.6em}
  \begin{itemize}[<+->]
    \item Outliers can be \textbf{errors} (wrong entry) or \textbf{real extremes}
    \item They can strongly affect mean, variance, and some models
    \item Use rules like IQR fences as a \textbf{screening} step
  \end{itemize}
\end{frame}

\begin{frame}{IQR Rule (Fences)}
  \[
    \text{Lower fence} = Q_1 - 1.5\times \mathrm{IQR},\quad
    \text{Upper fence} = Q_3 + 1.5\times \mathrm{IQR}
  \]
  \[
    \mathrm{IQR} = Q_3 - Q_1
  \]
  Values outside fences are \textit{possible} outliers.
\end{frame}

\begin{frame}{Exercise 4: IQR Outlier Check}
  \small
  Attendance (\%): 70, 75, 80, 85, 90, 95, 150 \\
  \vspace{0.4em}
  \normalsize
  \textbf{Task:} Compute $Q_1$, $Q_3$, IQR, fences, and decide if 150 is an outlier.
\end{frame}

\begin{frame}{Solution 4}
  \small
  Sorted data: 70, 75, 80, 85, 90, 95, 150 (n=7). Median = 85. \\
  Lower half: 70, 75, 80 $\Rightarrow Q_1=75$ \\
  Upper half: 90, 95, 150 $\Rightarrow Q_3=95$ \\
  IQR = $95-75=20$ \\
  Fences: $75-30=45$ and $95+30=125$ \\
  \normalsize
  \textbf{Conclusion:} 150 $>$ 125 $\Rightarrow$ outlier (by IQR rule).
\end{frame}

\begin{frame}{Cleaning Checklist (Fast)}
  \begin{itemize}[<+->]
    \item Check shape, column names, and data types
    \item Check missingness and duplicates
    \item Standardize categories (trim whitespace, normalize case)
    \item Check ranges and impossible values
    \item Save a cleaned version (do not overwrite raw file)
  \end{itemize}
\end{frame}

\section{Demo}
\label{sec:demo}

\begin{frame}{Mini Demo (Python)}
  Run from the lecture folder:
  \begin{center}
    \texttt{python demo/cleaning\_demo.py}
  \end{center}
  \vspace{0.4em}
  Outputs:
  \begin{itemize}
    \item \texttt{data/students\_clean.csv}
    \item plots in \texttt{images/} (missingness and outlier visual)
  \end{itemize}
\end{frame}

\begin{frame}{Demo Output (Example)}
  \begin{columns}[T]
    \begin{column}{0.49\textwidth}
      \textbf{Missingness}
      \vspace{0.2em}
      \begin{center}
      \IfFileExists{../images/missingness_before.png}{
        \includegraphics[width=\linewidth]{missingness_before.png}
      }{
        \small (Run the demo to generate: \texttt{missingness\_before.png})
      }
      \end{center}
    \end{column}
    \begin{column}{0.49\textwidth}
      \textbf{Attendance Outliers}
      \vspace{0.2em}
      \begin{center}
      \IfFileExists{../images/attendance_box_before.png}{
        \includegraphics[width=\linewidth]{attendance_box_before.png}
      }{
        \small (Run the demo to generate: \texttt{attendance\_box\_before.png})
      }
      \end{center}
    \end{column}
  \end{columns}
\end{frame}

\section{Summary}
\label{sec:summary}

\begin{frame}{Summary}
  \begin{itemize}[<+->]
    \item Data types and formats determine how we store and process data
    \item Different sources require different acquisition and validation steps
    \item Cleaning deals with missing values, duplicates, inconsistencies, and outliers
    \item Always save a cleaned dataset and document the rules you applied
  \end{itemize}
  \vspace{0.6em}
  \textbf{Exit question:} In one sentence, why can ``attendance 105\%'' be dangerous in analysis?
\end{frame}

\end{document}
