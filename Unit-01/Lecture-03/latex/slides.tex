\documentclass{beamer}

\usetheme{Berlin}
\usecolortheme{Orchid}
\useoutertheme{miniframes}
\setbeamertemplate{navigation symbols}{}

\usepackage{amsmath}
\usepackage{booktabs}
\usepackage{graphicx}
\graphicspath{{../images/}}

\title[Statistics and Data Analysis]{Statistics and Data Analysis}
\subtitle{Unit 01 -- Lecture 03: Preprocessing Pipelines and Exploratory Data Analysis (EDA)}
\author{Tofik Ali}
\institute{School of Computer Science, UPES Dehradun}
\date{\today}

\begin{document}

\begin{frame}
  \titlepage
  \vspace{-0.5em}
  \begin{center}
    \small \texttt{https://github.com/tali7c/Statistics-and-Data-Analysis}
  \end{center}
\end{frame}

\begin{frame}{Quick Links}
  \centering
  \hyperlink{sec:workflow}{\beamerbutton{Workflow}}\hspace{0.6em}
  \hyperlink{sec:eda}{\beamerbutton{EDA Checklist}}\hspace{0.6em}
  \hyperlink{sec:visuals}{\beamerbutton{Plots}}\hspace{0.6em}
  \hyperlink{sec:demo}{\beamerbutton{Demo}}\hspace{0.6em}
  \hyperlink{sec:summary}{\beamerbutton{Summary}}
\end{frame}

\begin{frame}{Agenda}
  \tableofcontents
\end{frame}

\section{Overview}

\begin{frame}{Learning Outcomes}
  \begin{itemize}[<+->]
    \item Explain what a preprocessing pipeline is and why it matters
    \item Apply a simple end-to-end workflow: load $\rightarrow$ clean $\rightarrow$ validate $\rightarrow$ summarize
    \item Perform basic EDA: missingness, summary stats, group summaries, correlations
    \item Choose appropriate plots for numeric and categorical variables
  \end{itemize}
\end{frame}

\section{Workflow and Pipelines}
\label{sec:workflow}

\begin{frame}{What is a Pipeline?}
  A pipeline is an ordered set of steps applied consistently to data.
  \vspace{0.6em}
  \begin{itemize}[<+->]
    \item Makes analysis \textbf{reproducible} (same input $\Rightarrow$ same output)
    \item Reduces mistakes (steps are documented and repeatable)
    \item Helps avoid \textbf{data leakage} (train/test separation)
  \end{itemize}
\end{frame}

\begin{frame}{Typical End-to-End Workflow (Practical)}
  \begin{enumerate}[<+->]
    \item Understand the question (what do you want to learn/decide?)
    \item Acquire data (files, DB, API)
    \item Inspect: shape, dtypes, missingness
    \item Clean: duplicates, invalid ranges, inconsistent categories
    \item Validate: check constraints (0--100\%, 0--10 CGPA, etc.)
    \item EDA: summary tables + plots + simple relationships
    \item Save outputs (cleaned dataset, plots, summary tables)
  \end{enumerate}
\end{frame}

\begin{frame}[fragile]{Example: ``Pipeline'' in Code (Concept)}
\small
\begin{verbatim}
df = read_raw()
df = clean_strings(df)
df = coerce_types(df)
df = range_check(df)
df = impute_missing(df)
save_clean(df)
eda_report(df)
\end{verbatim}
\normalsize
This is a simple pipeline: each step has a clear purpose.
\end{frame}

\begin{frame}{Exercise 1: Put Steps in Order}
  \small
  Arrange these steps in a reasonable order:
  \begin{enumerate}
    \item EDA plots
    \item Load raw data
    \item Fix data types + invalid ranges
    \item Save cleaned dataset
    \item Check missingness
  \end{enumerate}
\end{frame}

\begin{frame}{Solution 1}
  One reasonable order:
  \begin{enumerate}
    \item Load raw data
    \item Check missingness
    \item Fix data types + invalid ranges
    \item EDA plots
    \item Save cleaned dataset
  \end{enumerate}
\end{frame}

\section{EDA Checklist}
\label{sec:eda}

\begin{frame}{What is EDA?}
  Exploratory Data Analysis (EDA) is the first structured look at your data.
  \vspace{0.6em}
  \begin{itemize}[<+->]
    \item Understand distribution (shape, spread, outliers)
    \item Understand relationships (scatter plots, correlation)
    \item Compare groups (e.g., program-wise summaries)
    \item Identify issues early (missingness, strange values)
  \end{itemize}
\end{frame}

\begin{frame}{EDA Checklist (Minimum)}
  \begin{itemize}[<+->]
    \item \textbf{Data quality:} missingness \%, duplicates, invalid ranges
    \item \textbf{Univariate:} histograms/boxplots for numeric; bar charts for categorical
    \item \textbf{Bivariate:} scatter plot for numeric--numeric; boxplot for numeric by category
    \item \textbf{Multivariate (basic):} correlation matrix/heatmap for numeric columns
    \item \textbf{Group summaries:} mean/median/std by program or gender
  \end{itemize}
\end{frame}

\section{Plots}
\label{sec:visuals}

\begin{frame}{Plot Selection (Quick Guide)}
  \begin{itemize}[<+->]
    \item Numeric (one variable): histogram, boxplot
    \item Categorical (one variable): bar chart (counts)
    \item Numeric vs numeric: scatter plot
    \item Numeric vs categorical: boxplot (numeric grouped by category)
    \item Many numeric features: correlation heatmap
  \end{itemize}
\end{frame}

\begin{frame}{Exercise 2: Choose the Plot}
  \small
  Pick a good plot for each:
  \begin{enumerate}
    \item Distribution of \texttt{final\_marks} (numeric)
    \item Compare \texttt{final\_marks} across \texttt{program} (categorical)
    \item Relationship between \texttt{study\_hours\_week} and \texttt{final\_marks}
  \end{enumerate}
\end{frame}

\begin{frame}{Solution 2}
  \begin{itemize}
    \item (1) Histogram or boxplot
    \item (2) Boxplot of marks grouped by program
    \item (3) Scatter plot (hours vs marks)
  \end{itemize}
\end{frame}

\begin{frame}{Exercise 3: Spot Data Leakage}
  \small
  A student computes mean/std for scaling using the \textbf{entire dataset},
  then splits into train/test and trains a model.\\
  \vspace{0.4em}
  \normalsize
  \textbf{Question:} Is this correct? If not, what should be done instead?
\end{frame}

\begin{frame}{Solution 3}
  Not correct: it uses test information during training (\textbf{leakage}).\\
  Correct approach:
  \begin{itemize}
    \item split into train/test first
    \item compute scaling parameters on \textbf{train only}
    \item apply the same parameters to test
  \end{itemize}
\end{frame}

\section{Demo}
\label{sec:demo}

\begin{frame}{Mini Demo (Python)}
  Run from the lecture folder:
  \begin{center}
    \texttt{python demo/pipeline\_eda\_demo.py}
  \end{center}
  \vspace{0.4em}
  Outputs:
  \begin{itemize}
    \item \texttt{data/case\_study\_clean.csv}
    \item \texttt{data/summary\_by\_program.csv}
    \item \texttt{data/corr\_matrix.csv}
    \item plots in \texttt{images/} (histogram, boxplot, scatter, heatmap)
  \end{itemize}
\end{frame}

\begin{frame}{Demo Output (Example)}
  \begin{columns}[T]
    \begin{column}{0.49\textwidth}
      \textbf{Histogram}
      \begin{center}
      \IfFileExists{../images/final_marks_hist.png}{
        \includegraphics[width=\linewidth]{final_marks_hist.png}
      }{
        \small (Run demo to generate: \texttt{final\_marks\_hist.png})
      }
      \end{center}
    \end{column}
    \begin{column}{0.49\textwidth}
      \textbf{Correlation Heatmap}
      \begin{center}
      \IfFileExists{../images/corr_heatmap.png}{
        \includegraphics[width=\linewidth]{corr_heatmap.png}
      }{
        \small (Run demo to generate: \texttt{corr\_heatmap.png})
      }
      \end{center}
    \end{column}
  \end{columns}
\end{frame}

\section{Summary}
\label{sec:summary}

\begin{frame}{Summary}
  \begin{itemize}[<+->]
    \item Pipelines make preprocessing repeatable and reduce mistakes
    \item EDA is about understanding quality, distributions, and relationships
    \item Pick plots based on variable types (numeric vs categorical)
    \item Save cleaned data, plots, and summary tables as reusable artifacts
  \end{itemize}
  \vspace{0.6em}
  \textbf{Exit question:} Name two checks you must do before trusting a dataset for analysis.
\end{frame}

\end{document}
