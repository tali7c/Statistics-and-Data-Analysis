\documentclass[11pt]{article}
\usepackage[utf8]{inputenc}
\usepackage[T1]{fontenc}
\usepackage{geometry}
\usepackage{amsmath}
\usepackage{listings}
\usepackage{xcolor}
\usepackage{amssymb}
\usepackage{booktabs}
\usepackage{graphicx}
\graphicspath{{../images/}}
\usepackage{hyperref}
\geometry{margin=1in}

\title{Statistics and Data Analysis\\Unit 06 -- Lecture 05 Notes\\Stationarity and Non-stationarity}
\author{Tofik Ali}
\date{\today}

\begin{document}
\maketitle

\section*{Topic}
Stationarity concept; why it matters; fixes like differencing.
\section*{How to Use These Notes}
These notes are written for students who are seeing the topic for the first time. They
follow the slide order, but add the missing 'why', interpretation, and common mistakes. If
you get stuck, look at the worked exercises and then run the Python demo.

Course repository (slides, demos, datasets): \url{https://github.com/tali7c/Statistics-and-Data-Analysis}

\section*{Time Plan (55 minutes)}
\begin{itemize}
  \item 0--10 min: Attendance + recap of previous lecture
  \item 10--35 min: Core concepts (this lecture's sections)
  \item 35--45 min: Exercises (solve 1--2 in class, rest as practice)
  \item 45--50 min: Mini demo + interpretation of output
  \item 50--55 min: Buffer / wrap-up (leave 5 minutes early)
\end{itemize}

\section*{Slide-by-slide Notes}
\subsection*{Title Slide}
State the lecture title clearly and connect it to what students already know.
Tell students what they will be able to do by the end (not just what you will cover).

\subsection*{Quick Links / Agenda}
Explain the structure of the lecture and where the exercises and demo appear.
\begin{itemize}
  \item Overview
  \item Stationarity
  \item Fixes
  \item Exercises
  \item Demo
  \item Summary
\end{itemize}

\subsection*{Learning Outcomes}
\begin{itemize}
  \item Define stationarity (intuition)
  \item Recognize non-stationary patterns (trend/seasonality)
  \item Explain why stationarity matters for ARIMA-type models
  \item List basic fixes (differencing, transforms)
\end{itemize}
\paragraph{Why these outcomes matter.}
\textbf{Trend} is a long-term upward or downward movement. Trend changes the mean over time,
which often creates non-stationarity. Many forecasting models handle trend by differencing
or by explicitly modeling trend.
\textbf{Seasonality} is a repeating pattern with a fixed period (weekly, monthly, yearly).
You must account for it; otherwise forecasts systematically miss repeating rises/falls.
Seasonal differencing and SARIMA are common tools.

\subsection*{Stationarity: Key Points}
\begin{itemize}
  \item Mean/variance roughly constant
  \item Autocorrelation depends on lag only
  \item Trend/seasonality often implies non-stationarity
\end{itemize}
\paragraph{Explanation.}
\textbf{Correlation} measures the strength of a linear association between two variables. It
is symmetric (no X/Y direction) and does not imply causation. Outliers can inflate or hide
correlation, so always look at the scatter plot.
\textbf{Trend} is a long-term upward or downward movement. Trend changes the mean over time,
which often creates non-stationarity. Many forecasting models handle trend by differencing
or by explicitly modeling trend.
\textbf{Seasonality} is a repeating pattern with a fixed period (weekly, monthly, yearly).
You must account for it; otherwise forecasts systematically miss repeating rises/falls.
Seasonal differencing and SARIMA are common tools.

\subsection*{Fixes: Key Points}
\begin{itemize}
  \item Differencing removes trend
  \item Seasonal differencing removes seasonality
  \item Log transform can stabilize variance
\end{itemize}
\paragraph{Explanation.}
\textbf{Trend} is a long-term upward or downward movement. Trend changes the mean over time,
which often creates non-stationarity. Many forecasting models handle trend by differencing
or by explicitly modeling trend.
\textbf{Seasonality} is a repeating pattern with a fixed period (weekly, monthly, yearly).
You must account for it; otherwise forecasts systematically miss repeating rises/falls.
Seasonal differencing and SARIMA are common tools.
\textbf{Differencing} transforms a series by subtracting the previous value: $y_t -
y_{t-1}$. It removes trend and can help achieve stationarity. Over-differencing can add
noise, so use the smallest differencing order that works.

\subsection*{Exercises (with Solutions)}
Attempt the exercise first, then compare with the solution. Focus on interpretation, not
only arithmetic.

\subsection*{Exercise 1: Trend}
Is a strong upward trend likely stationary?
\subsubsection*{Solution}
\begin{itemize}
  \item No; mean changes over time.
\end{itemize}
\paragraph{Walkthrough.}
\textbf{Trend} is a long-term upward or downward movement. Trend changes the mean over time,
which often creates non-stationarity. Many forecasting models handle trend by differencing
or by explicitly modeling trend.
\textbf{Stationarity} (intuition) means the series behavior is stable over time: roughly
constant mean/variance and correlation structure. AR/MA/ARIMA models assume stationarity
(after differencing). If the process changes over time, parameters learned from the past may
not hold.

\subsection*{Exercise 2: Variance change}
If fluctuations grow over time, is variance constant?
\subsubsection*{Solution}
\begin{itemize}
  \item No; non-stationary variance.
\end{itemize}
\paragraph{Walkthrough.}
\textbf{Stationarity} (intuition) means the series behavior is stable over time: roughly
constant mean/variance and correlation structure. AR/MA/ARIMA models assume stationarity
(after differencing). If the process changes over time, parameters learned from the past may
not hold.

\subsection*{Exercise 3: Fix choice}
Name one fix for non-stationary mean.
\subsubsection*{Solution}
\begin{itemize}
  \item Differencing.
\end{itemize}
\paragraph{Walkthrough.}
\textbf{Differencing} transforms a series by subtracting the previous value: $y_t -
y_{t-1}$. It removes trend and can help achieve stationarity. Over-differencing can add
noise, so use the smallest differencing order that works.
\textbf{Stationarity} (intuition) means the series behavior is stable over time: roughly
constant mean/variance and correlation structure. AR/MA/ARIMA models assume stationarity
(after differencing). If the process changes over time, parameters learned from the past may
not hold.

\subsection*{Mini Demo (Python)}
Run from the lecture folder:
\begin{verbatim}
python demo/demo.py
\end{verbatim}

Output files:
\begin{itemize}
  \item \texttt{images/demo.png}
  \item \texttt{data/results.txt}
\end{itemize}
\paragraph{What to show and say.}
\begin{itemize}
  \item Generates non-stationary vs stationary examples and plots them.
  \item Demonstrates differencing / log transform as simple fixes.
  \item Use it to explain why ARIMA-type models assume stationarity after transforms.
\end{itemize}

\subsection*{Demo Output (Example)}
\begin{center}
\IfFileExists{../images/demo.png}{
  \includegraphics[width=0.95\linewidth]{../images/demo.png}
}{
  \small (Run the demo to generate \texttt{images/demo.png})
}
\end{center}

\subsection*{Summary}
\begin{itemize}
  \item Key definitions and the main formula.
  \item How to interpret results in context.
  \item How the demo connects to the theory.
\end{itemize}

\subsection*{Exit Question}
Why does non-stationarity make forecasting harder?
\paragraph{Suggested answer (for revision).}
If mean/variance change over time, patterns learned from the past may not hold; stationarity
makes modeling and forecasting more reliable.

\section*{References}
\begin{itemize}
  \item Montgomery, D. C., \& Runger, G. C. \textit{Applied Statistics and Probability for Engineers}, Wiley.
  \item Devore, J. L. \textit{Probability and Statistics for Engineering and the Sciences}, Cengage.
  \item McKinney, W. \textit{Python for Data Analysis}, O'Reilly.
\end{itemize}

% BEGIN SLIDE APPENDIX (AUTO-GENERATED)
\clearpage
\section*{Appendix: Slide Deck Content (Reference)}
\noindent The material below is a reference copy of the slide deck content. Exercise solutions are explained in the main notes where applicable.

\subsection*{Title Slide}
\titlepage
        \vspace{-0.5em}
        \begin{center}
          \small \texttt{https://github.com/tali7c/Statistics-and-Data-Analysis}
        \end{center}
\subsection*{Quick Links}
\centering
        \textbf{Overview}\hspace{0.6em}
\textbf{Stationarity}\hspace{0.6em}
\textbf{Fixes}\hspace{0.6em}
\textbf{Exercises}\hspace{0.6em}
\textbf{Demo}\hspace{0.6em}
\textbf{Summary}\hspace{0.6em}
\subsection*{Agenda}
\begin{itemize}
  \item Overview
  \item Stationarity
  \item Fixes
  \item Exercises
  \item Demo
  \item Summary
\end{itemize}
\subsection*{Learning Outcomes}
\begin{itemize}
        \item Define stationarity (intuition)
\item Recognize non-stationary patterns (trend/seasonality)
\item Explain why stationarity matters for ARIMA-type models
\item List basic fixes (differencing, transforms)
      \end{itemize}
\subsection*{Stationarity: Key Points}
\begin{itemize}
        \item Mean/variance roughly constant
\item Autocorrelation depends on lag only
\item Trend/seasonality often implies non-stationarity
      \end{itemize}
\subsection*{Fixes: Key Points}
\begin{itemize}
        \item Differencing removes trend
\item Seasonal differencing removes seasonality
\item Log transform can stabilize variance
      \end{itemize}
\subsection*{Exercise 1: Trend}
\small
  Is a strong upward trend likely stationary?
\subsection*{Solution 1}
\begin{itemize}
    \item No; mean changes over time.
  \end{itemize}
\subsection*{Exercise 2: Variance change}
\small
  If fluctuations grow over time, is variance constant?
\subsection*{Solution 2}
\begin{itemize}
    \item No; non-stationary variance.
  \end{itemize}
\subsection*{Exercise 3: Fix choice}
\small
  Name one fix for non-stationary mean.
\subsection*{Solution 3}
\begin{itemize}
    \item Differencing.
  \end{itemize}
\subsection*{Mini Demo (Python)}
Run from the lecture folder:
  \begin{center}
    \texttt{python demo/demo.py}
  \end{center}
  \vspace{0.4em}
  Outputs:
  \begin{itemize}
    \item \texttt{images/demo.png}
    \item \texttt{data/results.txt}
  \end{itemize}
\subsection*{Demo Output (Example)}
\begin{center}
  \IfFileExists{../images/demo.png}{
    \includegraphics[width=0.92\linewidth]{demo.png}
  }{
    \small (Run demo to generate: \texttt{demo.png})
  }
  \end{center}
\subsection*{Summary}
\begin{itemize}
        \item Key definitions and the main formula.
\item How to interpret results in context.
\item How the demo connects to the theory.
      \end{itemize}
\subsection*{Exit Question}
\small
  Why does non-stationarity make forecasting harder?
% END SLIDE APPENDIX (AUTO-GENERATED)

\end{document}
