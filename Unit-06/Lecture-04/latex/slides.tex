\documentclass{beamer}

      \usetheme{Berlin}
      \usecolortheme{Orchid}
      \useoutertheme{miniframes}
      \setbeamertemplate{navigation symbols}{}

      \usepackage{amsmath}
      \usepackage{amssymb}
      \usepackage{booktabs}
      \usepackage{graphicx}
      \graphicspath{{../images/}}

      \title[Statistics and Data Analysis]{Statistics and Data Analysis}
      \subtitle{Unit 06 -- Lecture 04: Forecasting Fundamentals and ARIMA}
      \author{Tofik Ali}
      \institute{School of Computer Science, UPES Dehradun}
      \date{\today}

      \begin{document}

      \begin{frame}
        \titlepage
        \vspace{-0.5em}
        \begin{center}
          \small \texttt{https://github.com/tali7c/Statistics-and-Data-Analysis}
        \end{center}
      \end{frame}

      \begin{frame}{Quick Links}
        \centering
        \hyperlink{sec:overview}{\beamerbutton{Overview}}\hspace{0.6em}
\hyperlink{sec:arima}{\beamerbutton{ARIMA}}\hspace{0.6em}
\hyperlink{sec:diff}{\beamerbutton{Differencing}}\hspace{0.6em}
\hyperlink{sec:exercises}{\beamerbutton{Exercises}}\hspace{0.6em}
\hyperlink{sec:demo}{\beamerbutton{Demo}}\hspace{0.6em}
\hyperlink{sec:summary}{\beamerbutton{Summary}}\hspace{0.6em}
      \end{frame}

      \begin{frame}{Agenda}
        \tableofcontents
      \end{frame}

\section{Overview}
\label{sec:overview}


\begin{frame}{Learning Outcomes}
      \begin{itemize}[<+->]
        \item Define ARIMA(p,d,q) at a high level
\item Explain differencing (d) to remove trend
\item Explain p and q meaning (AR and MA orders)
\item Describe time-based train/test split for forecasting
      \end{itemize}
    \end{frame}

\section{ARIMA}
\label{sec:arima}


\begin{frame}{ARIMA: Key Points}
      \begin{itemize}[<+->]
        \item p: AR order
\item d: differencing order
\item q: MA order
      \end{itemize}
    \end{frame}

\section{Differencing}
\label{sec:diff}


\begin{frame}{Differencing: Key Points}
      \begin{itemize}[<+->]
        \item First difference: y\_t - y\_{t-1}
\item Often stabilizes mean
\item Over-differencing adds noise
      \end{itemize}
    \end{frame}

\section{Exercises}
\label{sec:exercises}


\begin{frame}{Exercise 1: Meaning of d}
  \small
  What does d=1 mean?
\end{frame}

\begin{frame}{Solution 1}
  \begin{itemize}
    \item First differencing once.
  \end{itemize}
\end{frame}

\begin{frame}{Exercise 2: Chronological split}
  \small
  Why not random split in time series?
\end{frame}

\begin{frame}{Solution 2}
  \begin{itemize}
    \item Random split leaks future information.
  \end{itemize}
\end{frame}

\begin{frame}{Exercise 3: Trend fix}
  \small
  Series has strong upward trend. Name one simple step.
\end{frame}

\begin{frame}{Solution 3}
  \begin{itemize}
    \item First differencing.
  \end{itemize}
\end{frame}

\section{Demo}
\label{sec:demo}


\begin{frame}{Mini Demo (Python)}
  Run from the lecture folder:
  \begin{center}
    \texttt{python demo/demo.py}
  \end{center}
  \vspace{0.4em}
  Outputs:
  \begin{itemize}
    \item \texttt{images/demo.png}
    \item \texttt{data/results.txt}
  \end{itemize}
\end{frame}

\begin{frame}{Demo Output (Example)}
  \begin{center}
  \IfFileExists{../images/demo.png}{
    \includegraphics[width=0.92\linewidth]{demo.png}
  }{
    \small (Run demo to generate: \texttt{demo.png})
  }
  \end{center}
\end{frame}

\section{Summary}
\label{sec:summary}


\begin{frame}{Summary}
      \begin{itemize}[<+->]
        \item Key definitions and the main formula.
\item How to interpret results in context.
\item How the demo connects to the theory.
      \end{itemize}
    \end{frame}

\begin{frame}{Exit Question}
  \small
  Why do we check residuals after fitting an ARIMA model?
\end{frame}

\end{document}
