\documentclass[11pt]{article}
\usepackage[utf8]{inputenc}
\usepackage[T1]{fontenc}
\usepackage{geometry}
\usepackage{amsmath}
\usepackage{listings}
\usepackage{xcolor}
\usepackage{amssymb}
\usepackage{booktabs}
\usepackage{graphicx}
\graphicspath{{../images/}}
\usepackage{hyperref}
\geometry{margin=1in}

\title{Statistics and Data Analysis\\Unit 06 -- Lecture 04 Notes\\Forecasting Fundamentals and ARIMA}
\author{Tofik Ali}
\date{\today}

\begin{document}
\maketitle

\section*{Topic}
ARIMA models and forecasting workflow (overview).
\section*{How to Use These Notes}
These notes are written for students who are seeing the topic for the first time. They
follow the slide order, but add the missing 'why', interpretation, and common mistakes. If
you get stuck, look at the worked exercises and then run the Python demo.

Course repository (slides, demos, datasets): \url{https://github.com/tali7c/Statistics-and-Data-Analysis}

\section*{Time Plan (55 minutes)}
\begin{itemize}
  \item 0--10 min: Attendance + recap of previous lecture
  \item 10--35 min: Core concepts (this lecture's sections)
  \item 35--45 min: Exercises (solve 1--2 in class, rest as practice)
  \item 45--50 min: Mini demo + interpretation of output
  \item 50--55 min: Buffer / wrap-up (leave 5 minutes early)
\end{itemize}

\section*{Slide-by-slide Notes}
\subsection*{Title Slide}
State the lecture title clearly and connect it to what students already know.
Tell students what they will be able to do by the end (not just what you will cover).

\subsection*{Quick Links / Agenda}
Explain the structure of the lecture and where the exercises and demo appear.
\begin{itemize}
  \item Overview
  \item ARIMA
  \item Differencing
  \item Exercises
  \item Demo
  \item Summary
\end{itemize}

\subsection*{Learning Outcomes}
\begin{itemize}
  \item Define ARIMA(p,d,q) at a high level
  \item Explain differencing (d) to remove trend
  \item Explain p and q meaning (AR and MA orders)
  \item Describe time-based train/test split for forecasting
\end{itemize}
\paragraph{Why these outcomes matter.}
\textbf{Trend} is a long-term upward or downward movement. Trend changes the mean over time,
which often creates non-stationarity. Many forecasting models handle trend by differencing
or by explicitly modeling trend.
\textbf{Differencing} transforms a series by subtracting the previous value: $y_t -
y_{t-1}$. It removes trend and can help achieve stationarity. Over-differencing can add
noise, so use the smallest differencing order that works.

\subsection*{ARIMA: Key Points}
\begin{itemize}
  \item p: AR order
  \item d: differencing order
  \item q: MA order
\end{itemize}
\paragraph{Explanation.}
\textbf{Differencing} transforms a series by subtracting the previous value: $y_t -
y_{t-1}$. It removes trend and can help achieve stationarity. Over-differencing can add
noise, so use the smallest differencing order that works.

\subsection*{Differencing: Key Points}
\begin{itemize}
  \item First difference: y\_t - y\_{t-1}
  \item Often stabilizes mean
  \item Over-differencing adds noise
\end{itemize}
\paragraph{Explanation.}
\textbf{Differencing} transforms a series by subtracting the previous value: $y_t -
y_{t-1}$. It removes trend and can help achieve stationarity. Over-differencing can add
noise, so use the smallest differencing order that works.

\subsection*{Exercises (with Solutions)}
Attempt the exercise first, then compare with the solution. Focus on interpretation, not
only arithmetic.

\subsection*{Exercise 1: Meaning of d}
What does d=1 mean?
\subsubsection*{Solution}
\begin{itemize}
  \item First differencing once.
\end{itemize}
\paragraph{Walkthrough.}
\textbf{Differencing} transforms a series by subtracting the previous value: $y_t -
y_{t-1}$. It removes trend and can help achieve stationarity. Over-differencing can add
noise, so use the smallest differencing order that works.

\subsection*{Exercise 2: Chronological split}
Why not random split in time series?
\subsubsection*{Solution}
\begin{itemize}
  \item Random split leaks future information.
\end{itemize}
\paragraph{Walkthrough.}
A \textbf{time series} is data indexed by time (daily sales, hourly sensor readings). The
key difference from 'normal' data is that order matters and observations are often
correlated over time. Many standard ML assumptions (IID, random split) break for time
series.

\subsection*{Exercise 3: Trend fix}
Series has strong upward trend. Name one simple step.
\subsubsection*{Solution}
\begin{itemize}
  \item First differencing.
\end{itemize}
\paragraph{Walkthrough.}
\textbf{Trend} is a long-term upward or downward movement. Trend changes the mean over time,
which often creates non-stationarity. Many forecasting models handle trend by differencing
or by explicitly modeling trend.
\textbf{Differencing} transforms a series by subtracting the previous value: $y_t -
y_{t-1}$. It removes trend and can help achieve stationarity. Over-differencing can add
noise, so use the smallest differencing order that works.

\subsection*{Mini Demo (Python)}
Run from the lecture folder:
\begin{verbatim}
python demo/demo.py
\end{verbatim}

Output files:
\begin{itemize}
  \item \texttt{images/demo.png}
  \item \texttt{data/results.txt}
\end{itemize}
\paragraph{What to show and say.}
\begin{itemize}
  \item Creates a trending series, differences it, and fits an ARIMA-style model.
  \item Shows why differencing helps stabilize the mean (stationarity).
  \item Use residual output to motivate diagnostics after fitting.
\end{itemize}

\subsection*{Demo Output (Example)}
\begin{center}
\IfFileExists{../images/demo.png}{
  \includegraphics[width=0.95\linewidth]{../images/demo.png}
}{
  \small (Run the demo to generate \texttt{images/demo.png})
}
\end{center}

\subsection*{Summary}
\begin{itemize}
  \item Key definitions and the main formula.
  \item How to interpret results in context.
  \item How the demo connects to the theory.
\end{itemize}

\subsection*{Exit Question}
Why do we check residuals after fitting an ARIMA model?
\paragraph{Suggested answer (for revision).}
We check residuals to see if remaining structure (autocorrelation/patterns) is still
unexplained; good residuals look like white noise.

\section*{References}
\begin{itemize}
  \item Montgomery, D. C., \& Runger, G. C. \textit{Applied Statistics and Probability for Engineers}, Wiley.
  \item Devore, J. L. \textit{Probability and Statistics for Engineering and the Sciences}, Cengage.
  \item McKinney, W. \textit{Python for Data Analysis}, O'Reilly.
\end{itemize}

% BEGIN SLIDE APPENDIX (AUTO-GENERATED)
\clearpage
\section*{Appendix: Slide Deck Content (Reference)}
\noindent The material below is a reference copy of the slide deck content. Exercise solutions are explained in the main notes where applicable.

\subsection*{Title Slide}
\titlepage
        \vspace{-0.5em}
        \begin{center}
          \small \texttt{https://github.com/tali7c/Statistics-and-Data-Analysis}
        \end{center}
\subsection*{Quick Links}
\centering
        \textbf{Overview}\hspace{0.6em}
\textbf{ARIMA}\hspace{0.6em}
\textbf{Differencing}\hspace{0.6em}
\textbf{Exercises}\hspace{0.6em}
\textbf{Demo}\hspace{0.6em}
\textbf{Summary}\hspace{0.6em}
\subsection*{Agenda}
\begin{itemize}
  \item Overview
  \item ARIMA
  \item Differencing
  \item Exercises
  \item Demo
  \item Summary
\end{itemize}
\subsection*{Learning Outcomes}
\begin{itemize}
        \item Define ARIMA(p,d,q) at a high level
\item Explain differencing (d) to remove trend
\item Explain p and q meaning (AR and MA orders)
\item Describe time-based train/test split for forecasting
      \end{itemize}
\subsection*{ARIMA: Key Points}
\begin{itemize}
        \item p: AR order
\item d: differencing order
\item q: MA order
      \end{itemize}
\subsection*{Differencing: Key Points}
\begin{itemize}
        \item First difference: y\_t - y\_{t-1}
\item Often stabilizes mean
\item Over-differencing adds noise
      \end{itemize}
\subsection*{Exercise 1: Meaning of d}
\small
  What does d=1 mean?
\subsection*{Solution 1}
\begin{itemize}
    \item First differencing once.
  \end{itemize}
\subsection*{Exercise 2: Chronological split}
\small
  Why not random split in time series?
\subsection*{Solution 2}
\begin{itemize}
    \item Random split leaks future information.
  \end{itemize}
\subsection*{Exercise 3: Trend fix}
\small
  Series has strong upward trend. Name one simple step.
\subsection*{Solution 3}
\begin{itemize}
    \item First differencing.
  \end{itemize}
\subsection*{Mini Demo (Python)}
Run from the lecture folder:
  \begin{center}
    \texttt{python demo/demo.py}
  \end{center}
  \vspace{0.4em}
  Outputs:
  \begin{itemize}
    \item \texttt{images/demo.png}
    \item \texttt{data/results.txt}
  \end{itemize}
\subsection*{Demo Output (Example)}
\begin{center}
  \IfFileExists{../images/demo.png}{
    \includegraphics[width=0.92\linewidth]{demo.png}
  }{
    \small (Run demo to generate: \texttt{demo.png})
  }
  \end{center}
\subsection*{Summary}
\begin{itemize}
        \item Key definitions and the main formula.
\item How to interpret results in context.
\item How the demo connects to the theory.
      \end{itemize}
\subsection*{Exit Question}
\small
  Why do we check residuals after fitting an ARIMA model?
% END SLIDE APPENDIX (AUTO-GENERATED)

\end{document}
