\documentclass{beamer}

      \usetheme{Berlin}
      \usecolortheme{Orchid}
      \useoutertheme{miniframes}
      \setbeamertemplate{navigation symbols}{}

      \usepackage{amsmath}
      \usepackage{amssymb}
      \usepackage{booktabs}
      \usepackage{graphicx}
      \graphicspath{{../images/}}

      \title[Statistics and Data Analysis]{Statistics and Data Analysis}
      \subtitle{Unit 06 -- Lecture 02: Smoothing (Moving Average and Exponential Smoothing)}
      \author{Tofik Ali}
      \institute{School of Computer Science, UPES Dehradun}
      \date{\today}

      \begin{document}

      \begin{frame}
        \titlepage
        \vspace{-0.5em}
        \begin{center}
          \small \texttt{https://github.com/tali7c/Statistics-and-Data-Analysis}
        \end{center}
      \end{frame}

      \begin{frame}{Quick Links}
        \centering
        \hyperlink{sec:overview}{\beamerbutton{Overview}}\hspace{0.6em}
\hyperlink{sec:ma}{\beamerbutton{Moving Average}}\hspace{0.6em}
\hyperlink{sec:es}{\beamerbutton{Exponential Smoothing}}\hspace{0.6em}
\hyperlink{sec:exercises}{\beamerbutton{Exercises}}\hspace{0.6em}
\hyperlink{sec:demo}{\beamerbutton{Demo}}\hspace{0.6em}
\hyperlink{sec:summary}{\beamerbutton{Summary}}\hspace{0.6em}
      \end{frame}

      \begin{frame}{Agenda}
        \tableofcontents
      \end{frame}

\section{Overview}
\label{sec:overview}


\begin{frame}{Learning Outcomes}
      \begin{itemize}[<+->]
        \item Explain why smoothing is used (noise reduction)
\item Describe moving average and its window effect
\item Describe exponential smoothing and alpha effect
\item Discuss responsiveness vs smoothness trade-off
      \end{itemize}
    \end{frame}

\section{Moving Average}
\label{sec:ma}


\begin{frame}{Moving Average: Key Points}
      \begin{itemize}[<+->]
        \item Average last k points
\item Larger k -> smoother but more lag
\item Good for trend visualization
      \end{itemize}
    \end{frame}

\section{Exponential Smoothing}
\label{sec:es}


\begin{frame}{Exponential Smoothing: Key Points}
      \begin{itemize}[<+->]
        \item Weighted average with decay
\item Alpha near 1 -> responsive
\item Alpha near 0 -> smooth
      \end{itemize}
    \end{frame}

\begin{frame}{Exponential Smoothing: Key Formula}
  \[ s_t = \alpha x_t + (1-\alpha)s_{t-1} \]
\end{frame}

\section{Exercises}
\label{sec:exercises}


\begin{frame}{Exercise 1: Window effect}
  \small
  Increase window from 3 to 15: what happens?
\end{frame}

\begin{frame}{Solution 1}
  \begin{itemize}
    \item Smoother, more lag.
  \end{itemize}
\end{frame}

\begin{frame}{Exercise 2: Alpha}
  \small
  If alpha=0.9, smoothing is strong or weak?
\end{frame}

\begin{frame}{Solution 2}
  \begin{itemize}
    \item Weak smoothing (very responsive).
  \end{itemize}
\end{frame}

\begin{frame}{Exercise 3: Too much smoothing}
  \small
  Why can too much smoothing harm forecasting?
\end{frame}

\begin{frame}{Solution 3}
  \begin{itemize}
    \item It can hide real changes and add lag.
  \end{itemize}
\end{frame}

\section{Demo}
\label{sec:demo}


\begin{frame}{Mini Demo (Python)}
  Run from the lecture folder:
  \begin{center}
    \texttt{python demo/demo.py}
  \end{center}
  \vspace{0.4em}
  Outputs:
  \begin{itemize}
    \item \texttt{images/demo.png}
    \item \texttt{data/results.txt}
  \end{itemize}
\end{frame}

\begin{frame}{Demo Output (Example)}
  \begin{center}
  \IfFileExists{../images/demo.png}{
    \includegraphics[width=0.92\linewidth]{demo.png}
  }{
    \small (Run demo to generate: \texttt{demo.png})
  }
  \end{center}
\end{frame}

\section{Summary}
\label{sec:summary}


\begin{frame}{Summary}
      \begin{itemize}[<+->]
        \item Key definitions and the main formula.
\item How to interpret results in context.
\item How the demo connects to the theory.
      \end{itemize}
    \end{frame}

\begin{frame}{Exit Question}
  \small
  What is one sign that your smoothing window is too large?
\end{frame}

\end{document}
