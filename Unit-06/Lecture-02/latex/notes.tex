\documentclass[11pt]{article}
\usepackage[utf8]{inputenc}
\usepackage[T1]{fontenc}
\usepackage{geometry}
\usepackage{amsmath}
\usepackage{booktabs}
\usepackage{hyperref}
\geometry{margin=1in}

\title{Statistics and Data Analysis\\Unit 06 -- Lecture 02 Notes}
\author{Tofik Ali}
\date{\today}

\begin{document}
\maketitle

\section*{Topic}
Smoothing techniques for time series (moving average, exponential smoothing).

\subsection*{Learning Outcomes}
\begin{itemize}
  \item Explain why smoothing is used (noise reduction)
  \item Describe moving average and its window effect
  \item Describe exponential smoothing and alpha effect
  \item Discuss responsiveness vs smoothness trade-off
\end{itemize}

\section*{Detailed Notes}
These notes are designed to be read alongside the slides. They expand each slide bullet into
plain-language explanations, small worked examples, and common pitfalls. When a formula
appears, emphasize (1) what each symbol means, (2) the assumptions needed to use it, and (3)
how to interpret the final number in the problem context.

\section*{Moving Average}
\begin{itemize}
  \item Average last k points
  \item Larger k -> smoother but more lag
  \item Good for trend visualization
\end{itemize}

\section*{Exponential Smoothing}
\begin{itemize}
  \item Weighted average with decay
  \item Alpha near 1 -> responsive
  \item Alpha near 0 -> smooth
\end{itemize}

\section*{Exercises (with Solutions)}
\subsection*{Exercise 1: Window effect}
Increase window from 3 to 15: what happens?
\subsection*{Solution}
\begin{itemize}
  \item Smoother, more lag.
\end{itemize}

\subsection*{Exercise 2: Alpha}
If alpha=0.9, smoothing is strong or weak?
\subsection*{Solution}
\begin{itemize}
  \item Weak smoothing (very responsive).
\end{itemize}

\subsection*{Exercise 3: Too much smoothing}
Why can too much smoothing harm forecasting?
\subsection*{Solution}
\begin{itemize}
  \item It can hide real changes and add lag.
\end{itemize}

\section*{Exit Question}
What is one sign that your smoothing window is too large?

\section*{Demo (Python)}
Run from the lecture folder:
\begin{verbatim}
python demo/demo.py
\end{verbatim}

Output files:
\begin{itemize}
  \item \texttt{images/demo.png}
  \item \texttt{data/results.txt}
\end{itemize}

\section*{References}
\begin{itemize}
  \item Montgomery, D. C., \& Runger, G. C. \textit{Applied Statistics and Probability for Engineers}, Wiley.
  \item Devore, J. L. \textit{Probability and Statistics for Engineering and the Sciences}, Cengage.
  \item McKinney, W. \textit{Python for Data Analysis}, O'Reilly.
\end{itemize}
\end{document}
