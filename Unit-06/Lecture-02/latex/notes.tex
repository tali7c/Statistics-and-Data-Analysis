\documentclass[11pt]{article}
\usepackage[utf8]{inputenc}
\usepackage[T1]{fontenc}
\usepackage{geometry}
\usepackage{amsmath}
\usepackage{listings}
\usepackage{xcolor}
\usepackage{amssymb}
\usepackage{booktabs}
\usepackage{graphicx}
\graphicspath{{../images/}}
\usepackage{hyperref}
\geometry{margin=1in}

\title{Statistics and Data Analysis\\Unit 06 -- Lecture 02 Notes\\Smoothing (Moving Average and Exponential Smoothing)}
\author{Tofik Ali}
\date{\today}

\begin{document}
\maketitle

\section*{Topic}
Smoothing techniques for time series (moving average, exponential smoothing).
\section*{How to Use These Notes}
These notes are written for students who are seeing the topic for the first time. They
follow the slide order, but add the missing 'why', interpretation, and common mistakes. If
you get stuck, look at the worked exercises and then run the Python demo.

Course repository (slides, demos, datasets): \url{https://github.com/tali7c/Statistics-and-Data-Analysis}

\section*{Time Plan (55 minutes)}
\begin{itemize}
  \item 0--10 min: Attendance + recap of previous lecture
  \item 10--35 min: Core concepts (this lecture's sections)
  \item 35--45 min: Exercises (solve 1--2 in class, rest as practice)
  \item 45--50 min: Mini demo + interpretation of output
  \item 50--55 min: Buffer / wrap-up (leave 5 minutes early)
\end{itemize}

\section*{Slide-by-slide Notes}
\subsection*{Title Slide}
State the lecture title clearly and connect it to what students already know.
Tell students what they will be able to do by the end (not just what you will cover).

\subsection*{Quick Links / Agenda}
Explain the structure of the lecture and where the exercises and demo appear.
\begin{itemize}
  \item Overview
  \item Moving Average
  \item Exponential Smoothing
  \item Exercises
  \item Demo
  \item Summary
\end{itemize}

\subsection*{Learning Outcomes}
\begin{itemize}
  \item Explain why smoothing is used (noise reduction)
  \item Describe moving average and its window effect
  \item Describe exponential smoothing and alpha effect
  \item Discuss responsiveness vs smoothness trade-off
\end{itemize}
\paragraph{Why these outcomes matter.}
The \textbf{significance level} $\alpha$ is the maximum Type I error rate you are willing to
tolerate: the probability of rejecting $H_0$ when $H_0$ is actually true. Common choices are
0.05 or 0.01, but the right value depends on consequences of false alarms vs missed
detections.
A \textbf{moving average} smooths noise by averaging the last $k$ points. Larger windows
give smoother curves but introduce lag (the smooth curve reacts slowly to real changes). Use
smoothing for visualization and as a baseline forecasting idea, not as a magic fix.

\subsection*{Moving Average: Key Points}
\begin{itemize}
  \item Average last k points
  \item Larger k -> smoother but more lag
  \item Good for trend visualization
\end{itemize}
\paragraph{Explanation.}
\textbf{Trend} is a long-term upward or downward movement. Trend changes the mean over time,
which often creates non-stationarity. Many forecasting models handle trend by differencing
or by explicitly modeling trend.
A \textbf{moving average} smooths noise by averaging the last $k$ points. Larger windows
give smoother curves but introduce lag (the smooth curve reacts slowly to real changes). Use
smoothing for visualization and as a baseline forecasting idea, not as a magic fix.

\subsection*{Exponential Smoothing: Key Points}
\begin{itemize}
  \item Weighted average with decay
  \item Alpha near 1 -> responsive
  \item Alpha near 0 -> smooth
\end{itemize}
\paragraph{Explanation.}
The \textbf{significance level} $\alpha$ is the maximum Type I error rate you are willing to
tolerate: the probability of rejecting $H_0$ when $H_0$ is actually true. Common choices are
0.05 or 0.01, but the right value depends on consequences of false alarms vs missed
detections.
\textbf{Exponential smoothing} is a weighted average where recent observations get more
weight. The parameter $\alpha$ controls the trade-off: high $\alpha$ reacts quickly (less
smoothing), low $\alpha$ is smoother but slower.

\subsection*{Exponential Smoothing: Key Formula}
\[ s_t = \alpha x_t + (1-\alpha)s_{t-1} \]
\paragraph{How to read the formula.}
The \textbf{significance level} $\alpha$ is the maximum Type I error rate you are willing to
tolerate: the probability of rejecting $H_0$ when $H_0$ is actually true. Common choices are
0.05 or 0.01, but the right value depends on consequences of false alarms vs missed
detections.
\textbf{Exponential smoothing} is a weighted average where recent observations get more
weight. The parameter $\alpha$ controls the trade-off: high $\alpha$ reacts quickly (less
smoothing), low $\alpha$ is smoother but slower.

\subsection*{Exercises (with Solutions)}
Attempt the exercise first, then compare with the solution. Focus on interpretation, not
only arithmetic.

\subsection*{Exercise 1: Window effect}
Increase window from 3 to 15: what happens?
\subsubsection*{Solution}
\begin{itemize}
  \item Smoother, more lag.
\end{itemize}

\subsection*{Exercise 2: Alpha}
If alpha=0.9, smoothing is strong or weak?
\subsubsection*{Solution}
\begin{itemize}
  \item Weak smoothing (very responsive).
\end{itemize}
\paragraph{Walkthrough.}
The \textbf{significance level} $\alpha$ is the maximum Type I error rate you are willing to
tolerate: the probability of rejecting $H_0$ when $H_0$ is actually true. Common choices are
0.05 or 0.01, but the right value depends on consequences of false alarms vs missed
detections.

\subsection*{Exercise 3: Too much smoothing}
Why can too much smoothing harm forecasting?
\subsubsection*{Solution}
\begin{itemize}
  \item It can hide real changes and add lag.
\end{itemize}

\subsection*{Mini Demo (Python)}
Run from the lecture folder:
\begin{verbatim}
python demo/demo.py
\end{verbatim}

Output files:
\begin{itemize}
  \item \texttt{images/demo.png}
  \item \texttt{data/results.txt}
\end{itemize}
\paragraph{What to show and say.}
\begin{itemize}
  \item Applies moving average and exponential smoothing to a noisy series.
  \item Shows the lag vs smoothness trade-off for different window/alpha values.
  \item Use it as a baseline for forecasting and for visual trend extraction.
\end{itemize}

\subsection*{Demo Output (Example)}
\begin{center}
\IfFileExists{../images/demo.png}{
  \includegraphics[width=0.95\linewidth]{../images/demo.png}
}{
  \small (Run the demo to generate \texttt{images/demo.png})
}
\end{center}

\subsection*{Summary}
\begin{itemize}
  \item Key definitions and the main formula.
  \item How to interpret results in context.
  \item How the demo connects to the theory.
\end{itemize}

\subsection*{Exit Question}
What is one sign that your smoothing window is too large?
\paragraph{Suggested answer (for revision).}
If the smoothed curve reacts too slowly and misses real changes, your window is too large
(too much lag).

\section*{References}
\begin{itemize}
  \item Montgomery, D. C., \& Runger, G. C. \textit{Applied Statistics and Probability for Engineers}, Wiley.
  \item Devore, J. L. \textit{Probability and Statistics for Engineering and the Sciences}, Cengage.
  \item McKinney, W. \textit{Python for Data Analysis}, O'Reilly.
\end{itemize}

% BEGIN SLIDE APPENDIX (AUTO-GENERATED)
\clearpage
\section*{Appendix: Slide Deck Content (Reference)}
\noindent The material below is a reference copy of the slide deck content. Exercise solutions are explained in the main notes where applicable.

\subsection*{Title Slide}
\titlepage
        \vspace{-0.5em}
        \begin{center}
          \small \texttt{https://github.com/tali7c/Statistics-and-Data-Analysis}
        \end{center}
\subsection*{Quick Links}
\centering
        \textbf{Overview}\hspace{0.6em}
\textbf{Moving Average}\hspace{0.6em}
\textbf{Exponential Smoothing}\hspace{0.6em}
\textbf{Exercises}\hspace{0.6em}
\textbf{Demo}\hspace{0.6em}
\textbf{Summary}\hspace{0.6em}
\subsection*{Agenda}
\begin{itemize}
  \item Overview
  \item Moving Average
  \item Exponential Smoothing
  \item Exercises
  \item Demo
  \item Summary
\end{itemize}
\subsection*{Learning Outcomes}
\begin{itemize}
        \item Explain why smoothing is used (noise reduction)
\item Describe moving average and its window effect
\item Describe exponential smoothing and alpha effect
\item Discuss responsiveness vs smoothness trade-off
      \end{itemize}
\subsection*{Moving Average: Key Points}
\begin{itemize}
        \item Average last k points
\item Larger k -> smoother but more lag
\item Good for trend visualization
      \end{itemize}
\subsection*{Exponential Smoothing: Key Points}
\begin{itemize}
        \item Weighted average with decay
\item Alpha near 1 -> responsive
\item Alpha near 0 -> smooth
      \end{itemize}
\subsection*{Exponential Smoothing: Key Formula}
\[ s_t = \alpha x_t + (1-\alpha)s_{t-1} \]
\subsection*{Exercise 1: Window effect}
\small
  Increase window from 3 to 15: what happens?
\subsection*{Solution 1}
\begin{itemize}
    \item Smoother, more lag.
  \end{itemize}
\subsection*{Exercise 2: Alpha}
\small
  If alpha=0.9, smoothing is strong or weak?
\subsection*{Solution 2}
\begin{itemize}
    \item Weak smoothing (very responsive).
  \end{itemize}
\subsection*{Exercise 3: Too much smoothing}
\small
  Why can too much smoothing harm forecasting?
\subsection*{Solution 3}
\begin{itemize}
    \item It can hide real changes and add lag.
  \end{itemize}
\subsection*{Mini Demo (Python)}
Run from the lecture folder:
  \begin{center}
    \texttt{python demo/demo.py}
  \end{center}
  \vspace{0.4em}
  Outputs:
  \begin{itemize}
    \item \texttt{images/demo.png}
    \item \texttt{data/results.txt}
  \end{itemize}
\subsection*{Demo Output (Example)}
\begin{center}
  \IfFileExists{../images/demo.png}{
    \includegraphics[width=0.92\linewidth]{demo.png}
  }{
    \small (Run demo to generate: \texttt{demo.png})
  }
  \end{center}
\subsection*{Summary}
\begin{itemize}
        \item Key definitions and the main formula.
\item How to interpret results in context.
\item How the demo connects to the theory.
      \end{itemize}
\subsection*{Exit Question}
\small
  What is one sign that your smoothing window is too large?
% END SLIDE APPENDIX (AUTO-GENERATED)

\end{document}
