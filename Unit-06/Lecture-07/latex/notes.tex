\documentclass[11pt]{article}
\usepackage[utf8]{inputenc}
\usepackage[T1]{fontenc}
\usepackage{geometry}
\usepackage{amsmath}
\usepackage{listings}
\usepackage{xcolor}
\usepackage{amssymb}
\usepackage{booktabs}
\usepackage{graphicx}
\graphicspath{{../images/}}
\usepackage{hyperref}
\geometry{margin=1in}

\title{Statistics and Data Analysis\\Unit 06 -- Lecture 07 Notes\\ACF and PACF Interpretation}
\author{Tofik Ali}
\date{\today}

\begin{document}
\maketitle

\section*{Topic}
ACF/PACF interpretation to guide AR/MA orders (rough).
\section*{How to Use These Notes}
These notes are written for students who are seeing the topic for the first time. They
follow the slide order, but add the missing 'why', interpretation, and common mistakes. If
you get stuck, look at the worked exercises and then run the Python demo.

Course repository (slides, demos, datasets): \url{https://github.com/tali7c/Statistics-and-Data-Analysis}

\section*{Time Plan (55 minutes)}
\begin{itemize}
  \item 0--10 min: Attendance + recap of previous lecture
  \item 10--35 min: Core concepts (this lecture's sections)
  \item 35--45 min: Exercises (solve 1--2 in class, rest as practice)
  \item 45--50 min: Mini demo + interpretation of output
  \item 50--55 min: Buffer / wrap-up (leave 5 minutes early)
\end{itemize}

\section*{Slide-by-slide Notes}
\subsection*{Title Slide}
State the lecture title clearly and connect it to what students already know.
Tell students what they will be able to do by the end (not just what you will cover).

\subsection*{Quick Links / Agenda}
Explain the structure of the lecture and where the exercises and demo appear.
\begin{itemize}
  \item Overview
  \item ACF
  \item PACF
  \item Exercises
  \item Demo
  \item Summary
\end{itemize}

\subsection*{Learning Outcomes}
\begin{itemize}
  \item Define ACF and PACF (intuition)
  \item Use ACF/PACF patterns to guess AR and MA orders (rough)
  \item Recognize slow ACF decay as non-stationarity hint
  \item Explain why validation/diagnostics are still needed
\end{itemize}
\paragraph{Why these outcomes matter.}
\textbf{Stationarity} (intuition) means the series behavior is stable over time: roughly
constant mean/variance and correlation structure. AR/MA/ARIMA models assume stationarity
(after differencing). If the process changes over time, parameters learned from the past may
not hold.
\textbf{ACF} shows correlation of the series with its lagged versions. It helps identify MA-
like behavior and seasonality. \textbf{PACF} shows the correlation at a lag after removing
shorter-lag effects and helps identify AR-like behavior.

\subsection*{ACF: Key Points}
\begin{itemize}
  \item Corr(x\_t, x\_{t-k}) vs lag k
  \item Slow decay can suggest non-stationarity
  \item ACF cutoff after q lags can suggest MA(q) (rough)
\end{itemize}
\paragraph{Explanation.}
\textbf{Stationarity} (intuition) means the series behavior is stable over time: roughly
constant mean/variance and correlation structure. AR/MA/ARIMA models assume stationarity
(after differencing). If the process changes over time, parameters learned from the past may
not hold.
\textbf{ACF} shows correlation of the series with its lagged versions. It helps identify MA-
like behavior and seasonality. \textbf{PACF} shows the correlation at a lag after removing
shorter-lag effects and helps identify AR-like behavior.

\subsection*{PACF: Key Points}
\begin{itemize}
  \item Partial correlation after removing lower lags
  \item PACF cutoff after p lags can suggest AR(p) (rough)
  \item Use with caution and validate
\end{itemize}
\paragraph{Explanation.}
\textbf{Correlation} measures the strength of a linear association between two variables. It
is symmetric (no X/Y direction) and does not imply causation. Outliers can inflate or hide
correlation, so always look at the scatter plot.
\textbf{ACF} shows correlation of the series with its lagged versions. It helps identify MA-
like behavior and seasonality. \textbf{PACF} shows the correlation at a lag after removing
shorter-lag effects and helps identify AR-like behavior.

\subsection*{Exercises (with Solutions)}
Attempt the exercise first, then compare with the solution. Focus on interpretation, not
only arithmetic.

\subsection*{Exercise 1: ACF pattern}
If ACF stays significant for many lags, what might it suggest?
\subsubsection*{Solution}
\begin{itemize}
  \item Non-stationary; try differencing.
\end{itemize}
\paragraph{Walkthrough.}
\textbf{Differencing} transforms a series by subtracting the previous value: $y_t -
y_{t-1}$. It removes trend and can help achieve stationarity. Over-differencing can add
noise, so use the smallest differencing order that works.
\textbf{Stationarity} (intuition) means the series behavior is stable over time: roughly
constant mean/variance and correlation structure. AR/MA/ARIMA models assume stationarity
(after differencing). If the process changes over time, parameters learned from the past may
not hold.

\subsection*{Exercise 2: PACF AR hint}
PACF cuts off after lag 2: what AR order to try?
\subsubsection*{Solution}
\begin{itemize}
  \item Try AR(2) as starting point.
\end{itemize}
\paragraph{Walkthrough.}
\textbf{ACF} shows correlation of the series with its lagged versions. It helps identify MA-
like behavior and seasonality. \textbf{PACF} shows the correlation at a lag after removing
shorter-lag effects and helps identify AR-like behavior.

\subsection*{Exercise 3: Rules not perfect}
Why ACF/PACF rules are not perfect?
\subsubsection*{Solution}
\begin{itemize}
  \item Finite sample noise, seasonality, mixed ARMA.
\end{itemize}
\paragraph{Walkthrough.}
\textbf{Seasonality} is a repeating pattern with a fixed period (weekly, monthly, yearly).
You must account for it; otherwise forecasts systematically miss repeating rises/falls.
Seasonal differencing and SARIMA are common tools.
\textbf{ACF} shows correlation of the series with its lagged versions. It helps identify MA-
like behavior and seasonality. \textbf{PACF} shows the correlation at a lag after removing
shorter-lag effects and helps identify AR-like behavior.

\subsection*{Mini Demo (Python)}
Run from the lecture folder:
\begin{verbatim}
python demo/demo.py
\end{verbatim}

Output files:
\begin{itemize}
  \item \texttt{images/demo.png}
  \item \texttt{data/results.txt}
\end{itemize}
\paragraph{What to show and say.}
\begin{itemize}
  \item Computes and plots ACF/PACF-like behavior for a simulated process.
  \item Use it to practice reading cutoff/decay patterns (rough guidance).
  \item Emphasize validation and diagnostics because rules are noisy in finite samples.
\end{itemize}

\subsection*{Demo Output (Example)}
\begin{center}
\IfFileExists{../images/demo.png}{
  \includegraphics[width=0.95\linewidth]{../images/demo.png}
}{
  \small (Run the demo to generate \texttt{images/demo.png})
}
\end{center}

\subsection*{Summary}
\begin{itemize}
  \item Key definitions and the main formula.
  \item How to interpret results in context.
  \item How the demo connects to the theory.
\end{itemize}

\subsection*{Exit Question}
How do ACF and PACF help choose ARIMA orders p and q?
\paragraph{Suggested answer (for revision).}
ACF/PACF patterns give rough hints: PACF cutoff ~ AR(p), ACF cutoff ~ MA(q); use
diagnostics/validation because real data is noisy.

\section*{References}
\begin{itemize}
  \item Montgomery, D. C., \& Runger, G. C. \textit{Applied Statistics and Probability for Engineers}, Wiley.
  \item Devore, J. L. \textit{Probability and Statistics for Engineering and the Sciences}, Cengage.
  \item McKinney, W. \textit{Python for Data Analysis}, O'Reilly.
\end{itemize}

% BEGIN SLIDE APPENDIX (AUTO-GENERATED)
\clearpage
\section*{Appendix: Slide Deck Content (Reference)}
\noindent The material below is a reference copy of the slide deck content. Exercise solutions are explained in the main notes where applicable.

\subsection*{Title Slide}
\titlepage
        \vspace{-0.5em}
        \begin{center}
          \small \texttt{https://github.com/tali7c/Statistics-and-Data-Analysis}
        \end{center}
\subsection*{Quick Links}
\centering
        \textbf{Overview}\hspace{0.6em}
\textbf{ACF}\hspace{0.6em}
\textbf{PACF}\hspace{0.6em}
\textbf{Exercises}\hspace{0.6em}
\textbf{Demo}\hspace{0.6em}
\textbf{Summary}\hspace{0.6em}
\subsection*{Agenda}
\begin{itemize}
  \item Overview
  \item ACF
  \item PACF
  \item Exercises
  \item Demo
  \item Summary
\end{itemize}
\subsection*{Learning Outcomes}
\begin{itemize}
        \item Define ACF and PACF (intuition)
\item Use ACF/PACF patterns to guess AR and MA orders (rough)
\item Recognize slow ACF decay as non-stationarity hint
\item Explain why validation/diagnostics are still needed
      \end{itemize}
\subsection*{ACF: Key Points}
\begin{itemize}
        \item Corr(x\_t, x\_{t-k}) vs lag k
\item Slow decay can suggest non-stationarity
\item ACF cutoff after q lags can suggest MA(q) (rough)
      \end{itemize}
\subsection*{PACF: Key Points}
\begin{itemize}
        \item Partial correlation after removing lower lags
\item PACF cutoff after p lags can suggest AR(p) (rough)
\item Use with caution and validate
      \end{itemize}
\subsection*{Exercise 1: ACF pattern}
\small
  If ACF stays significant for many lags, what might it suggest?
\subsection*{Solution 1}
\begin{itemize}
    \item Non-stationary; try differencing.
  \end{itemize}
\subsection*{Exercise 2: PACF AR hint}
\small
  PACF cuts off after lag 2: what AR order to try?
\subsection*{Solution 2}
\begin{itemize}
    \item Try AR(2) as starting point.
  \end{itemize}
\subsection*{Exercise 3: Rules not perfect}
\small
  Why ACF/PACF rules are not perfect?
\subsection*{Solution 3}
\begin{itemize}
    \item Finite sample noise, seasonality, mixed ARMA.
  \end{itemize}
\subsection*{Mini Demo (Python)}
Run from the lecture folder:
  \begin{center}
    \texttt{python demo/demo.py}
  \end{center}
  \vspace{0.4em}
  Outputs:
  \begin{itemize}
    \item \texttt{images/demo.png}
    \item \texttt{data/results.txt}
  \end{itemize}
\subsection*{Demo Output (Example)}
\begin{center}
  \IfFileExists{../images/demo.png}{
    \includegraphics[width=0.92\linewidth]{demo.png}
  }{
    \small (Run demo to generate: \texttt{demo.png})
  }
  \end{center}
\subsection*{Summary}
\begin{itemize}
        \item Key definitions and the main formula.
\item How to interpret results in context.
\item How the demo connects to the theory.
      \end{itemize}
\subsection*{Exit Question}
\small
  How do ACF and PACF help choose ARIMA orders p and q?
% END SLIDE APPENDIX (AUTO-GENERATED)

\end{document}
