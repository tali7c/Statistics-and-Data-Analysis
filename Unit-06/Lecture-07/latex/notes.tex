\documentclass[11pt]{article}
\usepackage[utf8]{inputenc}
\usepackage[T1]{fontenc}
\usepackage{geometry}
\usepackage{amsmath}
\usepackage{booktabs}
\usepackage{hyperref}
\geometry{margin=1in}

\title{Statistics and Data Analysis\\Unit 06 -- Lecture 07 Notes}
\author{Tofik Ali}
\date{\today}

\begin{document}
\maketitle

\section*{Topic}
ACF/PACF interpretation to guide AR/MA orders (rough).

\subsection*{Learning Outcomes}
\begin{itemize}
  \item Define ACF and PACF (intuition)
  \item Use ACF/PACF patterns to guess AR and MA orders (rough)
  \item Recognize slow ACF decay as non-stationarity hint
  \item Explain why validation/diagnostics are still needed
\end{itemize}

\section*{Detailed Notes}
These notes are designed to be read alongside the slides. They expand each slide bullet into
plain-language explanations, small worked examples, and common pitfalls. When a formula
appears, emphasize (1) what each symbol means, (2) the assumptions needed to use it, and (3)
how to interpret the final number in the problem context.

\section*{ACF}
\begin{itemize}
  \item Corr(x\_t, x\_{t-k}) vs lag k
  \item Slow decay can suggest non-stationarity
  \item ACF cutoff after q lags can suggest MA(q) (rough)
\end{itemize}

\section*{PACF}
\begin{itemize}
  \item Partial correlation after removing lower lags
  \item PACF cutoff after p lags can suggest AR(p) (rough)
  \item Use with caution and validate
\end{itemize}

\section*{Exercises (with Solutions)}
\subsection*{Exercise 1: ACF pattern}
If ACF stays significant for many lags, what might it suggest?
\subsection*{Solution}
\begin{itemize}
  \item Non-stationary; try differencing.
\end{itemize}

\subsection*{Exercise 2: PACF AR hint}
PACF cuts off after lag 2: what AR order to try?
\subsection*{Solution}
\begin{itemize}
  \item Try AR(2) as starting point.
\end{itemize}

\subsection*{Exercise 3: Rules not perfect}
Why ACF/PACF rules are not perfect?
\subsection*{Solution}
\begin{itemize}
  \item Finite sample noise, seasonality, mixed ARMA.
\end{itemize}

\section*{Exit Question}
How do ACF and PACF help choose ARIMA orders p and q?

\section*{Demo (Python)}
Run from the lecture folder:
\begin{verbatim}
python demo/demo.py
\end{verbatim}

Output files:
\begin{itemize}
  \item \texttt{images/demo.png}
  \item \texttt{data/results.txt}
\end{itemize}

\section*{References}
\begin{itemize}
  \item Montgomery, D. C., \& Runger, G. C. \textit{Applied Statistics and Probability for Engineers}, Wiley.
  \item Devore, J. L. \textit{Probability and Statistics for Engineering and the Sciences}, Cengage.
  \item McKinney, W. \textit{Python for Data Analysis}, O'Reilly.
\end{itemize}
\end{document}
