\documentclass{beamer}

      \usetheme{Berlin}
      \usecolortheme{Orchid}
      \useoutertheme{miniframes}
      \setbeamertemplate{navigation symbols}{}

      \usepackage{amsmath}
      \usepackage{amssymb}
      \usepackage{booktabs}
      \usepackage{graphicx}
      \graphicspath{{../images/}}

      \title[Statistics and Data Analysis]{Statistics and Data Analysis}
      \subtitle{Unit 06 -- Lecture 01: Time-series Concepts (Trend, Seasonality, Autocorrelation)}
      \author{Tofik Ali}
      \institute{School of Computer Science, UPES Dehradun}
      \date{\today}

      \begin{document}

      \begin{frame}
        \titlepage
        \vspace{-0.5em}
        \begin{center}
          \small \texttt{https://github.com/tali7c/Statistics-and-Data-Analysis}
        \end{center}
      \end{frame}

      \begin{frame}{Quick Links}
        \centering
        \hyperlink{sec:overview}{\beamerbutton{Overview}}\hspace{0.6em}
\hyperlink{sec:comp}{\beamerbutton{Components}}\hspace{0.6em}
\hyperlink{sec:auto}{\beamerbutton{Autocorrelation}}\hspace{0.6em}
\hyperlink{sec:exercises}{\beamerbutton{Exercises}}\hspace{0.6em}
\hyperlink{sec:demo}{\beamerbutton{Demo}}\hspace{0.6em}
\hyperlink{sec:summary}{\beamerbutton{Summary}}\hspace{0.6em}
      \end{frame}

      \begin{frame}{Agenda}
        \tableofcontents
      \end{frame}

\section{Overview}
\label{sec:overview}


\begin{frame}{Learning Outcomes}
      \begin{itemize}[<+->]
        \item Define time series and why order matters
\item Identify trend, seasonality, and noise
\item Explain autocorrelation (intuition)
\item Explain why random shuffling breaks time series analysis
      \end{itemize}
    \end{frame}

\section{Components}
\label{sec:comp}


\begin{frame}{Components: Key Points}
      \begin{itemize}[<+->]
        \item Trend: long-term movement
\item Seasonality: repeating pattern
\item Noise: irregular fluctuations
      \end{itemize}
    \end{frame}

\section{Autocorrelation}
\label{sec:auto}


\begin{frame}{Autocorrelation: Key Points}
      \begin{itemize}[<+->]
        \item Correlation with past values
\item Important for AR/MA/ARIMA models
\item Shows persistence of shocks
      \end{itemize}
    \end{frame}

\section{Exercises}
\label{sec:exercises}


\begin{frame}{Exercise 1: Order matters}
  \small
  Why should train/test split be chronological for time series?
\end{frame}

\begin{frame}{Solution 1}
  \begin{itemize}
    \item To avoid future-to-past leakage.
  \end{itemize}
\end{frame}

\begin{frame}{Exercise 2: Seasonality example}
  \small
  Give one seasonal pattern in campus data.
\end{frame}

\begin{frame}{Solution 2}
  \begin{itemize}
    \item Weekly cafe sales (weekday vs weekend), etc.
  \end{itemize}
\end{frame}

\begin{frame}{Exercise 3: Autocorr meaning}
  \small
  If lag-1 autocorrelation is strong positive, what does it suggest?
\end{frame}

\begin{frame}{Solution 3}
  \begin{itemize}
    \item Values tend to persist from one step to next.
  \end{itemize}
\end{frame}

\section{Demo}
\label{sec:demo}


\begin{frame}{Mini Demo (Python)}
  Run from the lecture folder:
  \begin{center}
    \texttt{python demo/demo.py}
  \end{center}
  \vspace{0.4em}
  Outputs:
  \begin{itemize}
    \item \texttt{images/demo.png}
    \item \texttt{data/results.txt}
  \end{itemize}
\end{frame}

\begin{frame}{Demo Output (Example)}
  \begin{center}
  \IfFileExists{../images/demo.png}{
    \includegraphics[width=0.92\linewidth]{demo.png}
  }{
    \small (Run demo to generate: \texttt{demo.png})
  }
  \end{center}
\end{frame}

\section{Summary}
\label{sec:summary}


\begin{frame}{Summary}
      \begin{itemize}[<+->]
        \item Key definitions and the main formula.
\item How to interpret results in context.
\item How the demo connects to the theory.
      \end{itemize}
    \end{frame}

\begin{frame}{Exit Question}
  \small
  In one sentence: what is seasonality and why does it matter for forecasting?
\end{frame}

\end{document}
