\documentclass[11pt]{article}
\usepackage[utf8]{inputenc}
\usepackage[T1]{fontenc}
\usepackage{geometry}
\usepackage{amsmath}
\usepackage{booktabs}
\usepackage{hyperref}
\geometry{margin=1in}

\title{Statistics and Data Analysis\\Unit 06 -- Lecture 01 Notes}
\author{Tofik Ali}
\date{\today}

\begin{document}
\maketitle

\section*{Topic}
Time series basics: components and autocorrelation.

\subsection*{Learning Outcomes}
\begin{itemize}
  \item Define time series and why order matters
  \item Identify trend, seasonality, and noise
  \item Explain autocorrelation (intuition)
  \item Explain why random shuffling breaks time series analysis
\end{itemize}

\section*{Detailed Notes}
These notes are designed to be read alongside the slides. They expand each slide bullet into
plain-language explanations, small worked examples, and common pitfalls. When a formula
appears, emphasize (1) what each symbol means, (2) the assumptions needed to use it, and (3)
how to interpret the final number in the problem context.

\section*{Components}
\begin{itemize}
  \item Trend: long-term movement
  \item Seasonality: repeating pattern
  \item Noise: irregular fluctuations
\end{itemize}

\section*{Autocorrelation}
\begin{itemize}
  \item Correlation with past values
  \item Important for AR/MA/ARIMA models
  \item Shows persistence of shocks
\end{itemize}

\section*{Exercises (with Solutions)}
\subsection*{Exercise 1: Order matters}
Why should train/test split be chronological for time series?
\subsection*{Solution}
\begin{itemize}
  \item To avoid future-to-past leakage.
\end{itemize}

\subsection*{Exercise 2: Seasonality example}
Give one seasonal pattern in campus data.
\subsection*{Solution}
\begin{itemize}
  \item Weekly cafe sales (weekday vs weekend), etc.
\end{itemize}

\subsection*{Exercise 3: Autocorr meaning}
If lag-1 autocorrelation is strong positive, what does it suggest?
\subsection*{Solution}
\begin{itemize}
  \item Values tend to persist from one step to next.
\end{itemize}

\section*{Exit Question}
In one sentence: what is seasonality and why does it matter for forecasting?

\section*{Demo (Python)}
Run from the lecture folder:
\begin{verbatim}
python demo/demo.py
\end{verbatim}

Output files:
\begin{itemize}
  \item \texttt{images/demo.png}
  \item \texttt{data/results.txt}
\end{itemize}

\section*{References}
\begin{itemize}
  \item Montgomery, D. C., \& Runger, G. C. \textit{Applied Statistics and Probability for Engineers}, Wiley.
  \item Devore, J. L. \textit{Probability and Statistics for Engineering and the Sciences}, Cengage.
  \item McKinney, W. \textit{Python for Data Analysis}, O'Reilly.
\end{itemize}
\end{document}
