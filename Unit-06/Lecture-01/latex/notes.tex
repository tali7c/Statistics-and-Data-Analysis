\documentclass[11pt]{article}
\usepackage[utf8]{inputenc}
\usepackage[T1]{fontenc}
\usepackage{geometry}
\usepackage{amsmath}
\usepackage{listings}
\usepackage{xcolor}
\usepackage{amssymb}
\usepackage{booktabs}
\usepackage{graphicx}
\graphicspath{{../images/}}
\usepackage{hyperref}
\geometry{margin=1in}

\title{Statistics and Data Analysis\\Unit 06 -- Lecture 01 Notes\\Time-series Concepts (Trend, Seasonality, Autocorrelation)}
\author{Tofik Ali}
\date{\today}

\begin{document}
\maketitle

\section*{Topic}
Time series basics: components and autocorrelation.
\section*{How to Use These Notes}
These notes are written for students who are seeing the topic for the first time. They
follow the slide order, but add the missing 'why', interpretation, and common mistakes. If
you get stuck, look at the worked exercises and then run the Python demo.

Course repository (slides, demos, datasets): \url{https://github.com/tali7c/Statistics-and-Data-Analysis}

\section*{Time Plan (55 minutes)}
\begin{itemize}
  \item 0--10 min: Attendance + recap of previous lecture
  \item 10--35 min: Core concepts (this lecture's sections)
  \item 35--45 min: Exercises (solve 1--2 in class, rest as practice)
  \item 45--50 min: Mini demo + interpretation of output
  \item 50--55 min: Buffer / wrap-up (leave 5 minutes early)
\end{itemize}

\section*{Slide-by-slide Notes}
\subsection*{Title Slide}
State the lecture title clearly and connect it to what students already know.
Tell students what they will be able to do by the end (not just what you will cover).

\subsection*{Quick Links / Agenda}
Explain the structure of the lecture and where the exercises and demo appear.
\begin{itemize}
  \item Overview
  \item Components
  \item Autocorrelation
  \item Exercises
  \item Demo
  \item Summary
\end{itemize}

\subsection*{Learning Outcomes}
\begin{itemize}
  \item Define time series and why order matters
  \item Identify trend, seasonality, and noise
  \item Explain autocorrelation (intuition)
  \item Explain why random shuffling breaks time series analysis
\end{itemize}
\paragraph{Why these outcomes matter.}
\textbf{Correlation} measures the strength of a linear association between two variables. It
is symmetric (no X/Y direction) and does not imply causation. Outliers can inflate or hide
correlation, so always look at the scatter plot.
A \textbf{time series} is data indexed by time (daily sales, hourly sensor readings). The
key difference from 'normal' data is that order matters and observations are often
correlated over time. Many standard ML assumptions (IID, random split) break for time
series.

\subsection*{Components: Key Points}
\begin{itemize}
  \item Trend: long-term movement
  \item Seasonality: repeating pattern
  \item Noise: irregular fluctuations
\end{itemize}
\paragraph{Explanation.}
\textbf{Trend} is a long-term upward or downward movement. Trend changes the mean over time,
which often creates non-stationarity. Many forecasting models handle trend by differencing
or by explicitly modeling trend.
\textbf{Seasonality} is a repeating pattern with a fixed period (weekly, monthly, yearly).
You must account for it; otherwise forecasts systematically miss repeating rises/falls.
Seasonal differencing and SARIMA are common tools.

\subsection*{Autocorrelation: Key Points}
\begin{itemize}
  \item Correlation with past values
  \item Important for AR/MA/ARIMA models
  \item Shows persistence of shocks
\end{itemize}
\paragraph{Explanation.}
\textbf{Correlation} measures the strength of a linear association between two variables. It
is symmetric (no X/Y direction) and does not imply causation. Outliers can inflate or hide
correlation, so always look at the scatter plot.
\textbf{Autocorrelation} measures how strongly the series relates to its past values.
Positive autocorrelation means high values tend to follow high values; negative means
alternation. Autocorrelation is the reason AR/MA/ARIMA models work at all.

\subsection*{Exercises (with Solutions)}
Attempt the exercise first, then compare with the solution. Focus on interpretation, not
only arithmetic.

\subsection*{Exercise 1: Order matters}
Why should train/test split be chronological for time series?
\subsubsection*{Solution}
\begin{itemize}
  \item To avoid future-to-past leakage.
\end{itemize}
\paragraph{Walkthrough.}
\textbf{Data leakage} happens when information from the future or from the test set
influences training. Typical examples: scaling before splitting, using target-related
features, or using random splits for time series. Leakage can produce very good-looking
accuracy that disappears in real deployment.
A \textbf{time series} is data indexed by time (daily sales, hourly sensor readings). The
key difference from 'normal' data is that order matters and observations are often
correlated over time. Many standard ML assumptions (IID, random split) break for time
series.

\subsection*{Exercise 2: Seasonality example}
Give one seasonal pattern in campus data.
\subsubsection*{Solution}
\begin{itemize}
  \item Weekly cafe sales (weekday vs weekend), etc.
\end{itemize}
\paragraph{Walkthrough.}
\textbf{Seasonality} is a repeating pattern with a fixed period (weekly, monthly, yearly).
You must account for it; otherwise forecasts systematically miss repeating rises/falls.
Seasonal differencing and SARIMA are common tools.

\subsection*{Exercise 3: Autocorr meaning}
If lag-1 autocorrelation is strong positive, what does it suggest?
\subsubsection*{Solution}
\begin{itemize}
  \item Values tend to persist from one step to next.
\end{itemize}
\paragraph{Walkthrough.}
\textbf{Correlation} measures the strength of a linear association between two variables. It
is symmetric (no X/Y direction) and does not imply causation. Outliers can inflate or hide
correlation, so always look at the scatter plot.
\textbf{Autocorrelation} measures how strongly the series relates to its past values.
Positive autocorrelation means high values tend to follow high values; negative means
alternation. Autocorrelation is the reason AR/MA/ARIMA models work at all.

\subsection*{Mini Demo (Python)}
Run from the lecture folder:
\begin{verbatim}
python demo/demo.py
\end{verbatim}

Output files:
\begin{itemize}
  \item \texttt{images/demo.png}
  \item \texttt{data/results.txt}
\end{itemize}
\paragraph{What to show and say.}
\begin{itemize}
  \item Generates a time series with trend/seasonality and plots it.
  \item Shows how autocorrelation makes random shuffling invalid.
  \item Use it to motivate chronological splits and forecasting mindset.
\end{itemize}

\subsection*{Demo Output (Example)}
\begin{center}
\IfFileExists{../images/demo.png}{
  \includegraphics[width=0.95\linewidth]{../images/demo.png}
}{
  \small (Run the demo to generate \texttt{images/demo.png})
}
\end{center}

\subsection*{Summary}
\begin{itemize}
  \item Key definitions and the main formula.
  \item How to interpret results in context.
  \item How the demo connects to the theory.
\end{itemize}

\subsection*{Exit Question}
In one sentence: what is seasonality and why does it matter for forecasting?
\paragraph{Suggested answer (for revision).}
Seasonality is a repeating pattern with a fixed period; it matters because forecasts must
reproduce these cycles or they will be systematically wrong.

\section*{References}
\begin{itemize}
  \item Montgomery, D. C., \& Runger, G. C. \textit{Applied Statistics and Probability for Engineers}, Wiley.
  \item Devore, J. L. \textit{Probability and Statistics for Engineering and the Sciences}, Cengage.
  \item McKinney, W. \textit{Python for Data Analysis}, O'Reilly.
\end{itemize}

% BEGIN SLIDE APPENDIX (AUTO-GENERATED)
\clearpage
\section*{Appendix: Slide Deck Content (Reference)}
\noindent The material below is a reference copy of the slide deck content. Exercise solutions are explained in the main notes where applicable.

\subsection*{Title Slide}
\titlepage
        \vspace{-0.5em}
        \begin{center}
          \small \texttt{https://github.com/tali7c/Statistics-and-Data-Analysis}
        \end{center}
\subsection*{Quick Links}
\centering
        \textbf{Overview}\hspace{0.6em}
\textbf{Components}\hspace{0.6em}
\textbf{Autocorrelation}\hspace{0.6em}
\textbf{Exercises}\hspace{0.6em}
\textbf{Demo}\hspace{0.6em}
\textbf{Summary}\hspace{0.6em}
\subsection*{Agenda}
\begin{itemize}
  \item Overview
  \item Components
  \item Autocorrelation
  \item Exercises
  \item Demo
  \item Summary
\end{itemize}
\subsection*{Learning Outcomes}
\begin{itemize}
        \item Define time series and why order matters
\item Identify trend, seasonality, and noise
\item Explain autocorrelation (intuition)
\item Explain why random shuffling breaks time series analysis
      \end{itemize}
\subsection*{Components: Key Points}
\begin{itemize}
        \item Trend: long-term movement
\item Seasonality: repeating pattern
\item Noise: irregular fluctuations
      \end{itemize}
\subsection*{Autocorrelation: Key Points}
\begin{itemize}
        \item Correlation with past values
\item Important for AR/MA/ARIMA models
\item Shows persistence of shocks
      \end{itemize}
\subsection*{Exercise 1: Order matters}
\small
  Why should train/test split be chronological for time series?
\subsection*{Solution 1}
\begin{itemize}
    \item To avoid future-to-past leakage.
  \end{itemize}
\subsection*{Exercise 2: Seasonality example}
\small
  Give one seasonal pattern in campus data.
\subsection*{Solution 2}
\begin{itemize}
    \item Weekly cafe sales (weekday vs weekend), etc.
  \end{itemize}
\subsection*{Exercise 3: Autocorr meaning}
\small
  If lag-1 autocorrelation is strong positive, what does it suggest?
\subsection*{Solution 3}
\begin{itemize}
    \item Values tend to persist from one step to next.
  \end{itemize}
\subsection*{Mini Demo (Python)}
Run from the lecture folder:
  \begin{center}
    \texttt{python demo/demo.py}
  \end{center}
  \vspace{0.4em}
  Outputs:
  \begin{itemize}
    \item \texttt{images/demo.png}
    \item \texttt{data/results.txt}
  \end{itemize}
\subsection*{Demo Output (Example)}
\begin{center}
  \IfFileExists{../images/demo.png}{
    \includegraphics[width=0.92\linewidth]{demo.png}
  }{
    \small (Run demo to generate: \texttt{demo.png})
  }
  \end{center}
\subsection*{Summary}
\begin{itemize}
        \item Key definitions and the main formula.
\item How to interpret results in context.
\item How the demo connects to the theory.
      \end{itemize}
\subsection*{Exit Question}
\small
  In one sentence: what is seasonality and why does it matter for forecasting?
% END SLIDE APPENDIX (AUTO-GENERATED)

\end{document}
