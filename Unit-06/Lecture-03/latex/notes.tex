\documentclass[11pt]{article}
\usepackage[utf8]{inputenc}
\usepackage[T1]{fontenc}
\usepackage{geometry}
\usepackage{amsmath}
\usepackage{booktabs}
\usepackage{hyperref}
\geometry{margin=1in}

\title{Statistics and Data Analysis\\Unit 06 -- Lecture 03 Notes}
\author{Tofik Ali}
\date{\today}

\begin{document}
\maketitle

\section*{Topic}
Autoregressive (AR) and moving average (MA) models (intro).

\subsection*{Learning Outcomes}
\begin{itemize}
  \item Explain AR(p) model idea
  \item Explain MA(q) model idea
  \item Differentiate AR vs MA intuition
  \item Define white noise (basic)
\end{itemize}

\section*{Detailed Notes}
These notes are designed to be read alongside the slides. They expand each slide bullet into
plain-language explanations, small worked examples, and common pitfalls. When a formula
appears, emphasize (1) what each symbol means, (2) the assumptions needed to use it, and (3)
how to interpret the final number in the problem context.

\section*{AR}
\begin{itemize}
  \item Current value depends on past values
  \item AR(1): x\_t = c + phi x\_{t-1} + e\_t
  \item Phi controls persistence
\end{itemize}

\section*{MA}
\begin{itemize}
  \item Current value depends on past shocks
  \item MA(1): x\_t = mu + e\_t + theta e\_{t-1}
  \item Captures short-term shock effects
\end{itemize}

\section*{Exercises (with Solutions)}
\subsection*{Exercise 1: AR intuition}
If phi=0.8 and last value is high (ignore noise), what happens next?
\subsection*{Solution}
\begin{itemize}
  \item Next value tends to be high too.
\end{itemize}

\subsection*{Exercise 2: MA intuition}
In MA(1), what drives the series: past values or past shocks?
\subsection*{Solution}
\begin{itemize}
  \item Past shocks (errors).
\end{itemize}

\subsection*{Exercise 3: White noise}
What is white noise?
\subsection*{Solution}
\begin{itemize}
  \item Uncorrelated errors with mean 0 and constant variance.
\end{itemize}

\section*{Exit Question}
How are AR and MA models different in what they remember?

\section*{Demo (Python)}
Run from the lecture folder:
\begin{verbatim}
python demo/demo.py
\end{verbatim}

Output files:
\begin{itemize}
  \item \texttt{images/demo.png}
  \item \texttt{data/results.txt}
\end{itemize}

\section*{References}
\begin{itemize}
  \item Montgomery, D. C., \& Runger, G. C. \textit{Applied Statistics and Probability for Engineers}, Wiley.
  \item Devore, J. L. \textit{Probability and Statistics for Engineering and the Sciences}, Cengage.
  \item McKinney, W. \textit{Python for Data Analysis}, O'Reilly.
\end{itemize}
\end{document}
