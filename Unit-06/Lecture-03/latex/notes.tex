\documentclass[11pt]{article}
\usepackage[utf8]{inputenc}
\usepackage[T1]{fontenc}
\usepackage{geometry}
\usepackage{amsmath}
\usepackage{listings}
\usepackage{xcolor}
\usepackage{amssymb}
\usepackage{booktabs}
\usepackage{graphicx}
\graphicspath{{../images/}}
\usepackage{hyperref}
\geometry{margin=1in}

\title{Statistics and Data Analysis\\Unit 06 -- Lecture 03 Notes\\AR and MA Models}
\author{Tofik Ali}
\date{\today}

\begin{document}
\maketitle

\section*{Topic}
Autoregressive (AR) and moving average (MA) models (intro).
\section*{How to Use These Notes}
These notes are written for students who are seeing the topic for the first time. They
follow the slide order, but add the missing 'why', interpretation, and common mistakes. If
you get stuck, look at the worked exercises and then run the Python demo.

Course repository (slides, demos, datasets): \url{https://github.com/tali7c/Statistics-and-Data-Analysis}

\section*{Time Plan (55 minutes)}
\begin{itemize}
  \item 0--10 min: Attendance + recap of previous lecture
  \item 10--35 min: Core concepts (this lecture's sections)
  \item 35--45 min: Exercises (solve 1--2 in class, rest as practice)
  \item 45--50 min: Mini demo + interpretation of output
  \item 50--55 min: Buffer / wrap-up (leave 5 minutes early)
\end{itemize}

\section*{Slide-by-slide Notes}
\subsection*{Title Slide}
State the lecture title clearly and connect it to what students already know.
Tell students what they will be able to do by the end (not just what you will cover).

\subsection*{Quick Links / Agenda}
Explain the structure of the lecture and where the exercises and demo appear.
\begin{itemize}
  \item Overview
  \item AR
  \item MA
  \item Exercises
  \item Demo
  \item Summary
\end{itemize}

\subsection*{Learning Outcomes}
\begin{itemize}
  \item Explain AR(p) model idea
  \item Explain MA(q) model idea
  \item Differentiate AR vs MA intuition
  \item Define white noise (basic)
\end{itemize}

\subsection*{AR: Key Points}
\begin{itemize}
  \item Current value depends on past values
  \item AR(1): x\_t = c + phi x\_{t-1} + e\_t
  \item Phi controls persistence
\end{itemize}

\subsection*{AR: Key Formula}
\[ x_t = c + \phi x_{t-1} + \epsilon_t \]

\subsection*{MA: Key Points}
\begin{itemize}
  \item Current value depends on past shocks
  \item MA(1): x\_t = mu + e\_t + theta e\_{t-1}
  \item Captures short-term shock effects
\end{itemize}

\subsection*{MA: Key Formula}
\[ x_t = \mu + \epsilon_t + \theta\epsilon_{t-1} \]

\subsection*{Exercises (with Solutions)}
Attempt the exercise first, then compare with the solution. Focus on interpretation, not
only arithmetic.

\subsection*{Exercise 1: AR intuition}
If phi=0.8 and last value is high (ignore noise), what happens next?
\subsubsection*{Solution}
\begin{itemize}
  \item Next value tends to be high too.
\end{itemize}

\subsection*{Exercise 2: MA intuition}
In MA(1), what drives the series: past values or past shocks?
\subsubsection*{Solution}
\begin{itemize}
  \item Past shocks (errors).
\end{itemize}

\subsection*{Exercise 3: White noise}
What is white noise?
\subsubsection*{Solution}
\begin{itemize}
  \item Uncorrelated errors with mean 0 and constant variance.
\end{itemize}

\subsection*{Mini Demo (Python)}
Run from the lecture folder:
\begin{verbatim}
python demo/demo.py
\end{verbatim}

Output files:
\begin{itemize}
  \item \texttt{images/demo.png}
  \item \texttt{data/results.txt}
\end{itemize}
\paragraph{What to show and say.}
\begin{itemize}
  \item Simulates an AR(1) and an MA(1) process and plots the series.
  \item Use it to explain 'memory of past values' (AR) vs 'memory of shocks' (MA).
  \item Connect to ACF/PACF intuition for order selection.
\end{itemize}

\subsection*{Demo Output (Example)}
\begin{center}
\IfFileExists{../images/demo.png}{
  \includegraphics[width=0.95\linewidth]{../images/demo.png}
}{
  \small (Run the demo to generate \texttt{images/demo.png})
}
\end{center}

\subsection*{Summary}
\begin{itemize}
  \item Key definitions and the main formula.
  \item How to interpret results in context.
  \item How the demo connects to the theory.
\end{itemize}

\subsection*{Exit Question}
How are AR and MA models different in what they remember?
\paragraph{Suggested answer (for revision).}
AR remembers past values; MA remembers past shocks (errors).

\section*{References}
\begin{itemize}
  \item Montgomery, D. C., \& Runger, G. C. \textit{Applied Statistics and Probability for Engineers}, Wiley.
  \item Devore, J. L. \textit{Probability and Statistics for Engineering and the Sciences}, Cengage.
  \item McKinney, W. \textit{Python for Data Analysis}, O'Reilly.
\end{itemize}

% BEGIN SLIDE APPENDIX (AUTO-GENERATED)
\clearpage
\section*{Appendix: Slide Deck Content (Reference)}
\noindent The material below is a reference copy of the slide deck content. Exercise solutions are explained in the main notes where applicable.

\subsection*{Title Slide}
\titlepage
        \vspace{-0.5em}
        \begin{center}
          \small \texttt{https://github.com/tali7c/Statistics-and-Data-Analysis}
        \end{center}
\subsection*{Quick Links}
\centering
        \textbf{Overview}\hspace{0.6em}
\textbf{AR}\hspace{0.6em}
\textbf{MA}\hspace{0.6em}
\textbf{Exercises}\hspace{0.6em}
\textbf{Demo}\hspace{0.6em}
\textbf{Summary}\hspace{0.6em}
\subsection*{Agenda}
\begin{itemize}
  \item Overview
  \item AR
  \item MA
  \item Exercises
  \item Demo
  \item Summary
\end{itemize}
\subsection*{Learning Outcomes}
\begin{itemize}
        \item Explain AR(p) model idea
\item Explain MA(q) model idea
\item Differentiate AR vs MA intuition
\item Define white noise (basic)
      \end{itemize}
\subsection*{AR: Key Points}
\begin{itemize}
        \item Current value depends on past values
\item AR(1): x\_t = c + phi x\_{t-1} + e\_t
\item Phi controls persistence
      \end{itemize}
\subsection*{AR: Key Formula}
\[ x_t = c + \phi x_{t-1} + \epsilon_t \]
\subsection*{MA: Key Points}
\begin{itemize}
        \item Current value depends on past shocks
\item MA(1): x\_t = mu + e\_t + theta e\_{t-1}
\item Captures short-term shock effects
      \end{itemize}
\subsection*{MA: Key Formula}
\[ x_t = \mu + \epsilon_t + \theta\epsilon_{t-1} \]
\subsection*{Exercise 1: AR intuition}
\small
  If phi=0.8 and last value is high (ignore noise), what happens next?
\subsection*{Solution 1}
\begin{itemize}
    \item Next value tends to be high too.
  \end{itemize}
\subsection*{Exercise 2: MA intuition}
\small
  In MA(1), what drives the series: past values or past shocks?
\subsection*{Solution 2}
\begin{itemize}
    \item Past shocks (errors).
  \end{itemize}
\subsection*{Exercise 3: White noise}
\small
  What is white noise?
\subsection*{Solution 3}
\begin{itemize}
    \item Uncorrelated errors with mean 0 and constant variance.
  \end{itemize}
\subsection*{Mini Demo (Python)}
Run from the lecture folder:
  \begin{center}
    \texttt{python demo/demo.py}
  \end{center}
  \vspace{0.4em}
  Outputs:
  \begin{itemize}
    \item \texttt{images/demo.png}
    \item \texttt{data/results.txt}
  \end{itemize}
\subsection*{Demo Output (Example)}
\begin{center}
  \IfFileExists{../images/demo.png}{
    \includegraphics[width=0.92\linewidth]{demo.png}
  }{
    \small (Run demo to generate: \texttt{demo.png})
  }
  \end{center}
\subsection*{Summary}
\begin{itemize}
        \item Key definitions and the main formula.
\item How to interpret results in context.
\item How the demo connects to the theory.
      \end{itemize}
\subsection*{Exit Question}
\small
  How are AR and MA models different in what they remember?
% END SLIDE APPENDIX (AUTO-GENERATED)

\end{document}
