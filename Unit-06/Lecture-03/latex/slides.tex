\documentclass{beamer}

      \usetheme{Berlin}
      \usecolortheme{Orchid}
      \useoutertheme{miniframes}
      \setbeamertemplate{navigation symbols}{}

      \usepackage{amsmath}
      \usepackage{amssymb}
      \usepackage{booktabs}
      \usepackage{graphicx}
      \graphicspath{{../images/}}

      \title[Statistics and Data Analysis]{Statistics and Data Analysis}
      \subtitle{Unit 06 -- Lecture 03: AR and MA Models}
      \author{Tofik Ali}
      \institute{School of Computer Science, UPES Dehradun}
      \date{\today}

      \begin{document}

      \begin{frame}
        \titlepage
        \vspace{-0.5em}
        \begin{center}
          \small \texttt{https://github.com/tali7c/Statistics-and-Data-Analysis}
        \end{center}
      \end{frame}

      \begin{frame}{Quick Links}
        \centering
        \hyperlink{sec:overview}{\beamerbutton{Overview}}\hspace{0.6em}
\hyperlink{sec:ar}{\beamerbutton{AR}}\hspace{0.6em}
\hyperlink{sec:ma}{\beamerbutton{MA}}\hspace{0.6em}
\hyperlink{sec:exercises}{\beamerbutton{Exercises}}\hspace{0.6em}
\hyperlink{sec:demo}{\beamerbutton{Demo}}\hspace{0.6em}
\hyperlink{sec:summary}{\beamerbutton{Summary}}\hspace{0.6em}
      \end{frame}

      \begin{frame}{Agenda}
        \tableofcontents
      \end{frame}

\section{Overview}
\label{sec:overview}


\begin{frame}{Learning Outcomes}
      \begin{itemize}[<+->]
        \item Explain AR(p) model idea
\item Explain MA(q) model idea
\item Differentiate AR vs MA intuition
\item Define white noise (basic)
      \end{itemize}
    \end{frame}

\section{AR}
\label{sec:ar}


\begin{frame}{AR: Key Points}
      \begin{itemize}[<+->]
        \item Current value depends on past values
\item AR(1): x\_t = c + phi x\_{t-1} + e\_t
\item Phi controls persistence
      \end{itemize}
    \end{frame}

\begin{frame}{AR: Key Formula}
  \[ x_t = c + \phi x_{t-1} + \epsilon_t \]
\end{frame}

\section{MA}
\label{sec:ma}


\begin{frame}{MA: Key Points}
      \begin{itemize}[<+->]
        \item Current value depends on past shocks
\item MA(1): x\_t = mu + e\_t + theta e\_{t-1}
\item Captures short-term shock effects
      \end{itemize}
    \end{frame}

\begin{frame}{MA: Key Formula}
  \[ x_t = \mu + \epsilon_t + \theta\epsilon_{t-1} \]
\end{frame}

\section{Exercises}
\label{sec:exercises}


\begin{frame}{Exercise 1: AR intuition}
  \small
  If phi=0.8 and last value is high (ignore noise), what happens next?
\end{frame}

\begin{frame}{Solution 1}
  \begin{itemize}
    \item Next value tends to be high too.
  \end{itemize}
\end{frame}

\begin{frame}{Exercise 2: MA intuition}
  \small
  In MA(1), what drives the series: past values or past shocks?
\end{frame}

\begin{frame}{Solution 2}
  \begin{itemize}
    \item Past shocks (errors).
  \end{itemize}
\end{frame}

\begin{frame}{Exercise 3: White noise}
  \small
  What is white noise?
\end{frame}

\begin{frame}{Solution 3}
  \begin{itemize}
    \item Uncorrelated errors with mean 0 and constant variance.
  \end{itemize}
\end{frame}

\section{Demo}
\label{sec:demo}


\begin{frame}{Mini Demo (Python)}
  Run from the lecture folder:
  \begin{center}
    \texttt{python demo/demo.py}
  \end{center}
  \vspace{0.4em}
  Outputs:
  \begin{itemize}
    \item \texttt{images/demo.png}
    \item \texttt{data/results.txt}
  \end{itemize}
\end{frame}

\begin{frame}{Demo Output (Example)}
  \begin{center}
  \IfFileExists{../images/demo.png}{
    \includegraphics[width=0.92\linewidth]{demo.png}
  }{
    \small (Run demo to generate: \texttt{demo.png})
  }
  \end{center}
\end{frame}

\section{Summary}
\label{sec:summary}


\begin{frame}{Summary}
      \begin{itemize}[<+->]
        \item Key definitions and the main formula.
\item How to interpret results in context.
\item How the demo connects to the theory.
      \end{itemize}
    \end{frame}

\begin{frame}{Exit Question}
  \small
  How are AR and MA models different in what they remember?
\end{frame}

\end{document}
