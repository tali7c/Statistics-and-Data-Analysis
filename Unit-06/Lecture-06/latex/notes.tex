\documentclass[11pt]{article}
\usepackage[utf8]{inputenc}
\usepackage[T1]{fontenc}
\usepackage{geometry}
\usepackage{amsmath}
\usepackage{booktabs}
\usepackage{hyperref}
\geometry{margin=1in}

\title{Statistics and Data Analysis\\Unit 06 -- Lecture 06 Notes}
\author{Tofik Ali}
\date{\today}

\begin{document}
\maketitle

\section*{Topic}
ADF test (unit root) and interpretation.

\subsection*{Learning Outcomes}
\begin{itemize}
  \item State null and alternative of ADF test (unit root)
  \item Interpret ADF p-value for stationarity decision
  \item Apply ADF to original and differenced series (idea)
  \item Explain why tests are not the only evidence (plots matter)
\end{itemize}

\section*{Detailed Notes}
These notes are designed to be read alongside the slides. They expand each slide bullet into
plain-language explanations, small worked examples, and common pitfalls. When a formula
appears, emphasize (1) what each symbol means, (2) the assumptions needed to use it, and (3)
how to interpret the final number in the problem context.

\section*{ADF Test}
\begin{itemize}
  \item H0: unit root (non-stationary)
  \item H1: stationary
  \item Small p-value -> reject H0
\end{itemize}

\section*{Interpretation}
\begin{itemize}
  \item If non-stationary, difference and test again
  \item Seasonality can require seasonal differencing
  \item Use ACF/PACF + diagnostics too
\end{itemize}

\section*{Exercises (with Solutions)}
\subsection*{Exercise 1: ADF null}
What is H0 in ADF?
\subsection*{Solution}
\begin{itemize}
  \item Unit root; non-stationary.
\end{itemize}

\subsection*{Exercise 2: Decision}
If p=0.02 at alpha=0.05, what do you conclude?
\subsection*{Solution}
\begin{itemize}
  \item Reject H0; evidence of stationarity.
\end{itemize}

\subsection*{Exercise 3: Next step}
If p=0.6, what next step?
\subsection*{Solution}
\begin{itemize}
  \item Difference and test again; consider seasonal differencing.
\end{itemize}

\section*{Exit Question}
Why should we not rely on only one test to decide stationarity?

\section*{Demo (Python)}
Run from the lecture folder:
\begin{verbatim}
python demo/demo.py
\end{verbatim}

Output files:
\begin{itemize}
  \item \texttt{images/demo.png}
  \item \texttt{data/results.txt}
\end{itemize}

\section*{References}
\begin{itemize}
  \item Montgomery, D. C., \& Runger, G. C. \textit{Applied Statistics and Probability for Engineers}, Wiley.
  \item Devore, J. L. \textit{Probability and Statistics for Engineering and the Sciences}, Cengage.
  \item McKinney, W. \textit{Python for Data Analysis}, O'Reilly.
\end{itemize}
\end{document}
