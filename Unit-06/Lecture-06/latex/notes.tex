\documentclass[11pt]{article}
\usepackage[utf8]{inputenc}
\usepackage[T1]{fontenc}
\usepackage{geometry}
\usepackage{amsmath}
\usepackage{listings}
\usepackage{xcolor}
\usepackage{amssymb}
\usepackage{booktabs}
\usepackage{graphicx}
\graphicspath{{../images/}}
\usepackage{hyperref}
\geometry{margin=1in}

\title{Statistics and Data Analysis\\Unit 06 -- Lecture 06 Notes\\ADF Test for Stationarity}
\author{Tofik Ali}
\date{\today}

\begin{document}
\maketitle

\section*{Topic}
ADF test (unit root) and interpretation.
\section*{How to Use These Notes}
These notes are written for students who are seeing the topic for the first time. They
follow the slide order, but add the missing 'why', interpretation, and common mistakes. If
you get stuck, look at the worked exercises and then run the Python demo.

Course repository (slides, demos, datasets): \url{https://github.com/tali7c/Statistics-and-Data-Analysis}

\section*{Time Plan (55 minutes)}
\begin{itemize}
  \item 0--10 min: Attendance + recap of previous lecture
  \item 10--35 min: Core concepts (this lecture's sections)
  \item 35--45 min: Exercises (solve 1--2 in class, rest as practice)
  \item 45--50 min: Mini demo + interpretation of output
  \item 50--55 min: Buffer / wrap-up (leave 5 minutes early)
\end{itemize}

\section*{Slide-by-slide Notes}
\subsection*{Title Slide}
State the lecture title clearly and connect it to what students already know.
Tell students what they will be able to do by the end (not just what you will cover).

\subsection*{Quick Links / Agenda}
Explain the structure of the lecture and where the exercises and demo appear.
\begin{itemize}
  \item Overview
  \item ADF Test
  \item Interpretation
  \item Exercises
  \item Demo
  \item Summary
\end{itemize}

\subsection*{Learning Outcomes}
\begin{itemize}
  \item State null and alternative of ADF test (unit root)
  \item Interpret ADF p-value for stationarity decision
  \item Apply ADF to original and differenced series (idea)
  \item Explain why tests are not the only evidence (plots matter)
\end{itemize}
\paragraph{Why these outcomes matter.}
\textbf{Degrees of freedom (df)} roughly represent how much independent information is
available to estimate variability. For a one-sample t-test, $\mathrm{df}=n-1$ because one
constraint is used to estimate the sample mean. df affects the critical values and the shape
of the t-distribution (small df -> heavier tails).
A \textbf{p-value} is computed assuming the null hypothesis $H_0$ is true. It measures how
surprising the observed data (or something more extreme) would be under $H_0$. A small
p-value suggests the data is hard to explain by $H_0$ alone, but it does not tell you how
large the effect is or whether it is practically important.

\subsection*{ADF Test: Key Points}
\begin{itemize}
  \item H0: unit root (non-stationary)
  \item H1: stationary
  \item Small p-value -> reject H0
\end{itemize}
\paragraph{Explanation.}
The \textbf{null hypothesis $H_0$} usually represents 'no effect' or a baseline value (e.g.,
$\mu=60$). The \textbf{alternative $H_1$} represents the effect you are looking for (e.g.,
$\mu\neq 60$ or $\mu>60$). We compute a test statistic and a p-value assuming $H_0$ is true.
\textbf{Degrees of freedom (df)} roughly represent how much independent information is
available to estimate variability. For a one-sample t-test, $\mathrm{df}=n-1$ because one
constraint is used to estimate the sample mean. df affects the critical values and the shape
of the t-distribution (small df -> heavier tails).
A \textbf{p-value} is computed assuming the null hypothesis $H_0$ is true. It measures how
surprising the observed data (or something more extreme) would be under $H_0$. A small
p-value suggests the data is hard to explain by $H_0$ alone, but it does not tell you how
large the effect is or whether it is practically important.

\subsection*{Interpretation: Key Points}
\begin{itemize}
  \item If non-stationary, difference and test again
  \item Seasonality can require seasonal differencing
  \item Use ACF/PACF + diagnostics too
\end{itemize}
\paragraph{Explanation.}
\textbf{Seasonality} is a repeating pattern with a fixed period (weekly, monthly, yearly).
You must account for it; otherwise forecasts systematically miss repeating rises/falls.
Seasonal differencing and SARIMA are common tools.
\textbf{Differencing} transforms a series by subtracting the previous value: $y_t -
y_{t-1}$. It removes trend and can help achieve stationarity. Over-differencing can add
noise, so use the smallest differencing order that works.
\textbf{Stationarity} (intuition) means the series behavior is stable over time: roughly
constant mean/variance and correlation structure. AR/MA/ARIMA models assume stationarity
(after differencing). If the process changes over time, parameters learned from the past may
not hold.

\subsection*{Exercises (with Solutions)}
Attempt the exercise first, then compare with the solution. Focus on interpretation, not
only arithmetic.

\subsection*{Exercise 1: ADF null}
What is H0 in ADF?
\subsubsection*{Solution}
\begin{itemize}
  \item Unit root; non-stationary.
\end{itemize}
\paragraph{Walkthrough.}
The \textbf{null hypothesis $H_0$} usually represents 'no effect' or a baseline value (e.g.,
$\mu=60$). The \textbf{alternative $H_1$} represents the effect you are looking for (e.g.,
$\mu\neq 60$ or $\mu>60$). We compute a test statistic and a p-value assuming $H_0$ is true.
\textbf{Degrees of freedom (df)} roughly represent how much independent information is
available to estimate variability. For a one-sample t-test, $\mathrm{df}=n-1$ because one
constraint is used to estimate the sample mean. df affects the critical values and the shape
of the t-distribution (small df -> heavier tails).

\subsection*{Exercise 2: Decision}
If p=0.02 at alpha=0.05, what do you conclude?
\subsubsection*{Solution}
\begin{itemize}
  \item Reject H0; evidence of stationarity.
\end{itemize}
\paragraph{Walkthrough.}
The \textbf{null hypothesis $H_0$} usually represents 'no effect' or a baseline value (e.g.,
$\mu=60$). The \textbf{alternative $H_1$} represents the effect you are looking for (e.g.,
$\mu\neq 60$ or $\mu>60$). We compute a test statistic and a p-value assuming $H_0$ is true.
The \textbf{significance level} $\alpha$ is the maximum Type I error rate you are willing to
tolerate: the probability of rejecting $H_0$ when $H_0$ is actually true. Common choices are
0.05 or 0.01, but the right value depends on consequences of false alarms vs missed
detections.

\subsection*{Exercise 3: Next step}
If p=0.6, what next step?
\subsubsection*{Solution}
\begin{itemize}
  \item Difference and test again; consider seasonal differencing.
\end{itemize}
\paragraph{Walkthrough.}
\textbf{Seasonality} is a repeating pattern with a fixed period (weekly, monthly, yearly).
You must account for it; otherwise forecasts systematically miss repeating rises/falls.
Seasonal differencing and SARIMA are common tools.
\textbf{Differencing} transforms a series by subtracting the previous value: $y_t -
y_{t-1}$. It removes trend and can help achieve stationarity. Over-differencing can add
noise, so use the smallest differencing order that works.

\subsection*{Mini Demo (Python)}
Run from the lecture folder:
\begin{verbatim}
python demo/demo.py
\end{verbatim}

Output files:
\begin{itemize}
  \item \texttt{images/demo.png}
  \item \texttt{data/results.txt}
\end{itemize}
\paragraph{What to show and say.}
\begin{itemize}
  \item Runs an ADF test on an example series and reports the p-value.
  \item Shows how conclusions can change after differencing.
  \item Use it to stress: combine tests with plots/ACF/PACF, not tests alone.
\end{itemize}

\subsection*{Demo Output (Example)}
\begin{center}
\IfFileExists{../images/demo.png}{
  \includegraphics[width=0.95\linewidth]{../images/demo.png}
}{
  \small (Run the demo to generate \texttt{images/demo.png})
}
\end{center}

\subsection*{Summary}
\begin{itemize}
  \item Key definitions and the main formula.
  \item How to interpret results in context.
  \item How the demo connects to the theory.
\end{itemize}

\subsection*{Exit Question}
Why should we not rely on only one test to decide stationarity?
\paragraph{Suggested answer (for revision).}
Plots and ACF/PACF provide context; a single test can be noisy/misleading under seasonality,
breaks, or short samples.

\section*{References}
\begin{itemize}
  \item Montgomery, D. C., \& Runger, G. C. \textit{Applied Statistics and Probability for Engineers}, Wiley.
  \item Devore, J. L. \textit{Probability and Statistics for Engineering and the Sciences}, Cengage.
  \item McKinney, W. \textit{Python for Data Analysis}, O'Reilly.
\end{itemize}

% BEGIN SLIDE APPENDIX (AUTO-GENERATED)
\clearpage
\section*{Appendix: Slide Deck Content (Reference)}
\noindent The material below is a reference copy of the slide deck content. Exercise solutions are explained in the main notes where applicable.

\subsection*{Title Slide}
\titlepage
        \vspace{-0.5em}
        \begin{center}
          \small \texttt{https://github.com/tali7c/Statistics-and-Data-Analysis}
        \end{center}
\subsection*{Quick Links}
\centering
        \textbf{Overview}\hspace{0.6em}
\textbf{ADF Test}\hspace{0.6em}
\textbf{Interpretation}\hspace{0.6em}
\textbf{Exercises}\hspace{0.6em}
\textbf{Demo}\hspace{0.6em}
\textbf{Summary}\hspace{0.6em}
\subsection*{Agenda}
\begin{itemize}
  \item Overview
  \item ADF Test
  \item Interpretation
  \item Exercises
  \item Demo
  \item Summary
\end{itemize}
\subsection*{Learning Outcomes}
\begin{itemize}
        \item State null and alternative of ADF test (unit root)
\item Interpret ADF p-value for stationarity decision
\item Apply ADF to original and differenced series (idea)
\item Explain why tests are not the only evidence (plots matter)
      \end{itemize}
\subsection*{ADF Test: Key Points}
\begin{itemize}
        \item H0: unit root (non-stationary)
\item H1: stationary
\item Small p-value -> reject H0
      \end{itemize}
\subsection*{Interpretation: Key Points}
\begin{itemize}
        \item If non-stationary, difference and test again
\item Seasonality can require seasonal differencing
\item Use ACF/PACF + diagnostics too
      \end{itemize}
\subsection*{Exercise 1: ADF null}
\small
  What is H0 in ADF?
\subsection*{Solution 1}
\begin{itemize}
    \item Unit root; non-stationary.
  \end{itemize}
\subsection*{Exercise 2: Decision}
\small
  If p=0.02 at alpha=0.05, what do you conclude?
\subsection*{Solution 2}
\begin{itemize}
    \item Reject H0; evidence of stationarity.
  \end{itemize}
\subsection*{Exercise 3: Next step}
\small
  If p=0.6, what next step?
\subsection*{Solution 3}
\begin{itemize}
    \item Difference and test again; consider seasonal differencing.
  \end{itemize}
\subsection*{Mini Demo (Python)}
Run from the lecture folder:
  \begin{center}
    \texttt{python demo/demo.py}
  \end{center}
  \vspace{0.4em}
  Outputs:
  \begin{itemize}
    \item \texttt{images/demo.png}
    \item \texttt{data/results.txt}
  \end{itemize}
\subsection*{Demo Output (Example)}
\begin{center}
  \IfFileExists{../images/demo.png}{
    \includegraphics[width=0.92\linewidth]{demo.png}
  }{
    \small (Run demo to generate: \texttt{demo.png})
  }
  \end{center}
\subsection*{Summary}
\begin{itemize}
        \item Key definitions and the main formula.
\item How to interpret results in context.
\item How the demo connects to the theory.
      \end{itemize}
\subsection*{Exit Question}
\small
  Why should we not rely on only one test to decide stationarity?
% END SLIDE APPENDIX (AUTO-GENERATED)

\end{document}
