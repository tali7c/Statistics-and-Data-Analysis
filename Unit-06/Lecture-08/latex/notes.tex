\documentclass[11pt]{article}
\usepackage[utf8]{inputenc}
\usepackage[T1]{fontenc}
\usepackage{geometry}
\usepackage{amsmath}
\usepackage{listings}
\usepackage{xcolor}
\usepackage{amssymb}
\usepackage{booktabs}
\usepackage{graphicx}
\graphicspath{{../images/}}
\usepackage{hyperref}
\geometry{margin=1in}

\title{Statistics and Data Analysis\\Unit 06 -- Lecture 08 Notes\\Diagnostics and SARIMA Models}
\author{Tofik Ali}
\date{\today}

\begin{document}
\maketitle

\section*{Topic}
Time-series diagnostics and SARIMA seasonal modeling (overview).
\section*{How to Use These Notes}
These notes are written for students who are seeing the topic for the first time. They
follow the slide order, but add the missing 'why', interpretation, and common mistakes. If
you get stuck, look at the worked exercises and then run the Python demo.

Course repository (slides, demos, datasets): \url{https://github.com/tali7c/Statistics-and-Data-Analysis}

\section*{Time Plan (55 minutes)}
\begin{itemize}
  \item 0--10 min: Attendance + recap of previous lecture
  \item 10--35 min: Core concepts (this lecture's sections)
  \item 35--45 min: Exercises (solve 1--2 in class, rest as practice)
  \item 45--50 min: Mini demo + interpretation of output
  \item 50--55 min: Buffer / wrap-up (leave 5 minutes early)
\end{itemize}

\section*{Slide-by-slide Notes}
\subsection*{Title Slide}
State the lecture title clearly and connect it to what students already know.
Tell students what they will be able to do by the end (not just what you will cover).

\subsection*{Quick Links / Agenda}
Explain the structure of the lecture and where the exercises and demo appear.
\begin{itemize}
  \item Overview
  \item Diagnostics
  \item SARIMA
  \item Exercises
  \item Demo
  \item Summary
\end{itemize}

\subsection*{Learning Outcomes}
\begin{itemize}
  \item Explain why diagnostics are needed after fitting
  \item Recognize residual autocorrelation as model issue
  \item Explain SARIMA seasonal terms at a high level
  \item Choose seasonal period s (weekly/monthly/yearly)
\end{itemize}
\paragraph{Why these outcomes matter.}
\textbf{Correlation} measures the strength of a linear association between two variables. It
is symmetric (no X/Y direction) and does not imply causation. Outliers can inflate or hide
correlation, so always look at the scatter plot.
A \textbf{residual} is $y - \hat{y}$. Residual plots tell you what the model failed to
explain. Patterns in residuals (trend, curvature, changing variance) are warnings that your
model form is inadequate or assumptions are violated.

\subsection*{Diagnostics: Key Points}
\begin{itemize}
  \item Residuals should look like white noise
  \item Check residual ACF
  \item Check stability of variance
\end{itemize}
\paragraph{Explanation.}
A \textbf{residual} is $y - \hat{y}$. Residual plots tell you what the model failed to
explain. Patterns in residuals (trend, curvature, changing variance) are warnings that your
model form is inadequate or assumptions are violated.
\textbf{ACF} shows correlation of the series with its lagged versions. It helps identify MA-
like behavior and seasonality. \textbf{PACF} shows the correlation at a lag after removing
shorter-lag effects and helps identify AR-like behavior.

\subsection*{SARIMA: Key Points}
\begin{itemize}
  \item ARIMA(p,d,q) x (P,D,Q,s)
  \item Seasonal differencing D
  \item s is the seasonal period
\end{itemize}
\paragraph{Explanation.}
\textbf{Seasonality} is a repeating pattern with a fixed period (weekly, monthly, yearly).
You must account for it; otherwise forecasts systematically miss repeating rises/falls.
Seasonal differencing and SARIMA are common tools.
\textbf{Differencing} transforms a series by subtracting the previous value: $y_t -
y_{t-1}$. It removes trend and can help achieve stationarity. Over-differencing can add
noise, so use the smallest differencing order that works.
\textbf{SARIMA} extends ARIMA by adding seasonal AR/MA terms and seasonal differencing. It
is used when patterns repeat every $s$ steps (e.g., weekly seasonality with daily data).
Diagnostics on residuals are important: residuals should look like white noise.

\subsection*{Exercises (with Solutions)}
Attempt the exercise first, then compare with the solution. Focus on interpretation, not
only arithmetic.

\subsection*{Exercise 1: Residual goal}
After fitting, what should residuals look like ideally?
\subsubsection*{Solution}
\begin{itemize}
  \item White noise: no pattern, no autocorrelation.
\end{itemize}
\paragraph{Walkthrough.}
\textbf{Correlation} measures the strength of a linear association between two variables. It
is symmetric (no X/Y direction) and does not imply causation. Outliers can inflate or hide
correlation, so always look at the scatter plot.
A \textbf{residual} is $y - \hat{y}$. Residual plots tell you what the model failed to
explain. Patterns in residuals (trend, curvature, changing variance) are warnings that your
model form is inadequate or assumptions are violated.

\subsection*{Exercise 2: Seasonal period}
Daily data with weekly seasonality: what is s?
\subsubsection*{Solution}
\begin{itemize}
  \item s=7
\end{itemize}
\paragraph{Walkthrough.}
\textbf{Seasonality} is a repeating pattern with a fixed period (weekly, monthly, yearly).
You must account for it; otherwise forecasts systematically miss repeating rises/falls.
Seasonal differencing and SARIMA are common tools.

\subsection*{Exercise 3: Why seasonal}
Why add seasonal terms?
\subsubsection*{Solution}
\begin{itemize}
  \item To capture repeating seasonal dependence.
\end{itemize}
\paragraph{Walkthrough.}
\textbf{Seasonality} is a repeating pattern with a fixed period (weekly, monthly, yearly).
You must account for it; otherwise forecasts systematically miss repeating rises/falls.
Seasonal differencing and SARIMA are common tools.

\subsection*{Mini Demo (Python)}
Run from the lecture folder:
\begin{verbatim}
python demo/demo.py
\end{verbatim}

Output files:
\begin{itemize}
  \item \texttt{images/demo.png}
  \item \texttt{data/results.txt}
\end{itemize}
\paragraph{What to show and say.}
\begin{itemize}
  \item Fits a seasonal model idea and checks residual diagnostics conceptually.
  \item Shows how seasonal period s (e.g., 7 for weekly) enters the model.
  \item Use residual ACF idea to explain when the model is still missing structure.
\end{itemize}

\subsection*{Demo Output (Example)}
\begin{center}
\IfFileExists{../images/demo.png}{
  \includegraphics[width=0.95\linewidth]{../images/demo.png}
}{
  \small (Run the demo to generate \texttt{images/demo.png})
}
\end{center}

\subsection*{Summary}
\begin{itemize}
  \item Key definitions and the main formula.
  \item How to interpret results in context.
  \item How the demo connects to the theory.
\end{itemize}

\subsection*{Exit Question}
What is one residual symptom that suggests your model is inadequate?
\paragraph{Suggested answer (for revision).}
Residual autocorrelation (or visible patterns) suggests the model is missing structure; add
terms/seasonality until residuals resemble white noise.

\section*{References}
\begin{itemize}
  \item Montgomery, D. C., \& Runger, G. C. \textit{Applied Statistics and Probability for Engineers}, Wiley.
  \item Devore, J. L. \textit{Probability and Statistics for Engineering and the Sciences}, Cengage.
  \item McKinney, W. \textit{Python for Data Analysis}, O'Reilly.
\end{itemize}

% BEGIN SLIDE APPENDIX (AUTO-GENERATED)
\clearpage
\section*{Appendix: Slide Deck Content (Reference)}
\noindent The material below is a reference copy of the slide deck content. Exercise solutions are explained in the main notes where applicable.

\subsection*{Title Slide}
\titlepage
        \vspace{-0.5em}
        \begin{center}
          \small \texttt{https://github.com/tali7c/Statistics-and-Data-Analysis}
        \end{center}
\subsection*{Quick Links}
\centering
        \textbf{Overview}\hspace{0.6em}
\textbf{Diagnostics}\hspace{0.6em}
\textbf{SARIMA}\hspace{0.6em}
\textbf{Exercises}\hspace{0.6em}
\textbf{Demo}\hspace{0.6em}
\textbf{Summary}\hspace{0.6em}
\subsection*{Agenda}
\begin{itemize}
  \item Overview
  \item Diagnostics
  \item SARIMA
  \item Exercises
  \item Demo
  \item Summary
\end{itemize}
\subsection*{Learning Outcomes}
\begin{itemize}
        \item Explain why diagnostics are needed after fitting
\item Recognize residual autocorrelation as model issue
\item Explain SARIMA seasonal terms at a high level
\item Choose seasonal period s (weekly/monthly/yearly)
      \end{itemize}
\subsection*{Diagnostics: Key Points}
\begin{itemize}
        \item Residuals should look like white noise
\item Check residual ACF
\item Check stability of variance
      \end{itemize}
\subsection*{SARIMA: Key Points}
\begin{itemize}
        \item ARIMA(p,d,q) x (P,D,Q,s)
\item Seasonal differencing D
\item s is the seasonal period
      \end{itemize}
\subsection*{Exercise 1: Residual goal}
\small
  After fitting, what should residuals look like ideally?
\subsection*{Solution 1}
\begin{itemize}
    \item White noise: no pattern, no autocorrelation.
  \end{itemize}
\subsection*{Exercise 2: Seasonal period}
\small
  Daily data with weekly seasonality: what is s?
\subsection*{Solution 2}
\begin{itemize}
    \item s=7
  \end{itemize}
\subsection*{Exercise 3: Why seasonal}
\small
  Why add seasonal terms?
\subsection*{Solution 3}
\begin{itemize}
    \item To capture repeating seasonal dependence.
  \end{itemize}
\subsection*{Mini Demo (Python)}
Run from the lecture folder:
  \begin{center}
    \texttt{python demo/demo.py}
  \end{center}
  \vspace{0.4em}
  Outputs:
  \begin{itemize}
    \item \texttt{images/demo.png}
    \item \texttt{data/results.txt}
  \end{itemize}
\subsection*{Demo Output (Example)}
\begin{center}
  \IfFileExists{../images/demo.png}{
    \includegraphics[width=0.92\linewidth]{demo.png}
  }{
    \small (Run demo to generate: \texttt{demo.png})
  }
  \end{center}
\subsection*{Summary}
\begin{itemize}
        \item Key definitions and the main formula.
\item How to interpret results in context.
\item How the demo connects to the theory.
      \end{itemize}
\subsection*{Exit Question}
\small
  What is one residual symptom that suggests your model is inadequate?
% END SLIDE APPENDIX (AUTO-GENERATED)

\end{document}
