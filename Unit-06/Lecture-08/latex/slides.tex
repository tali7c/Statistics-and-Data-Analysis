\documentclass{beamer}

      \usetheme{Berlin}
      \usecolortheme{Orchid}
      \useoutertheme{miniframes}
      \setbeamertemplate{navigation symbols}{}

      \usepackage{amsmath}
      \usepackage{amssymb}
      \usepackage{booktabs}
      \usepackage{graphicx}
      \graphicspath{{../images/}}

      \title[Statistics and Data Analysis]{Statistics and Data Analysis}
      \subtitle{Unit 06 -- Lecture 08: Diagnostics and SARIMA Models}
      \author{Tofik Ali}
      \institute{School of Computer Science, UPES Dehradun}
      \date{\today}

      \begin{document}

      \begin{frame}
        \titlepage
        \vspace{-0.5em}
        \begin{center}
          \small \texttt{https://github.com/tali7c/Statistics-and-Data-Analysis}
        \end{center}
      \end{frame}

      \begin{frame}{Quick Links}
        \centering
        \hyperlink{sec:overview}{\beamerbutton{Overview}}\hspace{0.6em}
\hyperlink{sec:diag}{\beamerbutton{Diagnostics}}\hspace{0.6em}
\hyperlink{sec:sarima}{\beamerbutton{SARIMA}}\hspace{0.6em}
\hyperlink{sec:exercises}{\beamerbutton{Exercises}}\hspace{0.6em}
\hyperlink{sec:demo}{\beamerbutton{Demo}}\hspace{0.6em}
\hyperlink{sec:summary}{\beamerbutton{Summary}}\hspace{0.6em}
      \end{frame}

      \begin{frame}{Agenda}
        \tableofcontents
      \end{frame}

\section{Overview}
\label{sec:overview}


\begin{frame}{Learning Outcomes}
      \begin{itemize}[<+->]
        \item Explain why diagnostics are needed after fitting
\item Recognize residual autocorrelation as model issue
\item Explain SARIMA seasonal terms at a high level
\item Choose seasonal period s (weekly/monthly/yearly)
      \end{itemize}
    \end{frame}

\section{Diagnostics}
\label{sec:diag}


\begin{frame}{Diagnostics: Key Points}
      \begin{itemize}[<+->]
        \item Residuals should look like white noise
\item Check residual ACF
\item Check stability of variance
      \end{itemize}
    \end{frame}

\section{SARIMA}
\label{sec:sarima}


\begin{frame}{SARIMA: Key Points}
      \begin{itemize}[<+->]
        \item ARIMA(p,d,q) x (P,D,Q,s)
\item Seasonal differencing D
\item s is the seasonal period
      \end{itemize}
    \end{frame}

\section{Exercises}
\label{sec:exercises}


\begin{frame}{Exercise 1: Residual goal}
  \small
  After fitting, what should residuals look like ideally?
\end{frame}

\begin{frame}{Solution 1}
  \begin{itemize}
    \item White noise: no pattern, no autocorrelation.
  \end{itemize}
\end{frame}

\begin{frame}{Exercise 2: Seasonal period}
  \small
  Daily data with weekly seasonality: what is s?
\end{frame}

\begin{frame}{Solution 2}
  \begin{itemize}
    \item s=7
  \end{itemize}
\end{frame}

\begin{frame}{Exercise 3: Why seasonal}
  \small
  Why add seasonal terms?
\end{frame}

\begin{frame}{Solution 3}
  \begin{itemize}
    \item To capture repeating seasonal dependence.
  \end{itemize}
\end{frame}

\section{Demo}
\label{sec:demo}


\begin{frame}{Mini Demo (Python)}
  Run from the lecture folder:
  \begin{center}
    \texttt{python demo/demo.py}
  \end{center}
  \vspace{0.4em}
  Outputs:
  \begin{itemize}
    \item \texttt{images/demo.png}
    \item \texttt{data/results.txt}
  \end{itemize}
\end{frame}

\begin{frame}{Demo Output (Example)}
  \begin{center}
  \IfFileExists{../images/demo.png}{
    \includegraphics[width=0.92\linewidth]{demo.png}
  }{
    \small (Run demo to generate: \texttt{demo.png})
  }
  \end{center}
\end{frame}

\section{Summary}
\label{sec:summary}


\begin{frame}{Summary}
      \begin{itemize}[<+->]
        \item Key definitions and the main formula.
\item How to interpret results in context.
\item How the demo connects to the theory.
      \end{itemize}
    \end{frame}

\begin{frame}{Exit Question}
  \small
  What is one residual symptom that suggests your model is inadequate?
\end{frame}

\end{document}
